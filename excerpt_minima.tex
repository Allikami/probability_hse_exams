\documentclass[12pt]{article}

\usepackage{tikz} % картинки в tikz
\usepackage{microtype} % свешивание пунктуации

\usepackage{array} % для столбцов фиксированной ширины

\usepackage{indentfirst} % отступ в первом параграфе

\usepackage{sectsty} % для центрирования названий частей
\allsectionsfont{\centering}

\usepackage{amsmath} % куча стандартных математических плюшек

\usepackage{comment}
\usepackage{amsfonts}

\usepackage[top=2cm, left=1.2cm, right=1.2cm, bottom=2cm]{geometry} % размер текста на странице

\usepackage{lastpage} % чтобы узнать номер последней страницы

\usepackage{enumitem} % дополнительные плюшки для списков
%  например \begin{enumerate}[resume] позволяет продолжить нумерацию в новом списке
\usepackage{caption}

\usepackage{hyperref} % гиперссылки

\usepackage{multicol} % текст в несколько столбцов


\usepackage{fancyhdr} % весёлые колонтитулы
\pagestyle{fancy}
\lhead{Теория вероятностей}
\chead{}
\rhead{Минимумы к контрольным по ТВ и МС}
\lfoot{}
\cfoot{}
\rfoot{\thepage/\pageref{LastPage}}
\renewcommand{\headrulewidth}{0.4pt}
\renewcommand{\footrulewidth}{0.4pt}



\usepackage{todonotes} % для вставки в документ заметок о том, что осталось сделать
% \todo{Здесь надо коэффициенты исправить}
% \missingfigure{Здесь будет Последний день Помпеи}
% \listoftodos --- печатает все поставленные \todo'шки


% более красивые таблицы
\usepackage{booktabs}
% заповеди из докупентации:
% 1. Не используйте вертикальные линни
% 2. Не используйте двойные линии
% 3. Единицы измерения - в шапку таблицы
% 4. Не сокращайте .1 вместо 0.1
% 5. Повторяющееся значение повторяйте, а не говорите "то же"

\usepackage{ifthen} % для написания условий

\newboolean{excerpt}
\setboolean{excerpt}{true} % мы компилируем большой документ


\usepackage{fontspec}
\usepackage{polyglossia}

\setmainlanguage{russian}
\setotherlanguages{english}

% download "Linux Libertine" fonts:
% http://www.linuxlibertine.org/index.php?id=91&L=1
\setmainfont{Linux Libertine O} % or Helvetica, Arial, Cambria
% why do we need \newfontfamily:
% http://tex.stackexchange.com/questions/91507/
\newfontfamily{\cyrillicfonttt}{Linux Libertine O}

\AddEnumerateCounter{\asbuk}{\russian@alph}{щ} % для списков с русскими буквами
\setlist[enumerate, 2]{label=\asbuk*),ref=\asbuk*}

%% эконометрические сокращения
\DeclareMathOperator{\Cov}{Cov}
\DeclareMathOperator{\Corr}{Corr}
\DeclareMathOperator{\Var}{Var}
\DeclareMathOperator{\E}{E}
\def \hb{\hat{\beta}}
\def \hs{\hat{\sigma}}
\def \htheta{\hat{\theta}}
\def \s{\sigma}
\def \hy{\hat{y}}
\def \hY{\hat{Y}}
\def \v1{\vec{1}}
\def \e{\varepsilon}
\def \he{\hat{\e}}
\def \z{z}
\def \hVar{\widehat{\Var}}
\def \hCorr{\widehat{\Corr}}
\def \hCov{\widehat{\Cov}}
\def \cN{\mathcal{N}}
\def \P{\mathbb{P}}


\begin{document}

\section{Минимумы}



\subsection[Минимум к кр 1]{\hyperref[sec:sol_minimum_kr_01]{Минимум к кр 1}}
\label{sec:minimum_kr_01}

\subsubsection*{Теоретический минимум}


\begin{enumerate}
	\item Классическое определение вероятности
	\item Определение условной вероятности
	\item Определение независимости случайных событий
	\item Формула полной вероятности
	\item Формула Байеса
	\item Функция распределения случайной величины. Определение и свойства.
	\item Функция плотности. Определение и свойства.
	\item Математическое ожидание. Определения для дискретного и абсолютно непрерывного случаев. Свойства.
	\item Дисперсия. Определение и свойства.
	\item Законы распределений. Определение, $\E(X)$, $\Var(X)$:
	\begin{enumerate}
	\item Биномиальное распределение
	\item Распределение Пуассона
	\item Геометрическое распределение
	\item Равномерное распределение
	\item Экспоненциальное распределение
	\end{enumerate}
\end{enumerate}


\subsubsection*{Задачный минимум}

\begin{enumerate}
\item  Пусть $\P(A) = 0.3, \P(B) = 0.4, \P(A\cap B) = 0.1 $. Найдите
	\begin{enumerate}
		\item  $\P(A|B)$
		\item  $\P(A\cup B)$
		\item  Являются ли события $A$ и $B$ независимыми?
	\end{enumerate}



\item  Пусть $\P(A) = 0.5, \P(B) = 0.5, \P(A\cap B) = 0.25 $. Найдите
\begin{enumerate}
	\item  $\P(A|B)$
	\item  $\P(A\cup B)$
	\item  Являются ли события $A$ и $B$ независимыми?
\end{enumerate}



\item  Карлсон выложил кубиками слово КОМБИНАТОРИКА. Малыш выбирает наугад четыре кубика и выкладывает их в случайном порядке.
Найдите вероятность того, что при этом получится слово КОРТ.


\item  Карлсон выложил кубиками слово КОМБИНАТОРИКА. Малыш выбирает наугад четыре кубика и выкладывает их в случайном порядке.
Найдите вероятность того, что при этом получится слово РОТА.

\item  В первой урне 7 белых и 3 черных шара, во второй урне 8 белых и 4 черных
шара, в третьей урне 2 белых и 13 черных шаров. Из этих урн наугад выбирается одна урна. Какова вероятность того, что шар, взятый наугад из выбранной урны, окажется белым?


\item  В первой урне 7 белых и 3 черных шара, во второй урне 8 белых и 4 черных
шара, в третьей урне 2 белых и 13 черных шаров. Из этих урн наугад выбирается одна урна. Какова вероятность того, что была выбрана первая урна, если шар, взятый наугад из выбранной урны, оказался белым?


\item  В операционном отделе банка работает 80\% опытных сотрудников и 20\%
неопытных. Вероятность совершения ошибки при очередной банковской операции
опытным сотрудником равна 0.01, а неопытным — 0.1. Найдите вероятность совершения ошибки при очередной банковской операции в этом отделе.


\item  В операционном отделе банка работает 80\% опытных сотрудников и 20\%
неопытных. Вероятность совершения ошибки при очередной банковской операции
опытным сотрудником равна 0.01, а неопытным — 0.1. Известно, что при очередной банковской операции была допущена ошибка. Найдите вероятность того, что ошибку допустил неопытный сотрудник.

\item  Пусть случайная величина $X$ имеет таблицу распределения:

\begin{tabular}{ ll l l}
	\toprule
	$X$ & -1  & 0  & 1 \\
	$\P_X$ & 0.25  & c  & 0.25 \\
  \bottomrule
\end{tabular}

Найдите
	\begin{enumerate}
	\item константу $c$
	\item $\P(\{X \geq 0\})$
	\item $\P(\{X < -3\}])$
	\item $\P(\{X \in [-\frac{1}{2}; \frac{1}{2}]\})$
	\item функцию распределения случайной величины $X$
	\item имеет ли случайная величина $X$ плотность распределения?
	\end{enumerate}


\item  Пусть случайная величина $X$ имеет таблицу распределения:

\begin{tabular}{ llll}
\toprule
$X$ & -1  & 0  & 1 \\
$\P_X$ & 0.25  & c  & 0.25 \\
\bottomrule
\end{tabular}

Найдите
\begin{enumerate}
	\item константу $c$
	\item $\E(X)$
	\item $\E(X^2)$
	\item $\Var(X)$
	\item $\E(|X|)$
\end{enumerate}

\item  Пусть случайная величина $X$ имеет таблицу распределения:

\begin{tabular}{ lll l}
\toprule
$X$ & -1  & 0  & 1 \\
$\P_X$ & 0.25  & c  & 0.5 \\
\bottomrule
\end{tabular}

Найдите
	\begin{enumerate}
	\item константу $c$
	\item $\P(\{X \geq 0\})$
	\item $\P(\{X < -3\}])$
	\item $\P(\{X \in [-\frac{1}{2}; \frac{1}{2}]\})$
	\item функцию распределения случайной величины $X$
	\item имеет ли случайная величина $X$ плотность распределения?
\end{enumerate}

\item  Пусть случайная величина $X$ имеет таблицу распределения:

\begin{tabular}{ l l l l}
  \toprule
$X$ & -1  & 0  & 1 \\
$\P_X$ & 0.25  & c  & 0.5 \\
\bottomrule
\end{tabular}

Найдите
\begin{enumerate}
	\item константу $c$
	\item $\E(X)$
	\item $\E(X^2)$
	\item $\Var(X)$
	\item $\E(|X|)$
\end{enumerate}

\item Пусть случайная величина $X$ имеет биномиальное распределение с
параметрами $n = 4$ и $\P = \frac{3}{4}$.
 Найдите
\begin{enumerate}
	\item $\P(\{X = 0\})$
	\item $\P(\{X > 0\})$
	\item $\P(\{X < 0\})$
	\item $\E(X)$
	\item $\Var(X)$
	\item  наиболее вероятное значение, которое принимает случайная величина $X$
\end{enumerate}

\item Пусть случайная величина $X$ имеет биномиальное распределение с
параметрами $n = 5$ и $\P = \frac{2}{5}$.
Найдите
\begin{enumerate}
	\item $\P(\{X = 0\})$
	\item $\P(\{X > 0\})$
	\item $\P(\{X < 0\})$
	\item $\E(X)$
	\item $\Var(X)$
	\item  наиболее вероятное значение, которое принимает случайная величина $X$
\end{enumerate}


\item  Пусть случайная величина X имеет распределение Пуассона с параметром $\lambda = 100$ . Найдите
\begin{enumerate}
	\item $\P(\{X = 0\})$
	\item $\P(\{X > 0\})$
	\item $\P(\{X < 0\})$
	\item $\E(X)$
	\item $\Var(X)$
	\item  наиболее вероятное значение, которое принимает случайная величина $X$
\end{enumerate}


\item  Пусть случайная величина X имеет распределение Пуассона с параметром $\lambda = 101$ . Найдите
\begin{enumerate}
	\item $\P(\{X = 0\})$
	\item $\P(\{X > 0\})$
	\item $\P(\{X < 0\})$
	\item $\E(X)$
	\item $\Var(X)$
	\item  наиболее вероятное значение, которое принимает случайная величина $X$
\end{enumerate}


\item В лифт 10-этажного дома на первом этаже вошли 5 человек. Вычислите
вероятность того, что на 6-м этаже выйдет хотя бы один человек.


\item В лифт 10-этажного дома на первом этаже вошли 5 человек. Вычислите
вероятность того, что на 6-м этаже не выйдет ни один человек.


\item При работе некоторого устройства время от времени возникают сбои.
Количество сбоев за сутки имеет распределение Пуассона. Среднее количество сбоев за сутки равно 3. Найти вероятность того, что в течение суток произойдет хотя бы один сбой.


\item При работе некоторого устройства время от времени возникают сбои.
Количество сбоев за сутки имеет распределение Пуассона. Среднее количество сбоев за сутки равно 3. Найти вероятность того, что за двое суток не произойдет ни одного сбоя.


\item Пусть случайная величина $X$ имеет плотность распределения

\[
f_X(x) =
	\begin{cases}
	c,\text{ при }  x \in [-1; 1] \\
	0,\text{ при } x \notin  [-1; 1] \\
	\end{cases}
\]

Найдите
\begin{enumerate}
	\item константу $c$
	\item $\P(\{X \leq 0\})$
	\item $\P(\{X \in [\frac{1}{2}; \frac{3}{2}]\})$
	\item $\P(\{X \in [2;3]\}$
	\item $F_X(x)$
\end{enumerate}


\item Пусть случайная величина $X$ имеет плотность распределения

\[
f_X(x) =
	\begin{cases}
	c,\text{ при }  x \in [-1; 1] \\
	0,\text{ при } x \notin  [-1; 1] \\
	\end{cases}
\]

Найдите
\begin{enumerate}
	\item константу $c$
	\item $\E(X)$
	\item $\E(X^2)$
	\item $\Var(X)$
	\item $\E(|X|)$
\end{enumerate}


\item Пусть случайная величина $X$ имеет плотность распределения

\[
f_X(x) =
	\begin{cases}
	cx,\text{ при }  x \in [0; 1] \\
	0,\text{ при } x \notin  [0; 1] \\
	\end{cases}
\]

Найдите
\begin{enumerate}
	\item константу $c$
	\item $\P(\{X \leq \frac{1}{2}\})$
	\item $\P(\{X \in [\frac{1}{2}; \frac{3}{2}]\})$
	\item $\P(\{X \in [2;3]\}$
	\item $F_X(x)$
\end{enumerate}


\item Пусть случайная величина $X$ имеет плотность распределения

\[
f_X(x) =
	\begin{cases}
	cx,\text{ при }  x \in [0; 1] \\
	0,\text{ при } x \notin  [0; 1] \\
	\end{cases}
\]

Найдите
\begin{enumerate}
	\item константу $c$
	\item $\E(X)$
	\item $\E(X^2)$
	\item $\Var(X)$
	\item $\E(\sqrt{X})$
\end{enumerate}
\end{enumerate}

\thispagestyle{empty}
\section{Ответы к минимумам}

\subsection[Кр 1]{\hyperref[sec:minimum_kr_01]{Контрольная работа 1 — Задачный минимум}}
\label{sec:sol_minimum_kr_01}



\begin{multicols}{2}
\begin{enumerate}
	\item
			\begin{enumerate}
				\item $0.25$
				\item $0.6$
				\item нет
			\end{enumerate}
	\item
			\begin{enumerate}
				\item $0.5$
				\item $0.75$
				\item нет
			\end{enumerate}
	\item $\frac{4}{10 \cdot 11 \cdot 12 \cdot 13}$
	\item $\frac{4}{10 \cdot 11 \cdot 12 \cdot 13}$
	\item $0.5$
	\item $0.42$
	\item $0.028$
	\item $\frac{5}{7}$
	\item
			\begin{enumerate}
				\item $0.5$
				\item $0.75$
				\item $0$
				\item $0.5$
			\end{enumerate}
	\item
			\begin{enumerate}
				\item $0.5$
				\item $0$
				\item $0.5$
				\item $0.5$
				\item $0.5$
			\end{enumerate}
	\item
			\begin{enumerate}
				\item $0.25$
				\item $0.75$
				\item $0$
				\item $0.5$
			\end{enumerate}
	\item
			\begin{enumerate}
				\item $0.25$
				\item $0.25$
				\item $0.75$
				\item $0.5$
				\item $0.75$
			\end{enumerate}
	\item
			\begin{enumerate}
				\item $\left( \frac{1}{4} \right) ^4$
				\item $1 - \left( \frac{1}{4} \right) ^4$
				\item $0$
				\item $3$
				\item $0.75$
				\item $2$, $3$
			\end{enumerate}
	\item
			\begin{enumerate}
				\item $\left( \frac{3}{5} \right) ^5$
				\item $1 - \left( \frac{3}{5} \right) ^5$
				\item $0$
				\item $2$
				\item $1.2$
				\item $2$
			\end{enumerate}
	\item
			\begin{enumerate}
				\item $e^{-100}$
				\item $1 - e^{-100}$
				\item $0$
				\item $100$
				\item $100$
			\end{enumerate}
	\item
			\begin{enumerate}
				\item $e^{-101}$
				\item $1 - e^{-101}$
				\item $0$
				\item $101$
				\item $101$
			\end{enumerate}
	\item $1 - \frac{8^5}{9^5}$
	\item $\frac{8^5}{9^5}$
	\item $1 - e^{-3}$
	\item $e^{-3}$
	\item
			\begin{enumerate}
				\item $0.5$
				\item $0.25$
				\item $0.125$
				\item $1$
			\end{enumerate}
	\item
			\begin{enumerate}
				\item $0.5$
				\item $0.5$
				\item $\frac{1}{3}$
				\item $\frac{1}{12}$
				\item $1$
			\end{enumerate}
	\item
			\begin{enumerate}
				\item $2$
				\item $0.25$
				\item $\frac{3}{4}$
				\item $1$
			\end{enumerate}
	\item
			\begin{enumerate}
				\item $2$
				\item $0.5$
				\item $0.5$
				\item $0$
				\item $0.8$
			\end{enumerate}
\end{enumerate}
\end{multicols}


\subsection[Кр 2]{\hyperref[sec:minimum_kr_02]{Контрольная работа 2 — Задачный минимум}}
\label{sec:sol_minimum_kr_02}


\begin{multicols}{2}
\begin{enumerate}

\item
\begin{enumerate}
\item   $0.5 $
\item   $0.3$
\item   $0.2$
\item   нет
\item   $0.3$
\item
\begin{tabular}{lrr}
\toprule
$x$ & $-1$  & $1$   \\ \midrule
$\P(X=x)$ & $0.5$ & $0.5$ \\ \bottomrule
\end{tabular}
\item  $F_{X}(x) = \begin{cases}
0, & \text{при } x < -1 \\
0.5 , & \text{при } x \in [-1;1) \\
1, & \text{при }  x \geq 1
\end{cases}$
\end{enumerate}
\item
\begin{enumerate}
\item   $0.5$
\item   $0.4$
\item   $0.2$
\item   да
\item   $0.6$
\item
\begin{tabular}{lrrr}
\toprule
$y$ & $-1$  & $0$   & $1$   \\ \midrule
$\P(Y=y)$ & $0.4$ & $0.2$ & $0.4$ \\ \bottomrule
\end{tabular}
\item   $F_{Y}(y) = \begin{cases}
0, & \text{при } y < -1 \\
0.4 , & \text{при } y \in [-1;0) \\
0.6, & \text{при }  y \in [0;1)\\
1, & \text{при } y \geq 1
\end{cases}$
\end{enumerate}

\item
\begin{enumerate}
\item $0$
\item $1$
\item $1$
\item $0$
\item $0.6$
\item $0.6$
\item $0$
\item $0$
\item $0$
\item да, являются некоррелированными, но нельзя утверждать, что являются независимыми
\end{enumerate}

\item
\begin{enumerate}
\item $0$
\item $1$
\item $1$
\item $0$
\item $0.8$
\item $0.8$
\item $0$
\item $0$
\item $0$
\item да, являются некоррелированными, но нельзя утверждать, что являются независимыми
\end{enumerate}

\item
\begin{enumerate}
\item $0.25$
\item $0.2$
\item Обозначим $A = \{X = -1\}$

\begin{tabular}{lrrr}
\toprule
$y$           & $-1$  & $0$   & $1$   \\ \midrule
$\P(Y=y|A)$   & $0.4$ & $0.2$ & $0.4$ \\ \bottomrule
\end{tabular}
\item $0$
\item $0.8 $
\end{enumerate}
\item
\begin{enumerate}
\item $0.5$
\item $0.2$
\item Обозначим $A = \{X = 1\}$
\begin{tabular}{lrrr}
\toprule
$y$ & $-1$  & $0$   & $1$   \\ \midrule
$\P(Y=y|A)$             & $0.4$ & $0.2$ & $0.4$ \\ \bottomrule
\end{tabular}
\item $0$
\item $0.8$
\end{enumerate}

\item
\begin{enumerate}
\item $0 $
\item $36$
\item $9 $
\item $60 $
\item $-4$
\item $\frac{-1}{3\sqrt{5}}$
\item $\begin{pmatrix}
 3 & -1 \\
-1 & 4
\end{pmatrix}$
\end{enumerate}

\item
\begin{enumerate}
\item $-4$
\item $8 $
\item $1 $
\item $10$
\item $-6$
\item$ \frac{-1}{\sqrt{5}}$

\item $\begin{pmatrix}
 1 & 1 \\
 1 & 2
\end{pmatrix}$
\end{enumerate}
\item
\begin{enumerate}
\item $0.3413$
\item $0.0228$
\item $0.1915$
\end{enumerate}

\item
\begin{enumerate}
\item $0.6826$
\item $0.0228$
\item $0.1574$
\end{enumerate}

\item $0.4332$
\item $0.8185$
\item $0.4514$
\item $0.5328$
\item $\approx 0.8185$
\item $\approx 0.9115$
\item $\approx 0.6422$
\item $\approx 0.9606$

\item
\begin{enumerate}
\item $0.125$
\item $0.5$
\item $f_{X}(x) = \begin{cases} x+\frac{1}{2}, & \text{при } x \in [0;1] \\ 0 , & \text{при } x \not\in [0;1] \end{cases}$
\item $f_{Y}(y) = \begin{cases} y+\frac{1}{2}, & \text{при } y \in [0;1] \\ 0 , & \text{при } y \not\in [0;1] \end{cases}$
\item нет
\end{enumerate}

\item
\begin{enumerate}
\item $\frac{1}{16}$
\item $\frac{1}{2}$
\item $f_{X}(x) =
\begin{cases} 2x, & \text{при } x \in [0;1] \\
0 , & \text{при } x \not\in [0;1]
\end{cases}$
\item $f_{Y}(y) =
\begin{cases} 2y, & \text{при } y \in [0;1] \\
0 , & \text{при } y \not\in [0;1]
\end{cases}$
\item да
\end{enumerate}
\item
\begin{enumerate}
\item $\frac{7}{12}$
\item $\frac{7}{12}$
\item $\frac{1}{3}$
\item $-\frac{1}{144}$
\item $-\frac{1}{11}$
\end{enumerate}

\item
\begin{enumerate}
\item $\frac{2}{3}$
\item $\frac{2}{3}$
\item $\frac{4}{9}$
\item $0$
\item $0$
\end{enumerate}

\item
\begin{enumerate}
\item $f_{Y}(y) =
\begin{cases} y+\frac{1}{2}, & \text{при } y \in [0;1] \\
0 , & \text{при } y \not\in [0;1]
\end{cases}$
\item $f_{X|Y}(x|\frac{1}{2}) =
\begin{cases} x+\frac{1}{2}, & \text{при } x \in [0;1] \\
0 , & \text{при } x \not\in [0;1]
\end{cases}$
\item $\frac{7}{12}$
\item $\frac{11}{144}$
\end{enumerate}

\item
\begin{enumerate}
\item $f_{Y}(y) =
\begin{cases} 2y, & \text{при } y \in [0;1] \\
0 , & \text{при } y \not\in [0;1]
\end{cases}$
\item $f_{X|Y}(x|\frac{1}{2}) =
\begin{cases} 2x, & \text{при } x \in [0;1] \\
0 , & \text{при } x \not\in [0;1]
\end{cases}$
\item $\frac{2}{3}$
\item $\frac{1}{18}$
\end{enumerate}
\end{enumerate}
\end{multicols}



\subsection[Кр 3]{\hyperref[sec:minimum_kr_03]{Контрольная работа 3 — Задачный минимум}}
\label{sec:sol_minimum_kr_03}


\begin{multicols}{2}
\begin{enumerate}
\item
\begin{enumerate}
\item $\approx 0.15$
\item $U \sim \cN(101,29)$, $f(u) = \frac{1}{\sqrt{2\pi\cdot 29}}e^{-\frac{1}{2}\frac{(u-101)^2}{29}}$
\item $\approx 0.02$
\end{enumerate}
\item
\begin{enumerate}
\item $71.14$
\item $f(y|x=170) = \frac{1}{\sqrt{2\pi\cdot20}}e^{-\frac{1}{2}\frac{(y-71.14)^2}{20}}$
\item $\approx 0$
\end{enumerate}
\item
\begin{enumerate}
\item $0.25$
\item $0.6875$
\item $0.91(6)$
\item $0.75$
\item $-0.28125$
\end{enumerate}
\item
\begin{enumerate}
\item $-1, 0, 1, 1$
\item $-1$
\item $1$
\item $f(x) = \begin{cases}
0, & x < -1 \\
0.25, & -1 \leq x < 0 \\
0.5, & 0 \leq x < 1 \\
1, & x \geq 1
\end{cases}$
\end{enumerate}
\item
\begin{enumerate}
\item $\theta$
\item да
\end{enumerate}

\item
\begin{enumerate}
\item нет, оценка смещена
\item $c = 2$
\end{enumerate}
\item
\begin{enumerate}
\item все оценки несмещенные
\item $\hat{p}_3$ наиболее эффективная
\end{enumerate}
\item да
\item да
\item $\hat{\theta}_{MM} = \sqrt{\frac{\sum_{i=1}^n(X_i-\overline{X})^2\cdot20}{n}}$

\item $\hat{\theta}_{MM} = \frac{1}{5}\left(6 - \frac{1}{n}\sum_{i=1}^n X_i^2 \right)$, $\hat{\theta}_{MM} = 0.68$
\item $\hat{\theta}_{ML} = \frac{\sum_{i=1}^n x_i^2}{n}$
\item $\hat{p}_{ML} = \frac{\sum_{i=1}^n x_i}{n}$
\item да
\item $n_1 \approx 260$, $n_2 \approx 232$, $n_3 \approx 658$

\end{enumerate}
\end{multicols}






\subsection[Кр 4]{\hyperref[sec:minimum_kr_04]{Контрольная работа 4 — Задачный минимум}}
\label{sec:sol_minimum_kr_04}


\begin{enumerate}
\item $\left[-1.6 - 1.65 \cdot \frac{2}{\sqrt{3}}; -1.6 + 1.65 \cdot \frac{2}{\sqrt{3}} \right]$
\item $\left[-1.6 - 2.92 \cdot \sqrt{\frac{18.33}{3}}; -1.6 + 2.92 \cdot \sqrt{\frac{18.33}{3}} \right]$
\item $\left[\frac{17.43 \cdot 2}{4.61}; \frac{17.43 \cdot 2}{0.21} \right]$
\item $\left[-1.6 - (-2.6) - 1.96 \cdot \sqrt{\frac{2}{3} + \frac{1}{2}}; -1.6 - (-2.6) + 1.96 \cdot \sqrt{\frac{2}{3} + \frac{1}{2}} \right]$
\item $\left[1.04 - (-0.37) - 3.18 \cdot \sqrt{3.02} \sqrt{\frac{1}{3} + \frac{1}{2}}; 1.04 - (-0.37) + 3.18 \cdot \sqrt{3.02} \sqrt{\frac{1}{3} + \frac{1}{2}} \right]$
\item $\left[0.45 - 1.96 \cdot \sqrt{\frac{0.45 \cdot 0.55}{100}}; 0.45 + 1.96 \cdot \sqrt{\frac{0.45 \cdot 0.55}{100}} \right]$
\item $\left[0.6 - 0.4 - 1.96  \cdot \sqrt{\frac{0.6\cdot0.4}{100} + \frac{0.4 \cdot 0.6}{200}}; 0.6 - 0.4 + 1.96 \cdot \sqrt{\frac{0.6\cdot0.4}{100} + \frac{0.4 \cdot 0.6}{200}} \right]$
\item $\left[2.5 - 1.96 \cdot \sqrt{\frac{1}{40}}; 2.5 + 1.96 \cdot \sqrt{\frac{1}{40}} \right]$
\item $\left[\frac{1}{0.52} - 1.96 \cdot \sqrt{\frac{1}{100 \cdot 0.52^2}}; \frac{1}{0.52} + 1.96 \cdot \sqrt{\frac{1}{100 \cdot 0.52^2}} \right]$
\item
\begin{enumerate}
\item $\approx 0.02$
\item $\approx 0.02$
\item $\approx 0.98$
\end{enumerate}
\item $0.2$
\item $z_{obs} \approx -1.39 $, $z_{crit} = 1.28$, нет оснований отвергать $H_0$.
\item $t_{obs} \approx -0.65$, $t_{crit} = 1.89$, нет оснований отвергать $H_0$.
\item $z_{obs} \approx 0.93$, $z_{crit} = -1.65$, нет оснований отвергать $H_0$.
\item $t_{obs} \approx 0.89$, $t_{crit} = -2.35$, нет оснований отвергать $H_0$.
\item $F_{obs} \approx 95.37$, $F_{crit} = 199.5$, нет оснований отвергать $H_0$.
\item $z_{obs} \approx 2.04$, $z_{crit} = 1.65$, основная гипотеза отвергается.
\item $z_{obs} \approx 4.16$, $z_{crit} = 1.96$, основная гипотеза отвергается.
\item $\gamma_{obs} \approx 0.26$, $\gamma_{crit} = 5.99$, нет оснований отвергать $H_0$.
\item $\gamma_{obs} \approx 139.4$, $\gamma_{crit} = 3.84$, основная гипотеза отвергается.
\item $LR_{obs} \approx 5.5$, $LR_{crit} = 3.84$, основная гипотеза отвергается.
\end{enumerate}


\end{document}
