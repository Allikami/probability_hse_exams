
\element{probability1}{
  \begin{questionmult}{1} %2015ready
Крошка Джон  попадает в яблочко с вероятностью $0.8$. Его выстрелы независимы. Вероятность, что он попадёт хотя бы один раз из двух равна
\begin{multicols}{3}
   \begin{choices}
      \correctchoice{$0.96$}
      \wrongchoice{$0.8$}
      \wrongchoice{$0.64$}
      \wrongchoice{$0.36$}
      \wrongchoice{$0.9$}

       \end{choices}
  \end{multicols}
  \end{questionmult}
}

\element{probability1}{
  \begin{questionmult}{2} %2015ready
Крошка Джон попадает в яблочко с вероятностью $0.8$. Его выстрелы независимы. Вероятность, что он попал оба раза, если известно, что он попал хотя бы один раз из двух, равна
   \begin{multicols}{3}
   \begin{choices}
      \correctchoice{$2/3$}
      \wrongchoice{$1/3$}
      \wrongchoice{$1/2$}
      \wrongchoice{$3/4$}
      \wrongchoice{$1/4$}

       \end{choices}
  \end{multicols}
  \end{questionmult}
}

\element{probability1}{
  \begin{questionmult}{3} %2015ready
Имеется три монетки. Две «правильных» и одна — с «орлами» по обеим сторонам. Вася выбирает одну монетку наугад и подкидывает её два раза. Вероятность того, что оба раза выпадет орел равна
    \begin{multicols}{3}
   \begin{choices}
      \correctchoice{$1/2$}
      \wrongchoice{$2/3$}
      \wrongchoice{$1/3$}
      \wrongchoice{$1/4$}
      \wrongchoice{$3/4$}

       \end{choices}
  \end{multicols}
  \end{questionmult}
}

%
% \element{probability1}{
%   \begin{questionmult}{3} %2015ready
% Крошка Джон попадает в яблочко с вероятностью $0.8$. Его выстрелы независимы. Вероятность, что он попал во второй раз, если известно, что он попал хотя бы один раз из двух, равна
%   \begin{multicols}{3}
%    \begin{choices}
%       \correctchoice{$5/6$}
%       \wrongchoice{$3/4$}
%       \wrongchoice{$2/3$}
%       \wrongchoice{$1/2$}
%       \wrongchoice{$4/5$}
%
%        \end{choices}
%   \end{multicols}
%   \end{questionmult}
% }

\element{probability1}{
  \begin{questionmult}{4} %2015ready
    Если события $A$, $B$, $C$ попарно независимы, то
    \begin{choices}
      \wrongchoice{Событие $A\cup B\cup C$ обязательно произойдёт}
      \wrongchoice{События $A$, $B$, $C$ независимы в совокупности}
      \wrongchoice{События $A$, $B$, $C$ зависимы в совокупности}
      \wrongchoice{События $A$, $B$, $C$ несовместны}
      \wrongchoice{$\P(A\cap B\cap C)=\P(A)\P(B)\P(C)$}

    \end{choices}
  \end{questionmult}
}


\element{probability1}{
  \begin{questionmult}{5} %2015ready
Случайная величина $X$ равномерна на отрезке $[0;10]$. Вероятность $\P(X>3|X<7)$ равна
    \begin{multicols}{3}
   \begin{choices}
      \correctchoice{$4/7$}
      \wrongchoice{$3/10$}
      \wrongchoice{$7/10$}
      \wrongchoice{$3/7$}
      \wrongchoice{$0.21$}

       \end{choices}
  \end{multicols}
  \end{questionmult}
}


\element{probability}{
  \begin{questionmult}{6} %2015ready
Имеется три монетки. Две «правильных» и одна — с «орлами» по обеим сторонам. Вася выбирает одну монетку наугад и подкидывает её два раза. События $A = \{ \text{Орёл выпал при первом подбрасывании} \}$ и $B =\{\text{Орёл выпал при втором подбрасывании}\}$
  \begin{multicols}{2}
   \begin{choices}
      \correctchoice{удовлетворяют соотношению $\P(A|B)\geq \P(A)$}
      \wrongchoice{независимы}
      \wrongchoice{несовместны}
      \wrongchoice{образуют полную группу событий}
       \wrongchoice{удовлетворяют соотношению $\P(A\cap B)=\P(A)+\P(B) + \P(A\cup B)$}

       \end{choices}
  \end{multicols}
  \end{questionmult}
}


\element{probability}{
  \begin{questionmult}{7} %2015ready
В квадрат вписан круг. Наудачу в квадрат бросают восемь точек. Пусть $X$ — число точек, попавших в круг. Математическое ожидание величины $X$ равно
    \begin{multicols}{3}
   \begin{choices}
      \correctchoice{$2\pi$}
      \wrongchoice{$\pi$}
      \wrongchoice{$\pi / 2$}
      \wrongchoice{$\pi / 4$}
      \wrongchoice{$4 \pi$}

       \end{choices}
  \end{multicols}
  \end{questionmult}
}


\element{probability}{
  \begin{questionmult}{8} %2015ready
В квадрат вписан круг. Наудачу в квадрат бросают восемь точек. Пусть $X$ — число точек, попавших в круг. Дисперсия величины $X$ равна    \begin{multicols}{3}
   \begin{choices}
    \correctchoice{$2\pi - \pi^2 / 2$}
      \wrongchoice{$\pi^2$}
      \wrongchoice{$\pi^2 - 2 \pi$}
      \wrongchoice{$3\pi^2 - 4$}
       \wrongchoice{$3\pi^2 - 2$}

       \end{choices}
  \end{multicols}
  \end{questionmult}
}


\element{probability}{
  \begin{questionmult}{9} %2015ready
В квадрат вписан круг. Последовательно в квадрат наудачу бросают восемь точек. Пусть $Y$ — число точек, попавших в круг, при первых четырех бросаниях, а $Z$ — число точек, попавших в круг, при оставшихся четырех бросаниях. Ковариация $\Cov(Y,Z)$ равна
    \begin{multicols}{3}
   \begin{choices}
      \correctchoice{$0$}
      \wrongchoice{$\pi^2$}
      \wrongchoice{$-\pi^2$}
      \wrongchoice{$2\pi$}
      \wrongchoice{$-2\pi$}

      \end{choices}
  \end{multicols}
  \end{questionmult}
}


\element{probability}{
  \begin{questionmult}{10} %2015ready
В квадрат вписан круг. Последовательно в квадрат наудачу бросают восемь точек. Пусть $Y$ — число точек, попавших в круг, при первых четырех бросаниях, а $Z$ — число точек, попавших в круг, при оставшихся четырех бросаниях. Дисперсия $\Var(Y - Z)$ равна
\begin{multicols}{3}
   \begin{choices}
      \correctchoice{$2\pi - \pi^2 / 2$}
      \wrongchoice{$\pi^2$}
      \wrongchoice{$\pi^2 - 2 \pi$}
      \wrongchoice{$3\pi^2 - 4$}
      \wrongchoice{ $0$}

    \end{choices}
  \end{multicols}
  \end{questionmult}
}

\element{probability}{
  \begin{questionmult}{11} %2015ready
В квадрат вписан круг. Наудачу в квадрат бросают восемь точек. Наиболее вероятное число точек, попавших в круг, равно
\begin{multicols}{3}
   \begin{choices}
      \correctchoice{$7$}
      \wrongchoice{$2\pi$}
      \wrongchoice{$4$}
      \wrongchoice{$5$}
      \wrongchoice{$6$}

    \end{choices}
  \end{multicols}
  \end{questionmult}
}


\element{probability}{
  \begin{questionmult}{12} %2015ready
Всем известно, что Маша звонит Васе в среднем 10 раз в день. Число звонков, совершенных Машей, имеет распределение Пуассона. Вероятность того, что Маша ни разу не позвонит Васе в течение дня, равна
\begin{multicols}{3}
   \begin{choices}
      \correctchoice{$e^{-10}$}
      \wrongchoice{$1 - e^{10}$}
      \wrongchoice{$10\,e^{-10}$}
      \wrongchoice{$\tfrac{1}{10!}e^{-10}$}
      \wrongchoice{$1 - e^{-10}$}

    \end{choices}
  \end{multicols}
  \end{questionmult}
}

\element{commontext}{
%\newpage
\rule{\textwidth}{1pt} %2015ready
\textbf{В вопросах 13-16} совместное распределение пары величин $X$ и $Y$ задано таблицей:

\begin{center}
\begin{tabular}{c|cc}
 & $Y=-2$ & $Y=1$ \\
\hline
$X=-1$ & 0.1 & 0 \\
$X=0$ & 0.1 & 0.3 \\
$X=1$ & 0.2 & 0.3 \\
\end{tabular}
\end{center}
\vspace{0.2cm}
}

\element{1316}{
  \begin{questionmult}{13} %2015ready
Математическое ожидание величины $Y$ при условии, что $X=0$, равно
\begin{multicols}{3}
   \begin{choices}
      \correctchoice{$0.25$}
      \wrongchoice{$0$}
      \wrongchoice{$0.1$}
      \wrongchoice{$-0.1$}
      \wrongchoice{$0.2$}
      \wrongchoice{$-0.2$}

    \end{choices}
  \end{multicols}
  \end{questionmult}
}


\element{1316}{
  \begin{questionmult}{14} %2015ready
Дисперсия случайной величины $X$ равна
\begin{multicols}{3}
   \begin{choices}
      \correctchoice{$0.44$}
      \wrongchoice{$0.2$}
      \wrongchoice{$0.4$}
      \wrongchoice{$0.6$}
      \wrongchoice{$1.04$}

    \end{choices}
  \end{multicols}
  \end{questionmult}
}

\element{1316}{ %2015ready
  \begin{questionmult}{15}
Ковариация $\Cov(X, Y)$ равна
\begin{multicols}{3}
   \begin{choices}
      \correctchoice{$0.18$}
      \wrongchoice{$0.9$}
      \wrongchoice{$-0.7$}
      \wrongchoice{$-0.5$}
      \wrongchoice{$0.1$}
      \wrongchoice{$0.4$}

    \end{choices}
  \end{multicols}
  \end{questionmult}
}

\element{1316}{
  \begin{questionmult}{16} %2015ready
Вероятность того, что $Y = 1$ при условии, что $X > 0$ равна
\begin{multicols}{3}
   \begin{choices}
      \correctchoice{$0.6$}
      \wrongchoice{$0.5$}
      \wrongchoice{$0.2$}
      \wrongchoice{$0.3$}
      \wrongchoice{$0.4$}

    \end{choices}
  \end{multicols}
  \end{questionmult}
}


\element{commontext2}{ %2015ready
%\newpage
\rule{\textwidth}{1pt}
\textbf{В вопросах 17-19} Функция распределения абсолютно непрерывной случайной величины $X$ имеет вид
\[
F(x)=\begin{cases}
a, x<0,\\
b x^2+c, x \in [0,2],\\
d, x > 2.\\
\end{cases}
\]
\vspace{0.2cm}
}

\element{1719}{
  \begin{questionmult}{17} %2015ready
Величина $X$ равномерна от $0$ до $4$. Вероятность того, что $X$ примет значение 1, равна
\begin{multicols}{3}
   \begin{choices}
      \correctchoice{$0$}
      \wrongchoice{$0.25$}
      \wrongchoice{$0.4$}
      \wrongchoice{$0.5$}
      \wrongchoice{$0.8$}

    \end{choices}
  \end{multicols}
  \end{questionmult}
}

\element{1719}{
  \begin{questionmult}{18} %2015ready
Величина $X$ имеет функцию плотности $f(x)=x/2$ на отрезке $[0;2]$. Значение $\E(X)$  равно
\begin{multicols}{3}
   \begin{choices}
      \correctchoice{$4/3$}
      \wrongchoice{$0$}
      \wrongchoice{$1/2$}
      \wrongchoice{$1$}
      \wrongchoice{$2$}

    \end{choices}
  \end{multicols}
  \end{questionmult}
}

\element{1719}{
  \begin{questionmult}{19} %2015ready
Функция распределения абсолютно непрерывной случайной величины $X$ имеет вид
\[
F(x)=\begin{cases}
a, x<0,\\
b x^2+c, x \in [0,2],\\
d, x > 2.\\
\end{cases}
\]
Выражение $a+b+c+d$ равно
\begin{multicols}{3}
   \begin{choices}
      \correctchoice{$5/4$}
      \wrongchoice{$1/4$}
      \wrongchoice{$1/2$}
      \wrongchoice{$1$}
      \wrongchoice{$2$}

    \end{choices}
  \end{multicols}
  \end{questionmult}
}

\element{commontext3}{ %2015ready
%\newpage
\rule{\textwidth}{1pt}
\textbf{В вопросах 20-23} совместная функция плотности пары $X$ и $Y$ имеет вид
\[
f(x,y)=\begin{cases}
(x+y)/3, \; \text{ если } x\in[0;1], y\in [0;2] \\
0, \; \text{ иначе}
\end{cases}
\]

\vspace{0.5cm}

}

\element{2023}{
  \begin{questionmult}{20} %2015ready
Если функция $h(x,y)=c\cdot x\cdot f(x,y)$ также является совместной функцией плотности, то константа $c$ равна
\begin{multicols}{3}
   \begin{choices}
      \correctchoice{$9/5$}
      \wrongchoice{$5/9$}
      \wrongchoice{$5$}
      \wrongchoice{$9$}
      \wrongchoice{$1$}
    \end{choices}
  \end{multicols}
  \end{questionmult}
}

\element{2023}{
  \begin{questionmult}{21} %2015ready
Вероятность $\P(X<0.5, Y<1)$ равна
\begin{multicols}{3}
   \begin{choices}
      \correctchoice{$1/8$}
      \wrongchoice{$5/6$}
      \wrongchoice{$3/5$}
      \wrongchoice{$5/8$}
      \wrongchoice{$3/8$}

    \end{choices}
  \end{multicols}
  \end{questionmult}
}

\element{2023}{
  \begin{questionmult}{22} %2015ready
Условная функция плотности  $f_{X|Y=1}(x)$ равна
\begin{multicols}{2}
   \begin{choices}
      \correctchoice{$f_{X|Y=1}(x)=\begin{cases} (2x+2)/3\, \text{ если } x\in [0;1] \\ 0, \text{ иначе }    \end{cases}$}
      \wrongchoice{$f_{X|Y=1}(x)=\begin{cases} (x+2)/2\, \text{ если } x\in [0;1] \\ 0, \text{ иначе }    \end{cases}$}
      \wrongchoice{$f_{X|Y=1}(x)=\begin{cases} (2x+1)/2\, \text{ если } x\in [0;1] \\ 0, \text{ иначе }    \end{cases}$}
      \wrongchoice{$f_{X|Y=1}(x)=\begin{cases} (x+4)/2\, \text{ если } x\in [0;1] \\ 0, \text{ иначе }    \end{cases}$}
      \wrongchoice{$f_{X|Y=1}(x)=\begin{cases} (4x+2)/3\, \text{ если } x\in [0;1] \\ 0, \text{ иначе }    \end{cases}$}


    \end{choices}
  \end{multicols}
  \end{questionmult}
}

\element{2023}{
  \begin{questionmult}{23} %2015ready
Математическое ожидание $\E(Y)$ равно
\begin{multicols}{3}
   \begin{choices}
      \correctchoice{$11/9$}
      \wrongchoice{$2/3$}
      \wrongchoice{$4/3$}
      \wrongchoice{$6/5$}
      \wrongchoice{$13/7$}

    \end{choices}
  \end{multicols}
  \end{questionmult}
}

\element{commontext4}{ %2015ready
%\newpage
\rule{\textwidth}{1pt}
\textbf{В вопросах 24-25} известно, что $\E(X)=-1$, $\Var(X)=1$, $\E(Y)=-4$, $\Var(Y)=4$, $\Corr(X,Y)=-0.5$

\vspace{0.5cm}

}

\element{2425}{
  \begin{questionmult}{24} %2015ready
Ковариация $\Cov(2X+Y,X-3Y)$ равна
\begin{multicols}{3}
   \begin{choices}
      \correctchoice{$-5$}
      \wrongchoice{$0$}
      \wrongchoice{$5$}
      \wrongchoice{$1$}
      \wrongchoice{$-1$}

    \end{choices}
  \end{multicols}
  \end{questionmult}
}

\element{2425}{
  \begin{questionmult}{25} %2015ready
Корреляция $\Corr((1-X)/2,(Y+5)/2)$ равна
\begin{multicols}{3}
   \begin{choices}
      \correctchoice{$0.5$}
      \wrongchoice{$-0.5$}
      \wrongchoice{$1/8$}
      \wrongchoice{$-1/8$}
      \wrongchoice{$1$}

    \end{choices}
  \end{multicols}
  \end{questionmult}
}


\element{2630}{
  \begin{questionmult}{26} %2015ready
  \AMCnoCompleteMulti
У неотрицательной случайной величины $X$ известны $\E(X)=1$, $\Var(X)=4$. Вероятность $\P(X^2 \geq 25)$ обязательно попадает в интервал
\begin{multicols}{2}
   \begin{choices}
      \correctchoice{$[0;1/5]$ }
      \wrongchoice{$[0;1/25]$}
      \wrongchoice{$[1/25;1]$}
      \wrongchoice{$[0;4/625]$}
      \wrongchoice{$[0;4/25]$}
      \wrongchoice{$[4/25;1]$}
    \end{choices}
  \end{multicols}
  \end{questionmult}
}


\element{2630}{
  \begin{questionmult}{27} %2015ready
Если $\E(X)=0$, $\Var(X)=1$, то наиболее узкий интервал, в который гарантированно попадает вероятность $\P(|X| \geq 4)$, равен

\begin{multicols}{3}
   \begin{choices}
      \correctchoice{$[0; 0.0625]$ }
      \wrongchoice{$[0.0625; 1]$}
      \wrongchoice{$[0.25; 1]$}
      \wrongchoice{$[0; 0.25]$}
      \wrongchoice{$[0.5; 1]$}

    \end{choices}
  \end{multicols}
  \end{questionmult}
}


\element{2630}{
  \begin{questionmult}{28} %2015ready
  \AMCnoCompleteMulti
Дана последовательность независимых случайных величин, имеющих равномерное на $(-1,1)$ распределение.  \textbf{НЕВЕРНЫМ} является утверждение
%\begin{multicols}{3}
   \begin{choices}
      \correctchoice{  $\bar X$ сходится по распределению к равномерной на (-1,1) величине }
      \wrongchoice{Вероятность	$\P(\bar X>0)$ стремится к 0.5}
      \wrongchoice{$\bar X$ сходится по вероятности к нулю}
      \wrongchoice{	$\sqrt{3n}\bar X$ сходится по распределению к стандартной нормальной величине}
      \wrongchoice{Вероятность	$\P(\bar X = 0)$ стремится к 0}
      \wrongchoice{	$\P(|\bar X|<1/\sqrt{n})\leq 1/3$ }
    \end{choices}
%  \end{multicols}
  \end{questionmult}
}

\element{2630}{
  \begin{questionmult}{29} %2015ready
  \AMCnoCompleteMulti
Функция плотности случайной величины $X$ имеет вид
\[
f(x)=\frac{1}{\sqrt{8\pi}} e^{-(x-3)^2/8}
\]
 \textbf{НЕВЕРНЫМ} является утверждение
\begin{multicols}{3}
   \begin{choices}
      \correctchoice{$\Var(X)=8$ }
      \wrongchoice{$\max f(x) = \frac{1}{2\sqrt{2\pi}}$}
      \wrongchoice{$\E(X)=3$}
      \wrongchoice{$\P(X>3)=0.5$}
      \wrongchoice{$\P(X=0)=0$}
      \wrongchoice{$\P(X<0)>0$}
    \end{choices}
  \end{multicols}
  \end{questionmult}
}

\element{2630}{
  \begin{questionmult}{30}
Величины $X_1$, $X_2$, \ldots независимы и одинаково распределены с $\E(X_i)=\mu$, $\Var(X_i)=\sigma^2$. К стандартному нормальному распределению  сходится последовательность случайных величин
\begin{multicols}{3}
   \begin{choices}
      \correctchoice{$\sqrt{n}(\bar X - \mu) /\sigma$ }
      \wrongchoice{$\bar X$}
      \wrongchoice{$(\bar X - \mu) /\sigma$}
      \wrongchoice{$(\bar X - \mu) /(\sqrt{n}\sigma)$}
       \wrongchoice{$(\bar X - n\mu) /(\sqrt{n}\sigma)$}
    \end{choices}
  \end{multicols}
  \end{questionmult}
}

\element{ruler}{
\noindent\rule{\textwidth}{1pt}
}

\element{newpage}{
\newpage\null
}



\element{exam_15}{
  \begin{questionmult}{1}
Пусть $X_1$, \ldots, $X_n$ — выборка объема $n$ из равномерного на $[a, b]$ распределения. Оценка $X_1+X_2$ параметра $c=a+b$ является
\begin{multicols}{2}
   \begin{choices}
      \correctchoice{несмещенной и несостоятельной}
      \wrongchoice{несмещенной и состоятельной}
      \wrongchoice{смещенной и состоятельной}
      \wrongchoice{смещенной и несостоятельной}
      \wrongchoice{асимптотически несмещенной и состоятельной}
      \end{choices}
  \end{multicols}
  \end{questionmult}
}


\element{exam_15}{
  \begin{questionmult}{2}
Пусть $X_1$, \ldots, $X_n$ — выборка объема $n$ из некоторого распределения с конечным математическим ожиданием. Несмещенной и состоятельной оценкой математического ожидания является
\begin{multicols}{3}
   \begin{choices}
      \correctchoice{$\frac{X_1}{2 n}+\frac{X_2+\ldots+X_{n-1}}{n-2}-\frac{X_n}{2 n}$}
      \wrongchoice{$\frac{1}{3} X_1 + \frac{2}{3} X_2$}
      \wrongchoice{$\frac{X_1}{2 n}+\frac{X_2+\ldots+X_{n-2}}{n-2}+\frac{X_n}{2 n}$}
      \wrongchoice{$\frac{X_1}{2 n}+\frac{X_2+\ldots+X_{n-2}}{n-1}+\frac{X_n}{2 n}$}
      \wrongchoice{$\frac{X_1+X_2}{2}$}
      \end{choices}
  \end{multicols}
  \end{questionmult}
}

\element{exam_15}{
  \begin{questionmult}{3}
Пусть $X_1$,\ldots, $X_n$ — выборка объема $n$ из равномерного на $[0, \theta]$ распределения. Оценка параметра $\theta$ методом моментов по $k$-му моменту имеет вид:
\begin{multicols}{3}
   \begin{choices}
      \correctchoice{$\sqrt[k]{(k+1) \overline{X^k}}$}
      \wrongchoice{$\sqrt[k]{(k+1) \overline{X}^k}$}
      \wrongchoice{$\sqrt[k]{k \overline{X^k}}$}
      \wrongchoice{$\sqrt[k]{k \overline{X}^k}$}
       \wrongchoice{$\sqrt[k+1]{(k+1) \overline{X}^k}$}
      \end{choices}
  \end{multicols}
  \end{questionmult}
}


\element{exam_15}{
  \begin{questionmult}{4}
Пусть $X_1$, \ldots, $X_n$ — выборка объема $n$ из равномерного на $[0, \theta]$ распределения. Состоятельной оценкой параметра $\theta$ является:
\begin{multicols}{3}
   \begin{choices}[o] % не рандомизирует порядок ответов
      \wrongchoice{$X_{(n)}$}
      \wrongchoice{$X_{(n-1)}$}
      \wrongchoice{$\frac{n}{n+1} X_{(n-1)}$}
       \wrongchoice{$\frac{n^2}{n^2-n+3} X_{(n-3)}$}
       \correctchoice{все перечисленные случайные величины}
      \end{choices}
  \end{multicols}
  \end{questionmult}
}


\element{exam_15}{ % в фигурных скобках название группы вопросов
  \begin{questionmult}{5} % тип вопроса (questionmult — множественный выбор) и в фигурных — номер вопроса
Пусть $X_1$, \ldots, $X_{2 n}$ — выборка объема $2 n$ из некоторого распределения. Какая из нижеперечисленных оценок математического ожидания имеет наименьшую дисперсию?
\begin{multicols}{3} % располагаем ответы в 3 колонки
   \begin{choices} % опция [o] не рандомизирует порядок ответов
      \wrongchoice{$X_1$}
      \wrongchoice{$\frac{X_1+X_2}{2}$}
      \wrongchoice{$\frac{1}{n} \sum_{i=1}^n X_i$}
       \wrongchoice{$\frac{1}{n} \sum_{i=n+1}^{2 n} X_i$}
       \correctchoice{$\frac{1}{2 n} \sum_{i=1}^{2 n} X_i$}
      \end{choices}
  \end{multicols}
  \end{questionmult}
}


\element{exam_15}{ % в фигурных скобках название группы вопросов
  \begin{questionmult}{6} % тип вопроса (questionmult — множественный выбор) и в фигурных — номер вопроса
Пусть $X_1$, \ldots, $X_n$ — выборка объема $n$ из распределения Бернулли с параметром $p$. Статистика $X_2 X_{n-2}$ является
\begin{multicols}{2} % располагаем ответы в [k] колонки
   \begin{choices} % опция [o] не рандомизирует порядок ответов
      \wrongchoice{оценкой максимального правдоподобия}
      \wrongchoice{асимптотически нормальной оценкой $p^2$}
      \wrongchoice{эффективной оценкой $p^2$}
       \wrongchoice{состоятельной оценкой $p^2$}
       \correctchoice{несмещенной оценкой $p^2$}
      \end{choices}
  \end{multicols}
  \end{questionmult}
}

\element{exam_15}{ % в фигурных скобках название группы вопросов
  \begin{questionmult}{7} % тип вопроса (questionmult — множественный выбор) и в фигурных — номер вопроса
Пусть $X_1$, \ldots, $X_n$ — выборка объема $n$ из равномерного на $[a, b]$ распределения. Выберите наиболее точный ответ из предложенных. Оценка $\theta^*_n = X_{(n)}-X_{(1)}$ длины отрезка $[a,b]$ является
\begin{multicols}{3} % располагаем ответы в 3 колонки
   \begin{choices} % опция [o] не рандомизирует порядок ответов
      \wrongchoice{несмещенной}
      \wrongchoice{состоятельной и асимптотически смещённой}
      \wrongchoice{несостоятельной и асимптотически несмещенной}
       \wrongchoice{нормально распределённой}
       \correctchoice{состоятельной и асимптотически несмещенной}
      \end{choices}
  \end{multicols}
  \end{questionmult}
}


\element{exam_15}{ % в фигурных скобках название группы вопросов
  \begin{questionmult}{8} % тип вопроса (questionmult — множественный выбор) и в фигурных — номер вопроса
Мощностью теста называется
%\begin{multicols}{2} % располагаем ответы в [k] колонки
   \begin{choices} % опция [o] не рандомизирует порядок ответов
      \wrongchoice{Вероятность отвергнуть основную гипотезу, когда она верна}
      \wrongchoice{Вероятность отвергнуть альтернативную гипотезу, когда она верна}
      \wrongchoice{Вероятность принять неверную гипотезу}
       \wrongchoice{Единица минус  вероятность отвергнуть основную гипотезу, когда она верна}
       \correctchoice{Единица минус  вероятность отвергнуть альтернативную гипотезу, когда она верна}
      \end{choices}
%  \end{multicols}
  \end{questionmult}
}

\element{exam_15}{ % в фигурных скобках название группы вопросов
  \begin{questionmult}{9} % тип вопроса (questionmult — множественный выбор) и в фигурных — номер вопроса
Если P-значение (P-value) больше уровня значимости  $\alpha$, то гипотеза  $H_0: \; \sigma=1$
\begin{multicols}{2} % располагаем ответы в {k} колонки
   \begin{choices} % опция [o] не рандомизирует порядок ответов
      \wrongchoice{Отвергается}
      \wrongchoice{Отвергается, только если  $H_a: \; \sigma>1$}
      \wrongchoice{Отвергается, только если  $H_a: \; \sigma\neq 1$}
       \wrongchoice{ Отвергается, только если  $H_a: \; \sigma<1$}
       \correctchoice{Не отвергается}
      \end{choices}
  \end{multicols}
  \end{questionmult}
}


\element{exam_15}{ % в фигурных скобках название группы вопросов
  \begin{questionmult}{10} % тип вопроса (questionmult — множественный выбор) и в фигурных — номер вопроса
Имеется случайная выборка размера $n$ из нормального распределения. При проверке гипотезы о равенстве математического ожидания заданному значению при известной дисперсии используется статистика, имеющая распределение
\begin{multicols}{3} % располагаем ответы в {k} колонки
   \begin{choices} % опция [o] не рандомизирует порядок ответов
      \wrongchoice{$t_n$}
      \wrongchoice{ $t_{n-1}$}
      \wrongchoice{$\chi^2_n$}
       \wrongchoice{$\chi^2_{n-1}$}
       \correctchoice{$N(0,1)$}
      \end{choices}
  \end{multicols}
  \end{questionmult}
}




\element{exam_15}{ % в фигурных скобках название группы вопросов
  \begin{questionmult}{11} % тип вопроса (questionmult — множественный выбор) и в фигурных — номер вопроса
Имеется случайная выборка размера $n$ из нормального распределения. При проверке гипотезы о равенстве дисперсии заданному значению при неизвестном математическом ожидании используется статистика, имеющая распределение
\begin{multicols}{3} % располагаем ответы в {k} колонки
   \begin{choices} % опция [o] не рандомизирует порядок ответов
      \wrongchoice{$t_n$}
      \wrongchoice{ $t_{n-1}$}
      \wrongchoice{$\chi^2_n$}
       \wrongchoice{$N(0,1)$}
       \correctchoice{$\chi^2_{n-1}$}
      \end{choices}
  \end{multicols}
  \end{questionmult}
}


\element{exam_15}{ % в фигурных скобках название группы вопросов
  \begin{questionmult}{12} % тип вопроса (questionmult — множественный выбор) и в фигурных — номер вопроса
По случайной выборке из 100 наблюдений было оценено выборочное среднее $\bar{X}=20$  и несмещенная оценка дисперсии  $\hat{\sigma}^2=25$. В рамках проверки гипотезы $H_0: \; \mu=15$  против альтернативной гипотезы $H_a: \; \mu>15$  можно сделать следующее заключение
%\begin{multicols}{2} % располагаем ответы в {k} колонки
   \begin{choices} % опция [o] не рандомизирует порядок ответов
      \wrongchoice{Гипотеза $H_0$  отвергается на уровне значимости 5\%, но не  на уровне значимости 1\%}
      \wrongchoice{Гипотеза  $H_0$ отвергается на уровне значимости 10\%, но не на уровне значимости 5\%}
      \wrongchoice{Гипотеза  $H_0$ отвергается на уровне значимости 20\%, но не  на уровне значимости 10\%}
       \wrongchoice{ Гипотеза $H_0$  не отвергается на любом разумном уровне значимости}
       \correctchoice{Гипотеза $H_0$  отвергается на любом разумном уровне значимости}
      \end{choices}
%  \end{multicols}
  \end{questionmult}
}


\element{exam_15}{ % в фигурных скобках название группы вопросов
  \begin{questionmult}{13} % тип вопроса (questionmult — множественный выбор) и в фигурных — номер вопроса
На основе случайной выборки, содержащей одно наблюдение  $X_1$, тестируется гипотеза $H_0: \; X_1 \sim U[0;1]$  против альтернативной гипотезы  $H_a: \; X_1 \sim U[0.5;1.5]$. Рассматривается критерий: если $X_1>0.8$, то гипотеза $H_0$  отвергается в пользу гипотезы  $H_a$. Вероятность ошибки 2-го рода для этого критерия равна:
\begin{multicols}{3} % располагаем ответы в {k} колонки
   \begin{choices} % опция [o] не рандомизирует порядок ответов
      \wrongchoice{0.1}
      \wrongchoice{0.2}
      \wrongchoice{0.4}
       \wrongchoice{0.5}
       \correctchoice{0.3}
      \end{choices}
 \end{multicols}
  \end{questionmult}
}

\element{exam_15}{ % в фигурных скобках название группы вопросов
  \begin{questionmult}{14} % тип вопроса (questionmult — множественный выбор) и в фигурных — номер вопроса
Пусть $X_1$, $X_2$, \ldots, $X_n$ — случайная выборка размера 36 из нормального распределения $N(\mu, 9)$. Для тестирования основной гипотезы  $H_0: \; \mu=0$  против альтернативной $H_a: \; \mu=-2$   вы используете критерий: если  $\bar{X}\geq -1$, то вы не отвергаете гипотезу $H_0$, в противном случае вы отвергаете гипотезу  $H_0$ в пользу гипотезы  $H_a$. Мощность критерия равна
\begin{multicols}{3} % располагаем ответы в {k} колонки
   \begin{choices} % опция [o] не рандомизирует порядок ответов
      \wrongchoice{0.58}
      \wrongchoice{0.85}
      \wrongchoice{0.78}
       \wrongchoice{0.87}
       \correctchoice{0.98}
      \end{choices}
 \end{multicols}
  \end{questionmult}
}

\element{exam_15}{ % в фигурных скобках название группы вопросов
  \begin{questionmult}{15} % тип вопроса (questionmult — множественный выбор) и в фигурных — номер вопроса
Николай Коперник подбросил бутерброд 200 раз. Бутерброд упал маслом вниз 95 раз, а маслом вверх — 105 раз. Значение критерия $\chi^2$ Пирсона для проверки гипотезы о равной вероятности данных событий равно
\begin{multicols}{3} % располагаем ответы в {k} колонки
   \begin{choices} % опция [o] не рандомизирует порядок ответов
      \wrongchoice{0.25}
      \wrongchoice{0.75}
      \wrongchoice{0.5}
       \wrongchoice{7.5}
       \correctchoice{0.5}
      \end{choices}
 \end{multicols}
  \end{questionmult}
}

\element{exam_15}{ % в фигурных скобках название группы вопросов
  \begin{questionmult}{16} % тип вопроса (questionmult — множественный выбор) и в фигурных — номер вопроса
Каждое утро в 8:00 Иван Андреевич Крылов, либо завтракает, либо уже позавтракал. В это же время кухарка либо заглядывает к Крылову, либо нет. По таблице сопряженности вычислите  статистику $\chi^2$ Пирсона для тестирования гипотезы о том, что визиты кухарки не зависят от того, позавтракал ли уже Крылов или нет.
\begin{tabular}{c|cc}
Время 8:00 & кухарка заходит & кухарка не заходит \\
\hline
Крылов завтракает & 200 & 40 \\
Крылов уже позавтракал & 25 & 100 \\
\end{tabular}
\begin{multicols}{3} % располагаем ответы в {k} колонки
   \begin{choices} % опция [o] не рандомизирует порядок ответов
      \wrongchoice{39}
      \wrongchoice{79}
      \wrongchoice{100}
       \wrongchoice{179}
       \correctchoice{139}
      \end{choices}
 \end{multicols}
  \end{questionmult}
}

%% Боря Демешев:

\element{exam_15}{ % в фигурных скобках название группы вопросов
  \begin{questionmult}{17} % тип вопроса (questionmult — множественный выбор) и в фигурных — номер вопроса
Ковариационная матрица вектора $X=(X_1,X_2)$ имеет вид
\[
\begin{pmatrix}
10 & 3 \\
3 & 8
\end{pmatrix}
\]
Дисперсия разности элементов вектора, $\Var(X_1-X_2)$, равняется
\begin{multicols}{3} % располагаем ответы в {k} колонки
   \begin{choices} % опция [o] не рандомизирует порядок ответов
      \wrongchoice{18}
      \wrongchoice{15}
      \wrongchoice{2}
       \wrongchoice{6}
       \correctchoice{12}
      \end{choices}
 \end{multicols}
  \end{questionmult}
}


\element{exam_15}{ % в фигурных скобках название группы вопросов
  \begin{questionmult}{18} % тип вопроса (questionmult — множественный выбор) и в фигурных — номер вопроса
Все условия регулярности для применения метода максимального правдоподобия выполнены. Вторая производная лог-функции правдоподобия равна $\ell''(\hat{\theta})=-100$. Оценка стандартной ошибки для $\hat{\theta}$ равна
\begin{multicols}{3} % располагаем ответы в {k} колонки
   \begin{choices} % опция [o] не рандомизирует порядок ответов
      \wrongchoice{100}
      \wrongchoice{10}
      \wrongchoice{1}
       \wrongchoice{0.01}
       \correctchoice{0.1}
      \end{choices}
 \end{multicols}
  \end{questionmult}
}

\element{exam_15}{ % в фигурных скобках название группы вопросов
  \begin{questionmult}{19} % тип вопроса (questionmult — множественный выбор) и в фигурных — номер вопроса
Геродот Геликарнасский проверяет гипотезу $H_0: \; \mu=0, \; \sigma^2=1$ с помощью $LR$ статистики теста отношения правдоподобия. При подстановке оценок метода максимального правдоподобия в лог-функцию правдоподобия он получил $\ell=-177$, а при подстановке $\mu=0$ и $\sigma=1$ оказалось, что $\ell=-211$. Найдите значение $LR$ статистики и укажите её закон распределения при верной $H_0$
\begin{multicols}{3} % располагаем ответы в {k} колонки
   \begin{choices} % опция [o] не рандомизирует порядок ответов
      \wrongchoice{$LR=34$, $\chi^2_2$}
      \wrongchoice{$LR=34$, $\chi^2_{n-1}$}
      \wrongchoice{$LR=\ln 68$, $\chi^2_{n-2}$}
       \wrongchoice{$LR=\ln 34$, $\chi^2_{n-2}$}
       \correctchoice{$LR=68$, $\chi^2_2$}
      \end{choices}
 \end{multicols}
  \end{questionmult}
}


\element{exam_15}{ % в фигурных скобках название группы вопросов
  \begin{questionmult}{20} % тип вопроса (questionmult — множественный выбор) и в фигурных — номер вопроса
Геродот Геликарнасский проверяет гипотезу $H_0: \; \mu=2$. Лог-функция правдоподобия имеет вид $\ell(\mu,\nu)=-\frac{n}{2}\ln (2\pi)-\frac{n}{2}\ln \nu -\frac{\sum_{i=1}^n(x_i-\mu)^2}{2\nu}$. Оценка максимального правдоподобия для $\nu$ при предположении, что $H_0$ верна, равна
\begin{multicols}{3} % располагаем ответы в {k} колонки
   \begin{choices} % опция [o] не рандомизирует порядок ответов
      \wrongchoice{$\frac{\sum x_i^2 - 4\sum x_i}{n}$}
      \wrongchoice{$\frac{\sum x_i^2 - 4\sum x_i}{n}+1$}
      \wrongchoice{$\frac{\sum x_i^2 - 4\sum x_i}{n}+2$}
       \wrongchoice{$\frac{\sum x_i^2 - 4\sum x_i}{n}+3$}
       \correctchoice{$\frac{\sum x_i^2 - 4\sum x_i}{n}+4$}
      \end{choices}
 \end{multicols}
  \end{questionmult}
}


\element{exam_15}{ % в фигурных скобках название группы вопросов
  \begin{questionmult}{21} % тип вопроса (questionmult — множественный выбор) и в фигурных — номер вопроса
Ацтек Монтесума Илуикамина хочет оценить параметр $a$ методом максимального правдоподобия по выборке из неотрицательного распределения с функцией плотности $f(x)=\frac{1}{2}a^3x^2e^{-ax}$ при $x\geq 0$. Для этой цели ему достаточно максимизировать функцию
\begin{multicols}{3} % располагаем ответы в {k} колонки
   \begin{choices} % опция [o] не рандомизирует порядок ответов
      \wrongchoice{$3n\ln a - a \prod \ln x_i$}
      \wrongchoice{$3n\prod \ln a - a x^n$}
      \wrongchoice{$3n \ln a - an \ln x_i$}
       \wrongchoice{$3n \sum \ln a_i - a \sum \ln x_i$}
       \correctchoice{$3n \ln a - a \sum x_i$}
      \end{choices}
 \end{multicols}
  \end{questionmult}
}

\element{exam_15}{ % в фигурных скобках название группы вопросов
  \begin{questionmult}{22} % тип вопроса (questionmult — множественный выбор) и в фигурных — номер вопроса
Бессмертный гений поэзии Ли Бо оценивает математическое ожидание  по выборка размера $n$ из нормального распределения. Он построил оценку метода моментов, $\hat{\mu}_{MM}$, и оценку максимального правдоподобия, $\hat{\mu}_{ML}$. Про эти оценки можно утверждать, что
\begin{multicols}{2} % располагаем ответы в {k} колонки
   \begin{choices} % опция [o] не рандомизирует порядок ответов
      \wrongchoice{они не равны, но сближаются при $n\to \infty$}
      \wrongchoice{они не равны, и не сближаются при $n\to \infty$}
      \wrongchoice{ $\hat{\mu}_{MM}>\hat{\mu}_{ML}$}
       \wrongchoice{$\hat{\mu}_{MM}<\hat{\mu}_{ML}$ }
       \correctchoice{они равны}
      \end{choices}
 \end{multicols}
  \end{questionmult}
}


%% Ваня Станкевич

\element{exam_15}{ % в фигурных скобках название группы вопросов
  \begin{questionmult}{23} % тип вопроса (questionmult — множественный выбор) и в фигурных — номер вопроса
Проверяя гипотезу о равенстве дисперсий в двух выборках (размером в 3 и 5 наблюдений), Анаксимандр Милетский получил значение тестовой статистики 10. Если оценка дисперсии по первой выборке равна 8, то вторая оценка дисперсии может быть равна
\begin{multicols}{3} % располагаем ответы в {k} колонки
   \begin{choices} % опция [o] не рандомизирует порядок ответов
      \wrongchoice{$25$}
      \wrongchoice{$4/3$}
      \wrongchoice{$3/4$}
       \wrongchoice{$4$}
       \correctchoice{$80$}
      \end{choices}
 \end{multicols}
  \end{questionmult}
}

\element{exam_15}{ % в фигурных скобках название группы вопросов
  \begin{questionmult}{24} % тип вопроса (questionmult — множественный выбор) и в фигурных — номер вопроса
Пусть  $\hat{\sigma}^2_1$ — несмещенная оценка дисперсии, полученная по первой выборке размером $n_1$,   $\hat{\sigma}^2_2$ — несмещенная оценка дисперсии, полученная по второй выборке, с меньшим размером  $n_2$. Тогда статистика $\frac{\hat{\sigma}^2_1/n_1}{\hat{\sigma}^2_2/n_2}$  имеет распределение
\begin{multicols}{3} % располагаем ответы в {k} колонки
   \begin{choices} % опция [o] не рандомизирует порядок ответов
      \wrongchoice{$N(0;1)$}
      \wrongchoice{$\chi^2_{n_1+n_2}$}
      \wrongchoice{$F_{n_1,n_2}$}
       \wrongchoice{$t_{n_1+n_2-1}$}
       \wrongchoice{$F_{n_1-1,n_2-1}$}
      \end{choices}
 \end{multicols}
  \end{questionmult}
}

\element{exam_15}{ % в фигурных скобках название группы вопросов
  \begin{questionmult}{25} % тип вопроса (questionmult — множественный выбор) и в фигурных — номер вопроса
Зулус Чака каСензангакона проверяет гипотезу  о равенстве математических ожиданий в двух нормальных выборках небольших размеров $n_1$   и  $n_2$. Если дисперсии неизвестны, но равны, то тестовая статистика имеет распределение
\begin{multicols}{3} % располагаем ответы в {k} колонки
   \begin{choices} % опция [o] не рандомизирует порядок ответов
      \wrongchoice{$t_{n_1+n_2-2}$}
      \wrongchoice{$t_{n_1+n_2}$}
      \wrongchoice{$F_{n_1,n_2}$}
       \correctchoice{$t_{n_1+n_2-1}$}
       \wrongchoice{$\chi^2_{n_1+n_2-1}$}
      \end{choices}
 \end{multicols}
  \end{questionmult}
}

\element{exam_15}{ % в фигурных скобках название группы вопросов
  \begin{questionmult}{26} % тип вопроса (questionmult — множественный выбор) и в фигурных — номер вопроса
Критерий знаков проверяет нулевую гипотезу
%\begin{multicols}{3} % располагаем ответы в {k} колонки
   \begin{choices} % опция [o] не рандомизирует порядок ответов
      \wrongchoice{о равенстве математических ожиданий двух нормально распределенных случайных величин}
      \wrongchoice{о совпадении функции распределения случайной величины с заданной теоретической функцией распределения}
      \wrongchoice{о равенстве нулю вероятности того, что случайная величина $X$ окажется больше случайной величины $Y$, если альтернативная гипотеза записана как $\mu_X>\mu_Y$ }
       \correctchoice{о равенстве нулю вероятности того, что случайная величина $X$ окажется больше случайной величины $Y$, если альтернативная гипотеза записана как $\mu_X>\mu_Y$}
       \wrongchoice{о равенстве $1/2$ вероятности того, что случайная величина $X$ окажется больше случайной величины $Y$, если альтернативная гипотеза записана как $\mu_X>\mu_Y$}
      \end{choices}
% \end{multicols}
  \end{questionmult}
}


\element{exam_15}{ % в фигурных скобках название группы вопросов
  \begin{questionmult}{27} % тип вопроса (questionmult — множественный выбор) и в фигурных — номер вопроса
Вероятность ошибки первого рода, $\alpha$, и вероятность ошибки второго рода, $\beta$, всегда связаны соотношением
\begin{multicols}{3} % располагаем ответы в {k} колонки
   \begin{choices} % опция [o] не рандомизирует порядок ответов
      \wrongchoice{$\alpha+\beta=1$}
      \wrongchoice{$\alpha+\beta \leq 1$}
      \wrongchoice{$\alpha+\beta \geq 1$}
       \wrongchoice{$\alpha\leq \beta $}
       \wrongchoice{$\alpha\geq \beta $}
      \end{choices}
 \end{multicols}
  \end{questionmult}
}


\element{exam_15}{ % в фигурных скобках название группы вопросов
  \begin{questionmult}{28} % тип вопроса (questionmult — множественный выбор) и в фигурных — номер вопроса
Среди 100 случайно выбранных ацтеков 20 платят дань Кулуакану, а 80 — Аскапоцалько. Соответственно, оценка доли ацтеков, платящих дань Кулуакану, равна $\hat{p}=0.2$. Разумная оценка стандартного отклонения случайной величины $\hat{p}$ равна
\begin{multicols}{3} % располагаем ответы в {k} колонки
   \begin{choices} % опция [o] не рандомизирует порядок ответов
      \wrongchoice{$0.4$}
      \wrongchoice{$0.16$}
      \wrongchoice{$1.6$}
       \wrongchoice{$0.016$}
       \correctchoice{$0.04$}
      \end{choices}
 \end{multicols}
  \end{questionmult}
}



%% ЕВ Коссова

\element{exam_15}{ % в фигурных скобках название группы вопросов
  \begin{questionmult}{29} % тип вопроса (questionmult — множественный выбор) и в фигурных — номер вопроса
Датчик случайных чисел выдал следующие значения псевдо случайной величины: $0.78$, $0.48$. Вычислите значение критерия Колмогорова и проверьте гипотезу $H_0$ о соответствии распределения равномерному на $[0;1]$. Критическое значение статистики Колмогорова для уровня значимости 0.1 и двух наблюдений равно $0.776$.
\begin{multicols}{3} % располагаем ответы в {k} колонки
   \begin{choices} % опция [o] не рандомизирует порядок ответов
      \wrongchoice{0.3, $H_0$ не отвергается}
      \wrongchoice{1.26, $H_0$ отвергается}
      \wrongchoice{0.48, $H_0$ не отвергается}
       \wrongchoice{0.37, $H_0$ не отвергается}
       \correctchoice{0.78, $H_0$ отвергается}
      \end{choices}
 \end{multicols}
  \end{questionmult}
}

\element{exam_15}{ % в фигурных скобках название группы вопросов
  \begin{questionmult}{30} % тип вопроса (questionmult — множественный выбор) и в фигурных — номер вопроса
У пяти случайно выбранных студентов первого потока результаты за контрольную по статистике оказались равны  82, 47, 20, 43 и 73. У четырёх случайно выбранных студентов второго потока — 68, 83, 60 и 52. Вычислите статистику Вилкоксона и проверьте гипотезу $H_0$ об однородности результатов студентов двух потоков. Критические значения статистики Вилкоксона равны $T_L=12$ и $T_R=28$.
\begin{multicols}{3} % располагаем ответы в {k} колонки
   \begin{choices} % опция [o] не рандомизирует порядок ответов
      \wrongchoice{20, $H_0$ не отвергается}
      \wrongchoice{65.75, $H_0$ отвергается}
      \wrongchoice{53, $H_0$ отвергается}
       \wrongchoice{12.75, $H_0$ не отвергается}
       \correctchoice{24, $H_0$ не отвергается}
      \end{choices}
 \end{multicols}
  \end{questionmult}
}

\element{exam_15}{ % в фигурных скобках название группы вопросов
  \begin{questionmult}{31} % тип вопроса (questionmult — множественный выбор) и в фигурных — номер вопроса
 Производитель мороженного попросил оценить по 10-бальной шкале два вида мороженного: с кусочками шоколада и с орешками. Было опрошено 5 человек.


 \begin{tabular}{c|ccccc}
  & Евлампий & Аристарх & Капитолина & Аграфена & Эвридика \\
 \hline
С крошкой & 10 & 6 & 7 & 5 & 4 \\
С орехами & 9 & 8 & 8 & 7 & 6 \\
 \end{tabular}


Вычислите модуль значения статистики теста знаков. Используя нормальную аппроксимацию, проверьте на уровне значимости $0.05$ гипотезу об отсутствии предпочтения мороженного с орешками против альтернативы, что мороженное с орешками вкуснее.
\begin{multicols}{3} % располагаем ответы в {k} колонки
   \begin{choices} % опция [o] не рандомизирует порядок ответов
      \wrongchoice{1.65, $H_0$ отвергается}
      \wrongchoice{1.34, $H_0$ не отвергается}
      \wrongchoice{1.29, $H_0$ отвергается}
       \wrongchoice{1.29, $H_0$ не отвергается}
       \correctchoice{1.96, $H_0$ отвергается}
      \end{choices}
 \end{multicols}
  \end{questionmult}
}


\element{exam_15}{ % в фигурных скобках название группы вопросов
  \begin{questionmult}{32} % тип вопроса (questionmult — множественный выбор) и в фигурных — номер вопроса
По 10 наблюдениям проверяется гипотеза $H_0: \; \mu=10$ против $H_a: \; \mu \neq 10$ на выборке из нормального распределения с неизвестной дисперсией. Величина $\sqrt{n}\cdot (\bar{X}-\mu)/\hat{\sigma}$ оказалась равной $1$. P-значение примерно равно
\begin{multicols}{3} % располагаем ответы в {k} колонки
   \begin{choices} % опция [o] не рандомизирует порядок ответов
      \wrongchoice{$0.83$}
      \wrongchoice{$0.17$}
      \wrongchoice{$0.34$}
       \wrongchoice{$0.32$}
       \correctchoice{$0.16$}
      \end{choices}
 \end{multicols}
  \end{questionmult}
}
