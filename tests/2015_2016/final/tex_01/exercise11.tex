
\begin{question}
Величины \(X_1\), \(X_2\), \ldots, \(X_{2016}\) независимы и одинаково
распределены, \(\cN(\mu ; 42)\). Оказалось, что \(\bar X = -23\). Про
оценки метода моментов, \(\hat \mu_{MM}\), и метода максимального
правдоподобия, \(\hat \mu_{ML}\), можно утверждать, что
\begin{answerlist}
  \item \(\hat \mu_ML = -23\), \(\hat\mu_MM > -23\)
  \item \(\hat \mu_ML = -23\), \(\hat\mu_MM = -23\)
  \item \(\hat \mu_ML > -23\), \(\hat\mu_MM = -23\)
  \item \(\hat \mu_ML < -23\), \(\hat\mu_MM = -23\)
  \item \(\hat \mu_ML = -23\), \(\hat\mu_MM < -23\)
\end{answerlist}
\end{question}

\begin{solution}
\begin{answerlist}
  \item Неверно
  \item Отлично
  \item Неверно
  \item Неверно
  \item Неверно
\end{answerlist}
\end{solution}

