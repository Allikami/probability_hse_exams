
\begin{question}
Пусть \(X_1, \ldots , X_n\) — случайная выборка. Случайные величины
\(X_1, \ldots, X_n\) имеют дискретное распределение, которое задано при
помощи таблицы

\begin{center}
\begin{tabular}{lrrr} \toprule
$x$  & -3 & 0 & 2 \\
\midrule
$\P(X_i=x)$ & $\frac{2}{3} - \theta$ & $\frac{1}{3}$ & $\theta$\\
\bottomrule
\end{tabular}
\end{center}

При каком значении константы \(c\) оценка
\(\hat{\theta}_n = c (\bar{X} + 2)\) является несмещённой?
\begin{answerlist}
  \item \(\frac{1}{5}\)
  \item \(3\)
  \item \(5\)
  \item \(\frac{1}{3}\)
  \item \(1\)
\end{answerlist}
\end{question}

\begin{solution}
\begin{answerlist}
  \item Отлично
  \item Неверно
  \item Неверно
  \item Неверно
  \item Неверно
\end{answerlist}
\end{solution}

