
\begin{question}
Случайные величины \(X_1\), \(X_2\) и \(X_3\) независимы и одинаково
распределены,

\begin{center}
  \begin{tabular}{lrr} \toprule
  $X_i$ & 3 & 5 \\
  \midrule
  $\P(\cdot)$ & $p$ & $1-p$ \\
  \bottomrule
  \end{tabular}
\end{center}

Имеется выборка из трёх наблюдений: \(X_1=5\), \(X_2=3\), \(X_3=5\).
Оценка неизвестного \(p\), полученная методом максимального
правдоподобия, равна:
\begin{answerlist}
  \item \(1/2\)
  \item Метод неприменим
  \item \(1/4\)
  \item \(2/3\)
  \item \(1/3\)
\end{answerlist}
\end{question}

\begin{solution}
\begin{answerlist}
  \item Неверно
  \item Неверно
  \item Неверно
  \item Неверно
  \item Отлично
\end{answerlist}
\end{solution}

