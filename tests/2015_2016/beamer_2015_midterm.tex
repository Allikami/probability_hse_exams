\documentclass[t]{beamer}

\usetheme{Boadilla} 
 \usecolortheme{seahorse} 

\setbeamertemplate{footline}[frame number]{} 
 \setbeamertemplate{navigation symbols}{} 
 \setbeamertemplate{footline}{}
\usepackage{cmap} 

\usepackage{mathtext} 
 \usepackage{booktabs} 

\usepackage{amsmath,amsfonts,amssymb,amsthm,mathtools}
\usepackage[T2A]{fontenc} 

\usepackage[utf8]{inputenc} 

\usepackage[english,russian]{babel} 

\DeclareMathOperator{\Lin}{\mathrm{Lin}} 
 \DeclareMathOperator{\Linp}{\Lin^{\perp}} 
 \DeclareMathOperator*\plim{plim}

 \DeclareMathOperator{\grad}{grad} 
 \DeclareMathOperator{\card}{card} 
 \DeclareMathOperator{\sgn}{sign} 
 \DeclareMathOperator{\sign}{sign} 
 \DeclareMathOperator*{\argmin}{arg\,min} 
 \DeclareMathOperator*{\argmax}{arg\,max} 
 \DeclareMathOperator*{\amn}{arg\,min} 
 \DeclareMathOperator*{\amx}{arg\,max} 
 \DeclareMathOperator{\cov}{Cov} 

\DeclareMathOperator{\Var}{Var} 
 \DeclareMathOperator{\Cov}{Cov} 
 \DeclareMathOperator{\Corr}{Corr} 
 \DeclareMathOperator{\E}{\mathbb{E}} 
 \let\P\relax 

\DeclareMathOperator{\P}{\mathbb{P}} 
 \newcommand{\cN}{\mathcal{N}} 
 \def \R{\mathbb{R}} 
 \def \N{\mathbb{N}} 
 \def \Z{\mathbb{Z}} 

\title{Midterm 2015} 
 \subtitle{Теория вероятностей и математическая статистика} 
 \author{Обратная связь: \url{https://github.com/bdemeshev/probability_hse_exams}} 
 \date{Последнее обновление: \today}
\begin{document} 

\frame[plain]{\titlepage}

 \begin{frame} \label{1} 
\begin{block}{1} 

Крошка Джон  попадает в яблочко с вероятностью $0.8$. Его выстрелы независимы. Вероятность, что он попадёт хотя бы один раз из двух равна
 


 \end{block} 
\begin{enumerate} 
\item[] \hyperlink{1-No}{\beamergotobutton{} $0.64$}
\item[] \hyperlink{1-No}{\beamergotobutton{} $0.36$}
\item[] \hyperlink{1-Yes}{\beamergotobutton{} $0.96$}
\item[] \hyperlink{1-No}{\beamergotobutton{} $0.9$
}
\item[] \hyperlink{1-No}{\beamergotobutton{} $0.8$}
\end{enumerate} 
\end{frame} 


 \begin{frame} \label{2} 
\begin{block}{2} 

Крошка Джон попадает в яблочко с вероятностью $0.8$. Его выстрелы независимы. Вероятность, что он попал оба раза, если известно, что он попал хотя бы один раз из двух, равна
    


 \end{block} 
\begin{enumerate} 
\item[] \hyperlink{2-No}{\beamergotobutton{} $1/3$}
\item[] \hyperlink{2-No}{\beamergotobutton{} $1/4$
}
\item[] \hyperlink{2-No}{\beamergotobutton{} $1/2$}
\item[] \hyperlink{2-Yes}{\beamergotobutton{} $2/3$}
\item[] \hyperlink{2-No}{\beamergotobutton{} $3/4$}
\end{enumerate} 
\end{frame} 


 \begin{frame} \label{3} 
\begin{block}{3} 

Имеется три монетки. Две «правильных» и одна — с «орлами» по обеим сторонам. Вася выбирает одну монетку наугад и подкидывает её два раза. Вероятность того, что оба раза выпадет орел равна
     


 \end{block} 
\begin{enumerate} 
\item[] \hyperlink{3-Yes}{\beamergotobutton{} $1/2$}
\item[] \hyperlink{3-No}{\beamergotobutton{} $3/4$
}
\item[] \hyperlink{3-No}{\beamergotobutton{} $1/3$}
\item[] \hyperlink{3-No}{\beamergotobutton{} $1/4$}
\item[] \hyperlink{3-No}{\beamergotobutton{} $2/3$}
\end{enumerate} 
\end{frame} 


 \begin{frame} \label{4} 
\begin{block}{4} 

Крошка Джон попадает в яблочко с вероятностью $0.8$. Его выстрелы независимы. Вероятность, что он попал во второй раз, если известно, что он попал хотя бы один раз из двух, равна


 \end{block} 
\begin{enumerate} 
\item[] \hyperlink{4-No}{\beamergotobutton{} $4/5$}
\item[] \hyperlink{4-No}{\beamergotobutton{} $3/4$}
\item[] \hyperlink{4-Yes}{\beamergotobutton{} $5/6$}
\item[] \hyperlink{4-No}{\beamergotobutton{} $1/2$}
\item[] \hyperlink{4-No}{\beamergotobutton{} $2/3$}
\end{enumerate} 
\end{frame} 


 \begin{frame} \label{5} 
\begin{block}{5} 

Если события $A$, $B$, $C$ попарно независимы, то


 \end{block} 
\begin{enumerate} 
\item[] \hyperlink{5-No}{\beamergotobutton{} События $A$, $B$, $C$ несовместны}
\item[] \hyperlink{5-No}{\beamergotobutton{} События $A$, $B$, $C$ зависимы в совокупности}
\item[] \hyperlink{5-No}{\beamergotobutton{} $\P(A\cap B\cap C)=\P(A)\P(B)\P(C)$
}
\item[] \hyperlink{5-No}{\beamergotobutton{} Событие $A\cup B\cup C$ обязательно произойдёт}
\item[] \hyperlink{5-No}{\beamergotobutton{} События $A$, $B$, $C$ независимы в совокупности}
\end{enumerate} 
\end{frame} 


 \begin{frame} \label{6} 
\begin{block}{6} 

Случайная величина $X$ равномерна на отрезке $[0;10]$. Вероятность $\P(X>3|X<7)$ равна
     


 \end{block} 
\begin{enumerate} 
\item[] \hyperlink{6-Yes}{\beamergotobutton{} $4/7$}
\item[] \hyperlink{6-No}{\beamergotobutton{} $7/10$}
\item[] \hyperlink{6-No}{\beamergotobutton{} $3/7$}
\item[] \hyperlink{6-No}{\beamergotobutton{} $3/10$}
\item[] \hyperlink{6-No}{\beamergotobutton{} $0.21$
}
\end{enumerate} 
\end{frame} 


 \begin{frame} \label{7} 
\begin{block}{7} 

Имеется три монетки. Две «правильных» и одна — с «орлами» по обеим сторонам. Вася выбирает одну монетку наугад и подкидывает её два раза. События $A = \{ \text{Орёл выпал при первом подбрасывании} \}$ и $B =\{\text{Орёл выпал при втором подбрасывании}\}$
   


 \end{block} 
\begin{enumerate} 
\item[] \hyperlink{7-No}{\beamergotobutton{} образуют полную группу событий}
\item[] \hyperlink{7-No}{\beamergotobutton{} удовлетворяют соотношению $\P(A\cap B)=\P(A)+\P(B) + \P(A\cup B)$
}
\item[] \hyperlink{7-No}{\beamergotobutton{} независимы}
\item[] \hyperlink{7-Yes}{\beamergotobutton{} удовлетворяют соотношению $\P(A|B)\geq \P(A)$}
\item[] \hyperlink{7-No}{\beamergotobutton{} несовместны}
\end{enumerate} 
\end{frame} 


 \begin{frame} \label{8} 
\begin{block}{8} 

В квадрат вписан круг. Наудачу в квадрат бросают восемь точек. Пусть $X$ — число точек, попавших в круг. Математическое ожидание величины $X$ равно
     


 \end{block} 
\begin{enumerate} 
\item[] \hyperlink{8-No}{\beamergotobutton{} $4 \pi$
}
\item[] \hyperlink{8-No}{\beamergotobutton{} $\pi / 2$}
\item[] \hyperlink{8-No}{\beamergotobutton{} $\pi / 4$}
\item[] \hyperlink{8-No}{\beamergotobutton{} $\pi$}
\item[] \hyperlink{8-Yes}{\beamergotobutton{} $2\pi$}
\end{enumerate} 
\end{frame} 


 \begin{frame} \label{9} 
\begin{block}{9} 

В квадрат вписан круг. Наудачу в квадрат бросают восемь точек. Пусть $X$ — число точек, попавших в круг. Дисперсия величины $X$ равна     


 \end{block} 
\begin{enumerate} 
\item[] \hyperlink{9-No}{\beamergotobutton{} $\pi^2 - 2 \pi$}
\item[] \hyperlink{9-No}{\beamergotobutton{} $3\pi^2 - 4$}
\item[] \hyperlink{9-No}{\beamergotobutton{} $\pi^2$}
\item[] \hyperlink{9-No}{\beamergotobutton{} $3\pi^2 - 2$
}
\item[] \hyperlink{9-Yes}{\beamergotobutton{} $2\pi - \pi^2 / 2$}
\end{enumerate} 
\end{frame} 


 \begin{frame} \label{10} 
\begin{block}{10} 

В квадрат вписан круг. Последовательно в квадрат наудачу бросают восемь точек. Пусть $Y$ — число точек, попавших в круг, при первых четырех бросаниях, а $Z$ — число точек, попавших в круг, при оставшихся четырех бросаниях. Ковариация $\Cov(Y,Z)$ равна
     


 \end{block} 
\begin{enumerate} 
\item[] \hyperlink{10-No}{\beamergotobutton{} $-\pi^2$}
\item[] \hyperlink{10-No}{\beamergotobutton{} $-2\pi$
}
\item[] \hyperlink{10-No}{\beamergotobutton{} $2\pi$}
\item[] \hyperlink{10-No}{\beamergotobutton{} $\pi^2$}
\item[] \hyperlink{10-Yes}{\beamergotobutton{} $0$}
\end{enumerate} 
\end{frame} 


 \begin{frame} \label{11} 
\begin{block}{11} 

В квадрат вписан круг. Последовательно в квадрат наудачу бросают восемь точек. Пусть $Y$ — число точек, попавших в круг, при первых четырех бросаниях, а $Z$ — число точек, попавших в круг, при оставшихся четырех бросаниях. Дисперсия $\Var(Y - Z)$ равна
 


 \end{block} 
\begin{enumerate} 
\item[] \hyperlink{11-No}{\beamergotobutton{}  $0$
}
\item[] \hyperlink{11-No}{\beamergotobutton{} $\pi^2$}
\item[] \hyperlink{11-No}{\beamergotobutton{} $3\pi^2 - 4$}
\item[] \hyperlink{11-Yes}{\beamergotobutton{} $2\pi - \pi^2 / 2$}
\item[] \hyperlink{11-No}{\beamergotobutton{} $\pi^2 - 2 \pi$}
\end{enumerate} 
\end{frame} 


 \begin{frame} \label{12} 
\begin{block}{12} 

В квадрат вписан круг. Наудачу в квадрат бросают восемь точек. Наиболее вероятное число точек, попавших в круг, равно
 


 \end{block} 
\begin{enumerate} 
\item[] \hyperlink{12-No}{\beamergotobutton{} $6$
}
\item[] \hyperlink{12-Yes}{\beamergotobutton{} $7$}
\item[] \hyperlink{12-No}{\beamergotobutton{} $4$}
\item[] \hyperlink{12-No}{\beamergotobutton{} $2\pi$}
\item[] \hyperlink{12-No}{\beamergotobutton{} $5$}
\end{enumerate} 
\end{frame} 


 \begin{frame} \label{13} 
\begin{block}{13} 

Совместное распределение пары величин $X$ и $Y$ задано таблицей:

\begin{center}
\begin{tabular}{c|cc}
 & $Y=-2$ & $Y=1$ \\
\hline
$X=-1$ & 0.1 & 0 \\
$X=0$ & 0.1 & 0.3 \\
$X=1$ & 0.2 & 0.3 \\
\end{tabular}
\end{center}
\vspace{0.2cm} 
 
 Всем известно, что Маша звонит Васе в среднем 10 раз в день. Число звонков, совершенных Машей, имеет распределение Пуассона. Вероятность того, что Маша ни разу не позвонит Васе в течение дня, равна

 \end{block} 
\begin{enumerate} 
\item[] \hyperlink{13-No}{\beamergotobutton{} $1 - e^{10}$}
\item[] \hyperlink{13-Yes}{\beamergotobutton{} $e^{-10}$}
\item[] \hyperlink{13-No}{\beamergotobutton{} $10\,e^{-10}$}
\item[] \hyperlink{13-No}{\beamergotobutton{} $\tfrac{1}{10!}e^{-10}$}
\item[] \hyperlink{13-No}{\beamergotobutton{} $1 - e^{-10}$}
\end{enumerate} 
\end{frame} 


 \begin{frame} \label{14} 
\begin{block}{14} 

Совместное распределение пары величин $X$ и $Y$ задано таблицей:

\begin{center}
\begin{tabular}{c|cc}
 & $Y=-2$ & $Y=1$ \\
\hline
$X=-1$ & 0.1 & 0 \\
$X=0$ & 0.1 & 0.3 \\
$X=1$ & 0.2 & 0.3 \\
\end{tabular}
\end{center}
\vspace{0.2cm} 
 
 Математическое ожидание величины $Y$ при условии, что $X=0$, равно
 


 \end{block} 
\begin{enumerate} 
\item[] \hyperlink{14-No}{\beamergotobutton{} $-0.1$}
\item[] \hyperlink{14-Yes}{\beamergotobutton{} $0.25$}
\item[] \hyperlink{14-No}{\beamergotobutton{} $0.2$}
\item[] \hyperlink{14-No}{\beamergotobutton{} $0$}
\item[] \hyperlink{14-No}{\beamergotobutton{} $0.1$}
\end{enumerate} 
\end{frame} 


 \begin{frame} \label{15} 
\begin{block}{15} 

Совместное распределение пары величин $X$ и $Y$ задано таблицей:

\begin{center}
\begin{tabular}{c|cc}
 & $Y=-2$ & $Y=1$ \\
\hline
$X=-1$ & 0.1 & 0 \\
$X=0$ & 0.1 & 0.3 \\
$X=1$ & 0.2 & 0.3 \\
\end{tabular}
\end{center}
\vspace{0.2cm} 
 
 Дисперсия случайной величины $X$ равна
 


 \end{block} 
\begin{enumerate} 
\item[] \hyperlink{15-No}{\beamergotobutton{} $0.6$}
\item[] \hyperlink{15-Yes}{\beamergotobutton{} $0.44$}
\item[] \hyperlink{15-No}{\beamergotobutton{} $1.04$
}
\item[] \hyperlink{15-No}{\beamergotobutton{} $0.4$}
\item[] \hyperlink{15-No}{\beamergotobutton{} $0.2$}
\end{enumerate} 
\end{frame} 


 \begin{frame} \label{16} 
\begin{block}{16} 

Совместное распределение пары величин $X$ и $Y$ задано таблицей:

\begin{center}
\begin{tabular}{c|cc}
 & $Y=-2$ & $Y=1$ \\
\hline
$X=-1$ & 0.1 & 0 \\
$X=0$ & 0.1 & 0.3 \\
$X=1$ & 0.2 & 0.3 \\
\end{tabular}
\end{center}
\vspace{0.2cm} 
 
 
Ковариация $\Cov(X, Y)$ равна
 


 \end{block} 
\begin{enumerate} 
\item[] \hyperlink{16-Yes}{\beamergotobutton{} $0.18$}
\item[] \hyperlink{16-No}{\beamergotobutton{} $0.1$}
\item[] \hyperlink{16-No}{\beamergotobutton{} $0.4$}
\item[] \hyperlink{16-No}{\beamergotobutton{} $0.9$}
\item[] \hyperlink{16-No}{\beamergotobutton{} $-0.7$}
\item[] \hyperlink{16-No}{\beamergotobutton{} $-0.5$}
\end{enumerate} 
\end{frame} 


 \begin{frame} \label{17} 
\begin{block}{17} 

Функция распределения абсолютно непрерывной случайной величины $X$ имеет вид
\[
F(x)=\begin{cases}
a, x<0,\\
b x^2+c, x \in [0,2],\\
d, x > 2.\\
\end{cases}
\]
\vspace{0.2cm} 
 
 Вероятность того, что $Y = 1$ при условии, что $X > 0$ равна
 


 \end{block} 
\begin{enumerate} 
\item[] \hyperlink{17-No}{\beamergotobutton{} $0.3$}
\item[] \hyperlink{17-Yes}{\beamergotobutton{} $0.6$}
\item[] \hyperlink{17-No}{\beamergotobutton{} $0.5$}
\item[] \hyperlink{17-No}{\beamergotobutton{} $0.4$
}
\item[] \hyperlink{17-No}{\beamergotobutton{} $0.2$}
\end{enumerate} 
\end{frame} 


 \begin{frame} \label{18} 
\begin{block}{18} 

Функция распределения абсолютно непрерывной случайной величины $X$ имеет вид
\[
F(x)=\begin{cases}
a, x<0,\\
b x^2+c, x \in [0,2],\\
d, x > 2.\\
\end{cases}
\]
\vspace{0.2cm} 
 
 Величина $X$ равномерна от $0$ до $4$. Вероятность того, что $X$ примет значение 1, равна
 


 \end{block} 
\begin{enumerate} 
\item[] \hyperlink{18-Yes}{\beamergotobutton{} $0$}
\item[] \hyperlink{18-No}{\beamergotobutton{} $0.25$}
\item[] \hyperlink{18-No}{\beamergotobutton{} $0.4$}
\item[] \hyperlink{18-No}{\beamergotobutton{} $0.8$
}
\item[] \hyperlink{18-No}{\beamergotobutton{} $0.5$}
\end{enumerate} 
\end{frame} 


 \begin{frame} \label{19} 
\begin{block}{19} 

Функция распределения абсолютно непрерывной случайной величины $X$ имеет вид
\[
F(x)=\begin{cases}
a, x<0,\\
b x^2+c, x \in [0,2],\\
d, x > 2.\\
\end{cases}
\]
\vspace{0.2cm} 
 
Величина $X$ имеет функцию плотности $f(x)=x/2$ на отрезке $[0;2]$. Значение $\E(X)$  равно
 


 \end{block} 
\begin{enumerate} 
\item[] \hyperlink{19-No}{\beamergotobutton{} $1/2$}
\item[] \hyperlink{19-Yes}{\beamergotobutton{} $4/3$}
\item[] \hyperlink{19-No}{\beamergotobutton{} $2$
}
\item[] \hyperlink{19-No}{\beamergotobutton{} $1$}
\item[] \hyperlink{19-No}{\beamergotobutton{} $0$}
\end{enumerate} 
\end{frame} 


 \begin{frame} \label{20} 
\begin{block}{20} 

Совместная функция плотности пары $X$ и $Y$ имеет вид
\[
f(x,y)=\begin{cases}
(x+y)/3, \; \text{ если } x\in[0;1], y\in [0;2] \\
0, \; \text{ иначе}
\end{cases}
\]
Функция распределения абсолютно непрерывной случайной величины $X$ имеет вид
\[
F(x)=\begin{cases}
a, x<0,\\
b x^2+c, x \in [0,2],\\
d, x > 2.\\
\end{cases}
\]
Выражение $a+b+c+d$ равно
 


 \end{block} 
\begin{enumerate} 
\item[] \hyperlink{20-No}{\beamergotobutton{} $2$
}
\item[] \hyperlink{20-No}{\beamergotobutton{} $1/4$}
\item[] \hyperlink{20-Yes}{\beamergotobutton{} $5/4$}
\item[] \hyperlink{20-No}{\beamergotobutton{} $1/2$}
\item[] \hyperlink{20-No}{\beamergotobutton{} $1$}
\end{enumerate} 
\end{frame} 


 \begin{frame} \label{21} 
\begin{block}{21} 

Совместная функция плотности пары $X$ и $Y$ имеет вид
\[
f(x,y)=\begin{cases}
(x+y)/3, \; \text{ если } x\in[0;1], y\in [0;2] \\
0, \; \text{ иначе}
\end{cases}
\]

\vspace{0.5cm} 
 
Если функция $h(x,y)=c\cdot x\cdot f(x,y)$ также является совместной функцией плотности, то константа $c$ равна
 


 \end{block} 
\begin{enumerate} 
\item[] \hyperlink{21-No}{\beamergotobutton{} $9$}
\item[] \hyperlink{21-No}{\beamergotobutton{} $5$}
\item[] \hyperlink{21-No}{\beamergotobutton{} $5/9$}
\item[] \hyperlink{21-Yes}{\beamergotobutton{} $9/5$}
\item[] \hyperlink{21-No}{\beamergotobutton{} $1$}
\end{enumerate} 
\end{frame} 


 \begin{frame} \label{22} 
\begin{block}{22} 

Совместная функция плотности пары $X$ и $Y$ имеет вид
\[
f(x,y)=\begin{cases}
(x+y)/3, \; \text{ если } x\in[0;1], y\in [0;2] \\
0, \; \text{ иначе}
\end{cases}
\]

\vspace{0.5cm} 
 
 Вероятность $\P(X<0.5, Y<1)$ равна
 


 \end{block} 
\begin{enumerate} 
\item[] \hyperlink{22-Yes}{\beamergotobutton{} $1/8$}
\item[] \hyperlink{22-No}{\beamergotobutton{} $5/8$}
\item[] \hyperlink{22-No}{\beamergotobutton{} $3/5$}
\item[] \hyperlink{22-No}{\beamergotobutton{} $5/6$}
\item[] \hyperlink{22-No}{\beamergotobutton{} $3/8$
}
\end{enumerate} 
\end{frame} 


 \begin{frame} \label{23} 
\begin{block}{23} 

Совместная функция плотности пары $X$ и $Y$ имеет вид
\[
f(x,y)=\begin{cases}
(x+y)/3, \; \text{ если } x\in[0;1], y\in [0;2] \\
0, \; \text{ иначе}
\end{cases}
\]
Условная функция плотности  $f_{X|Y=1}(x)$ равна

 \end{block} 
\begin{enumerate} 
\item[] \hyperlink{23-No}{\beamergotobutton{} $f_{X|Y=1}(x)=\begin{cases} (x+2)/2\, \text{ если } x\in [0;1] \\ 0, \text{ иначе }    \end{cases}$}
\item[] \hyperlink{23-No}{\beamergotobutton{} $f_{X|Y=1}(x)=\begin{cases} (2x+1)/2\, \text{ если } x\in [0;1] \\ 0, \text{ иначе }    \end{cases}$}
\item[] \hyperlink{23-Yes}{\beamergotobutton{} $f_{X|Y=1}(x)=\begin{cases} (2x+2)/3\, \text{ если } x\in [0;1] \\ 0, \text{ иначе }    \end{cases}$}
\item[] \hyperlink{23-No}{\beamergotobutton{} $f_{X|Y=1}(x)=\begin{cases} (x+4)/2\, \text{ если } x\in [0;1] \\ 0, \text{ иначе }    \end{cases}$}
%\item[] \hyperlink{23-No}{\beamergotobutton{} $f_{X|Y=1}(x)=\begin{cases} (4x+2)/3\, \text{ если } x\in [0;1] \\ 0, \text{ иначе }    \end{cases}$}
\end{enumerate} 
\end{frame} 


 \begin{frame} \label{24} 
\begin{block}{24} 

Известно, что $\E(X)=-1$, $\Var(X)=1$, $\E(Y)=-4$, $\Var(Y)=4$, $\Corr(X,Y)=-0.5$

\vspace{0.5cm} 
 
Математическое ожидание $\E(Y)$ равно
 


 \end{block} 
\begin{enumerate} 
\item[] \hyperlink{24-No}{\beamergotobutton{} $13/7$
}
\item[] \hyperlink{24-Yes}{\beamergotobutton{} $11/9$}
\item[] \hyperlink{24-No}{\beamergotobutton{} $2/3$}
\item[] \hyperlink{24-No}{\beamergotobutton{} $4/3$}
\item[] \hyperlink{24-No}{\beamergotobutton{} $6/5$}
\end{enumerate} 
\end{frame} 


 \begin{frame} \label{25} 
\begin{block}{25} 

Известно, что $\E(X)=-1$, $\Var(X)=1$, $\E(Y)=-4$, $\Var(Y)=4$, $\Corr(X,Y)=-0.5$

\vspace{0.5cm} 
 
Ковариация $\Cov(2X+Y,X-3Y)$ равна
 


 \end{block} 
\begin{enumerate} 
\item[] \hyperlink{25-No}{\beamergotobutton{} $-1$
}
\item[] \hyperlink{25-Yes}{\beamergotobutton{} $-5$}
\item[] \hyperlink{25-No}{\beamergotobutton{} $0$}
\item[] \hyperlink{25-No}{\beamergotobutton{} $5$}
\item[] \hyperlink{25-No}{\beamergotobutton{} $1$}
\end{enumerate} 
\end{frame} 


 \begin{frame} \label{26} 
\begin{block}{26} 

Корреляция $\Corr((1-X)/2,(Y+5)/2)$ равна
 


 \end{block} 
\begin{enumerate} 
\item[] \hyperlink{26-Yes}{\beamergotobutton{} $0.5$}
\item[] \hyperlink{26-No}{\beamergotobutton{} $1$
}
\item[] \hyperlink{26-No}{\beamergotobutton{} $-0.5$}
\item[] \hyperlink{26-No}{\beamergotobutton{} $-1/8$}
\item[] \hyperlink{26-No}{\beamergotobutton{} $1/8$}
\end{enumerate} 
\end{frame} 


 \begin{frame} \label{27} 
\begin{block}{27} 
У неотрицательной случайной величины $X$ известны $\E(X)=1$, $\Var(X)=4$. Вероятность $\P(X^2 \geq 25)$ обязательно попадает в интервал
 


 \end{block} 
\begin{enumerate} 
\item[] \hyperlink{27-No}{\beamergotobutton{} $[0;4/25]$}
\item[] \hyperlink{27-No}{\beamergotobutton{} $[0;4/625]$}
\item[] \hyperlink{27-No}{\beamergotobutton{} $[1/25;1]$}
\item[] \hyperlink{27-No}{\beamergotobutton{} $[0;1/25]$}
\item[] \hyperlink{27-Yes}{\beamergotobutton{} $[0;1/5]$}
\end{enumerate} 
\end{frame} 


 \begin{frame} \label{28} 
\begin{block}{28} 

Если $\E(X)=0$, $\Var(X)=1$, то наиболее узкий интервал, в который гарантированно попадает вероятность $\P(|X| \geq 4)$, равен


 \end{block} 
\begin{enumerate} 
\item[] \hyperlink{28-No}{\beamergotobutton{} $[0.5; 1]$
}
\item[] \hyperlink{28-No}{\beamergotobutton{} $[0.0625; 1]$}
\item[] \hyperlink{28-No}{\beamergotobutton{} $[0.25; 1]$}
\item[] \hyperlink{28-No}{\beamergotobutton{} $[0; 0.25]$}
\item[] \hyperlink{28-Yes}{\beamergotobutton{} $[0; 0.0625]$ }
\end{enumerate} 
\end{frame} 


 \begin{frame} \label{29} 
\begin{block}{29} 

Дана последовательность независимых случайных величин, имеющих равномерное на $(-1,1)$ распределение.  \textbf{НЕВЕРНЫМ} является утверждение

 \end{block} 
\begin{enumerate} 
\item[] \hyperlink{29-No}{\beamergotobutton{} 	$\sqrt3n\bar X$ сходится по распределению к стандартной нормальной величине}
\item[] \hyperlink{29-No}{\beamergotobutton{} Вероятность	$\P(\bar X>0)$ стремится к 0.5}
\item[] \hyperlink{29-Yes}{\beamergotobutton{}   $\bar X$ сходится по распределению к равномерной на (-1,1) величине }
\item[] \hyperlink{29-No}{\beamergotobutton{} $\bar X$ сходится по вероятности к нулю}
\item[] \hyperlink{29-No}{\beamergotobutton{} Вероятность	$\P(\bar X = 0)$ стремится к}
\end{enumerate} 
\end{frame} 


 \begin{frame} \label{30} 
\begin{block}{30} 

Функция плотности случайной величины $X$ имеет вид
\[
f(x)=\frac{1}{\sqrt{8\pi}} e^{-(x-3)^2/8}
\]
 \textbf{НЕВЕРНЫМ} является утверждение
 


 \end{block} 
\begin{enumerate} 
\item[] \hyperlink{30-No}{\beamergotobutton{} $\P(X=0)=0$}
\item[] \hyperlink{30-No}{\beamergotobutton{} $\P(X>3)=0.5$}
\item[] \hyperlink{30-No}{\beamergotobutton{} $\P(X<0)>0$}
\item[] \hyperlink{30-Yes}{\beamergotobutton{} $\Var(X)=8$ }
\item[] \hyperlink{30-No}{\beamergotobutton{} $\E(X)=3$}
\item[] \hyperlink{30-No}{\beamergotobutton{} $\max f(x) = \frac{1}{2\sqrt{2\pi}}$}
\end{enumerate} 
\end{frame} 


 \begin{frame} \label{31} 
\begin{block}{31} 

Величины $X_1$, $X_2$, \ldots независимы и одинаково распределены с $\E(X_i)=\mu$, $\Var(X_i)=\sigma^2$. К стандартному нормальному распределению  сходится последовательность случайных величин
 


 \end{block} 
\begin{enumerate} 
\item[] \hyperlink{31-No}{\beamergotobutton{} $(\bar X - \mu) /(\sqrt{n}\sigma)$}
\item[] \hyperlink{31-No}{\beamergotobutton{} $(\bar X - n\mu) /(\sqrt{n}\sigma)$}
\item[] \hyperlink{31-No}{\beamergotobutton{} $(\bar X - \mu) /\sigma$}
\item[] \hyperlink{31-Yes}{\beamergotobutton{} $\sqrt{n}(\bar X - \mu) /\sigma$ }
\item[] \hyperlink{31-No}{\beamergotobutton{} $\bar X$}
\end{enumerate} 
\end{frame} 


 \begin{frame} \label{32} 
\begin{block}{32} 

Пусть $X_1$, \ldots, $X_n$ — выборка объема $n$ из равномерного на $[a, b]$ распределения. Оценка $X_1+X_2$ параметра $c=a+b$ является
 


 \end{block} 
\begin{enumerate} 
\item[] \hyperlink{32-No}{\beamergotobutton{} смещенной и несостоятельной}
\item[] \hyperlink{32-Yes}{\beamergotobutton{} несмещенной и несостоятельной}
\item[] \hyperlink{32-No}{\beamergotobutton{} смещенной и состоятельной}
\item[] \hyperlink{32-No}{\beamergotobutton{} асимптотически несмещенной и состоятельной}
\item[] \hyperlink{32-No}{\beamergotobutton{} несмещенной и состоятельной}
\end{enumerate} 
\end{frame} 


 \begin{frame} \label{33} 
\begin{block}{33} 

Пусть $X_1$, \ldots, $X_n$ — выборка объема $n$ из некоторого распределения с конечным математическим ожиданием. Несмещенной и состоятельной оценкой математического ожидания является
 


 \end{block} 
\begin{enumerate} 
\item[] \hyperlink{33-Yes}{\beamergotobutton{} $\frac{X_1}{2n}+\frac{X_2+\ldots+X_{n-1}}{n-2}-\frac{X_n}{2 n}$}
\item[] \hyperlink{33-No}{\beamergotobutton{} $\frac{1}{3} X_1 + \frac{2}{3} X_2$}
\item[] \hyperlink{33-No}{\beamergotobutton{} $\frac{X_1}{2 n}+\frac{X_2+\ldots+X_{n-2}}{n-2}+\frac{X_n}{2 n}$}
\item[] \hyperlink{33-No}{\beamergotobutton{} $\frac{X_1}{2 n}+\frac{X_2+\ldots+X_{n-2}}{n-1}+\frac{X_n}{2 n}$}
\item[] \hyperlink{33-No}{\beamergotobutton{} $\frac{X_1+X_2}{2}$}
\end{enumerate} 
\end{frame} 


 \begin{frame} \label{34} 
\begin{block}{34} 

Пусть $X_1$,\ldots, $X_n$ — выборка объема $n$ из равномерного на $[0, \theta]$ распределения. Оценка параметра $\theta$ методом моментов по $k$-му моменту имеет вид:
 


 \end{block} 
\begin{enumerate} 
\item[] \hyperlink{34-No}{\beamergotobutton{} $\sqrt[k]k \overline{X^k}$}
\item[] \hyperlink{34-No}{\beamergotobutton{} $\sqrt[k](k+1) \overline{X^k}$}
\item[] \hyperlink{34-No}{\beamergotobutton{} $\sqrt[k]k \overline{X^k}$}
\item[] \hyperlink{34-Yes}{\beamergotobutton{} $\sqrt[k](k+1) \overline{X^k}$}
\item[] \hyperlink{34-No}{\beamergotobutton{} $\sqrt[k+1](k+1) \overline{X^k}$}
\end{enumerate} 
\end{frame} 


 \begin{frame} \label{35} 
\begin{block}{35} 

Пусть $X_1$, \ldots, $X_n$ — выборка объема $n$ из равномерного на $[0, \theta]$ распределения. Состоятельной оценкой параметра $\theta$ является:
 


 \end{block} 
\begin{enumerate} 
\item[] \hyperlink{35-No}{\beamergotobutton{} $X_(n)$}
\item[] \hyperlink{35-No}{\beamergotobutton{} $X_(n-1)$}
\item[] \hyperlink{35-No}{\beamergotobutton{} $\frac{n}{n+1} X_{(n-1)}$}
\item[] \hyperlink{35-No}{\beamergotobutton{} $\frac{n^2}{n^2-n+3} X_{(n-3)}$}
\item[] \hyperlink{35-Yes}{\beamergotobutton{} все перечисленные случайные величины}
\end{enumerate} 
\end{frame} 


 \begin{frame} \label{36} 
\begin{block}{36} 

Пусть $X_1$, \ldots, $X_{2 n}$ — выборка объема $2 n$ из некоторого распределения. Какая из нижеперечисленных оценок математического ожидания имеет наименьшую дисперсию?
 


 \end{block} 
\begin{enumerate} 
\item[] \hyperlink{36-No}{\beamergotobutton{} $X_1$}
\item[] \hyperlink{36-No}{\beamergotobutton{} $\frac{X_1+X_2}{2}$}
\item[] \hyperlink{36-No}{\beamergotobutton{} $\frac{1}{n} \sum_{i=1}^n X_i$}
\item[] \hyperlink{36-No}{\beamergotobutton{} $\frac{1}{n} \sum_{i=n+1}^{2 n} X_i$}
\item[] \hyperlink{36-Yes}{\beamergotobutton{} $\frac{1}{2 n} \sum_{i=1}^{2 n} X_i$}
\end{enumerate} 
\end{frame} 


 \begin{frame} \label{37} 
\begin{block}{37} 

Пусть $X_1$, \ldots, $X_n$ — выборка объема $n$ из распределения Бернулли с параметром $p$. Статистика $X_2 X_{n-2}$ является
 


 \end{block} 
\begin{enumerate} 
\item[] \hyperlink{37-Yes}{\beamergotobutton{} несмещенной оценкой $p^2$}
\item[] \hyperlink{37-No}{\beamergotobutton{} состоятельной оценкой $p^2$}
\item[] \hyperlink{37-No}{\beamergotobutton{} эффективной оценкой $p^2$}
\item[] \hyperlink{37-No}{\beamergotobutton{} асимптотически нормальной оценкой $p^2$}
\item[] \hyperlink{37-No}{\beamergotobutton{} оценкой максимального правдоподобия}
\end{enumerate} 
\end{frame} 


 \begin{frame} \label{38} 
\begin{block}{38} 

Пусть $X_1$, \ldots, $X_n$ — выборка объема $n$ из равномерного на $[a, b]$ распределения. Выберите наиболее точный ответ из предложенных. Оценка $\theta^*_n = X_{(n)}-X_{(1)}$ длины отрезка $[a,b]$ является
 


 \end{block} 
\begin{enumerate} 
\item[] \hyperlink{38-No}{\beamergotobutton{} состоятельной и асимптотически смещённой}
\item[] \hyperlink{38-No}{\beamergotobutton{} несостоятельной и асимптотически несмещенной}
\item[] \hyperlink{38-Yes}{\beamergotobutton{} состоятельной и асимптотически несмещенной}
\item[] \hyperlink{38-No}{\beamergotobutton{} нормально распределённой}
\item[] \hyperlink{38-No}{\beamergotobutton{} несмещенной}
\end{enumerate} 
\end{frame} 


 \begin{frame} \label{39} 
\begin{block}{39} 

Мощностью теста называется
 


 \end{block} 
\begin{enumerate} 
\item[] \hyperlink{39-No}{\beamergotobutton{} Вероятность принять неверную гипотезу}
\item[] \hyperlink{39-No}{\beamergotobutton{} Единица минус  вероятность отвергнуть основную гипотезу, когда она верна}
\item[] \hyperlink{39-Yes}{\beamergotobutton{} Единица минус  вероятность отвергнуть альтернативную гипотезу, когда она верна}
\item[] \hyperlink{39-No}{\beamergotobutton{} Вероятность отвергнуть альтернативную гипотезу, когда она верна}
\item[] \hyperlink{39-No}{\beamergotobutton{} Вероятность отвергнуть основную гипотезу, когда она верна}
\end{enumerate} 
\end{frame} 


 \begin{frame} \label{40} 
\begin{block}{40} 

Если P-значение (P-value) больше уровня значимости  $\alpha$, то гипотеза  $H_0: \; \sigma=1$


 \end{block} 
\begin{enumerate} 
\item[] \hyperlink{40-No}{\beamergotobutton{} Отвергается, только если  $H_a: \; \sigma<1$}
\item[] \hyperlink{40-No}{\beamergotobutton{} Отвергается}
\item[] \hyperlink{40-No}{\beamergotobutton{} Отвергается, только если  $H_a: \; \sigma>1$}
\item[] \hyperlink{40-Yes}{\beamergotobutton{} Не отвергается}
\item[] \hyperlink{40-No}{\beamergotobutton{} Отвергается, только если  $H_a: \; \sigma\neq 1$}
\end{enumerate} 
\end{frame} 


 \begin{frame} \label{41} 
\begin{block}{41} 

Имеется случайная выборка размера $n$ из нормального распределения. При проверке гипотезы о равенстве математического ожидания заданному значению при известной дисперсии используется статистика, имеющая распределение
 


 \end{block} 
\begin{enumerate} 
\item[] \hyperlink{41-No}{\beamergotobutton{}  $t_n-1$}
\item[] \hyperlink{41-No}{\beamergotobutton{} $\chi^2_n$}
\item[] \hyperlink{41-Yes}{\beamergotobutton{} $N(0,1)$}
\item[] \hyperlink{41-No}{\beamergotobutton{} $\chi^2_n-1$}
\item[] \hyperlink{41-No}{\beamergotobutton{} $t_n$}
\end{enumerate} 
\end{frame} 


 \begin{frame} \label{42} 
\begin{block}{42} 

Имеется случайная выборка размера $n$ из нормального распределения. При проверке гипотезы о равенстве дисперсии заданному значению при неизвестном математическом ожидании используется статистика, имеющая распределение
 


 \end{block} 
\begin{enumerate} 
\item[] \hyperlink{42-Yes}{\beamergotobutton{} $\chi^2_n-1$}
\item[] \hyperlink{42-No}{\beamergotobutton{} $t_n$}
\item[] \hyperlink{42-No}{\beamergotobutton{}  $t_n-1$}
\item[] \hyperlink{42-No}{\beamergotobutton{} $N(0,1)$}
\item[] \hyperlink{42-No}{\beamergotobutton{} $\chi^2_n$}
\end{enumerate} 
\end{frame} 


 \begin{frame} \label{43} 
\begin{block}{43} 

По случайной выборке из 100 наблюдений было оценено выборочное среднее $\bar{X}=20$  и несмещенная оценка дисперсии  $\hat{\sigma}^2=25$. В рамках проверки гипотезы $H_0: \; \mu=15$  против альтернативной гипотезы $H_a: \; \mu>15$  можно сделать следующее заключение
 


 \end{block} 
\begin{enumerate} 
\item[] \hyperlink{43-No}{\beamergotobutton{} Гипотеза $H_0$  не отвергается на любом разумном уровне значимости}
\item[] \hyperlink{43-No}{\beamergotobutton{} Гипотеза $H_0$  отвергается на уровне значимости 5\%, но не  на уровне значимости 1\%}
\item[] \hyperlink{43-No}{\beamergotobutton{} Гипотеза  $H_0$ отвергается на уровне значимости 10\%, но не на уровне значимости 5\%}
\item[] \hyperlink{43-Yes}{\beamergotobutton{} Гипотеза $H_0$  отвергается на любом разумном уровне значимости}
\item[] \hyperlink{43-No}{\beamergotobutton{} Гипотеза  $H_0$ отвергается на уровне значимости 20\%, но не  на уровне значимости 10\%}
\end{enumerate} 
\end{frame} 


 \begin{frame} \label{44} 
\begin{block}{44} 

На основе случайной выборки, содержащей одно наблюдение  $X_1$, тестируется гипотеза $H_0: \; X_1 \sim U[0;1]$  против альтернативной гипотезы  $H_a: \; X_1 \sim U[0.5;1.5]$. Рассматривается критерий: если $X_1>0.8$, то гипотеза $H_0$  отвергается в пользу гипотезы  $H_a$. Вероятность ошибки 2-го рода для этого критерия равна:
 


 \end{block} 
\begin{enumerate} 
\item[] \hyperlink{44-Yes}{\beamergotobutton{} 0.3}
\item[] \hyperlink{44-No}{\beamergotobutton{} 0.1}
\item[] \hyperlink{44-No}{\beamergotobutton{} 0.2}
\item[] \hyperlink{44-No}{\beamergotobutton{} 0.5}
\item[] \hyperlink{44-No}{\beamergotobutton{} 0.4}
\end{enumerate} 
\end{frame} 


 \begin{frame} \label{45} 
\begin{block}{45} 

Пусть $X_1$, $X_2$, \ldots, $X_n$ — случайная выборка размера 36 из нормального распределения $N(\mu, 9)$. Для тестирования основной гипотезы  $H_0: \; \mu=0$  против альтернативной $H_a: \; \mu=-2$   вы используете критерий: если  $\bar{X}\geq -1$, то вы не отвергаете гипотезу $H_0$, в противном случае вы отвергаете гипотезу  $H_0$ в пользу гипотезы  $H_a$. Мощность критерия равна
 


 \end{block} 
\begin{enumerate} 
\item[] \hyperlink{45-No}{\beamergotobutton{} 0.87}
\item[] \hyperlink{45-Yes}{\beamergotobutton{} 0.98}
\item[] \hyperlink{45-No}{\beamergotobutton{} 0.78}
\item[] \hyperlink{45-No}{\beamergotobutton{} 0.58}
\item[] \hyperlink{45-No}{\beamergotobutton{} 0.85}
\end{enumerate} 
\end{frame} 


 \begin{frame} \label{46} 
\begin{block}{46} 

Николай Коперник подбросил бутерброд 200 раз. Бутерброд упал маслом вниз 95 раз, а маслом вверх — 105 раз. Значение критерия $\chi^2$ Пирсона для проверки гипотезы о равной вероятности данных событий равно
 


 \end{block} 
\begin{enumerate} 
\item[] \hyperlink{46-No}{\beamergotobutton{} 0.75}
\item[] \hyperlink{46-No}{\beamergotobutton{} 7.5}
\item[] \hyperlink{46-No}{\beamergotobutton{} 0.5}
\item[] \hyperlink{46-No}{\beamergotobutton{} 0.25}
\item[] \hyperlink{46-Yes}{\beamergotobutton{} 0.5}
\end{enumerate} 
\end{frame} 


 \begin{frame} \label{47} 
\begin{block}{47} 

Каждое утро в 8:00 Иван Андреевич Крылов, либо завтракает, либо уже позавтракал. В это же время кухарка либо заглядывает к Крылову, либо нет. По таблице сопряженности вычислите  статистику $\chi^2$ Пирсона для тестирования гипотезы о том, что визиты кухарки не зависят от того, позавтракал ли уже Крылов или нет.
\begin{tabular}{c|cc}
Время 8:00 & кухарка заходит & кухарка не заходит \\
\hline
Крылов завтракает & 200 & 40 \\
Крылов уже позавтракал & 25 & 100 \\
\end{tabular}
 


 \end{block} 
\begin{enumerate} 
\item[] \hyperlink{47-No}{\beamergotobutton{} 79}
\item[] \hyperlink{47-No}{\beamergotobutton{} 100}
\item[] \hyperlink{47-Yes}{\beamergotobutton{} 139}
\item[] \hyperlink{47-No}{\beamergotobutton{} 39}
\item[] \hyperlink{47-No}{\beamergotobutton{} 179}
\end{enumerate} 
\end{frame} 


 \begin{frame} \label{48} 
\begin{block}{48} 

Ковариационная матрица вектора $X=(X_1,X_2)$ имеет вид
\[
\begin{pmatrix}
10 & 3 \\
3 & 8
\end{pmatrix}
\]
Дисперсия разности элементов вектора, $\Var(X_1-X_2)$, равняется
 


 \end{block} 
\begin{enumerate} 
\item[] \hyperlink{48-No}{\beamergotobutton{} 15}
\item[] \hyperlink{48-No}{\beamergotobutton{} 2}
\item[] \hyperlink{48-Yes}{\beamergotobutton{} 12}
\item[] \hyperlink{48-No}{\beamergotobutton{} 18}
\item[] \hyperlink{48-No}{\beamergotobutton{} 6}
\end{enumerate} 
\end{frame} 


 \begin{frame} \label{49} 
\begin{block}{49} 

Все условия регулярности для применения метода максимального правдоподобия выполнены. Вторая производная лог-функции правдоподобия равна $\ell''(\hat{\theta})=-100$. Оценка стандартной ошибки для $\hat{\theta}$ равна
 


 \end{block} 
\begin{enumerate} 
\item[] \hyperlink{49-No}{\beamergotobutton{} 1}
\item[] \hyperlink{49-No}{\beamergotobutton{} 100}
\item[] \hyperlink{49-Yes}{\beamergotobutton{} 0.1}
\item[] \hyperlink{49-No}{\beamergotobutton{} 0.01}
\item[] \hyperlink{49-No}{\beamergotobutton{} 10}
\end{enumerate} 
\end{frame} 


 \begin{frame} \label{50} 
\begin{block}{50} 

Геродот Геликарнасский проверяет гипотезу $H_0: \; \mu=0, \; \sigma^2=1$ с помощью $LR$ статистики теста отношения правдоподобия. При подстановке оценок метода максимального правдоподобия в лог-функцию правдоподобия он получил $\ell=-177$, а при подстановке $\mu=0$ и $\sigma=1$ оказалось, что $\ell=-211$. Найдите значение $LR$ статистики и укажите её закон распределения при верной $H_0$
 


 \end{block} 
\begin{enumerate} 
\item[] \hyperlink{50-No}{\beamergotobutton{} $LR=\ln 68$, $\chi^2_n-2$}
\item[] \hyperlink{50-No}{\beamergotobutton{} $LR=34$, $\chi^2_n-1$}
\item[] \hyperlink{50-No}{\beamergotobutton{} $LR=34$, $\chi^2_2$}
\item[] \hyperlink{50-No}{\beamergotobutton{} $LR=\ln 34$, $\chi^2_n-2$}
\item[] \hyperlink{50-Yes}{\beamergotobutton{} $LR=68$, $\chi^2_2$}
\end{enumerate} 
\end{frame} 


 \begin{frame} \label{51} 
\begin{block}{51} 

Геродот Геликарнасский проверяет гипотезу $H_0: \; \mu=2$. Лог-функция правдоподобия имеет вид $\ell(\mu,\nu)=-\frac{n}{2}\ln (2\pi)-\frac{n}{2}\ln \nu -\frac{\sum_{i=1}^n(x_i-\mu)^2}{2\nu}$. Оценка максимального правдоподобия для $\nu$ при предположении, что $H_0$ верна, равна
 


 \end{block} 
\begin{enumerate} 
\item[] \hyperlink{51-Yes}{\beamergotobutton{}$\frac{\sum x_i^2 - 4\sum x_i}{n}+4$}
\item[] \hyperlink{51-No}{\beamergotobutton{} $\frac{\sum x_i^2 - 4\sum x_i}{n}+3$}
\item[] \hyperlink{51-No}{\beamergotobutton{} $\frac{\sum x_i^2 - 4\sum x_i}{n}+2$}
\item[] \hyperlink{51-No}{\beamergotobutton{} $\frac{\sum x_i^2 - 4\sum x_i}{n}+1$}
\item[] \hyperlink{51-No}{\beamergotobutton{} $\frac{\sum x_i^2 - 4\sum x_i}{n}$}
\end{enumerate} 
\end{frame} 


 \begin{frame} \label{52} 
\begin{block}{52} 

Ацтек Монтесума Илуикамина хочет оценить параметр $a$ методом максимального правдоподобия по выборке из неотрицательного распределения с функцией плотности $f(x)=\frac{1}{2}a^3x^2e^{-ax}$ при $x\geq 0$. Для этой цели ему достаточно максимизировать функцию
 


 \end{block} 
\begin{enumerate} 
\item[] \hyperlink{52-Yes}{\beamergotobutton{} $3n \ln a - a \sum x_i$}
\item[] \hyperlink{52-No}{\beamergotobutton{} $3n\prod \ln a - a x^n$}
\item[] \hyperlink{52-No}{\beamergotobutton{} $3n\ln a - a \prod \ln x_i$}
\item[] \hyperlink{52-No}{\beamergotobutton{} $3n \ln a - an \ln x_i$}
\item[] \hyperlink{52-No}{\beamergotobutton{} $3n \sum \ln a_i - a \sum \ln x_i$}
\end{enumerate} 
\end{frame} 


 \begin{frame} \label{53} 
\begin{block}{53} 

Бессмертный гений поэзии Ли Бо оценивает математическое ожидание  по выборка размера $n$ из нормального распределения. Он построил оценку метода моментов, $\hat{\mu}_{MM}$, и оценку максимального правдоподобия, $\hat{\mu}_{ML}$. Про эти оценки можно утверждать, что
 


 \end{block} 
\begin{enumerate} 
\item[] \hyperlink{53-No}{\beamergotobutton{}  $\hat\mu_MM>\hat\mu_ML$}
\item[] \hyperlink{53-Yes}{\beamergotobutton{} они равны}
\item[] \hyperlink{53-No}{\beamergotobutton{} $\hat\mu_MM<\hat\mu_ML$ }
\item[] \hyperlink{53-No}{\beamergotobutton{} они не равны, но сближаются при $n\to \infty$}
\item[] \hyperlink{53-No}{\beamergotobutton{} они не равны, и не сближаются при $n\to \infty$}
\end{enumerate} 
\end{frame} 


 \begin{frame} \label{54} 
\begin{block}{54} 

Проверяя гипотезу о равенстве дисперсий в двух выборках (размером в 3 и 5 наблюдений), Анаксимандр Милетский получил значение тестовой статистики 10. Если оценка дисперсии по первой выборке равна 8, то вторая оценка дисперсии может быть равна
 


 \end{block} 
\begin{enumerate} 
\item[] \hyperlink{54-No}{\beamergotobutton{} $25$}
\item[] \hyperlink{54-No}{\beamergotobutton{} $4/3$}
\item[] \hyperlink{54-Yes}{\beamergotobutton{} $80$}
\item[] \hyperlink{54-No}{\beamergotobutton{} $3/4$}
\item[] \hyperlink{54-No}{\beamergotobutton{} $4$}
\end{enumerate} 
\end{frame} 


 \begin{frame} \label{55} 
\begin{block}{55} 

Пусть  $\hat{\sigma}^2_1$ — несмещенная оценка дисперсии, полученная по первой выборке размером $n_1$,   $\hat{\sigma}^2_2$ — несмещенная оценка дисперсии, полученная по второй выборке, с меньшим размером  $n_2$. Тогда статистика $\frac{\hat{\sigma}^2_1/n_1}{\hat{\sigma}^2_2/n_2}$  имеет распределение
 


 \end{block} 
\begin{enumerate} 
\item[] \hyperlink{55-No}{\beamergotobutton{} $\chi^2_{n_1+n_2}$}
\item[] \hyperlink{55-No}{\beamergotobutton{} $F_{n_1,n_2}$}
\item[] \hyperlink{55-No}{\beamergotobutton{} $F_{n_1-1,n_2-1}$}
\item[] \hyperlink{55-No}{\beamergotobutton{} $t_{n_1+n_2-1}$}
\item[] \hyperlink{55-No}{\beamergotobutton{} $N(0;1)$}
\end{enumerate} 
\end{frame} 


 \begin{frame} \label{56} 
\begin{block}{56} 

Зулус Чака каСензангакона проверяет гипотезу  о равенстве математических ожиданий в двух нормальных выборках небольших размеров $n_1$   и  $n_2$. Если дисперсии неизвестны, но равны, то тестовая статистика имеет распределение
 


 \end{block} 
\begin{enumerate} 
\item[] \hyperlink{56-No}{\beamergotobutton{} $F_{n_1,n_2}$}
\item[] \hyperlink{56-Yes}{\beamergotobutton{} $t_{n_1+n_2-1}$}
\item[] \hyperlink{56-No}{\beamergotobutton{} $t_{n_1+n_2}$}
\item[] \hyperlink{56-No}{\beamergotobutton{} $t_{n_1+n_2-2}$}
\item[] \hyperlink{56-No}{\beamergotobutton{} $\chi^2_{n_1+n_2-1}$}
\end{enumerate} 
\end{frame} 


 \begin{frame} \label{57} 
\begin{block}{57} 

Критерий знаков проверяет нулевую гипотезу
 


 \end{block} 
\begin{enumerate} 
\item[] \hyperlink{57-Yes}{\beamergotobutton{} о равенстве нулю вероятности того, что случайная величина $X$ окажется больше случайной величины $Y$, если альтернативная гипотеза записана как $\mu_X>\mu_Y$}
\item[] \hyperlink{57-No}{\beamergotobutton{} о равенстве нулю вероятности того, что случайная величина $X$ окажется больше случайной величины $Y$, если альтернативная гипотеза записана как $\mu_X>\mu_Y$ }
\item[] \hyperlink{57-No}{\beamergotobutton{} о равенстве математических ожиданий двух нормально распределенных случайных величин}
\item[] \hyperlink{57-No}{\beamergotobutton{} о совпадении функции распределения случайной величины с заданной теоретической функцией распределения}
\item[] \hyperlink{57-No}{\beamergotobutton{} о равенстве $1/2$ вероятности того, что случайная величина $X$ окажется больше случайной величины $Y$, если альтернативная гипотеза записана как $\mu_X>\mu_Y$}
\end{enumerate} 
\end{frame} 


 \begin{frame} \label{58} 
\begin{block}{58} 

Вероятность ошибки первого рода, $\alpha$, и вероятность ошибки второго рода, $\beta$, всегда связаны соотношением


 \end{block} 
\begin{enumerate} 
\item[] \hyperlink{58-No}{\beamergotobutton{} $\alpha+\beta \leq 1$}
\item[] \hyperlink{58-No}{\beamergotobutton{} $\alpha+\beta \geq 1$}
\item[] \hyperlink{58-No}{\beamergotobutton{} $\alpha\geq \beta $}
\item[] \hyperlink{58-No}{\beamergotobutton{} $\alpha+\beta=1$}
\item[] \hyperlink{58-No}{\beamergotobutton{} $\alpha\leq \beta $}
\end{enumerate} 
\end{frame} 


 \begin{frame} \label{59} 
\begin{block}{59} 

Среди 100 случайно выбранных ацтеков 20 платят дань Кулуакану, а 80 — Аскапоцалько. Соответственно, оценка доли ацтеков, платящих дань Кулуакану, равна $\hat{p}=0.2$. Разумная оценка стандартного отклонения случайной величины $\hat{p}$ равна
 


 \end{block} 
\begin{enumerate} 
\item[] \hyperlink{59-No}{\beamergotobutton{} $0.4$}
\item[] \hyperlink{59-No}{\beamergotobutton{} $1.6$}
\item[] \hyperlink{59-Yes}{\beamergotobutton{} $0.04$}
\item[] \hyperlink{59-No}{\beamergotobutton{} $0.16$}
\item[] \hyperlink{59-No}{\beamergotobutton{} $0.016$}
\end{enumerate} 
\end{frame} 


 \begin{frame} \label{60} 
\begin{block}{60} 

Датчик случайных чисел выдал следующие значения псевдо случайной величины: $0.78$, $0.48$. Вычислите значение критерия Колмогорова и проверьте гипотезу $H_0$ о соответствии распределения равномерному на $[0;1]$. Критическое значение статистики Колмогорова для уровня значимости 0.1 и двух наблюдений равно $0.776$.
 


 \end{block} 
\begin{enumerate} 
\item[] \hyperlink{60-No}{\beamergotobutton{} 1.26, $H_0$ отвергается}
\item[] \hyperlink{60-No}{\beamergotobutton{} 0.3, $H_0$ не отвергается}
\item[] \hyperlink{60-Yes}{\beamergotobutton{} 0.78, $H_0$ отвергается}
\item[] \hyperlink{60-No}{\beamergotobutton{} 0.48, $H_0$ не отвергается}
\item[] \hyperlink{60-No}{\beamergotobutton{} 0.37, $H_0$ не отвергается}
\end{enumerate} 
\end{frame} 


 \begin{frame} \label{61} 
\begin{block}{61} 

У пяти случайно выбранных студентов первого потока результаты за контрольную по статистике оказались равны  82, 47, 20, 43 и 73. У четырёх случайно выбранных студентов второго потока — 68, 83, 60 и 52. Вычислите статистику Вилкоксона и проверьте гипотезу $H_0$ об однородности результатов студентов двух потоков. Критические значения статистики Вилкоксона равны $T_L=12$ и $T_R=28$.
 


 \end{block} 
\begin{enumerate} 
\item[] \hyperlink{61-No}{\beamergotobutton{} 53, $H_0$ отвергается}
\item[] \hyperlink{61-No}{\beamergotobutton{} 20, $H_0$ не отвергается}
\item[] \hyperlink{61-No}{\beamergotobutton{} 65.75, $H_0$ отвергается}
\item[] \hyperlink{61-No}{\beamergotobutton{} 12.75, $H_0$ не отвергается}
\item[] \hyperlink{61-Yes}{\beamergotobutton{} 24, $H_0$ не отвергается}
\end{enumerate} 
\end{frame} 


 \begin{frame} \label{62} 
\begin{block}{62} 

 Производитель мороженного попросил оценить по 10-бальной шкале два вида мороженного: с кусочками шоколада и с орешками. Было опрошено 5 человек.


 \begin{tabular}{c|ccccc}
  & Евлампий & Аристарх & Капитолина & Аграфена & Эвридика \\
 \hline
С крошкой & 10 & 6 & 7 & 5 & 4 \\
С орехами & 9 & 8 & 8 & 7 & 6 \\
 \end{tabular}


Вычислите модуль значения статистики теста знаков. Используя нормальную аппроксимацию, проверьте на уровне значимости $0.05$ гипотезу об отсутствии предпочтения мороженного с орешками против альтернативы, что мороженное с орешками вкуснее.
 


 \end{block} 
\begin{enumerate} 
\item[] \hyperlink{62-No}{\beamergotobutton{} 1.29, $H_0$ не отвергается}
\item[] \hyperlink{62-No}{\beamergotobutton{} 1.34, $H_0$ не отвергается}
\item[] \hyperlink{62-No}{\beamergotobutton{} 1.65, $H_0$ отвергается}
\item[] \hyperlink{62-Yes}{\beamergotobutton{} 1.96, $H_0$ отвергается}
\item[] \hyperlink{62-No}{\beamergotobutton{} 1.29, $H_0$ отвергается}
\end{enumerate} 
\end{frame} 


 \begin{frame} \label{63} 
\begin{block}{63} 

По 10 наблюдениям проверяется гипотеза $H_0: \; \mu=10$ против $H_a: \; \mu \neq 10$ на выборке из нормального распределения с неизвестной дисперсией. Величина $\sqrt{n}\cdot (\bar{X}-\mu)/\hat{\sigma}$ оказалась равной $1$. P-значение примерно равно
 


 \end{block} 
\begin{enumerate} 
\item[] \hyperlink{63-No}{\beamergotobutton{} $0.32$}
\item[] \hyperlink{63-No}{\beamergotobutton{} $0.17$}
\item[] \hyperlink{63-Yes}{\beamergotobutton{} $0.16$}
\item[] \hyperlink{63-No}{\beamergotobutton{} $0.34$}
\item[] \hyperlink{63-No}{\beamergotobutton{} $0.83$}
\end{enumerate} 
\end{frame} 


 \begin{frame} \label{1-Yes} 
\begin{block}{1} 

Крошка Джон  попадает в яблочко с вероятностью $0.8$. Его выстрелы независимы. Вероятность, что он попадёт хотя бы один раз из двух равна
 


 \end{block} 
\begin{enumerate} 
\item[] \hyperlink{1-No}{\beamergotobutton{} $0.64$}
\item[] \hyperlink{1-No}{\beamergotobutton{} $0.36$}
\item[] \hyperlink{1-Yes}{\beamergotobutton{} $0.96$}
\item[] \hyperlink{1-No}{\beamergotobutton{} $0.9$
}
\item[] \hyperlink{1-No}{\beamergotobutton{} $0.8$}
\end{enumerate} 

 \textbf{Да!} 
 \hyperlink{2}{\beamerbutton{Следующий вопрос}}\end{frame} 


 \begin{frame} \label{2-Yes} 
\begin{block}{2} 

Крошка Джон попадает в яблочко с вероятностью $0.8$. Его выстрелы независимы. Вероятность, что он попал оба раза, если известно, что он попал хотя бы один раз из двух, равна
    


 \end{block} 
\begin{enumerate} 
\item[] \hyperlink{2-No}{\beamergotobutton{} $1/3$}
\item[] \hyperlink{2-No}{\beamergotobutton{} $1/4$
}
\item[] \hyperlink{2-No}{\beamergotobutton{} $1/2$}
\item[] \hyperlink{2-Yes}{\beamergotobutton{} $2/3$}
\item[] \hyperlink{2-No}{\beamergotobutton{} $3/4$}
\end{enumerate} 

 \textbf{Да!} 
 \hyperlink{3}{\beamerbutton{Следующий вопрос}}\end{frame} 


 \begin{frame} \label{3-Yes} 
\begin{block}{3} 

Имеется три монетки. Две «правильных» и одна — с «орлами» по обеим сторонам. Вася выбирает одну монетку наугад и подкидывает её два раза. Вероятность того, что оба раза выпадет орел равна
     


 \end{block} 
\begin{enumerate} 
\item[] \hyperlink{3-Yes}{\beamergotobutton{} $1/2$}
\item[] \hyperlink{3-No}{\beamergotobutton{} $3/4$
}
\item[] \hyperlink{3-No}{\beamergotobutton{} $1/3$}
\item[] \hyperlink{3-No}{\beamergotobutton{} $1/4$}
\item[] \hyperlink{3-No}{\beamergotobutton{} $2/3$}
\end{enumerate} 

 \textbf{Да!} 
 \hyperlink{4}{\beamerbutton{Следующий вопрос}}\end{frame} 


 \begin{frame} \label{4-Yes} 
\begin{block}{4} 

Крошка Джон попадает в яблочко с вероятностью $0.8$. Его выстрелы независимы. Вероятность, что он попал во второй раз, если известно, что он попал хотя бы один раз из двух, равна


 \end{block} 
\begin{enumerate} 
\item[] \hyperlink{4-No}{\beamergotobutton{} $4/5$}
\item[] \hyperlink{4-No}{\beamergotobutton{} $3/4$}
\item[] \hyperlink{4-Yes}{\beamergotobutton{} $5/6$}
\item[] \hyperlink{4-No}{\beamergotobutton{} $1/2$}
\item[] \hyperlink{4-No}{\beamergotobutton{} $2/3$}
\end{enumerate} 

 \textbf{Да!} 
 \hyperlink{5}{\beamerbutton{Следующий вопрос}}\end{frame} 


 \begin{frame} \label{5-Yes} 
\begin{block}{5} 

Если события $A$, $B$, $C$ попарно независимы, то


 \end{block} 
\begin{enumerate} 
\item[] \hyperlink{5-No}{\beamergotobutton{} События $A$, $B$, $C$ несовместны}
\item[] \hyperlink{5-No}{\beamergotobutton{} События $A$, $B$, $C$ зависимы в совокупности}
\item[] \hyperlink{5-No}{\beamergotobutton{} $\P(A\cap B\cap C)=\P(A)\P(B)\P(C)$
}
\item[] \hyperlink{5-No}{\beamergotobutton{} Событие $A\cup B\cup C$ обязательно произойдёт}
\item[] \hyperlink{5-No}{\beamergotobutton{} События $A$, $B$, $C$ независимы в совокупности}
\end{enumerate} 

 \textbf{Да!} 
 \hyperlink{6}{\beamerbutton{Следующий вопрос}}\end{frame} 


 \begin{frame} \label{6-Yes} 
\begin{block}{6} 

Случайная величина $X$ равномерна на отрезке $[0;10]$. Вероятность $\P(X>3|X<7)$ равна
     


 \end{block} 
\begin{enumerate} 
\item[] \hyperlink{6-Yes}{\beamergotobutton{} $4/7$}
\item[] \hyperlink{6-No}{\beamergotobutton{} $7/10$}
\item[] \hyperlink{6-No}{\beamergotobutton{} $3/7$}
\item[] \hyperlink{6-No}{\beamergotobutton{} $3/10$}
\item[] \hyperlink{6-No}{\beamergotobutton{} $0.21$
}
\end{enumerate} 

 \textbf{Да!} 
 \hyperlink{7}{\beamerbutton{Следующий вопрос}}\end{frame} 


 \begin{frame} \label{7-Yes} 
\begin{block}{7} 

Имеется три монетки. Две «правильных» и одна — с «орлами» по обеим сторонам. Вася выбирает одну монетку наугад и подкидывает её два раза. События $A = \{ \text{Орёл выпал при первом подбрасывании} \}$ и $B =\{\text{Орёл выпал при втором подбрасывании}\}$
   


 \end{block} 
\begin{enumerate} 
\item[] \hyperlink{7-No}{\beamergotobutton{} образуют полную группу событий}
\item[] \hyperlink{7-No}{\beamergotobutton{} удовлетворяют соотношению $\P(A\cap B)=\P(A)+\P(B) + \P(A\cup B)$
}
\item[] \hyperlink{7-No}{\beamergotobutton{} независимы}
\item[] \hyperlink{7-Yes}{\beamergotobutton{} удовлетворяют соотношению $\P(A|B)\geq \P(A)$}
\item[] \hyperlink{7-No}{\beamergotobutton{} несовместны}
\end{enumerate} 

 \textbf{Да!} 
 \hyperlink{8}{\beamerbutton{Следующий вопрос}}\end{frame} 


 \begin{frame} \label{8-Yes} 
\begin{block}{8} 

В квадрат вписан круг. Наудачу в квадрат бросают восемь точек. Пусть $X$ — число точек, попавших в круг. Математическое ожидание величины $X$ равно
     


 \end{block} 
\begin{enumerate} 
\item[] \hyperlink{8-No}{\beamergotobutton{} $4 \pi$
}
\item[] \hyperlink{8-No}{\beamergotobutton{} $\pi / 2$}
\item[] \hyperlink{8-No}{\beamergotobutton{} $\pi / 4$}
\item[] \hyperlink{8-No}{\beamergotobutton{} $\pi$}
\item[] \hyperlink{8-Yes}{\beamergotobutton{} $2\pi$}
\end{enumerate} 

 \textbf{Да!} 
 \hyperlink{9}{\beamerbutton{Следующий вопрос}}\end{frame} 


 \begin{frame} \label{9-Yes} 
\begin{block}{9} 

В квадрат вписан круг. Наудачу в квадрат бросают восемь точек. Пусть $X$ — число точек, попавших в круг. Дисперсия величины $X$ равна     


 \end{block} 
\begin{enumerate} 
\item[] \hyperlink{9-No}{\beamergotobutton{} $\pi^2 - 2 \pi$}
\item[] \hyperlink{9-No}{\beamergotobutton{} $3\pi^2 - 4$}
\item[] \hyperlink{9-No}{\beamergotobutton{} $\pi^2$}
\item[] \hyperlink{9-No}{\beamergotobutton{} $3\pi^2 - 2$
}
\item[] \hyperlink{9-Yes}{\beamergotobutton{} $2\pi - \pi^2 / 2$}
\end{enumerate} 

 \textbf{Да!} 
 \hyperlink{10}{\beamerbutton{Следующий вопрос}}\end{frame} 


 \begin{frame} \label{10-Yes} 
\begin{block}{10} 

В квадрат вписан круг. Последовательно в квадрат наудачу бросают восемь точек. Пусть $Y$ — число точек, попавших в круг, при первых четырех бросаниях, а $Z$ — число точек, попавших в круг, при оставшихся четырех бросаниях. Ковариация $\Cov(Y,Z)$ равна
     


 \end{block} 
\begin{enumerate} 
\item[] \hyperlink{10-No}{\beamergotobutton{} $-\pi^2$}
\item[] \hyperlink{10-No}{\beamergotobutton{} $-2\pi$
}
\item[] \hyperlink{10-No}{\beamergotobutton{} $2\pi$}
\item[] \hyperlink{10-No}{\beamergotobutton{} $\pi^2$}
\item[] \hyperlink{10-Yes}{\beamergotobutton{} $0$}
\end{enumerate} 

 \textbf{Да!} 
 \hyperlink{11}{\beamerbutton{Следующий вопрос}}\end{frame} 


 \begin{frame} \label{11-Yes} 
\begin{block}{11} 

В квадрат вписан круг. Последовательно в квадрат наудачу бросают восемь точек. Пусть $Y$ — число точек, попавших в круг, при первых четырех бросаниях, а $Z$ — число точек, попавших в круг, при оставшихся четырех бросаниях. Дисперсия $\Var(Y - Z)$ равна
 


 \end{block} 
\begin{enumerate} 
\item[] \hyperlink{11-No}{\beamergotobutton{}  $0$
}
\item[] \hyperlink{11-No}{\beamergotobutton{} $\pi^2$}
\item[] \hyperlink{11-No}{\beamergotobutton{} $3\pi^2 - 4$}
\item[] \hyperlink{11-Yes}{\beamergotobutton{} $2\pi - \pi^2 / 2$}
\item[] \hyperlink{11-No}{\beamergotobutton{} $\pi^2 - 2 \pi$}
\end{enumerate} 

 \textbf{Да!} 
 \hyperlink{12}{\beamerbutton{Следующий вопрос}}\end{frame} 


 \begin{frame} \label{12-Yes} 
\begin{block}{12} 

В квадрат вписан круг. Наудачу в квадрат бросают восемь точек. Наиболее вероятное число точек, попавших в круг, равно
 


 \end{block} 
\begin{enumerate} 
\item[] \hyperlink{12-No}{\beamergotobutton{} $6$
}
\item[] \hyperlink{12-Yes}{\beamergotobutton{} $7$}
\item[] \hyperlink{12-No}{\beamergotobutton{} $4$}
\item[] \hyperlink{12-No}{\beamergotobutton{} $2\pi$}
\item[] \hyperlink{12-No}{\beamergotobutton{} $5$}
\end{enumerate} 

 \textbf{Да!} 
 \hyperlink{13}{\beamerbutton{Следующий вопрос}}\end{frame} 


 \begin{frame} \label{13-Yes} 
\begin{block}{13} 

Совместное распределение пары величин $X$ и $Y$ задано таблицей:

\begin{center}
\begin{tabular}{c|cc}
 & $Y=-2$ & $Y=1$ \\
\hline
$X=-1$ & 0.1 & 0 \\
$X=0$ & 0.1 & 0.3 \\
$X=1$ & 0.2 & 0.3 \\
\end{tabular}
\end{center}
\vspace{0.2cm} 
 
 Всем известно, что Маша звонит Васе в среднем 10 раз в день. Число звонков, совершенных Машей, имеет распределение Пуассона. Вероятность того, что Маша ни разу не позвонит Васе в течение дня, равна
 


 \end{block} 
\begin{enumerate} 
\item[] \hyperlink{13-No}{\beamergotobutton{} $1 - e^{10}$}
\item[] \hyperlink{13-Yes}{\beamergotobutton{} $e^{-10}$}
\item[] \hyperlink{13-No}{\beamergotobutton{} $10\,e^{-10}$}
\item[] \hyperlink{13-No}{\beamergotobutton{} $\tfrac{1}{10!}e^{-10}$}
\item[] \hyperlink{13-No}{\beamergotobutton{} $1 - e^{-10}$}
\end{enumerate} 

 \textbf{Да!} 
 \hyperlink{14}{\beamerbutton{Следующий вопрос}}\end{frame} 


 \begin{frame} \label{14-Yes} 
\begin{block}{14} 

Совместное распределение пары величин $X$ и $Y$ задано таблицей:

\begin{center}
\begin{tabular}{c|cc}
 & $Y=-2$ & $Y=1$ \\
\hline
$X=-1$ & 0.1 & 0 \\
$X=0$ & 0.1 & 0.3 \\
$X=1$ & 0.2 & 0.3 \\
\end{tabular}
\end{center}
\vspace{0.2cm} 
 
 Математическое ожидание величины $Y$ при условии, что $X=0$, равно
 


 \end{block} 
\begin{enumerate} 
\item[] \hyperlink{14-No}{\beamergotobutton{} $-0.1$}
\item[] \hyperlink{14-Yes}{\beamergotobutton{} $0.25$}
\item[] \hyperlink{14-No}{\beamergotobutton{} $0.2$}
\item[] \hyperlink{14-No}{\beamergotobutton{} $0$}
\item[] \hyperlink{14-No}{\beamergotobutton{} $0.1$}
\end{enumerate} 

 \textbf{Да!} 
 \hyperlink{15}{\beamerbutton{Следующий вопрос}}\end{frame} 


 \begin{frame} \label{15-Yes} 
\begin{block}{15} 

Совместное распределение пары величин $X$ и $Y$ задано таблицей:

\begin{center}
\begin{tabular}{c|cc}
 & $Y=-2$ & $Y=1$ \\
\hline
$X=-1$ & 0.1 & 0 \\
$X=0$ & 0.1 & 0.3 \\
$X=1$ & 0.2 & 0.3 \\
\end{tabular}
\end{center}
\vspace{0.2cm} 
 
 Дисперсия случайной величины $X$ равна
 


 \end{block} 
\begin{enumerate} 
\item[] \hyperlink{15-No}{\beamergotobutton{} $0.6$}
\item[] \hyperlink{15-Yes}{\beamergotobutton{} $0.44$}
\item[] \hyperlink{15-No}{\beamergotobutton{} $1.04$
}
\item[] \hyperlink{15-No}{\beamergotobutton{} $0.4$}
\item[] \hyperlink{15-No}{\beamergotobutton{} $0.2$}
\end{enumerate} 

 \textbf{Да!} 
 \hyperlink{16}{\beamerbutton{Следующий вопрос}}\end{frame} 


 \begin{frame} \label{16-Yes} 
\begin{block}{16} 

Совместное распределение пары величин $X$ и $Y$ задано таблицей:

\begin{center}
\begin{tabular}{c|cc}
 & $Y=-2$ & $Y=1$ \\
\hline
$X=-1$ & 0.1 & 0 \\
$X=0$ & 0.1 & 0.3 \\
$X=1$ & 0.2 & 0.3 \\
\end{tabular}
\end{center}
\vspace{0.2cm} 
 
 
Ковариация $\Cov(X, Y)$ равна
 


 \end{block} 
\begin{enumerate} 
\item[] \hyperlink{16-Yes}{\beamergotobutton{} $0.18$}
\item[] \hyperlink{16-No}{\beamergotobutton{} $0.1$}
\item[] \hyperlink{16-No}{\beamergotobutton{} $0.4$}
\item[] \hyperlink{16-No}{\beamergotobutton{} $0.9$}
\item[] \hyperlink{16-No}{\beamergotobutton{} $-0.7$}
\item[] \hyperlink{16-No}{\beamergotobutton{} $-0.5$}
\end{enumerate} 

 \textbf{Да!} 
 \hyperlink{17}{\beamerbutton{Следующий вопрос}}\end{frame} 


 \begin{frame} \label{17-Yes} 
\begin{block}{17} 

Функция распределения абсолютно непрерывной случайной величины $X$ имеет вид
\[
F(x)=\begin{cases}
a, x<0,\\
b x^2+c, x \in [0,2],\\
d, x > 2.\\
\end{cases}
\]
\vspace{0.2cm} 
 
 Вероятность того, что $Y = 1$ при условии, что $X > 0$ равна
 


 \end{block} 
\begin{enumerate} 
\item[] \hyperlink{17-No}{\beamergotobutton{} $0.3$}
\item[] \hyperlink{17-Yes}{\beamergotobutton{} $0.6$}
\item[] \hyperlink{17-No}{\beamergotobutton{} $0.5$}
\item[] \hyperlink{17-No}{\beamergotobutton{} $0.4$
}
\item[] \hyperlink{17-No}{\beamergotobutton{} $0.2$}
\end{enumerate} 

 \textbf{Да!} 
 \hyperlink{18}{\beamerbutton{Следующий вопрос}}\end{frame} 


 \begin{frame} \label{18-Yes} 
\begin{block}{18} 

Функция распределения абсолютно непрерывной случайной величины $X$ имеет вид
\[
F(x)=\begin{cases}
a, x<0,\\
b x^2+c, x \in [0,2],\\
d, x > 2.\\
\end{cases}
\]
\vspace{0.2cm} 
 
 Величина $X$ равномерна от $0$ до $4$. Вероятность того, что $X$ примет значение 1, равна
 


 \end{block} 
\begin{enumerate} 
\item[] \hyperlink{18-Yes}{\beamergotobutton{} $0$}
\item[] \hyperlink{18-No}{\beamergotobutton{} $0.25$}
\item[] \hyperlink{18-No}{\beamergotobutton{} $0.4$}
\item[] \hyperlink{18-No}{\beamergotobutton{} $0.8$
}
\item[] \hyperlink{18-No}{\beamergotobutton{} $0.5$}
\end{enumerate} 

 \textbf{Да!} 
 \hyperlink{19}{\beamerbutton{Следующий вопрос}}\end{frame} 


 \begin{frame} \label{19-Yes} 
\begin{block}{19} 

Функция распределения абсолютно непрерывной случайной величины $X$ имеет вид
\[
F(x)=\begin{cases}
a, x<0,\\
b x^2+c, x \in [0,2],\\
d, x > 2.\\
\end{cases}
\]
\vspace{0.2cm} 
 
Величина $X$ имеет функцию плотности $f(x)=x/2$ на отрезке $[0;2]$. Значение $\E(X)$  равно
 


 \end{block} 
\begin{enumerate} 
\item[] \hyperlink{19-No}{\beamergotobutton{} $1/2$}
\item[] \hyperlink{19-Yes}{\beamergotobutton{} $4/3$}
\item[] \hyperlink{19-No}{\beamergotobutton{} $2$
}
\item[] \hyperlink{19-No}{\beamergotobutton{} $1$}
\item[] \hyperlink{19-No}{\beamergotobutton{} $0$}
\end{enumerate} 

 \textbf{Да!} 
 \hyperlink{20}{\beamerbutton{Следующий вопрос}}\end{frame} 


 \begin{frame} \label{20-Yes} 
\begin{block}{20  \textbf{Да!} \hyperlink{21}{\beamerbutton{Следующий вопрос}}} 

Совместная функция плотности пары $X$ и $Y$ имеет вид
\[
f(x,y)=\begin{cases}
(x+y)/3, \; \text{ если } x\in[0;1], y\in [0;2] \\
0, \; \text{ иначе}
\end{cases}
\] 
Функция распределения абсолютно непрерывной случайной величины $X$ имеет вид
\[
F(x)=\begin{cases}
a, x<0,\\
b x^2+c, x \in [0,2],\\
d, x > 2.\\
\end{cases}
\]
Выражение $a+b+c+d$ равно
 


\end{block} 
\begin{enumerate} 
\item[] \hyperlink{20-No}{\beamergotobutton{} $2$
}
\item[] \hyperlink{20-No}{\beamergotobutton{} $1/4$}
\item[] \hyperlink{20-Yes}{\beamergotobutton{} $5/4$}
\item[] \hyperlink{20-No}{\beamergotobutton{} $1/2$}
\item[] \hyperlink{20-No}{\beamergotobutton{} $1$}
\end{enumerate} 

\end{frame} 


 \begin{frame} \label{21-Yes} 
\begin{block}{21} 

Совместная функция плотности пары $X$ и $Y$ имеет вид
\[
f(x,y)=\begin{cases}
(x+y)/3, \; \text{ если } x\in[0;1], y\in [0;2] \\
0, \; \text{ иначе}
\end{cases}
\]

\vspace{0.5cm} 
 
Если функция $h(x,y)=c\cdot x\cdot f(x,y)$ также является совместной функцией плотности, то константа $c$ равна
 


 \end{block} 
\begin{enumerate} 
\item[] \hyperlink{21-No}{\beamergotobutton{} $9$}
\item[] \hyperlink{21-No}{\beamergotobutton{} $5$}
\item[] \hyperlink{21-No}{\beamergotobutton{} $5/9$}
\item[] \hyperlink{21-Yes}{\beamergotobutton{} $9/5$}
\item[] \hyperlink{21-No}{\beamergotobutton{} $1$}
\end{enumerate} 

 \textbf{Да!} 
 \hyperlink{22}{\beamerbutton{Следующий вопрос}}\end{frame} 


 \begin{frame} \label{22-Yes} 
\begin{block}{22} 

Совместная функция плотности пары $X$ и $Y$ имеет вид
\[
f(x,y)=\begin{cases}
(x+y)/3, \; \text{ если } x\in[0;1], y\in [0;2] \\
0, \; \text{ иначе}
\end{cases}
\]

\vspace{0.5cm} 
 
 Вероятность $\P(X<0.5, Y<1)$ равна
 


 \end{block} 
\begin{enumerate} 
\item[] \hyperlink{22-Yes}{\beamergotobutton{} $1/8$}
\item[] \hyperlink{22-No}{\beamergotobutton{} $5/8$}
\item[] \hyperlink{22-No}{\beamergotobutton{} $3/5$}
\item[] \hyperlink{22-No}{\beamergotobutton{} $5/6$}
\item[] \hyperlink{22-No}{\beamergotobutton{} $3/8$
}
\end{enumerate} 

 \textbf{Да!} 
 \hyperlink{23}{\beamerbutton{Следующий вопрос}}\end{frame} 


 \begin{frame} \label{23-Yes} 
\begin{block}{23  \textbf{Да!} 
		\hyperlink{24}{\beamerbutton{Следующий вопрос}}} 

Совместная функция плотности пары $X$ и $Y$ имеет вид
\[
f(x,y)=\begin{cases}
(x+y)/3, \; \text{ если } x\in[0;1], y\in [0;2] \\
0, \; \text{ иначе}
\end{cases}
\]
Условная функция плотности  $f_{X|Y=1}(x)$ равна
 


 \end{block} 
\begin{enumerate} 
\item[] \hyperlink{23-No}{\beamergotobutton{} $f_{X|Y=1}(x)=\begin{cases} (x+2)/2\, \text{ если } x\in [0;1] \\ 0, \text{ иначе }    \end{cases}$}
\item[] \hyperlink{23-No}{\beamergotobutton{} $f_{X|Y=1}(x)=\begin{cases} (2x+1)/2\, \text{ если } x\in [0;1] \\ 0, \text{ иначе }    \end{cases}$}
\item[] \hyperlink{23-Yes}{\beamergotobutton{} $f_{X|Y=1}(x)=\begin{cases} (2x+2)/3\, \text{ если } x\in [0;1] \\ 0, \text{ иначе }    \end{cases}$}
\item[] \hyperlink{23-No}{\beamergotobutton{} $f_{X|Y=1}(x)=\begin{cases} (x+4)/2\, \text{ если } x\in [0;1] \\ 0, \text{ иначе }    \end{cases}$}
\end{enumerate} 

\end{frame} 

 \begin{frame} \label{24-Yes} 
\begin{block}{24} 

Известно, что $\E(X)=-1$, $\Var(X)=1$, $\E(Y)=-4$, $\Var(Y)=4$, $\Corr(X,Y)=-0.5$

\vspace{0.5cm} 
 
Математическое ожидание $\E(Y)$ равно
 


 \end{block} 
\begin{enumerate} 
\item[] \hyperlink{24-No}{\beamergotobutton{} $13/7$
}
\item[] \hyperlink{24-Yes}{\beamergotobutton{} $11/9$}
\item[] \hyperlink{24-No}{\beamergotobutton{} $2/3$}
\item[] \hyperlink{24-No}{\beamergotobutton{} $4/3$}
\item[] \hyperlink{24-No}{\beamergotobutton{} $6/5$}
\end{enumerate} 

 \textbf{Да!} 
 \hyperlink{25}{\beamerbutton{Следующий вопрос}}\end{frame} 


 \begin{frame} \label{25-Yes} 
\begin{block}{25} 

Известно, что $\E(X)=-1$, $\Var(X)=1$, $\E(Y)=-4$, $\Var(Y)=4$, $\Corr(X,Y)=-0.5$

\vspace{0.5cm} 
 
Ковариация $\Cov(2X+Y,X-3Y)$ равна
 


 \end{block} 
\begin{enumerate} 
\item[] \hyperlink{25-No}{\beamergotobutton{} $-1$
}
\item[] \hyperlink{25-Yes}{\beamergotobutton{} $-5$}
\item[] \hyperlink{25-No}{\beamergotobutton{} $0$}
\item[] \hyperlink{25-No}{\beamergotobutton{} $5$}
\item[] \hyperlink{25-No}{\beamergotobutton{} $1$}
\end{enumerate} 

 \textbf{Да!} 
 \hyperlink{26}{\beamerbutton{Следующий вопрос}}\end{frame} 


 \begin{frame} \label{26-Yes} 
\begin{block}{26} 

Корреляция $\Corr((1-X)/2,(Y+5)/2)$ равна
 


 \end{block} 
\begin{enumerate} 
\item[] \hyperlink{26-Yes}{\beamergotobutton{} $0.5$}
\item[] \hyperlink{26-No}{\beamergotobutton{} $1$
}
\item[] \hyperlink{26-No}{\beamergotobutton{} $-0.5$}
\item[] \hyperlink{26-No}{\beamergotobutton{} $-1/8$}
\item[] \hyperlink{26-No}{\beamergotobutton{} $1/8$}
\end{enumerate} 

 \textbf{Да!} 
 \hyperlink{27}{\beamerbutton{Следующий вопрос}}\end{frame} 


 \begin{frame} \label{27-Yes} 
\begin{block}{27} 
	
У неотрицательной случайной величины $X$ известны $\E(X)=1$, $\Var(X)=4$. Вероятность $\P(X^2 \geq 25)$ обязательно попадает в интервал
 
 \end{block} 
\begin{enumerate} 
\item[] \hyperlink{27-No}{\beamergotobutton{} $[0;4/25]$}
\item[] \hyperlink{27-No}{\beamergotobutton{} $[0;4/625]$}
\item[] \hyperlink{27-No}{\beamergotobutton{} $[1/25;1]$}
\item[] \hyperlink{27-No}{\beamergotobutton{} $[0;1/25]$}
\item[] \hyperlink{27-Yes}{\beamergotobutton{} $[0;1/5]$}
\end{enumerate} 

 \textbf{Да!} 
 \hyperlink{28}{\beamerbutton{Следующий вопрос}}\end{frame} 


 \begin{frame} \label{28-Yes} 
\begin{block}{28} 

Если $\E(X)=0$, $\Var(X)=1$, то наиболее узкий интервал, в который гарантированно попадает вероятность $\P(|X| \geq 4)$, равен


 \end{block} 
\begin{enumerate} 
\item[] \hyperlink{28-No}{\beamergotobutton{} $[0.5; 1]$
}
\item[] \hyperlink{28-No}{\beamergotobutton{} $[0.0625; 1]$}
\item[] \hyperlink{28-No}{\beamergotobutton{} $[0.25; 1]$}
\item[] \hyperlink{28-No}{\beamergotobutton{} $[0; 0.25]$}
\item[] \hyperlink{28-Yes}{\beamergotobutton{} $[0; 0.0625]$ }
\end{enumerate} 

 \textbf{Да!} 
 \hyperlink{29}{\beamerbutton{Следующий вопрос}}\end{frame} 


 \begin{frame} \label{29-Yes} 
\begin{block}{29} 


Дана последовательность независимых случайных величин, имеющих равномерное на $(-1,1)$ распределение.  \textbf{НЕВЕРНЫМ} является утверждение
 


 \end{block} 
\begin{enumerate} 
\item[] \hyperlink{29-No}{\beamergotobutton{} 	$\sqrt3n\bar X$ сходится по распределению к стандартной нормальной величине}
\item[] \hyperlink{29-No}{\beamergotobutton{} Вероятность	$\P(\bar X>0)$ стремится к 0.5}
\item[] \hyperlink{29-Yes}{\beamergotobutton{}   $\bar X$ сходится по распределению к равномерной на (-1,1) величине }
\item[] \hyperlink{29-No}{\beamergotobutton{} $\bar X$ сходится по вероятности к нулю}
\item[] \hyperlink{29-No}{\beamergotobutton{} Вероятность	$\P(\bar X = 0)$ стремится к}
\end{enumerate} 

 \textbf{Да!} 
 \hyperlink{30}{\beamerbutton{Следующий вопрос}}\end{frame} 


 \begin{frame} \label{30-Yes} 
\begin{block}{30} 


Функция плотности случайной величины $X$ имеет вид
\[
f(x)=\frac{1}{\sqrt{8\pi}} e^{-(x-3)^2/8}
\]
 \textbf{НЕВЕРНЫМ} является утверждение
 


 \end{block} 
\begin{enumerate} 
\item[] \hyperlink{30-No}{\beamergotobutton{} $\P(X=0)=0$}
\item[] \hyperlink{30-No}{\beamergotobutton{} $\P(X>3)=0.5$}
\item[] \hyperlink{30-No}{\beamergotobutton{} $\P(X<0)>0$}
\item[] \hyperlink{30-Yes}{\beamergotobutton{} $\Var(X)=8$ }
\item[] \hyperlink{30-No}{\beamergotobutton{} $\E(X)=3$}
\item[] \hyperlink{30-No}{\beamergotobutton{} $\max f(x) = \frac{1}{2\sqrt{2\pi}}$}
\end{enumerate} 

 \textbf{Да!} 
 \hyperlink{31}{\beamerbutton{Следующий вопрос}}\end{frame} 


 \begin{frame} \label{31-Yes} 
\begin{block}{31} 

Величины $X_1$, $X_2$, \ldots независимы и одинаково распределены с $\E(X_i)=\mu$, $\Var(X_i)=\sigma^2$. К стандартному нормальному распределению  сходится последовательность случайных величин
 


 \end{block} 
\begin{enumerate} 
\item[] \hyperlink{31-No}{\beamergotobutton{} $(\bar X - \mu) /(\sqrt{n}\sigma)$}
\item[] \hyperlink{31-No}{\beamergotobutton{} $(\bar X - n\mu) /(\sqrt{n}\sigma)$}
\item[] \hyperlink{31-No}{\beamergotobutton{} $(\bar X - \mu) /\sigma$}
\item[] \hyperlink{31-Yes}{\beamergotobutton{} $\sqrt{n}(\bar X - \mu) /\sigma$ }
\item[] \hyperlink{31-No}{\beamergotobutton{} $\bar X$}
\end{enumerate} 

 \textbf{Да!} 
 \hyperlink{32}{\beamerbutton{Следующий вопрос}}\end{frame} 


 \begin{frame} \label{32-Yes} 
\begin{block}{32} 

Пусть $X_1$, \ldots, $X_n$ — выборка объема $n$ из равномерного на $[a, b]$ распределения. Оценка $X_1+X_2$ параметра $c=a+b$ является
 


 \end{block} 
\begin{enumerate} 
\item[] \hyperlink{32-No}{\beamergotobutton{} смещенной и несостоятельной}
\item[] \hyperlink{32-Yes}{\beamergotobutton{} несмещенной и несостоятельной}
\item[] \hyperlink{32-No}{\beamergotobutton{} смещенной и состоятельной}
\item[] \hyperlink{32-No}{\beamergotobutton{} асимптотически несмещенной и состоятельной}
\item[] \hyperlink{32-No}{\beamergotobutton{} несмещенной и состоятельной}
\end{enumerate} 

 \textbf{Да!} 
 \hyperlink{33}{\beamerbutton{Следующий вопрос}}\end{frame} 


 \begin{frame} \label{33-Yes} 
\begin{block}{33} 

Пусть $X_1$, \ldots, $X_n$ — выборка объема $n$ из некоторого распределения с конечным математическим ожиданием. Несмещенной и состоятельной оценкой математического ожидания является
 


 \end{block} 
\begin{enumerate} 
\item[] \hyperlink{33-Yes}{\beamergotobutton{} $\frac{X_1}{2n}+\frac{X_2+\ldots+X_{n-1}}{n-2}-\frac{X_n}{2 n}$}
\item[] \hyperlink{33-No}{\beamergotobutton{} $\frac{1}{3} X_1 + \frac{2}{3} X_2$}
\item[] \hyperlink{33-No}{\beamergotobutton{} $\frac{X_1}{2 n}+\frac{X_2+\ldots+X_{n-2}}{n-2}+\frac{X_n}{2 n}$}
\item[] \hyperlink{33-No}{\beamergotobutton{} $\frac{X_1}{2 n}+\frac{X_2+\ldots+X_{n-2}}{n-1}+\frac{X_n}{2 n}$}
\item[] \hyperlink{33-No}{\beamergotobutton{} $\frac{X_1+X_2}{2}$}
\end{enumerate} 

 \textbf{Да!} 
 \hyperlink{34}{\beamerbutton{Следующий вопрос}}\end{frame} 


 \begin{frame} \label{34-Yes} 
\begin{block}{34} 

Пусть $X_1$,\ldots, $X_n$ — выборка объема $n$ из равномерного на $[0, \theta]$ распределения. Оценка параметра $\theta$ методом моментов по $k$-му моменту имеет вид:
 


 \end{block} 
\begin{enumerate} 
\item[] \hyperlink{34-No}{\beamergotobutton{} $\sqrt[k]k \overline{X^k}$}
\item[] \hyperlink{34-No}{\beamergotobutton{} $\sqrt[k](k+1) \overline{X^k}$}
\item[] \hyperlink{34-No}{\beamergotobutton{} $\sqrt[k]k \overline{X^k}$}
\item[] \hyperlink{34-Yes}{\beamergotobutton{} $\sqrt[k](k+1) \overline{X^k}$}
\item[] \hyperlink{34-No}{\beamergotobutton{} $\sqrt[k+1](k+1) \overline{X^k}$}
\end{enumerate} 

 \textbf{Да!} 
 \hyperlink{35}{\beamerbutton{Следующий вопрос}}\end{frame} 


 \begin{frame} \label{35-Yes} 
\begin{block}{35} 

Пусть $X_1$, \ldots, $X_n$ — выборка объема $n$ из равномерного на $[0, \theta]$ распределения. Состоятельной оценкой параметра $\theta$ является:
 


 \end{block} 
\begin{enumerate} 
\item[] \hyperlink{35-No}{\beamergotobutton{} $X_(n)$}
\item[] \hyperlink{35-No}{\beamergotobutton{} $X_(n-1)$}
\item[] \hyperlink{35-No}{\beamergotobutton{} $\frac{n}{n+1} X_{(n-1)}$}
\item[] \hyperlink{35-No}{\beamergotobutton{} $\frac{n^2}{n^2-n+3} X_{(n-3)}$}
\item[] \hyperlink{35-Yes}{\beamergotobutton{} все перечисленные случайные величины}
\end{enumerate} 

 \textbf{Да!} 
 \hyperlink{36}{\beamerbutton{Следующий вопрос}}\end{frame} 


 \begin{frame} \label{36-Yes} 
\begin{block}{36} 

Пусть $X_1$, \ldots, $X_{2 n}$ — выборка объема $2 n$ из некоторого распределения. Какая из нижеперечисленных оценок математического ожидания имеет наименьшую дисперсию?
 


 \end{block} 
\begin{enumerate} 
\item[] \hyperlink{36-No}{\beamergotobutton{} $X_1$}
\item[] \hyperlink{36-No}{\beamergotobutton{} $\frac{X_1+X_2}{2}$}
\item[] \hyperlink{36-No}{\beamergotobutton{} $\frac{1}{n} \sum_{i=1}^n X_i$}
\item[] \hyperlink{36-No}{\beamergotobutton{} $\frac{1}{n} \sum_{i=n+1}^{2 n} X_i$}
\item[] \hyperlink{36-Yes}{\beamergotobutton{} $\frac{1}{2 n} \sum_{i=1}^{2 n} X_i$}
\end{enumerate} 

 \textbf{Да!} 
 \hyperlink{37}{\beamerbutton{Следующий вопрос}}\end{frame} 


 \begin{frame} \label{37-Yes} 
\begin{block}{37} 

Пусть $X_1$, \ldots, $X_n$ — выборка объема $n$ из распределения Бернулли с параметром $p$. Статистика $X_2 X_{n-2}$ является
 


 \end{block} 
\begin{enumerate} 
\item[] \hyperlink{37-Yes}{\beamergotobutton{} несмещенной оценкой $p^2$}
\item[] \hyperlink{37-No}{\beamergotobutton{} состоятельной оценкой $p^2$}
\item[] \hyperlink{37-No}{\beamergotobutton{} эффективной оценкой $p^2$}
\item[] \hyperlink{37-No}{\beamergotobutton{} асимптотически нормальной оценкой $p^2$}
\item[] \hyperlink{37-No}{\beamergotobutton{} оценкой максимального правдоподобия}
\end{enumerate} 

 \textbf{Да!} 
 \hyperlink{38}{\beamerbutton{Следующий вопрос}}\end{frame} 


 \begin{frame} \label{38-Yes} 
\begin{block}{38} 

Пусть $X_1$, \ldots, $X_n$ — выборка объема $n$ из равномерного на $[a, b]$ распределения. Выберите наиболее точный ответ из предложенных. Оценка $\theta^*_n = X_{(n)}-X_{(1)}$ длины отрезка $[a,b]$ является
 


 \end{block} 
\begin{enumerate} 
\item[] \hyperlink{38-No}{\beamergotobutton{} состоятельной и асимптотически смещённой}
\item[] \hyperlink{38-No}{\beamergotobutton{} несостоятельной и асимптотически несмещенной}
\item[] \hyperlink{38-Yes}{\beamergotobutton{} состоятельной и асимптотически несмещенной}
\item[] \hyperlink{38-No}{\beamergotobutton{} нормально распределённой}
\item[] \hyperlink{38-No}{\beamergotobutton{} несмещенной}
\end{enumerate} 

 \textbf{Да!} 
 \hyperlink{39}{\beamerbutton{Следующий вопрос}}\end{frame} 


 \begin{frame} \label{39-Yes} 
\begin{block}{39} 

Мощностью теста называется
 


 \end{block} 
\begin{enumerate} 
\item[] \hyperlink{39-No}{\beamergotobutton{} Вероятность принять неверную гипотезу}
\item[] \hyperlink{39-No}{\beamergotobutton{} Единица минус  вероятность отвергнуть основную гипотезу, когда она верна}
\item[] \hyperlink{39-Yes}{\beamergotobutton{} Единица минус  вероятность отвергнуть альтернативную гипотезу, когда она верна}
\item[] \hyperlink{39-No}{\beamergotobutton{} Вероятность отвергнуть альтернативную гипотезу, когда она верна}
\item[] \hyperlink{39-No}{\beamergotobutton{} Вероятность отвергнуть основную гипотезу, когда она верна}
\end{enumerate} 

 \textbf{Да!} 
 \hyperlink{40}{\beamerbutton{Следующий вопрос}}\end{frame} 


 \begin{frame} \label{40-Yes} 
\begin{block}{40} 

Если P-значение (P-value) больше уровня значимости  $\alpha$, то гипотеза  $H_0: \; \sigma=1$


 \end{block} 
\begin{enumerate} 
\item[] \hyperlink{40-No}{\beamergotobutton{} Отвергается, только если  $H_a: \; \sigma<1$}
\item[] \hyperlink{40-No}{\beamergotobutton{} Отвергается}
\item[] \hyperlink{40-No}{\beamergotobutton{} Отвергается, только если  $H_a: \; \sigma>1$}
\item[] \hyperlink{40-Yes}{\beamergotobutton{} Не отвергается}
\item[] \hyperlink{40-No}{\beamergotobutton{} Отвергается, только если  $H_a: \; \sigma\neq 1$}
\end{enumerate} 

 \textbf{Да!} 
 \hyperlink{41}{\beamerbutton{Следующий вопрос}}\end{frame} 


 \begin{frame} \label{41-Yes} 
\begin{block}{41} 

Имеется случайная выборка размера $n$ из нормального распределения. При проверке гипотезы о равенстве математического ожидания заданному значению при известной дисперсии используется статистика, имеющая распределение
 


 \end{block} 
\begin{enumerate} 
\item[] \hyperlink{41-No}{\beamergotobutton{}  $t_n-1$}
\item[] \hyperlink{41-No}{\beamergotobutton{} $\chi^2_n$}
\item[] \hyperlink{41-Yes}{\beamergotobutton{} $N(0,1)$}
\item[] \hyperlink{41-No}{\beamergotobutton{} $\chi^2_n-1$}
\item[] \hyperlink{41-No}{\beamergotobutton{} $t_n$}
\end{enumerate} 

 \textbf{Да!} 
 \hyperlink{42}{\beamerbutton{Следующий вопрос}}\end{frame} 


 \begin{frame} \label{42-Yes} 
\begin{block}{42} 

Имеется случайная выборка размера $n$ из нормального распределения. При проверке гипотезы о равенстве дисперсии заданному значению при неизвестном математическом ожидании используется статистика, имеющая распределение
 


 \end{block} 
\begin{enumerate} 
\item[] \hyperlink{42-Yes}{\beamergotobutton{} $\chi^2_n-1$}
\item[] \hyperlink{42-No}{\beamergotobutton{} $t_n$}
\item[] \hyperlink{42-No}{\beamergotobutton{}  $t_n-1$}
\item[] \hyperlink{42-No}{\beamergotobutton{} $N(0,1)$}
\item[] \hyperlink{42-No}{\beamergotobutton{} $\chi^2_n$}
\end{enumerate} 

 \textbf{Да!} 
 \hyperlink{43}{\beamerbutton{Следующий вопрос}}\end{frame} 


 \begin{frame} \label{43-Yes} 
\begin{block}{43} 

По случайной выборке из 100 наблюдений было оценено выборочное среднее $\bar{X}=20$  и несмещенная оценка дисперсии  $\hat{\sigma}^2=25$. В рамках проверки гипотезы $H_0: \; \mu=15$  против альтернативной гипотезы $H_a: \; \mu>15$  можно сделать следующее заключение
 


 \end{block} 
\begin{enumerate} 
\item[] \hyperlink{43-No}{\beamergotobutton{} Гипотеза $H_0$  не отвергается на любом разумном уровне значимости}
\item[] \hyperlink{43-No}{\beamergotobutton{} Гипотеза $H_0$  отвергается на уровне значимости 5\%, но не  на уровне значимости 1\%}
\item[] \hyperlink{43-No}{\beamergotobutton{} Гипотеза  $H_0$ отвергается на уровне значимости 10\%, но не на уровне значимости 5\%}
\item[] \hyperlink{43-Yes}{\beamergotobutton{} Гипотеза $H_0$  отвергается на любом разумном уровне значимости}
\item[] \hyperlink{43-No}{\beamergotobutton{} Гипотеза  $H_0$ отвергается на уровне значимости 20\%, но не  на уровне значимости 10\%}
\end{enumerate} 

 \textbf{Да!} 
 \hyperlink{44}{\beamerbutton{Следующий вопрос}}\end{frame} 


 \begin{frame} \label{44-Yes} 
\begin{block}{44} 

На основе случайной выборки, содержащей одно наблюдение  $X_1$, тестируется гипотеза $H_0: \; X_1 \sim U[0;1]$  против альтернативной гипотезы  $H_a: \; X_1 \sim U[0.5;1.5]$. Рассматривается критерий: если $X_1>0.8$, то гипотеза $H_0$  отвергается в пользу гипотезы  $H_a$. Вероятность ошибки 2-го рода для этого критерия равна:
 


 \end{block} 
\begin{enumerate} 
\item[] \hyperlink{44-Yes}{\beamergotobutton{} 0.3}
\item[] \hyperlink{44-No}{\beamergotobutton{} 0.1}
\item[] \hyperlink{44-No}{\beamergotobutton{} 0.2}
\item[] \hyperlink{44-No}{\beamergotobutton{} 0.5}
\item[] \hyperlink{44-No}{\beamergotobutton{} 0.4}
\end{enumerate} 

 \textbf{Да!} 
 \hyperlink{45}{\beamerbutton{Следующий вопрос}}\end{frame} 


 \begin{frame} \label{45-Yes} 
\begin{block}{45} 

Пусть $X_1$, $X_2$, \ldots, $X_n$ — случайная выборка размера 36 из нормального распределения $N(\mu, 9)$. Для тестирования основной гипотезы  $H_0: \; \mu=0$  против альтернативной $H_a: \; \mu=-2$   вы используете критерий: если  $\bar{X}\geq -1$, то вы не отвергаете гипотезу $H_0$, в противном случае вы отвергаете гипотезу  $H_0$ в пользу гипотезы  $H_a$. Мощность критерия равна
 


 \end{block} 
\begin{enumerate} 
\item[] \hyperlink{45-No}{\beamergotobutton{} 0.87}
\item[] \hyperlink{45-Yes}{\beamergotobutton{} 0.98}
\item[] \hyperlink{45-No}{\beamergotobutton{} 0.78}
\item[] \hyperlink{45-No}{\beamergotobutton{} 0.58}
\item[] \hyperlink{45-No}{\beamergotobutton{} 0.85}
\end{enumerate} 

 \textbf{Да!} 
 \hyperlink{46}{\beamerbutton{Следующий вопрос}}\end{frame} 


 \begin{frame} \label{46-Yes} 
\begin{block}{46} 

Николай Коперник подбросил бутерброд 200 раз. Бутерброд упал маслом вниз 95 раз, а маслом вверх — 105 раз. Значение критерия $\chi^2$ Пирсона для проверки гипотезы о равной вероятности данных событий равно
 


 \end{block} 
\begin{enumerate} 
\item[] \hyperlink{46-No}{\beamergotobutton{} 0.75}
\item[] \hyperlink{46-No}{\beamergotobutton{} 7.5}
\item[] \hyperlink{46-No}{\beamergotobutton{} 0.5}
\item[] \hyperlink{46-No}{\beamergotobutton{} 0.25}
\item[] \hyperlink{46-Yes}{\beamergotobutton{} 0.5}
\end{enumerate} 

 \textbf{Да!} 
 \hyperlink{47}{\beamerbutton{Следующий вопрос}}\end{frame} 


 \begin{frame} \label{47-Yes} 
\begin{block}{47} 

Каждое утро в 8:00 Иван Андреевич Крылов, либо завтракает, либо уже позавтракал. В это же время кухарка либо заглядывает к Крылову, либо нет. По таблице сопряженности вычислите  статистику $\chi^2$ Пирсона для тестирования гипотезы о том, что визиты кухарки не зависят от того, позавтракал ли уже Крылов или нет.
\begin{tabular}{c|cc}
Время 8:00 & кухарка заходит & кухарка не заходит \\
\hline
Крылов завтракает & 200 & 40 \\
Крылов уже позавтракал & 25 & 100 \\
\end{tabular}
 


 \end{block} 
\begin{enumerate} 
\item[] \hyperlink{47-No}{\beamergotobutton{} 79}
\item[] \hyperlink{47-No}{\beamergotobutton{} 100}
\item[] \hyperlink{47-Yes}{\beamergotobutton{} 139}
\item[] \hyperlink{47-No}{\beamergotobutton{} 39}
\item[] \hyperlink{47-No}{\beamergotobutton{} 179}
\end{enumerate} 

 \textbf{Да!} 
 \hyperlink{48}{\beamerbutton{Следующий вопрос}}\end{frame} 


 \begin{frame} \label{48-Yes} 
\begin{block}{48} 

Ковариационная матрица вектора $X=(X_1,X_2)$ имеет вид
\[
\begin{pmatrix}
10 & 3 \\
3 & 8
\end{pmatrix}
\]
Дисперсия разности элементов вектора, $\Var(X_1-X_2)$, равняется
 


 \end{block} 
\begin{enumerate} 
\item[] \hyperlink{48-No}{\beamergotobutton{} 15}
\item[] \hyperlink{48-No}{\beamergotobutton{} 2}
\item[] \hyperlink{48-Yes}{\beamergotobutton{} 12}
\item[] \hyperlink{48-No}{\beamergotobutton{} 18}
\item[] \hyperlink{48-No}{\beamergotobutton{} 6}
\end{enumerate} 

 \textbf{Да!} 
 \hyperlink{49}{\beamerbutton{Следующий вопрос}}\end{frame} 


 \begin{frame} \label{49-Yes} 
\begin{block}{49} 

Все условия регулярности для применения метода максимального правдоподобия выполнены. Вторая производная лог-функции правдоподобия равна $\ell''(\hat{\theta})=-100$. Оценка стандартной ошибки для $\hat{\theta}$ равна
 


 \end{block} 
\begin{enumerate} 
\item[] \hyperlink{49-No}{\beamergotobutton{} 1}
\item[] \hyperlink{49-No}{\beamergotobutton{} 100}
\item[] \hyperlink{49-Yes}{\beamergotobutton{} 0.1}
\item[] \hyperlink{49-No}{\beamergotobutton{} 0.01}
\item[] \hyperlink{49-No}{\beamergotobutton{} 10}
\end{enumerate} 

 \textbf{Да!} 
 \hyperlink{50}{\beamerbutton{Следующий вопрос}}\end{frame} 


 \begin{frame} \label{50-Yes} 
\begin{block}{50} 

Геродот Геликарнасский проверяет гипотезу $H_0: \; \mu=0, \; \sigma^2=1$ с помощью $LR$ статистики теста отношения правдоподобия. При подстановке оценок метода максимального правдоподобия в лог-функцию правдоподобия он получил $\ell=-177$, а при подстановке $\mu=0$ и $\sigma=1$ оказалось, что $\ell=-211$. Найдите значение $LR$ статистики и укажите её закон распределения при верной $H_0$
 


 \end{block} 
\begin{enumerate} 
\item[] \hyperlink{50-No}{\beamergotobutton{} $LR=\ln 68$, $\chi^2_n-2$}
\item[] \hyperlink{50-No}{\beamergotobutton{} $LR=34$, $\chi^2_n-1$}
\item[] \hyperlink{50-No}{\beamergotobutton{} $LR=34$, $\chi^2_2$}
\item[] \hyperlink{50-No}{\beamergotobutton{} $LR=\ln 34$, $\chi^2_n-2$}
\item[] \hyperlink{50-Yes}{\beamergotobutton{} $LR=68$, $\chi^2_2$}
\end{enumerate} 

 \textbf{Да!} 
 \hyperlink{51}{\beamerbutton{Следующий вопрос}}\end{frame} 


 \begin{frame} \label{51-Yes} 
\begin{block}{51} 

Геродот Геликарнасский проверяет гипотезу $H_0: \; \mu=2$. Лог-функция правдоподобия имеет вид $\ell(\mu,\nu)=-\frac{n}{2}\ln (2\pi)-\frac{n}{2}\ln \nu -\frac{\sum_{i=1}^n(x_i-\mu)^2}{2\nu}$. Оценка максимального правдоподобия для $\nu$ при предположении, что $H_0$ верна, равна
 


 \end{block} 
\begin{enumerate} 
\item[] \hyperlink{51-Yes}{\beamergotobutton{}$\frac{\sum x_i^2 - 4\sum x_i}{n}+4$}
\item[] \hyperlink{51-No}{\beamergotobutton{} $\frac{\sum x_i^2 - 4\sum x_i}{n}+3$}
\item[] \hyperlink{51-No}{\beamergotobutton{} $\frac{\sum x_i^2 - 4\sum x_i}{n}+2$}
\item[] \hyperlink{51-No}{\beamergotobutton{} $\frac{\sum x_i^2 - 4\sum x_i}{n}+1$}
\item[] \hyperlink{51-No}{\beamergotobutton{} $\frac{\sum x_i^2 - 4\sum x_i}{n}$}
\end{enumerate} 

 \textbf{Да!} 
 \hyperlink{52}{\beamerbutton{Следующий вопрос}}\end{frame} 


 \begin{frame} \label{52-Yes} 
\begin{block}{52} 

Ацтек Монтесума Илуикамина хочет оценить параметр $a$ методом максимального правдоподобия по выборке из неотрицательного распределения с функцией плотности $f(x)=\frac{1}{2}a^3x^2e^{-ax}$ при $x\geq 0$. Для этой цели ему достаточно максимизировать функцию
 


 \end{block} 
\begin{enumerate} 
\item[] \hyperlink{52-Yes}{\beamergotobutton{} $3n \ln a - a \sum x_i$}
\item[] \hyperlink{52-No}{\beamergotobutton{} $3n\prod \ln a - a x^n$}
\item[] \hyperlink{52-No}{\beamergotobutton{} $3n\ln a - a \prod \ln x_i$}
\item[] \hyperlink{52-No}{\beamergotobutton{} $3n \ln a - an \ln x_i$}
\item[] \hyperlink{52-No}{\beamergotobutton{} $3n \sum \ln a_i - a \sum \ln x_i$}
\end{enumerate} 

 \textbf{Да!} 
 \hyperlink{53}{\beamerbutton{Следующий вопрос}}\end{frame} 


 \begin{frame} \label{53-Yes} 
\begin{block}{53} 

Бессмертный гений поэзии Ли Бо оценивает математическое ожидание  по выборка размера $n$ из нормального распределения. Он построил оценку метода моментов, $\hat{\mu}_{MM}$, и оценку максимального правдоподобия, $\hat{\mu}_{ML}$. Про эти оценки можно утверждать, что
 


 \end{block} 
\begin{enumerate} 
\item[] \hyperlink{53-No}{\beamergotobutton{}  $\hat\mu_MM>\hat\mu_ML$}
\item[] \hyperlink{53-Yes}{\beamergotobutton{} они равны}
\item[] \hyperlink{53-No}{\beamergotobutton{} $\hat\mu_MM<\hat\mu_ML$ }
\item[] \hyperlink{53-No}{\beamergotobutton{} они не равны, но сближаются при $n\to \infty$}
\item[] \hyperlink{53-No}{\beamergotobutton{} они не равны, и не сближаются при $n\to \infty$}
\end{enumerate} 

 \textbf{Да!} 
 \hyperlink{54}{\beamerbutton{Следующий вопрос}}\end{frame} 


 \begin{frame} \label{54-Yes} 
\begin{block}{54} 

Проверяя гипотезу о равенстве дисперсий в двух выборках (размером в 3 и 5 наблюдений), Анаксимандр Милетский получил значение тестовой статистики 10. Если оценка дисперсии по первой выборке равна 8, то вторая оценка дисперсии может быть равна
 


 \end{block} 
\begin{enumerate} 
\item[] \hyperlink{54-No}{\beamergotobutton{} $25$}
\item[] \hyperlink{54-No}{\beamergotobutton{} $4/3$}
\item[] \hyperlink{54-Yes}{\beamergotobutton{} $80$}
\item[] \hyperlink{54-No}{\beamergotobutton{} $3/4$}
\item[] \hyperlink{54-No}{\beamergotobutton{} $4$}
\end{enumerate} 

 \textbf{Да!} 
 \hyperlink{55}{\beamerbutton{Следующий вопрос}}\end{frame} 


 \begin{frame} \label{55-Yes} 
\begin{block}{55} 

Пусть  $\hat{\sigma}^2_1$ — несмещенная оценка дисперсии, полученная по первой выборке размером $n_1$,   $\hat{\sigma}^2_2$ — несмещенная оценка дисперсии, полученная по второй выборке, с меньшим размером  $n_2$. Тогда статистика $\frac{\hat{\sigma}^2_1/n_1}{\hat{\sigma}^2_2/n_2}$  имеет распределение
 


 \end{block} 
\begin{enumerate} 
\item[] \hyperlink{55-No}{\beamergotobutton{} $\chi^2_{n_1+n_2}$}
\item[] \hyperlink{55-No}{\beamergotobutton{} $F_{n_1,n_2}$}
\item[] \hyperlink{55-No}{\beamergotobutton{} $F_{n_1-1,n_2-1}$}
\item[] \hyperlink{55-No}{\beamergotobutton{} $t_{n_1+n_2-1}$}
\item[] \hyperlink{55-No}{\beamergotobutton{} $N(0;1)$}
\end{enumerate} 

 \textbf{Да!} 
 \hyperlink{56}{\beamerbutton{Следующий вопрос}}\end{frame} 


 \begin{frame} \label{56-Yes} 
\begin{block}{56} 

Зулус Чака каСензангакона проверяет гипотезу  о равенстве математических ожиданий в двух нормальных выборках небольших размеров $n_1$   и  $n_2$. Если дисперсии неизвестны, но равны, то тестовая статистика имеет распределение
 


 \end{block} 
\begin{enumerate} 
\item[] \hyperlink{56-No}{\beamergotobutton{} $F_{n_1,n_2}$}
\item[] \hyperlink{56-Yes}{\beamergotobutton{} $t_{n_1+n_2-1}$}
\item[] \hyperlink{56-No}{\beamergotobutton{} $t_{n_1+n_2}$}
\item[] \hyperlink{56-No}{\beamergotobutton{} $t_{n_1+n_2-2}$}
\item[] \hyperlink{56-No}{\beamergotobutton{} $\chi^2_{n_1+n_2-1}$}
\end{enumerate} 

 \textbf{Да!} 
 \hyperlink{57}{\beamerbutton{Следующий вопрос}}\end{frame} 


 \begin{frame} \label{57-Yes} 
\begin{block}{57} 

Критерий знаков проверяет нулевую гипотезу
 


 \end{block} 
\begin{enumerate} 
\item[] \hyperlink{57-Yes}{\beamergotobutton{} о равенстве нулю вероятности того, что случайная величина $X$ окажется больше случайной величины $Y$, если альтернативная гипотеза записана как $\mu_X>\mu_Y$}
\item[] \hyperlink{57-No}{\beamergotobutton{} о равенстве нулю вероятности того, что случайная величина $X$ окажется больше случайной величины $Y$, если альтернативная гипотеза записана как $\mu_X>\mu_Y$ }
\item[] \hyperlink{57-No}{\beamergotobutton{} о равенстве математических ожиданий двух нормально распределенных случайных величин}
\item[] \hyperlink{57-No}{\beamergotobutton{} о совпадении функции распределения случайной величины с заданной теоретической функцией распределения}
\item[] \hyperlink{57-No}{\beamergotobutton{} о равенстве $1/2$ вероятности того, что случайная величина $X$ окажется больше случайной величины $Y$, если альтернативная гипотеза записана как $\mu_X>\mu_Y$}
\end{enumerate} 

 \textbf{Да!} 
 \hyperlink{58}{\beamerbutton{Следующий вопрос}}\end{frame} 


 \begin{frame} \label{58-Yes} 
\begin{block}{58} 

Вероятность ошибки первого рода, $\alpha$, и вероятность ошибки второго рода, $\beta$, всегда связаны соотношением


 \end{block} 
\begin{enumerate} 
\item[] \hyperlink{58-No}{\beamergotobutton{} $\alpha+\beta \leq 1$}
\item[] \hyperlink{58-No}{\beamergotobutton{} $\alpha+\beta \geq 1$}
\item[] \hyperlink{58-No}{\beamergotobutton{} $\alpha\geq \beta $}
\item[] \hyperlink{58-No}{\beamergotobutton{} $\alpha+\beta=1$}
\item[] \hyperlink{58-No}{\beamergotobutton{} $\alpha\leq \beta $}
\end{enumerate} 

 \textbf{Да!} 
 \hyperlink{59}{\beamerbutton{Следующий вопрос}}\end{frame} 


 \begin{frame} \label{59-Yes} 
\begin{block}{59} 

Среди 100 случайно выбранных ацтеков 20 платят дань Кулуакану, а 80 — Аскапоцалько. Соответственно, оценка доли ацтеков, платящих дань Кулуакану, равна $\hat{p}=0.2$. Разумная оценка стандартного отклонения случайной величины $\hat{p}$ равна
 


 \end{block} 
\begin{enumerate} 
\item[] \hyperlink{59-No}{\beamergotobutton{} $0.4$}
\item[] \hyperlink{59-No}{\beamergotobutton{} $1.6$}
\item[] \hyperlink{59-Yes}{\beamergotobutton{} $0.04$}
\item[] \hyperlink{59-No}{\beamergotobutton{} $0.16$}
\item[] \hyperlink{59-No}{\beamergotobutton{} $0.016$}
\end{enumerate} 

 \textbf{Да!} 
 \hyperlink{60}{\beamerbutton{Следующий вопрос}}\end{frame} 


 \begin{frame} \label{60-Yes} 
\begin{block}{60} 

Датчик случайных чисел выдал следующие значения псевдо случайной величины: $0.78$, $0.48$. Вычислите значение критерия Колмогорова и проверьте гипотезу $H_0$ о соответствии распределения равномерному на $[0;1]$. Критическое значение статистики Колмогорова для уровня значимости 0.1 и двух наблюдений равно $0.776$.
 


 \end{block} 
\begin{enumerate} 
\item[] \hyperlink{60-No}{\beamergotobutton{} 1.26, $H_0$ отвергается}
\item[] \hyperlink{60-No}{\beamergotobutton{} 0.3, $H_0$ не отвергается}
\item[] \hyperlink{60-Yes}{\beamergotobutton{} 0.78, $H_0$ отвергается}
\item[] \hyperlink{60-No}{\beamergotobutton{} 0.48, $H_0$ не отвергается}
\item[] \hyperlink{60-No}{\beamergotobutton{} 0.37, $H_0$ не отвергается}
\end{enumerate} 

 \textbf{Да!} 
 \hyperlink{61}{\beamerbutton{Следующий вопрос}}\end{frame} 


 \begin{frame} \label{61-Yes} 
\begin{block}{61} 

У пяти случайно выбранных студентов первого потока результаты за контрольную по статистике оказались равны  82, 47, 20, 43 и 73. У четырёх случайно выбранных студентов второго потока — 68, 83, 60 и 52. Вычислите статистику Вилкоксона и проверьте гипотезу $H_0$ об однородности результатов студентов двух потоков. Критические значения статистики Вилкоксона равны $T_L=12$ и $T_R=28$.
 


 \end{block} 
\begin{enumerate} 
\item[] \hyperlink{61-No}{\beamergotobutton{} 53, $H_0$ отвергается}
\item[] \hyperlink{61-No}{\beamergotobutton{} 20, $H_0$ не отвергается}
\item[] \hyperlink{61-No}{\beamergotobutton{} 65.75, $H_0$ отвергается}
\item[] \hyperlink{61-No}{\beamergotobutton{} 12.75, $H_0$ не отвергается}
\item[] \hyperlink{61-Yes}{\beamergotobutton{} 24, $H_0$ не отвергается}
\end{enumerate} 

 \textbf{Да!} 
 \hyperlink{62}{\beamerbutton{Следующий вопрос}}\end{frame} 


 \begin{frame} \label{62-Yes} 
\begin{block}{62} 

 Производитель мороженного попросил оценить по 10-бальной шкале два вида мороженного: с кусочками шоколада и с орешками. Было опрошено 5 человек.


 \begin{tabular}{c|ccccc}
  & Евлампий & Аристарх & Капитолина & Аграфена & Эвридика \\
 \hline
С крошкой & 10 & 6 & 7 & 5 & 4 \\
С орехами & 9 & 8 & 8 & 7 & 6 \\
 \end{tabular}


Вычислите модуль значения статистики теста знаков. Используя нормальную аппроксимацию, проверьте на уровне значимости $0.05$ гипотезу об отсутствии предпочтения мороженного с орешками против альтернативы, что мороженное с орешками вкуснее.
 


 \end{block} 
\begin{enumerate} 
\item[] \hyperlink{62-No}{\beamergotobutton{} 1.29, $H_0$ не отвергается}
\item[] \hyperlink{62-No}{\beamergotobutton{} 1.34, $H_0$ не отвергается}
\item[] \hyperlink{62-No}{\beamergotobutton{} 1.65, $H_0$ отвергается}
\item[] \hyperlink{62-Yes}{\beamergotobutton{} 1.96, $H_0$ отвергается}
\item[] \hyperlink{62-No}{\beamergotobutton{} 1.29, $H_0$ отвергается}
\end{enumerate} 

 \textbf{Да!} 
 \hyperlink{63}{\beamerbutton{Следующий вопрос}}\end{frame} 


 \begin{frame} \label{63-Yes} 
\begin{block}{63} 

По 10 наблюдениям проверяется гипотеза $H_0: \; \mu=10$ против $H_a: \; \mu \neq 10$ на выборке из нормального распределения с неизвестной дисперсией. Величина $\sqrt{n}\cdot (\bar{X}-\mu)/\hat{\sigma}$ оказалась равной $1$. P-значение примерно равно
 


 \end{block} 
\begin{enumerate} 
\item[] \hyperlink{63-No}{\beamergotobutton{} $0.32$}
\item[] \hyperlink{63-No}{\beamergotobutton{} $0.17$}
\item[] \hyperlink{63-Yes}{\beamergotobutton{} $0.16$}
\item[] \hyperlink{63-No}{\beamergotobutton{} $0.34$}
\item[] \hyperlink{63-No}{\beamergotobutton{} $0.83$}
\end{enumerate} 

 \textbf{Да!} 
 \hyperlink{64}{\beamerbutton{Следующий вопрос}}\end{frame} 


 \begin{frame} \label{1-No} 
\begin{block}{1} 

Крошка Джон  попадает в яблочко с вероятностью $0.8$. Его выстрелы независимы. Вероятность, что он попадёт хотя бы один раз из двух равна
 


 \end{block} 
\begin{enumerate} 
\item[] \hyperlink{1-No}{\beamergotobutton{} $0.64$}
\item[] \hyperlink{1-No}{\beamergotobutton{} $0.36$}
\item[] \hyperlink{1-Yes}{\beamergotobutton{} $0.96$}
\item[] \hyperlink{1-No}{\beamergotobutton{} $0.9$
}
\item[] \hyperlink{1-No}{\beamergotobutton{} $0.8$}
\end{enumerate} 

 \alert{Нет!} 
\end{frame} 


 \begin{frame} \label{2-No} 
\begin{block}{2} 

Крошка Джон попадает в яблочко с вероятностью $0.8$. Его выстрелы независимы. Вероятность, что он попал оба раза, если известно, что он попал хотя бы один раз из двух, равна
    


 \end{block} 
\begin{enumerate} 
\item[] \hyperlink{2-No}{\beamergotobutton{} $1/3$}
\item[] \hyperlink{2-No}{\beamergotobutton{} $1/4$
}
\item[] \hyperlink{2-No}{\beamergotobutton{} $1/2$}
\item[] \hyperlink{2-Yes}{\beamergotobutton{} $2/3$}
\item[] \hyperlink{2-No}{\beamergotobutton{} $3/4$}
\end{enumerate} 

 \alert{Нет!} 
\end{frame} 


 \begin{frame} \label{3-No} 
\begin{block}{3} 

Имеется три монетки. Две «правильных» и одна — с «орлами» по обеим сторонам. Вася выбирает одну монетку наугад и подкидывает её два раза. Вероятность того, что оба раза выпадет орел равна
     


 \end{block} 
\begin{enumerate} 
\item[] \hyperlink{3-Yes}{\beamergotobutton{} $1/2$}
\item[] \hyperlink{3-No}{\beamergotobutton{} $3/4$
}
\item[] \hyperlink{3-No}{\beamergotobutton{} $1/3$}
\item[] \hyperlink{3-No}{\beamergotobutton{} $1/4$}
\item[] \hyperlink{3-No}{\beamergotobutton{} $2/3$}
\end{enumerate} 

 \alert{Нет!} 
\end{frame} 


 \begin{frame} \label{4-No} 
\begin{block}{4} 

Крошка Джон попадает в яблочко с вероятностью $0.8$. Его выстрелы независимы. Вероятность, что он попал во второй раз, если известно, что он попал хотя бы один раз из двух, равна


 \end{block} 
\begin{enumerate} 
\item[] \hyperlink{4-No}{\beamergotobutton{} $4/5$}
\item[] \hyperlink{4-No}{\beamergotobutton{} $3/4$}
\item[] \hyperlink{4-Yes}{\beamergotobutton{} $5/6$}
\item[] \hyperlink{4-No}{\beamergotobutton{} $1/2$}
\item[] \hyperlink{4-No}{\beamergotobutton{} $2/3$}
\end{enumerate} 

 \alert{Нет!} 
\end{frame} 


 \begin{frame} \label{5-No} 
\begin{block}{5} 

Если события $A$, $B$, $C$ попарно независимы, то


 \end{block} 
\begin{enumerate} 
\item[] \hyperlink{5-No}{\beamergotobutton{} События $A$, $B$, $C$ несовместны}
\item[] \hyperlink{5-No}{\beamergotobutton{} События $A$, $B$, $C$ зависимы в совокупности}
\item[] \hyperlink{5-No}{\beamergotobutton{} $\P(A\cap B\cap C)=\P(A)\P(B)\P(C)$
}
\item[] \hyperlink{5-No}{\beamergotobutton{} Событие $A\cup B\cup C$ обязательно произойдёт}
\item[] \hyperlink{5-No}{\beamergotobutton{} События $A$, $B$, $C$ независимы в совокупности}
\end{enumerate} 

 \alert{Нет!} 
\end{frame} 


 \begin{frame} \label{6-No} 
\begin{block}{6} 

Случайная величина $X$ равномерна на отрезке $[0;10]$. Вероятность $\P(X>3|X<7)$ равна
     


 \end{block} 
\begin{enumerate} 
\item[] \hyperlink{6-Yes}{\beamergotobutton{} $4/7$}
\item[] \hyperlink{6-No}{\beamergotobutton{} $7/10$}
\item[] \hyperlink{6-No}{\beamergotobutton{} $3/7$}
\item[] \hyperlink{6-No}{\beamergotobutton{} $3/10$}
\item[] \hyperlink{6-No}{\beamergotobutton{} $0.21$
}
\end{enumerate} 

 \alert{Нет!} 
\end{frame} 


 \begin{frame} \label{7-No} 
\begin{block}{7} 

Имеется три монетки. Две «правильных» и одна — с «орлами» по обеим сторонам. Вася выбирает одну монетку наугад и подкидывает её два раза. События $A = \{ \text{Орёл выпал при первом подбрасывании} \}$ и $B =\{\text{Орёл выпал при втором подбрасывании}\}$
   


 \end{block} 
\begin{enumerate} 
\item[] \hyperlink{7-No}{\beamergotobutton{} образуют полную группу событий}
\item[] \hyperlink{7-No}{\beamergotobutton{} удовлетворяют соотношению $\P(A\cap B)=\P(A)+\P(B) + \P(A\cup B)$
}
\item[] \hyperlink{7-No}{\beamergotobutton{} независимы}
\item[] \hyperlink{7-Yes}{\beamergotobutton{} удовлетворяют соотношению $\P(A|B)\geq \P(A)$}
\item[] \hyperlink{7-No}{\beamergotobutton{} несовместны}
\end{enumerate} 

 \alert{Нет!} 
\end{frame} 


 \begin{frame} \label{8-No} 
\begin{block}{8} 

В квадрат вписан круг. Наудачу в квадрат бросают восемь точек. Пусть $X$ — число точек, попавших в круг. Математическое ожидание величины $X$ равно
     


 \end{block} 
\begin{enumerate} 
\item[] \hyperlink{8-No}{\beamergotobutton{} $4 \pi$
}
\item[] \hyperlink{8-No}{\beamergotobutton{} $\pi / 2$}
\item[] \hyperlink{8-No}{\beamergotobutton{} $\pi / 4$}
\item[] \hyperlink{8-No}{\beamergotobutton{} $\pi$}
\item[] \hyperlink{8-Yes}{\beamergotobutton{} $2\pi$}
\end{enumerate} 

 \alert{Нет!} 
\end{frame} 


 \begin{frame} \label{9-No} 
\begin{block}{9} 

В квадрат вписан круг. Наудачу в квадрат бросают восемь точек. Пусть $X$ — число точек, попавших в круг. Дисперсия величины $X$ равна     


 \end{block} 
\begin{enumerate} 
\item[] \hyperlink{9-No}{\beamergotobutton{} $\pi^2 - 2 \pi$}
\item[] \hyperlink{9-No}{\beamergotobutton{} $3\pi^2 - 4$}
\item[] \hyperlink{9-No}{\beamergotobutton{} $\pi^2$}
\item[] \hyperlink{9-No}{\beamergotobutton{} $3\pi^2 - 2$
}
\item[] \hyperlink{9-Yes}{\beamergotobutton{} $2\pi - \pi^2 / 2$}
\end{enumerate} 

 \alert{Нет!} 
\end{frame} 


 \begin{frame} \label{10-No} 
\begin{block}{10} 

В квадрат вписан круг. Последовательно в квадрат наудачу бросают восемь точек. Пусть $Y$ — число точек, попавших в круг, при первых четырех бросаниях, а $Z$ — число точек, попавших в круг, при оставшихся четырех бросаниях. Ковариация $\Cov(Y,Z)$ равна
     


 \end{block} 
\begin{enumerate} 
\item[] \hyperlink{10-No}{\beamergotobutton{} $-\pi^2$}
\item[] \hyperlink{10-No}{\beamergotobutton{} $-2\pi$
}
\item[] \hyperlink{10-No}{\beamergotobutton{} $2\pi$}
\item[] \hyperlink{10-No}{\beamergotobutton{} $\pi^2$}
\item[] \hyperlink{10-Yes}{\beamergotobutton{} $0$}
\end{enumerate} 

 \alert{Нет!} 
\end{frame} 


 \begin{frame} \label{11-No} 
\begin{block}{11} 

В квадрат вписан круг. Последовательно в квадрат наудачу бросают восемь точек. Пусть $Y$ — число точек, попавших в круг, при первых четырех бросаниях, а $Z$ — число точек, попавших в круг, при оставшихся четырех бросаниях. Дисперсия $\Var(Y - Z)$ равна
 


 \end{block} 
\begin{enumerate} 
\item[] \hyperlink{11-No}{\beamergotobutton{}  $0$
}
\item[] \hyperlink{11-No}{\beamergotobutton{} $\pi^2$}
\item[] \hyperlink{11-No}{\beamergotobutton{} $3\pi^2 - 4$}
\item[] \hyperlink{11-Yes}{\beamergotobutton{} $2\pi - \pi^2 / 2$}
\item[] \hyperlink{11-No}{\beamergotobutton{} $\pi^2 - 2 \pi$}
\end{enumerate} 

 \alert{Нет!} 
\end{frame} 


 \begin{frame} \label{12-No} 
\begin{block}{12} 

В квадрат вписан круг. Наудачу в квадрат бросают восемь точек. Наиболее вероятное число точек, попавших в круг, равно
 


 \end{block} 
\begin{enumerate} 
\item[] \hyperlink{12-No}{\beamergotobutton{} $6$
}
\item[] \hyperlink{12-Yes}{\beamergotobutton{} $7$}
\item[] \hyperlink{12-No}{\beamergotobutton{} $4$}
\item[] \hyperlink{12-No}{\beamergotobutton{} $2\pi$}
\item[] \hyperlink{12-No}{\beamergotobutton{} $5$}
\end{enumerate} 

 \alert{Нет!} 
\end{frame} 


 \begin{frame} \label{13-No} 
\begin{block}{13} 

Совместное распределение пары величин $X$ и $Y$ задано таблицей:

\begin{center}
\begin{tabular}{c|cc}
 & $Y=-2$ & $Y=1$ \\
\hline
$X=-1$ & 0.1 & 0 \\
$X=0$ & 0.1 & 0.3 \\
$X=1$ & 0.2 & 0.3 \\
\end{tabular}
\end{center}
\vspace{0.2cm} 
 
 Всем известно, что Маша звонит Васе в среднем 10 раз в день. Число звонков, совершенных Машей, имеет распределение Пуассона. Вероятность того, что Маша ни разу не позвонит Васе в течение дня, равна
 


 \end{block} 
\begin{enumerate} 
\item[] \hyperlink{13-No}{\beamergotobutton{} $1 - e^{10}$}
\item[] \hyperlink{13-Yes}{\beamergotobutton{} $e^{-10}$}
\item[] \hyperlink{13-No}{\beamergotobutton{} $10\,e^{-10}$}
\item[] \hyperlink{13-No}{\beamergotobutton{} $\tfrac{1}{10!}e^{-10}$}
\item[] \hyperlink{13-No}{\beamergotobutton{} $1 - e^{-10}$}
\end{enumerate} 

 \alert{Нет!} 
\end{frame} 


 \begin{frame} \label{14-No} 
\begin{block}{14} 

Совместное распределение пары величин $X$ и $Y$ задано таблицей:

\begin{center}
\begin{tabular}{c|cc}
 & $Y=-2$ & $Y=1$ \\
\hline
$X=-1$ & 0.1 & 0 \\
$X=0$ & 0.1 & 0.3 \\
$X=1$ & 0.2 & 0.3 \\
\end{tabular}
\end{center}
\vspace{0.2cm} 
 
 Математическое ожидание величины $Y$ при условии, что $X=0$, равно
 
 \end{block} 
\begin{enumerate} 
\item[] \hyperlink{14-No}{\beamergotobutton{} $-0.1$}
\item[] \hyperlink{14-Yes}{\beamergotobutton{} $0.25$}
\item[] \hyperlink{14-No}{\beamergotobutton{} $0.2$}
\item[] \hyperlink{14-No}{\beamergotobutton{} $0$}
\item[] \hyperlink{14-No}{\beamergotobutton{} $0.1$}
\end{enumerate} 

 \alert{Нет!} 
\end{frame} 


 \begin{frame} \label{15-No} 
\begin{block}{15} 

Совместное распределение пары величин $X$ и $Y$ задано таблицей:

\begin{center}
\begin{tabular}{c|cc}
 & $Y=-2$ & $Y=1$ \\
\hline
$X=-1$ & 0.1 & 0 \\
$X=0$ & 0.1 & 0.3 \\
$X=1$ & 0.2 & 0.3 \\
\end{tabular}
\end{center}
\vspace{0.2cm} 
 
 Дисперсия случайной величины $X$ равна
 


 \end{block} 
\begin{enumerate} 
\item[] \hyperlink{15-No}{\beamergotobutton{} $0.6$}
\item[] \hyperlink{15-Yes}{\beamergotobutton{} $0.44$}
\item[] \hyperlink{15-No}{\beamergotobutton{} $1.04$
}
\item[] \hyperlink{15-No}{\beamergotobutton{} $0.4$}
\item[] \hyperlink{15-No}{\beamergotobutton{} $0.2$}
\end{enumerate} 

 \alert{Нет!} 
\end{frame} 


 \begin{frame} \label{16-No} 
\begin{block}{16} 

Совместное распределение пары величин $X$ и $Y$ задано таблицей:

\begin{center}
\begin{tabular}{c|cc}
 & $Y=-2$ & $Y=1$ \\
\hline
$X=-1$ & 0.1 & 0 \\
$X=0$ & 0.1 & 0.3 \\
$X=1$ & 0.2 & 0.3 \\
\end{tabular}
\end{center}
\vspace{0.2cm} 
 
 
Ковариация $\Cov(X, Y)$ равна
 


 \end{block} 
\begin{enumerate} 
\item[] \hyperlink{16-Yes}{\beamergotobutton{} $0.18$}
\item[] \hyperlink{16-No}{\beamergotobutton{} $0.1$}
\item[] \hyperlink{16-No}{\beamergotobutton{} $0.4$}
\item[] \hyperlink{16-No}{\beamergotobutton{} $0.9$}
\item[] \hyperlink{16-No}{\beamergotobutton{} $-0.7$}
\item[] \hyperlink{16-No}{\beamergotobutton{} $-0.5$}
\end{enumerate} 

 \alert{Нет!} 
\end{frame} 


 \begin{frame} \label{17-No} 
\begin{block}{17} 

Функция распределения абсолютно непрерывной случайной величины $X$ имеет вид
\[
F(x)=\begin{cases}
a, x<0,\\
b x^2+c, x \in [0,2],\\
d, x > 2.\\
\end{cases}
\]
\vspace{0.2cm} 
 
 Вероятность того, что $Y = 1$ при условии, что $X > 0$ равна
 


 \end{block} 
\begin{enumerate} 
\item[] \hyperlink{17-No}{\beamergotobutton{} $0.3$}
\item[] \hyperlink{17-Yes}{\beamergotobutton{} $0.6$}
\item[] \hyperlink{17-No}{\beamergotobutton{} $0.5$}
\item[] \hyperlink{17-No}{\beamergotobutton{} $0.4$
}
\item[] \hyperlink{17-No}{\beamergotobutton{} $0.2$}
\end{enumerate} 

 \alert{Нет!} 
\end{frame} 


 \begin{frame} \label{18-No} 
\begin{block}{18} 

Функция распределения абсолютно непрерывной случайной величины $X$ имеет вид
\[
F(x)=\begin{cases}
a, x<0,\\
b x^2+c, x \in [0,2],\\
d, x > 2.\\
\end{cases}
\]
\vspace{0.2cm} 
 
 Величина $X$ равномерна от $0$ до $4$. Вероятность того, что $X$ примет значение 1, равна
 


 \end{block} 
\begin{enumerate} 
\item[] \hyperlink{18-Yes}{\beamergotobutton{} $0$}
\item[] \hyperlink{18-No}{\beamergotobutton{} $0.25$}
\item[] \hyperlink{18-No}{\beamergotobutton{} $0.4$}
\item[] \hyperlink{18-No}{\beamergotobutton{} $0.8$
}
\item[] \hyperlink{18-No}{\beamergotobutton{} $0.5$}
\end{enumerate} 

 \alert{Нет!} 
\end{frame} 


 \begin{frame} \label{19-No} 
\begin{block}{19} 

Функция распределения абсолютно непрерывной случайной величины $X$ имеет вид
\[
F(x)=\begin{cases}
a, x<0,\\
b x^2+c, x \in [0,2],\\
d, x > 2.\\
\end{cases}
\]
\vspace{0.2cm} 
 
Величина $X$ имеет функцию плотности $f(x)=x/2$ на отрезке $[0;2]$. Значение $\E(X)$  равно
 


 \end{block} 
\begin{enumerate} 
\item[] \hyperlink{19-No}{\beamergotobutton{} $1/2$}
\item[] \hyperlink{19-Yes}{\beamergotobutton{} $4/3$}
\item[] \hyperlink{19-No}{\beamergotobutton{} $2$
}
\item[] \hyperlink{19-No}{\beamergotobutton{} $1$}
\item[] \hyperlink{19-No}{\beamergotobutton{} $0$}
\end{enumerate} 

 \alert{Нет!} 
\end{frame} 


 \begin{frame} \label{20-No} 
\begin{block}{20  \alert{Нет!} } 

Совместная функция плотности пары $X$ и $Y$ имеет вид
\[
f(x,y)=\begin{cases}
(x+y)/3, \; \text{ если } x\in[0;1], y\in [0;2] \\
0, \; \text{ иначе}
\end{cases}
\]
Функция распределения абсолютно непрерывной случайной величины $X$ имеет вид
\[
F(x)=\begin{cases}
a, x<0,\\
b x^2+c, x \in [0,2],\\
d, x > 2.\\
\end{cases}
\]
Выражение $a+b+c+d$ равно
 


 \end{block} 
\begin{enumerate} 
\item[] \hyperlink{20-No}{\beamergotobutton{} $2$
}
\item[] \hyperlink{20-No}{\beamergotobutton{} $1/4$}
\item[] \hyperlink{20-Yes}{\beamergotobutton{} $5/4$}
\item[] \hyperlink{20-No}{\beamergotobutton{} $1/2$}
\item[] \hyperlink{20-No}{\beamergotobutton{} $1$}
\end{enumerate} 
\end{frame} 


 \begin{frame} \label{21-No} 
\begin{block}{21} 

Совместная функция плотности пары $X$ и $Y$ имеет вид
\[
f(x,y)=\begin{cases}
(x+y)/3, \; \text{ если } x\in[0;1], y\in [0;2] \\
0, \; \text{ иначе}
\end{cases}
\]

\vspace{0.5cm} 
 
Если функция $h(x,y)=c\cdot x\cdot f(x,y)$ также является совместной функцией плотности, то константа $c$ равна
 


 \end{block} 
\begin{enumerate} 
\item[] \hyperlink{21-No}{\beamergotobutton{} $9$}
\item[] \hyperlink{21-No}{\beamergotobutton{} $5$}
\item[] \hyperlink{21-No}{\beamergotobutton{} $5/9$}
\item[] \hyperlink{21-Yes}{\beamergotobutton{} $9/5$}
\item[] \hyperlink{21-No}{\beamergotobutton{} $1$}
\end{enumerate} 

 \alert{Нет!} 
\end{frame} 


 \begin{frame} \label{22-No} 
\begin{block}{22} 

Совместная функция плотности пары $X$ и $Y$ имеет вид
\[
f(x,y)=\begin{cases}
(x+y)/3, \; \text{ если } x\in[0;1], y\in [0;2] \\
0, \; \text{ иначе}
\end{cases}
\]

\vspace{0.5cm} 
 
 Вероятность $\P(X<0.5, Y<1)$ равна
 


 \end{block} 
\begin{enumerate} 
\item[] \hyperlink{22-Yes}{\beamergotobutton{} $1/8$}
\item[] \hyperlink{22-No}{\beamergotobutton{} $5/8$}
\item[] \hyperlink{22-No}{\beamergotobutton{} $3/5$}
\item[] \hyperlink{22-No}{\beamergotobutton{} $5/6$}
\item[] \hyperlink{22-No}{\beamergotobutton{} $3/8$
}
\end{enumerate} 

 \alert{Нет!} 
\end{frame} 


 \begin{frame} \label{23-No} 
\begin{block}{23  \alert{Нет!} } 

Совместная функция плотности пары $X$ и $Y$ имеет вид
\[
f(x,y)=\begin{cases}
(x+y)/3, \; \text{ если } x\in[0;1], y\in [0;2] \\
0, \; \text{ иначе}
\end{cases}
\]

Условная функция плотности  $f_{X|Y=1}(x)$ равна
 \end{block} 
\begin{enumerate} 
\item[] \hyperlink{23-No}{\beamergotobutton{} $f_{X|Y=1}(x)=\begin{cases} (x+2)/2\, \text{ если } x\in [0;1] \\ 0, \text{ иначе }    \end{cases}$}
\item[] \hyperlink{23-No}{\beamergotobutton{} $f_{X|Y=1}(x)=\begin{cases} (2x+1)/2\, \text{ если } x\in [0;1] \\ 0, \text{ иначе }    \end{cases}$}
\item[] \hyperlink{23-Yes}{\beamergotobutton{} $f_{X|Y=1}(x)=\begin{cases} (2x+2)/3\, \text{ если } x\in [0;1] \\ 0, \text{ иначе }    \end{cases}$}
\item[] \hyperlink{23-No}{\beamergotobutton{} $f_{X|Y=1}(x)=\begin{cases} (x+4)/2\, \text{ если } x\in [0;1] \\ 0, \text{ иначе }    \end{cases}$}
\end{enumerate} 
\end{frame} 


 \begin{frame} \label{24-No} 
\begin{block}{24} 

Известно, что $\E(X)=-1$, $\Var(X)=1$, $\E(Y)=-4$, $\Var(Y)=4$, $\Corr(X,Y)=-0.5$

\vspace{0.5cm} 
 
Математическое ожидание $\E(Y)$ равно
 


 \end{block} 
\begin{enumerate} 
\item[] \hyperlink{24-No}{\beamergotobutton{} $13/7$
}
\item[] \hyperlink{24-Yes}{\beamergotobutton{} $11/9$}
\item[] \hyperlink{24-No}{\beamergotobutton{} $2/3$}
\item[] \hyperlink{24-No}{\beamergotobutton{} $4/3$}
\item[] \hyperlink{24-No}{\beamergotobutton{} $6/5$}
\end{enumerate} 

 \alert{Нет!} 
\end{frame} 


 \begin{frame} \label{25-No} 
\begin{block}{25} 

Известно, что $\E(X)=-1$, $\Var(X)=1$, $\E(Y)=-4$, $\Var(Y)=4$, $\Corr(X,Y)=-0.5$

\vspace{0.5cm} 
 
Ковариация $\Cov(2X+Y,X-3Y)$ равна
 


 \end{block} 
\begin{enumerate} 
\item[] \hyperlink{25-No}{\beamergotobutton{} $-1$
}
\item[] \hyperlink{25-Yes}{\beamergotobutton{} $-5$}
\item[] \hyperlink{25-No}{\beamergotobutton{} $0$}
\item[] \hyperlink{25-No}{\beamergotobutton{} $5$}
\item[] \hyperlink{25-No}{\beamergotobutton{} $1$}
\end{enumerate} 

 \alert{Нет!} 
\end{frame} 


 \begin{frame} \label{26-No} 
\begin{block}{26} 

Корреляция $\Corr((1-X)/2,(Y+5)/2)$ равна
 


 \end{block} 
\begin{enumerate} 
\item[] \hyperlink{26-Yes}{\beamergotobutton{} $0.5$}
\item[] \hyperlink{26-No}{\beamergotobutton{} $1$
}
\item[] \hyperlink{26-No}{\beamergotobutton{} $-0.5$}
\item[] \hyperlink{26-No}{\beamergotobutton{} $-1/8$}
\item[] \hyperlink{26-No}{\beamergotobutton{} $1/8$}
\end{enumerate} 

 \alert{Нет!} 
\end{frame} 


 \begin{frame} \label{27-No} 
\begin{block}{27} 

У неотрицательной случайной величины $X$ известны $\E(X)=1$, $\Var(X)=4$. Вероятность $\P(X^2 \geq 25)$ обязательно попадает в интервал
 


 \end{block} 
\begin{enumerate} 
\item[] \hyperlink{27-No}{\beamergotobutton{} $[0;4/25]$}
\item[] \hyperlink{27-No}{\beamergotobutton{} $[0;4/625]$}
\item[] \hyperlink{27-No}{\beamergotobutton{} $[1/25;1]$}
\item[] \hyperlink{27-No}{\beamergotobutton{} $[0;1/25]$}
\item[] \hyperlink{27-Yes}{\beamergotobutton{} $[0;1/5]$}
\end{enumerate} 

 \alert{Нет!} 
\end{frame} 


 \begin{frame} \label{28-No} 
\begin{block}{28} 

Если $\E(X)=0$, $\Var(X)=1$, то наиболее узкий интервал, в который гарантированно попадает вероятность $\P(|X| \geq 4)$, равен


 \end{block} 
\begin{enumerate} 
\item[] \hyperlink{28-No}{\beamergotobutton{} $[0.5; 1]$
}
\item[] \hyperlink{28-No}{\beamergotobutton{} $[0.0625; 1]$}
\item[] \hyperlink{28-No}{\beamergotobutton{} $[0.25; 1]$}
\item[] \hyperlink{28-No}{\beamergotobutton{} $[0; 0.25]$}
\item[] \hyperlink{28-Yes}{\beamergotobutton{} $[0; 0.0625]$ }
\end{enumerate} 

 \alert{Нет!} 
\end{frame} 


 \begin{frame} \label{29-No} 
\begin{block}{29} 


Дана последовательность независимых случайных величин, имеющих равномерное на $(-1,1)$ распределение.  \textbf{НЕВЕРНЫМ} является утверждение
 


 \end{block} 
\begin{enumerate} 
\item[] \hyperlink{29-No}{\beamergotobutton{} 	$\sqrt3n\bar X$ сходится по распределению к стандартной нормальной величине}
\item[] \hyperlink{29-No}{\beamergotobutton{} Вероятность	$\P(\bar X>0)$ стремится к 0.5}
\item[] \hyperlink{29-Yes}{\beamergotobutton{}   $\bar X$ сходится по распределению к равномерной на (-1,1) величине }
\item[] \hyperlink{29-No}{\beamergotobutton{} $\bar X$ сходится по вероятности к нулю}
\item[] \hyperlink{29-No}{\beamergotobutton{} Вероятность	$\P(\bar X = 0)$ стремится к}
\end{enumerate} 

 \alert{Нет!} 
\end{frame} 


 \begin{frame} \label{30-No} 
\begin{block}{30} 


Функция плотности случайной величины $X$ имеет вид
\[
f(x)=\frac{1}{\sqrt{8\pi}} e^{-(x-3)^2/8}
\]
 \textbf{НЕВЕРНЫМ} является утверждение
 


 \end{block} 
\begin{enumerate} 
\item[] \hyperlink{30-No}{\beamergotobutton{} $\P(X=0)=0$}
\item[] \hyperlink{30-No}{\beamergotobutton{} $\P(X>3)=0.5$}
\item[] \hyperlink{30-No}{\beamergotobutton{} $\P(X<0)>0$}
\item[] \hyperlink{30-Yes}{\beamergotobutton{} $\Var(X)=8$ }
\item[] \hyperlink{30-No}{\beamergotobutton{} $\E(X)=3$}
\item[] \hyperlink{30-No}{\beamergotobutton{} $\max f(x) = \frac{1}{2\sqrt{2\pi}}$}
\end{enumerate} 

 \alert{Нет!} 
\end{frame} 


 \begin{frame} \label{31-No} 
\begin{block}{31} 

Величины $X_1$, $X_2$, \ldots независимы и одинаково распределены с $\E(X_i)=\mu$, $\Var(X_i)=\sigma^2$. К стандартному нормальному распределению  сходится последовательность случайных величин
 


 \end{block} 
\begin{enumerate} 
\item[] \hyperlink{31-No}{\beamergotobutton{} $(\bar X - \mu) /(\sqrt{n}\sigma)$}
\item[] \hyperlink{31-No}{\beamergotobutton{} $(\bar X - n\mu) /(\sqrt{n}\sigma)$}
\item[] \hyperlink{31-No}{\beamergotobutton{} $(\bar X - \mu) /\sigma$}
\item[] \hyperlink{31-Yes}{\beamergotobutton{} $\sqrt{n}(\bar X - \mu) /\sigma$ }
\item[] \hyperlink{31-No}{\beamergotobutton{} $\bar X$}
\end{enumerate} 

 \alert{Нет!} 
\end{frame} 


 \begin{frame} \label{32-No} 
\begin{block}{32} 

Пусть $X_1$, \ldots, $X_n$ — выборка объема $n$ из равномерного на $[a, b]$ распределения. Оценка $X_1+X_2$ параметра $c=a+b$ является
 


 \end{block} 
\begin{enumerate} 
\item[] \hyperlink{32-No}{\beamergotobutton{} смещенной и несостоятельной}
\item[] \hyperlink{32-Yes}{\beamergotobutton{} несмещенной и несостоятельной}
\item[] \hyperlink{32-No}{\beamergotobutton{} смещенной и состоятельной}
\item[] \hyperlink{32-No}{\beamergotobutton{} асимптотически несмещенной и состоятельной}
\item[] \hyperlink{32-No}{\beamergotobutton{} несмещенной и состоятельной}
\end{enumerate} 

 \alert{Нет!} 
\end{frame} 


 \begin{frame} \label{33-No} 
\begin{block}{33} 

Пусть $X_1$, \ldots, $X_n$ — выборка объема $n$ из некоторого распределения с конечным математическим ожиданием. Несмещенной и состоятельной оценкой математического ожидания является
 


 \end{block} 
\begin{enumerate} 
\item[] \hyperlink{33-Yes}{\beamergotobutton{} $\frac{X_1}{2n}+\frac{X_2+\ldots+X_{n-1}}{n-2}-\frac{X_n}{2 n}$}
\item[] \hyperlink{33-No}{\beamergotobutton{} $\frac{1}{3} X_1 + \frac{2}{3} X_2$}
\item[] \hyperlink{33-No}{\beamergotobutton{} $\frac{X_1}{2 n}+\frac{X_2+\ldots+X_{n-2}}{n-2}+\frac{X_n}{2 n}$}
\item[] \hyperlink{33-No}{\beamergotobutton{} $\frac{X_1}{2 n}+\frac{X_2+\ldots+X_{n-2}}{n-1}+\frac{X_n}{2 n}$}
\item[] \hyperlink{33-No}{\beamergotobutton{} $\frac{X_1+X_2}{2}$}
\end{enumerate} 

 \alert{Нет!} 
\end{frame} 


 \begin{frame} \label{34-No} 
\begin{block}{34} 

Пусть $X_1$,\ldots, $X_n$ — выборка объема $n$ из равномерного на $[0, \theta]$ распределения. Оценка параметра $\theta$ методом моментов по $k$-му моменту имеет вид:
 


 \end{block} 
\begin{enumerate} 
\item[] \hyperlink{34-No}{\beamergotobutton{} $\sqrt[k]k \overline{X^k}$}
\item[] \hyperlink{34-No}{\beamergotobutton{} $\sqrt[k](k+1) \overline{X^k}$}
\item[] \hyperlink{34-No}{\beamergotobutton{} $\sqrt[k]k \overline{X^k}$}
\item[] \hyperlink{34-Yes}{\beamergotobutton{} $\sqrt[k](k+1) \overline{X^k}$}
\item[] \hyperlink{34-No}{\beamergotobutton{} $\sqrt[k+1](k+1) \overline{X^k}$}
\end{enumerate} 

 \alert{Нет!} 
\end{frame} 


 \begin{frame} \label{35-No} 
\begin{block}{35} 

Пусть $X_1$, \ldots, $X_n$ — выборка объема $n$ из равномерного на $[0, \theta]$ распределения. Состоятельной оценкой параметра $\theta$ является:
 


 \end{block} 
\begin{enumerate} 
\item[] \hyperlink{35-No}{\beamergotobutton{} $X_(n)$}
\item[] \hyperlink{35-No}{\beamergotobutton{} $X_(n-1)$}
\item[] \hyperlink{35-No}{\beamergotobutton{} $\frac{n}{n+1} X_{(n-1)}$}
\item[] \hyperlink{35-No}{\beamergotobutton{} $\frac{n^2}{n^2-n+3} X_{(n-3)}$}
\item[] \hyperlink{35-Yes}{\beamergotobutton{} все перечисленные случайные величины}
\end{enumerate} 

 \alert{Нет!} 
\end{frame} 


 \begin{frame} \label{36-No} 
\begin{block}{36} 

Пусть $X_1$, \ldots, $X_{2 n}$ — выборка объема $2 n$ из некоторого распределения. Какая из нижеперечисленных оценок математического ожидания имеет наименьшую дисперсию?
 


 \end{block} 
\begin{enumerate} 
\item[] \hyperlink{36-No}{\beamergotobutton{} $X_1$}
\item[] \hyperlink{36-No}{\beamergotobutton{} $\frac{X_1+X_2}{2}$}
\item[] \hyperlink{36-No}{\beamergotobutton{} $\frac{1}{n} \sum_{i=1}^n X_i$}
\item[] \hyperlink{36-No}{\beamergotobutton{} $\frac{1}{n} \sum_{i=n+1}^{2 n} X_i$}
\item[] \hyperlink{36-Yes}{\beamergotobutton{} $\frac{1}{2 n} \sum_{i=1}^{2 n} X_i$}
\end{enumerate} 

 \alert{Нет!} 
\end{frame} 


 \begin{frame} \label{37-No} 
\begin{block}{37} 

Пусть $X_1$, \ldots, $X_n$ — выборка объема $n$ из распределения Бернулли с параметром $p$. Статистика $X_2 X_{n-2}$ является
 


 \end{block} 
\begin{enumerate} 
\item[] \hyperlink{37-Yes}{\beamergotobutton{} несмещенной оценкой $p^2$}
\item[] \hyperlink{37-No}{\beamergotobutton{} состоятельной оценкой $p^2$}
\item[] \hyperlink{37-No}{\beamergotobutton{} эффективной оценкой $p^2$}
\item[] \hyperlink{37-No}{\beamergotobutton{} асимптотически нормальной оценкой $p^2$}
\item[] \hyperlink{37-No}{\beamergotobutton{} оценкой максимального правдоподобия}
\end{enumerate} 

 \alert{Нет!} 
\end{frame} 


 \begin{frame} \label{38-No} 
\begin{block}{38} 

Пусть $X_1$, \ldots, $X_n$ — выборка объема $n$ из равномерного на $[a, b]$ распределения. Выберите наиболее точный ответ из предложенных. Оценка $\theta^*_n = X_{(n)}-X_{(1)}$ длины отрезка $[a,b]$ является
 


 \end{block} 
\begin{enumerate} 
\item[] \hyperlink{38-No}{\beamergotobutton{} состоятельной и асимптотически смещённой}
\item[] \hyperlink{38-No}{\beamergotobutton{} несостоятельной и асимптотически несмещенной}
\item[] \hyperlink{38-Yes}{\beamergotobutton{} состоятельной и асимптотически несмещенной}
\item[] \hyperlink{38-No}{\beamergotobutton{} нормально распределённой}
\item[] \hyperlink{38-No}{\beamergotobutton{} несмещенной}
\end{enumerate} 

 \alert{Нет!} 
\end{frame} 


 \begin{frame} \label{39-No} 
\begin{block}{39} 

Мощностью теста называется
 


 \end{block} 
\begin{enumerate} 
\item[] \hyperlink{39-No}{\beamergotobutton{} Вероятность принять неверную гипотезу}
\item[] \hyperlink{39-No}{\beamergotobutton{} Единица минус  вероятность отвергнуть основную гипотезу, когда она верна}
\item[] \hyperlink{39-Yes}{\beamergotobutton{} Единица минус  вероятность отвергнуть альтернативную гипотезу, когда она верна}
\item[] \hyperlink{39-No}{\beamergotobutton{} Вероятность отвергнуть альтернативную гипотезу, когда она верна}
\item[] \hyperlink{39-No}{\beamergotobutton{} Вероятность отвергнуть основную гипотезу, когда она верна}
\end{enumerate} 

 \alert{Нет!} 
\end{frame} 


 \begin{frame} \label{40-No} 
\begin{block}{40} 

Если P-значение (P-value) больше уровня значимости  $\alpha$, то гипотеза  $H_0: \; \sigma=1$


 \end{block} 
\begin{enumerate} 
\item[] \hyperlink{40-No}{\beamergotobutton{} Отвергается, только если  $H_a: \; \sigma<1$}
\item[] \hyperlink{40-No}{\beamergotobutton{} Отвергается}
\item[] \hyperlink{40-No}{\beamergotobutton{} Отвергается, только если  $H_a: \; \sigma>1$}
\item[] \hyperlink{40-Yes}{\beamergotobutton{} Не отвергается}
\item[] \hyperlink{40-No}{\beamergotobutton{} Отвергается, только если  $H_a: \; \sigma\neq 1$}
\end{enumerate} 

 \alert{Нет!} 
\end{frame} 


 \begin{frame} \label{41-No} 
\begin{block}{41} 

Имеется случайная выборка размера $n$ из нормального распределения. При проверке гипотезы о равенстве математического ожидания заданному значению при известной дисперсии используется статистика, имеющая распределение
 


 \end{block} 
\begin{enumerate} 
\item[] \hyperlink{41-No}{\beamergotobutton{}  $t_n-1$}
\item[] \hyperlink{41-No}{\beamergotobutton{} $\chi^2_n$}
\item[] \hyperlink{41-Yes}{\beamergotobutton{} $N(0,1)$}
\item[] \hyperlink{41-No}{\beamergotobutton{} $\chi^2_n-1$}
\item[] \hyperlink{41-No}{\beamergotobutton{} $t_n$}
\end{enumerate} 

 \alert{Нет!} 
\end{frame} 


 \begin{frame} \label{42-No} 
\begin{block}{42} 

Имеется случайная выборка размера $n$ из нормального распределения. При проверке гипотезы о равенстве дисперсии заданному значению при неизвестном математическом ожидании используется статистика, имеющая распределение
 


 \end{block} 
\begin{enumerate} 
\item[] \hyperlink{42-Yes}{\beamergotobutton{} $\chi^2_n-1$}
\item[] \hyperlink{42-No}{\beamergotobutton{} $t_n$}
\item[] \hyperlink{42-No}{\beamergotobutton{}  $t_n-1$}
\item[] \hyperlink{42-No}{\beamergotobutton{} $N(0,1)$}
\item[] \hyperlink{42-No}{\beamergotobutton{} $\chi^2_n$}
\end{enumerate} 

 \alert{Нет!} 
\end{frame} 


 \begin{frame} \label{43-No} 
\begin{block}{43} 

По случайной выборке из 100 наблюдений было оценено выборочное среднее $\bar{X}=20$  и несмещенная оценка дисперсии  $\hat{\sigma}^2=25$. В рамках проверки гипотезы $H_0: \; \mu=15$  против альтернативной гипотезы $H_a: \; \mu>15$  можно сделать следующее заключение
 


 \end{block} 
\begin{enumerate} 
\item[] \hyperlink{43-No}{\beamergotobutton{} Гипотеза $H_0$  не отвергается на любом разумном уровне значимости}
\item[] \hyperlink{43-No}{\beamergotobutton{} Гипотеза $H_0$  отвергается на уровне значимости 5\%, но не  на уровне значимости 1\%}
\item[] \hyperlink{43-No}{\beamergotobutton{} Гипотеза  $H_0$ отвергается на уровне значимости 10\%, но не на уровне значимости 5\%}
\item[] \hyperlink{43-Yes}{\beamergotobutton{} Гипотеза $H_0$  отвергается на любом разумном уровне значимости}
\item[] \hyperlink{43-No}{\beamergotobutton{} Гипотеза  $H_0$ отвергается на уровне значимости 20\%, но не  на уровне значимости 10\%}
\end{enumerate} 

 \alert{Нет!} 
\end{frame} 


 \begin{frame} \label{44-No} 
\begin{block}{44} 

На основе случайной выборки, содержащей одно наблюдение  $X_1$, тестируется гипотеза $H_0: \; X_1 \sim U[0;1]$  против альтернативной гипотезы  $H_a: \; X_1 \sim U[0.5;1.5]$. Рассматривается критерий: если $X_1>0.8$, то гипотеза $H_0$  отвергается в пользу гипотезы  $H_a$. Вероятность ошибки 2-го рода для этого критерия равна:
 


 \end{block} 
\begin{enumerate} 
\item[] \hyperlink{44-Yes}{\beamergotobutton{} 0.3}
\item[] \hyperlink{44-No}{\beamergotobutton{} 0.1}
\item[] \hyperlink{44-No}{\beamergotobutton{} 0.2}
\item[] \hyperlink{44-No}{\beamergotobutton{} 0.5}
\item[] \hyperlink{44-No}{\beamergotobutton{} 0.4}
\end{enumerate} 

 \alert{Нет!} 
\end{frame} 


 \begin{frame} \label{45-No} 
\begin{block}{45} 

Пусть $X_1$, $X_2$, \ldots, $X_n$ — случайная выборка размера 36 из нормального распределения $N(\mu, 9)$. Для тестирования основной гипотезы  $H_0: \; \mu=0$  против альтернативной $H_a: \; \mu=-2$   вы используете критерий: если  $\bar{X}\geq -1$, то вы не отвергаете гипотезу $H_0$, в противном случае вы отвергаете гипотезу  $H_0$ в пользу гипотезы  $H_a$. Мощность критерия равна
 


 \end{block} 
\begin{enumerate} 
\item[] \hyperlink{45-No}{\beamergotobutton{} 0.87}
\item[] \hyperlink{45-Yes}{\beamergotobutton{} 0.98}
\item[] \hyperlink{45-No}{\beamergotobutton{} 0.78}
\item[] \hyperlink{45-No}{\beamergotobutton{} 0.58}
\item[] \hyperlink{45-No}{\beamergotobutton{} 0.85}
\end{enumerate} 

 \alert{Нет!} 
\end{frame} 


 \begin{frame} \label{46-No} 
\begin{block}{46} 

Николай Коперник подбросил бутерброд 200 раз. Бутерброд упал маслом вниз 95 раз, а маслом вверх — 105 раз. Значение критерия $\chi^2$ Пирсона для проверки гипотезы о равной вероятности данных событий равно
 


 \end{block} 
\begin{enumerate} 
\item[] \hyperlink{46-No}{\beamergotobutton{} 0.75}
\item[] \hyperlink{46-No}{\beamergotobutton{} 7.5}
\item[] \hyperlink{46-No}{\beamergotobutton{} 0.5}
\item[] \hyperlink{46-No}{\beamergotobutton{} 0.25}
\item[] \hyperlink{46-Yes}{\beamergotobutton{} 0.5}
\end{enumerate} 

 \alert{Нет!} 
\end{frame} 


 \begin{frame} \label{47-No} 
\begin{block}{47} 

Каждое утро в 8:00 Иван Андреевич Крылов, либо завтракает, либо уже позавтракал. В это же время кухарка либо заглядывает к Крылову, либо нет. По таблице сопряженности вычислите  статистику $\chi^2$ Пирсона для тестирования гипотезы о том, что визиты кухарки не зависят от того, позавтракал ли уже Крылов или нет.
\begin{tabular}{c|cc}
Время 8:00 & кухарка заходит & кухарка не заходит \\
\hline
Крылов завтракает & 200 & 40 \\
Крылов уже позавтракал & 25 & 100 \\
\end{tabular}
 


 \end{block} 
\begin{enumerate} 
\item[] \hyperlink{47-No}{\beamergotobutton{} 79}
\item[] \hyperlink{47-No}{\beamergotobutton{} 100}
\item[] \hyperlink{47-Yes}{\beamergotobutton{} 139}
\item[] \hyperlink{47-No}{\beamergotobutton{} 39}
\item[] \hyperlink{47-No}{\beamergotobutton{} 179}
\end{enumerate} 

 \alert{Нет!} 
\end{frame} 


 \begin{frame} \label{48-No} 
\begin{block}{48} 

Ковариационная матрица вектора $X=(X_1,X_2)$ имеет вид
\[
\begin{pmatrix}
10 & 3 \\
3 & 8
\end{pmatrix}
\]
Дисперсия разности элементов вектора, $\Var(X_1-X_2)$, равняется
 


 \end{block} 
\begin{enumerate} 
\item[] \hyperlink{48-No}{\beamergotobutton{} 15}
\item[] \hyperlink{48-No}{\beamergotobutton{} 2}
\item[] \hyperlink{48-Yes}{\beamergotobutton{} 12}
\item[] \hyperlink{48-No}{\beamergotobutton{} 18}
\item[] \hyperlink{48-No}{\beamergotobutton{} 6}
\end{enumerate} 

 \alert{Нет!} 
\end{frame} 


 \begin{frame} \label{49-No} 
\begin{block}{49} 

Все условия регулярности для применения метода максимального правдоподобия выполнены. Вторая производная лог-функции правдоподобия равна $\ell''(\hat{\theta})=-100$. Оценка стандартной ошибки для $\hat{\theta}$ равна
 


 \end{block} 
\begin{enumerate} 
\item[] \hyperlink{49-No}{\beamergotobutton{} 1}
\item[] \hyperlink{49-No}{\beamergotobutton{} 100}
\item[] \hyperlink{49-Yes}{\beamergotobutton{} 0.1}
\item[] \hyperlink{49-No}{\beamergotobutton{} 0.01}
\item[] \hyperlink{49-No}{\beamergotobutton{} 10}
\end{enumerate} 

 \alert{Нет!} 
\end{frame} 


 \begin{frame} \label{50-No} 
\begin{block}{50} 

Геродот Геликарнасский проверяет гипотезу $H_0: \; \mu=0, \; \sigma^2=1$ с помощью $LR$ статистики теста отношения правдоподобия. При подстановке оценок метода максимального правдоподобия в лог-функцию правдоподобия он получил $\ell=-177$, а при подстановке $\mu=0$ и $\sigma=1$ оказалось, что $\ell=-211$. Найдите значение $LR$ статистики и укажите её закон распределения при верной $H_0$
 


 \end{block} 
\begin{enumerate} 
\item[] \hyperlink{50-No}{\beamergotobutton{} $LR=\ln 68$, $\chi^2_n-2$}
\item[] \hyperlink{50-No}{\beamergotobutton{} $LR=34$, $\chi^2_n-1$}
\item[] \hyperlink{50-No}{\beamergotobutton{} $LR=34$, $\chi^2_2$}
\item[] \hyperlink{50-No}{\beamergotobutton{} $LR=\ln 34$, $\chi^2_n-2$}
\item[] \hyperlink{50-Yes}{\beamergotobutton{} $LR=68$, $\chi^2_2$}
\end{enumerate} 

 \alert{Нет!} 
\end{frame} 


 \begin{frame} \label{51-No} 
\begin{block}{51} 

Геродот Геликарнасский проверяет гипотезу $H_0: \; \mu=2$. Лог-функция правдоподобия имеет вид $\ell(\mu,\nu)=-\frac{n}{2}\ln (2\pi)-\frac{n}{2}\ln \nu -\frac{\sum_{i=1}^n(x_i-\mu)^2}{2\nu}$. Оценка максимального правдоподобия для $\nu$ при предположении, что $H_0$ верна, равна
 


 \end{block} 
\begin{enumerate} 
\item[] \hyperlink{51-Yes}{\beamergotobutton{}$\frac{\sum x_i^2 - 4\sum x_i}{n}+4$}
\item[] \hyperlink{51-No}{\beamergotobutton{} $\frac{\sum x_i^2 - 4\sum x_i}{n}+3$}
\item[] \hyperlink{51-No}{\beamergotobutton{} $\frac{\sum x_i^2 - 4\sum x_i}{n}+2$}
\item[] \hyperlink{51-No}{\beamergotobutton{} $\frac{\sum x_i^2 - 4\sum x_i}{n}+1$}
\item[] \hyperlink{51-No}{\beamergotobutton{} $\frac{\sum x_i^2 - 4\sum x_i}{n}$}
\end{enumerate} 

 \alert{Нет!} 
\end{frame} 


 \begin{frame} \label{52-No} 
\begin{block}{52} 

Ацтек Монтесума Илуикамина хочет оценить параметр $a$ методом максимального правдоподобия по выборке из неотрицательного распределения с функцией плотности $f(x)=\frac{1}{2}a^3x^2e^{-ax}$ при $x\geq 0$. Для этой цели ему достаточно максимизировать функцию
 


 \end{block} 
\begin{enumerate} 
\item[] \hyperlink{52-Yes}{\beamergotobutton{} $3n \ln a - a \sum x_i$}
\item[] \hyperlink{52-No}{\beamergotobutton{} $3n\prod \ln a - a x^n$}
\item[] \hyperlink{52-No}{\beamergotobutton{} $3n\ln a - a \prod \ln x_i$}
\item[] \hyperlink{52-No}{\beamergotobutton{} $3n \ln a - an \ln x_i$}
\item[] \hyperlink{52-No}{\beamergotobutton{} $3n \sum \ln a_i - a \sum \ln x_i$}
\end{enumerate} 

 \alert{Нет!} 
\end{frame} 


 \begin{frame} \label{53-No} 
\begin{block}{53} 

Бессмертный гений поэзии Ли Бо оценивает математическое ожидание  по выборка размера $n$ из нормального распределения. Он построил оценку метода моментов, $\hat{\mu}_{MM}$, и оценку максимального правдоподобия, $\hat{\mu}_{ML}$. Про эти оценки можно утверждать, что
 


 \end{block} 
\begin{enumerate} 
\item[] \hyperlink{53-No}{\beamergotobutton{}  $\hat\mu_MM>\hat\mu_ML$}
\item[] \hyperlink{53-Yes}{\beamergotobutton{} они равны}
\item[] \hyperlink{53-No}{\beamergotobutton{} $\hat\mu_MM<\hat\mu_ML$ }
\item[] \hyperlink{53-No}{\beamergotobutton{} они не равны, но сближаются при $n\to \infty$}
\item[] \hyperlink{53-No}{\beamergotobutton{} они не равны, и не сближаются при $n\to \infty$}
\end{enumerate} 

 \alert{Нет!} 
\end{frame} 


 \begin{frame} \label{54-No} 
\begin{block}{54} 

Проверяя гипотезу о равенстве дисперсий в двух выборках (размером в 3 и 5 наблюдений), Анаксимандр Милетский получил значение тестовой статистики 10. Если оценка дисперсии по первой выборке равна 8, то вторая оценка дисперсии может быть равна
 


 \end{block} 
\begin{enumerate} 
\item[] \hyperlink{54-No}{\beamergotobutton{} $25$}
\item[] \hyperlink{54-No}{\beamergotobutton{} $4/3$}
\item[] \hyperlink{54-Yes}{\beamergotobutton{} $80$}
\item[] \hyperlink{54-No}{\beamergotobutton{} $3/4$}
\item[] \hyperlink{54-No}{\beamergotobutton{} $4$}
\end{enumerate} 

 \alert{Нет!} 
\end{frame} 


 \begin{frame} \label{55-No} 
\begin{block}{55} 

Пусть  $\hat{\sigma}^2_1$ — несмещенная оценка дисперсии, полученная по первой выборке размером $n_1$,   $\hat{\sigma}^2_2$ — несмещенная оценка дисперсии, полученная по второй выборке, с меньшим размером  $n_2$. Тогда статистика $\frac{\hat{\sigma}^2_1/n_1}{\hat{\sigma}^2_2/n_2}$  имеет распределение
 


 \end{block} 
\begin{enumerate} 
\item[] \hyperlink{55-No}{\beamergotobutton{} $\chi^2_{n_1+n_2}$}
\item[] \hyperlink{55-No}{\beamergotobutton{} $F_{n_1,n_2}$}
\item[] \hyperlink{55-No}{\beamergotobutton{} $F_{n_1-1,n_2-1}$}
\item[] \hyperlink{55-No}{\beamergotobutton{} $t_{n_1+n_2-1}$}
\item[] \hyperlink{55-No}{\beamergotobutton{} $N(0;1)$}
\end{enumerate} 

 \alert{Нет!} 
\end{frame} 


 \begin{frame} \label{56-No} 
\begin{block}{56} 

Зулус Чака каСензангакона проверяет гипотезу  о равенстве математических ожиданий в двух нормальных выборках небольших размеров $n_1$   и  $n_2$. Если дисперсии неизвестны, но равны, то тестовая статистика имеет распределение
 


 \end{block} 
\begin{enumerate} 
\item[] \hyperlink{56-No}{\beamergotobutton{} $F_{n_1,n_2}$}
\item[] \hyperlink{56-Yes}{\beamergotobutton{} $t_{n_1+n_2-1}$}
\item[] \hyperlink{56-No}{\beamergotobutton{} $t_{n_1+n_2}$}
\item[] \hyperlink{56-No}{\beamergotobutton{} $t_{n_1+n_2-2}$}
\item[] \hyperlink{56-No}{\beamergotobutton{} $\chi^2_{n_1+n_2-1}$}
\end{enumerate} 

 \alert{Нет!} 
\end{frame} 


 \begin{frame} \label{57-No} 
\begin{block}{57} 

Критерий знаков проверяет нулевую гипотезу
 


 \end{block} 
\begin{enumerate} 
\item[] \hyperlink{57-Yes}{\beamergotobutton{} о равенстве нулю вероятности того, что случайная величина $X$ окажется больше случайной величины $Y$, если альтернативная гипотеза записана как $\mu_X>\mu_Y$}
\item[] \hyperlink{57-No}{\beamergotobutton{} о равенстве нулю вероятности того, что случайная величина $X$ окажется больше случайной величины $Y$, если альтернативная гипотеза записана как $\mu_X>\mu_Y$ }
\item[] \hyperlink{57-No}{\beamergotobutton{} о равенстве математических ожиданий двух нормально распределенных случайных величин}
\item[] \hyperlink{57-No}{\beamergotobutton{} о совпадении функции распределения случайной величины с заданной теоретической функцией распределения}
\item[] \hyperlink{57-No}{\beamergotobutton{} о равенстве $1/2$ вероятности того, что случайная величина $X$ окажется больше случайной величины $Y$, если альтернативная гипотеза записана как $\mu_X>\mu_Y$}
\end{enumerate} 

 \alert{Нет!} 
\end{frame} 


 \begin{frame} \label{58-No} 
\begin{block}{58} 

Вероятность ошибки первого рода, $\alpha$, и вероятность ошибки второго рода, $\beta$, всегда связаны соотношением


 \end{block} 
\begin{enumerate} 
\item[] \hyperlink{58-No}{\beamergotobutton{} $\alpha+\beta \leq 1$}
\item[] \hyperlink{58-No}{\beamergotobutton{} $\alpha+\beta \geq 1$}
\item[] \hyperlink{58-No}{\beamergotobutton{} $\alpha\geq \beta $}
\item[] \hyperlink{58-No}{\beamergotobutton{} $\alpha+\beta=1$}
\item[] \hyperlink{58-No}{\beamergotobutton{} $\alpha\leq \beta $}
\end{enumerate} 

 \alert{Нет!} 
\end{frame} 


 \begin{frame} \label{59-No} 
\begin{block}{59} 

Среди 100 случайно выбранных ацтеков 20 платят дань Кулуакану, а 80 — Аскапоцалько. Соответственно, оценка доли ацтеков, платящих дань Кулуакану, равна $\hat{p}=0.2$. Разумная оценка стандартного отклонения случайной величины $\hat{p}$ равна
 


 \end{block} 
\begin{enumerate} 
\item[] \hyperlink{59-No}{\beamergotobutton{} $0.4$}
\item[] \hyperlink{59-No}{\beamergotobutton{} $1.6$}
\item[] \hyperlink{59-Yes}{\beamergotobutton{} $0.04$}
\item[] \hyperlink{59-No}{\beamergotobutton{} $0.16$}
\item[] \hyperlink{59-No}{\beamergotobutton{} $0.016$}
\end{enumerate} 

 \alert{Нет!} 
\end{frame} 


 \begin{frame} \label{60-No} 
\begin{block}{60} 

Датчик случайных чисел выдал следующие значения псевдо случайной величины: $0.78$, $0.48$. Вычислите значение критерия Колмогорова и проверьте гипотезу $H_0$ о соответствии распределения равномерному на $[0;1]$. Критическое значение статистики Колмогорова для уровня значимости 0.1 и двух наблюдений равно $0.776$.
 


 \end{block} 
\begin{enumerate} 
\item[] \hyperlink{60-No}{\beamergotobutton{} 1.26, $H_0$ отвергается}
\item[] \hyperlink{60-No}{\beamergotobutton{} 0.3, $H_0$ не отвергается}
\item[] \hyperlink{60-Yes}{\beamergotobutton{} 0.78, $H_0$ отвергается}
\item[] \hyperlink{60-No}{\beamergotobutton{} 0.48, $H_0$ не отвергается}
\item[] \hyperlink{60-No}{\beamergotobutton{} 0.37, $H_0$ не отвергается}
\end{enumerate} 

 \alert{Нет!} 
\end{frame} 


 \begin{frame} \label{61-No} 
\begin{block}{61} 

У пяти случайно выбранных студентов первого потока результаты за контрольную по статистике оказались равны  82, 47, 20, 43 и 73. У четырёх случайно выбранных студентов второго потока — 68, 83, 60 и 52. Вычислите статистику Вилкоксона и проверьте гипотезу $H_0$ об однородности результатов студентов двух потоков. Критические значения статистики Вилкоксона равны $T_L=12$ и $T_R=28$.
 


 \end{block} 
\begin{enumerate} 
\item[] \hyperlink{61-No}{\beamergotobutton{} 53, $H_0$ отвергается}
\item[] \hyperlink{61-No}{\beamergotobutton{} 20, $H_0$ не отвергается}
\item[] \hyperlink{61-No}{\beamergotobutton{} 65.75, $H_0$ отвергается}
\item[] \hyperlink{61-No}{\beamergotobutton{} 12.75, $H_0$ не отвергается}
\item[] \hyperlink{61-Yes}{\beamergotobutton{} 24, $H_0$ не отвергается}
\end{enumerate} 

 \alert{Нет!} 
\end{frame} 


 \begin{frame} \label{62-No} 
\begin{block}{62} 

 Производитель мороженного попросил оценить по 10-бальной шкале два вида мороженного: с кусочками шоколада и с орешками. Было опрошено 5 человек.


 \begin{tabular}{c|ccccc}
  & Евлампий & Аристарх & Капитолина & Аграфена & Эвридика \\
 \hline
С крошкой & 10 & 6 & 7 & 5 & 4 \\
С орехами & 9 & 8 & 8 & 7 & 6 \\
 \end{tabular}


Вычислите модуль значения статистики теста знаков. Используя нормальную аппроксимацию, проверьте на уровне значимости $0.05$ гипотезу об отсутствии предпочтения мороженного с орешками против альтернативы, что мороженное с орешками вкуснее.
 


 \end{block} 
\begin{enumerate} 
\item[] \hyperlink{62-No}{\beamergotobutton{} 1.29, $H_0$ не отвергается}
\item[] \hyperlink{62-No}{\beamergotobutton{} 1.34, $H_0$ не отвергается}
\item[] \hyperlink{62-No}{\beamergotobutton{} 1.65, $H_0$ отвергается}
\item[] \hyperlink{62-Yes}{\beamergotobutton{} 1.96, $H_0$ отвергается}
\item[] \hyperlink{62-No}{\beamergotobutton{} 1.29, $H_0$ отвергается}
\end{enumerate} 

 \alert{Нет!} 
\end{frame} 


 \begin{frame} \label{63-No} 
\begin{block}{63} 

По 10 наблюдениям проверяется гипотеза $H_0: \; \mu=10$ против $H_a: \; \mu \neq 10$ на выборке из нормального распределения с неизвестной дисперсией. Величина $\sqrt{n}\cdot (\bar{X}-\mu)/\hat{\sigma}$ оказалась равной $1$. P-значение примерно равно
 


 \end{block} 
\begin{enumerate} 
\item[] \hyperlink{63-No}{\beamergotobutton{} $0.32$}
\item[] \hyperlink{63-No}{\beamergotobutton{} $0.17$}
\item[] \hyperlink{63-Yes}{\beamergotobutton{} $0.16$}
\item[] \hyperlink{63-No}{\beamergotobutton{} $0.34$}
\item[] \hyperlink{63-No}{\beamergotobutton{} $0.83$}
\end{enumerate} 

 \alert{Нет!} 
\end{frame} 

\end{document}
