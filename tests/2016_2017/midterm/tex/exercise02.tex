
\begin{question}
Функцией плотности случайной величины может являться функция
\begin{answerlist}
  \item \(f(x) = \begin{cases} \frac{1}{x^2}, x \in [1,+ \infty) \\ 0,\text{ иначе} \end{cases}\)
  \item \(f(x) = \begin{cases} -1, x \in [-1, 0] \\ 0,\text{ иначе} \end{cases}\)
  \item \(f(x) = \begin{cases} x^2, x \in [0,2] \\ 0,\text{ иначе} \end{cases}\)
  \item \(f(x) = \frac{1}{\sqrt{2\pi}} e^{-x^2}\)
  \item \(f(x) = \begin{cases} x - 1, x \in [0,1+\sqrt{3}] \\ 0,\text{ иначе} \end{cases}\)
\end{answerlist}
\end{question}

\begin{solution}
\begin{answerlist}
  \item Good answer :)
  \item Bad answer :(
  \item Bad answer :(
  \item Bad answer :(
  \item Bad answer :(
\end{answerlist}
\end{solution}

