
\begin{question}
Монетку подбрасывают три раза. Рассмотрим три события: \(A\) — «хотя
бы один раз выпала решка»; \(B\) — «хотя бы один раз выпал орёл»;
\(C\) — «все три раза выпал орёл».
\begin{answerlist}
  \item События \(A\) и \(B\) совместны, события \(A\) и \(C\) несовместны.
  \item События \(A\) и \(B\) несовместны, события \(A\) и \(C\) совместны.
  \item События \(A\) и \(B\) несовместны, события \(B\) и \(C\) совместны.
  \item События \(A\) и \(B\) совместны, события \(A\) и \(C\) совместны.
  \item События \(A\) и \(B\) несовместны, события \(B\) и \(C\) несовместны.
\end{answerlist}
\end{question}

\begin{solution}
\begin{answerlist}
  \item Good answer :)
  \item Bad answer :(
  \item Bad answer :(
  \item Bad answer :(
  \item Bad answer :(
\end{answerlist}
\end{solution}

