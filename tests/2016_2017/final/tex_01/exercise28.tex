
\begin{question}
Экзамен принимают два преподавателя: Б.Б.\textasciitilde{}Злой и
Е.В.\textasciitilde{}Добрая. Они поставили следующие оценки:

\vspace{5mm}
\begin{tabular}{lccccc}
\toprule
Е.В. Добрая & 6 & 4 & 7  & 8 &   \\
Б.Б. Злой   & 2 & 3 & 10 & 8 & 3 \\
\bottomrule
\end{tabular}
\vspace{5mm}

Значение статистики Вилкоксона для гипотезы о совпадении распределений
оценок равно
\begin{answerlist}
  \item \(20.5\)
  \item \(7.5\)
  \item \(20\)
  \item \(19\)
  \item \(22.5\)
\end{answerlist}
\end{question}

\begin{solution}
\begin{answerlist}
  \item Bad answer :(
  \item Bad answer :(
  \item Bad answer :(
  \item Bad answer :(
  \item Good answer :)
\end{answerlist}
\end{solution}

