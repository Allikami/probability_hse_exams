
\begin{question}
Экзамен принимают два преподавателя: Злой и Добрый. Злой поставил оценки
2, 3, 10, 8, 1. А Добрый — оценки 6, 4, 7, 9. Значение статистики
критерия Вилкоксона о совпадении распределений оценок может быть равно
\begin{answerlist}
  \item \(25\)
  \item \(23\)
  \item \(26\)
  \item \(22\)
  \item \(24\)
\end{answerlist}
\end{question}

\begin{solution}
\begin{answerlist}
  \item Тоже ересь
  \item Ураа!!!
  \item Не туда!
  \item Не угадал
  \item Неверно
\end{answerlist}
\end{solution}

