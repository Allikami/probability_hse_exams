\documentclass[t]{beamer}

\usetheme{Boadilla} 
 \usecolortheme{seahorse} 

\setbeamertemplate{footline}[frame number]{} 
 \setbeamertemplate{navigation symbols}{} 
 \setbeamertemplate{footline}{}
\usepackage{cmap} 

\usepackage{mathtext} 

\usepackage{amsmath,amsfonts,amssymb,amsthm,mathtools}
\usepackage[T2A]{fontenc} 

\usepackage[utf8]{inputenc} 
\usepackage{booktabs}
\usepackage[english,russian]{babel} 

\DeclareMathOperator{\Lin}{\mathrm{Lin}} 
 \DeclareMathOperator{\Linp}{\Lin^{\perp}} 
 \DeclareMathOperator*\plim{plim}

 \DeclareMathOperator{\grad}{grad} 
 \DeclareMathOperator{\card}{card} 
 \DeclareMathOperator{\sgn}{sign} 
 \DeclareMathOperator{\sign}{sign} 
 \DeclareMathOperator*{\argmin}{arg\,min} 
 \DeclareMathOperator*{\argmax}{arg\,max} 
 \DeclareMathOperator*{\amn}{arg\,min} 
 \DeclareMathOperator*{\amx}{arg\,max} 
 \DeclareMathOperator{\cov}{Cov} 

\DeclareMathOperator{\Var}{Var} 
 \DeclareMathOperator{\Cov}{Cov} 
 \DeclareMathOperator{\Corr}{Corr} 
 \DeclareMathOperator{\E}{\mathbb{E}} 
 \let\P\relax 

\DeclareMathOperator{\P}{\mathbb{P}} 
 \newcommand{\cN}{\mathcal{N}} 
 \def \R{\mathbb{R}} 
 \def \N{\mathbb{N}} 
 \def \Z{\mathbb{Z}} 

\title{Midterm 2014} 
 \subtitle{Теория вероятностей и математическая статистика} 
 \author{Обратная связь: \url{https://github.com/bdemeshev/probability_hse_exams}} 
 \date{Последнее обновление: \today}
\begin{document} 

\frame[plain]{\titlepage}

 \begin{frame} \label{1} 
\begin{block}{1} 

Случайным образом выбирается семья с двумя детьми. Событие $A$ — в семье старший ребенок — мальчик,  событие $B$ — в семье только один из детей — мальчик, событие $C$ — в семье хотя бы один из детей — мальчик.  Вероятность $\P(C)$ равна
 


 \end{block} 
\begin{enumerate} 
\item[] \hyperlink{1-Yes}{\beamergotobutton{} $3/4$}
\item[] \hyperlink{1-No}{\beamergotobutton{} $1/4$}
\item[] \hyperlink{1-No}{\beamergotobutton{} $1/2$}
\item[] \hyperlink{1-No}{\beamergotobutton{} $2/3$
}
\item[] \hyperlink{1-No}{\beamergotobutton{} $1$}
\end{enumerate} 
\end{frame} 


 \begin{frame} \label{2} 
\begin{block}{2} 

Случайным образом выбирается семья с двумя детьми. Событие $A$ — в семье старший ребенок — мальчик,  событие $B$ — в семье только один из детей — мальчик, событие $C$ — в семье хотя бы один из детей — мальчик.  Вероятность $\P(A \cup C)$ равна
    


 \end{block} 
\begin{enumerate} 
\item[] \hyperlink{2-No}{\beamergotobutton{} $2/3$}
\item[] \hyperlink{2-No}{\beamergotobutton{} $1/2$
}
\item[] \hyperlink{2-No}{\beamergotobutton{} $1$}
\item[] \hyperlink{2-Yes}{\beamergotobutton{} $3/4$}
\item[] \hyperlink{2-No}{\beamergotobutton{} $3/8$}
\end{enumerate} 
\end{frame} 


 \begin{frame} \label{3} 
\begin{block}{3} 

Случайным образом выбирается семья с двумя детьми. Событие $A$ — в семье старший ребенок — мальчик,  событие $B$ — в семье только один из детей — мальчик, событие $C$ — в семье хотя бы один из детей — мальчик.  Вероятность $\P(A | C)$ равна
   


 \end{block} 
\begin{enumerate} 
\item[] \hyperlink{3-No}{\beamergotobutton{} $1/4$}
\item[] \hyperlink{3-No}{\beamergotobutton{} $3/4$
}
\item[] \hyperlink{3-Yes}{\beamergotobutton{} $2/3$}
\item[] \hyperlink{3-No}{\beamergotobutton{} $1$}
\item[] \hyperlink{3-No}{\beamergotobutton{} $1/2$}
\end{enumerate} 
\end{frame} 


 \begin{frame} \label{4} 
\begin{block}{4} 

Случайным образом выбирается семья с двумя детьми. Событие $A$ — в семье старший ребенок — мальчик,  событие $B$ — в семье только один из детей — мальчик, событие $C$ — в семье хотя бы один из детей — мальчик.


 \end{block} 
\begin{enumerate} 
\item[] \hyperlink{4-Yes}{\beamergotobutton{} $A$ и $B$ — независимы, $A$ и $C$ — зависимы, $B$ и $C$ — зависимы}
\item[] \hyperlink{4-No}{\beamergotobutton{} Любые два события из $A$, $B$, $C$ — зависимы}
\item[] \hyperlink{4-No}{\beamergotobutton{} $\P(A\cap B\cap C)=\P(A)\P(B)\P(C)$
}
\item[] \hyperlink{4-No}{\beamergotobutton{} События $A$, $B$, $C$ — независимы в совокупности}
\item[] \hyperlink{4-No}{\beamergotobutton{} События $A$, $B$, $C$ — независимы попарно, но зависимы в совокупности}
\end{enumerate} 
\end{frame} 


 \begin{frame} \label{5} 
\begin{block}{5} 

Имеется три монетки. Две «правильных» и одна — с «орлами» по обеим сторонам. Вася выбирает одну монетку наугад и подкидывает ее один раз. Вероятность того, что выпадет орел равна
     


 \end{block} 
\begin{enumerate} 
\item[] \hyperlink{5-No}{\beamergotobutton{} $1/3$}
\item[] \hyperlink{5-No}{\beamergotobutton{} $1/2$}
\item[] \hyperlink{5-No}{\beamergotobutton{} $3/5$}
\item[] \hyperlink{5-Yes}{\beamergotobutton{} $2/3$}
\item[] \hyperlink{5-No}{\beamergotobutton{} $2/5$
}
\end{enumerate} 
\end{frame} 


 \begin{frame} \label{6} 
\begin{block}{6} 

Имеется три монетки. Две «правильных» и одна — с «орлами» по обеим сторонам. Вася выбирает одну монетку наугад и подкидывает ее один раз. Вероятность того, что была выбрана неправильная монетка, если выпал орел, равна
   


 \end{block} 
\begin{enumerate} 
\item[] \hyperlink{6-No}{\beamergotobutton{} $2/3$}
\item[] \hyperlink{6-No}{\beamergotobutton{} $3/5$}
\item[] \hyperlink{6-No}{\beamergotobutton{} $1/3$}
\item[] \hyperlink{6-Yes}{\beamergotobutton{} $1/2$}
\item[] \hyperlink{6-No}{\beamergotobutton{} $3/2$
}
\end{enumerate} 
\end{frame} 


 \begin{frame} \label{7} 
\begin{block}{7} 

Вася бросает 7 правильных игральных кубиков. Наиболее вероятное количество выпавших шестёрок равно
     


 \end{block} 
\begin{enumerate} 
\item[] \hyperlink{7-No}{\beamergotobutton{} $0$}
\item[] \hyperlink{7-No}{\beamergotobutton{} $2$
}
\item[] \hyperlink{7-Yes}{\beamergotobutton{} $1$}
\item[] \hyperlink{7-No}{\beamergotobutton{} $7/6$}
\item[] \hyperlink{7-No}{\beamergotobutton{} $6/7$}
\end{enumerate} 
\end{frame} 


 \begin{frame} \label{8} 
\begin{block}{8} 

Вася бросает 7 правильных игральных кубиков. Вероятность того, что ровно на пяти из кубиков выпадет шестёрка равна
     


 \end{block} 
\begin{enumerate} 
\item[] \hyperlink{8-No}{\beamergotobutton{} $\left(\frac{1}{6}\right)^7$}
\item[] \hyperlink{8-No}{\beamergotobutton{} $\left(\frac{1}{6}\right)^5$}
\item[] \hyperlink{8-No}{\beamergotobutton{} $\frac{525}{12}\left(\frac{1}{6}\right)^7$}
\item[] \hyperlink{8-No}{\beamergotobutton{} $\frac{7}{12}\left(\frac{1}{6}\right)^5$}
\item[] \hyperlink{8-Yes}{\beamergotobutton{} $525\left(\frac{1}{6}\right)^7$}
\end{enumerate} 
\end{frame} 


 \begin{frame} \label{9} 
\begin{block}{9} 

Вася бросает 7 правильных игральных кубиков. Математическое ожидание суммы выпавших очков равно
     


 \end{block} 
\begin{enumerate} 
\item[] \hyperlink{9-No}{\beamergotobutton{} $7/6$}
\item[] \hyperlink{9-No}{\beamergotobutton{} $42$
}
\item[] \hyperlink{9-No}{\beamergotobutton{} $21$}
\item[] \hyperlink{9-No}{\beamergotobutton{} $30$}
\item[] \hyperlink{9-Yes}{\beamergotobutton{} $24.5$}
\end{enumerate} 
\end{frame} 


 \begin{frame} \label{10} 
\begin{block}{10} 

Вася бросает 7 правильных игральных кубиков. Дисперсия суммы выпавших очков равна
 


 \end{block} 
\begin{enumerate} 
\item[] \hyperlink{10-No}{\beamergotobutton{} $35/36$}
\item[] \hyperlink{10-No}{\beamergotobutton{} $7/6$}
\item[] \hyperlink{10-Yes}{\beamergotobutton{} $7\cdot\frac{35}{12}$}
\item[] \hyperlink{10-No}{\beamergotobutton{} $7\cdot \frac{35}{36}$}
\item[] \hyperlink{10-No}{\beamergotobutton{} $7$}
\end{enumerate} 
\end{frame} 


 \begin{frame} \label{11} 
\begin{block}{11} 

Вася бросает 7 правильных игральных кубиков. Пусть величина  $X$ — сумма очков, выпавших на первых двух кубиках, а величина  $Y$ — сумма очков, выпавших на следующих пяти кубиках. Ковариация $\Cov(X,Y)$ равна
 


 \end{block} 
\begin{enumerate} 
\item[] \hyperlink{11-Yes}{\beamergotobutton{} $0$}
\item[] \hyperlink{11-No}{\beamergotobutton{} $-2/5$
}
\item[] \hyperlink{11-No}{\beamergotobutton{} $0.5$}
\item[] \hyperlink{11-No}{\beamergotobutton{} $1$}
\item[] \hyperlink{11-No}{\beamergotobutton{} $2/5$}
\end{enumerate} 
\end{frame} 


 \begin{frame} \label{12} 
\begin{block}{12} 

Число изюминок в булочке — случайная величина, имеющая распределение Пуассона. Известно, что в среднем каждая булочка содержит 13 изюминок. Вероятность того, что в случайно выбранной булочке окажется только одна изюминка равна:
 


 \end{block} 
\begin{enumerate} 
\item[] \hyperlink{12-No}{\beamergotobutton{} $1/13$}
\item[] \hyperlink{12-Yes}{\beamergotobutton{} $13e^{-13}$}
\item[] \hyperlink{12-No}{\beamergotobutton{} $e^{-13}/13$}
\item[] \hyperlink{12-No}{\beamergotobutton{} $e^{13}/13!$}
\item[] \hyperlink{12-No}{\beamergotobutton{} $e^{-13}$}
\end{enumerate} 
\end{frame} 


 \begin{frame} \label{13} 
\begin{block}{13} 

Совместное распределение пары величин $X$ и $Y$ задано таблицей:
\begin{center}
\begin{tabular}{@{}c|ccc@{}}
\toprule
       & $Y=-1$ & $Y=0$ & $Y=1$ \\ \midrule
$X=-1$ & $1/4$  & $0$   & $1/4$ \\
$X=1$  & $1/6$  & $1/6$ & $1/6$ \\ \bottomrule
\end{tabular}
\end{center}

\vspace{0.5cm} 
 
 
Математическое ожидание случайной величины $X$ при условии, что $Y=-1$ равно
 


 \end{block} 
\begin{enumerate} 
\item[] \hyperlink{13-Yes}{\beamergotobutton{} $-1/5$}
\item[] \hyperlink{13-No}{\beamergotobutton{} $-1/3$}
\item[] \hyperlink{13-No}{\beamergotobutton{} $1/10$
}
\item[] \hyperlink{13-No}{\beamergotobutton{} $-1/12$}
\item[] \hyperlink{13-No}{\beamergotobutton{} $0$}
\end{enumerate} 
\end{frame} 


 \begin{frame} \label{14} 
\begin{block}{14} 

Совместное распределение пары величин $X$ и $Y$ задано таблицей:
\begin{center}
\begin{tabular}{@{}c|ccc@{}}
\toprule
       & $Y=-1$ & $Y=0$ & $Y=1$ \\ \midrule
$X=-1$ & $1/4$  & $0$   & $1/4$ \\
$X=1$  & $1/6$  & $1/6$ & $1/6$ \\ \bottomrule
\end{tabular}
\end{center}
\vspace{0.5cm} 
 
 
Вероятность того, что $X=1$ при условии, что $Y<0$ равна
 


 \end{block} 
\begin{enumerate} 
\item[] \hyperlink{14-No}{\beamergotobutton{} $1/12$}
\item[] \hyperlink{14-Yes}{\beamergotobutton{} $2/5$}
\item[] \hyperlink{14-No}{\beamergotobutton{} $1/6$}
\item[] \hyperlink{14-No}{\beamergotobutton{} $1/3$
}
\item[] \hyperlink{14-No}{\beamergotobutton{} $5/12$}
\end{enumerate} 
\end{frame} 


 \begin{frame} \label{15} 
\begin{block}{15} 

Совместное распределение пары величин $X$ и $Y$ задано таблицей:
\begin{center}
\begin{tabular}{@{}c|ccc@{}}
\toprule
       & $Y=-1$ & $Y=0$ & $Y=1$ \\ \midrule
$X=-1$ & $1/4$  & $0$   & $1/4$ \\
$X=1$  & $1/6$  & $1/6$ & $1/6$ \\ \bottomrule
\end{tabular}
\end{center}
\vspace{0.5cm} 
 
 
Дисперсия случайной величины $Y$  равна
 


 \end{block} 
\begin{enumerate} 
\item[] \hyperlink{15-No}{\beamergotobutton{} $1/3$}
\item[] \hyperlink{15-No}{\beamergotobutton{} $5/12$}
\item[] \hyperlink{15-No}{\beamergotobutton{} $1/2$}
\item[] \hyperlink{15-Yes}{\beamergotobutton{} $5/6$}
\item[] \hyperlink{15-No}{\beamergotobutton{} $12/5$
}
\end{enumerate} 
\end{frame} 


 \begin{frame} \label{16} 
\begin{block}{16} 

Совместное распределение пары величин $X$ и $Y$ задано таблицей:

\begin{center}
\begin{tabular}{@{}c|ccc@{}}
\toprule
       & $Y=-1$ & $Y=0$ & $Y=1$ \\ \midrule
$X=-1$ & $1/4$  & $0$   & $1/4$ \\
$X=1$  & $1/6$  & $1/6$ & $1/6$ \\ \bottomrule
\end{tabular}
\end{center}

\vspace{0.5cm} 
 
 
Ковариация, $\Cov(X,Y)$, равна
 


 \end{block} 
\begin{enumerate} 
\item[] \hyperlink{16-Yes}{\beamergotobutton{} $0$}
\item[] \hyperlink{16-No}{\beamergotobutton{} $0.5$}
\item[] \hyperlink{16-No}{\beamergotobutton{} $1$}
\item[] \hyperlink{16-No}{\beamergotobutton{} $-1$
}
\item[] \hyperlink{16-No}{\beamergotobutton{} $-0.5$}
\end{enumerate} 
\end{frame} 


 \begin{frame} \label{17} 
\begin{block}{17} 

Функция распределения случайной величины $X$ имеет вид
\[
F(x)=\begin{cases}
0, \; \text{ если } x<0 \\
cx^2, \; \text{ если } x\in [0;1] \\
1, \; \text{ если } x>1
\end{cases}
\]

\vspace{0.5cm} 
 
 
Константа $c$ равна
 


 \end{block} 
\begin{enumerate} 
\item[] \hyperlink{17-Yes}{\beamergotobutton{} $1$}
\item[] \hyperlink{17-No}{\beamergotobutton{} $0.5$}
\item[] \hyperlink{17-No}{\beamergotobutton{} $2/3$
}
\item[] \hyperlink{17-No}{\beamergotobutton{} $2$}
\item[] \hyperlink{17-No}{\beamergotobutton{} $1.5$}
\end{enumerate} 
\end{frame} 


 \begin{frame} \label{18} 
\begin{block}{18} 

Функция распределения случайной величины $X$ имеет вид
\[
F(x)=\begin{cases}
0, \; \text{ если } x<0 \\
cx^2, \; \text{ если } x\in [0;1] \\
1, \; \text{ если } x>1
\end{cases}
\]

\vspace{0.5cm} 
 
 
Вероятность того, что величина $X$ примет значение из интервала  $[0.5, 1.5]$ равна
 


 \end{block} 
\begin{enumerate} 
\item[] \hyperlink{18-No}{\beamergotobutton{} $1/2$}
\item[] \hyperlink{18-No}{\beamergotobutton{} $3/2$
}
\item[] \hyperlink{18-No}{\beamergotobutton{} $1$}
\item[] \hyperlink{18-No}{\beamergotobutton{} $2/3$}
\item[] \hyperlink{18-Yes}{\beamergotobutton{} $3/4$}
\end{enumerate} 
\end{frame} 


 \begin{frame} \label{19} 
\begin{block}{19} 

Функция распределения случайной величины $X$ имеет вид
\[
F(x)=\begin{cases}
0, \; \text{ если } x<0 \\
cx^2, \; \text{ если } x\in [0;1] \\
1, \; \text{ если } x>1
\end{cases}
\]

\vspace{0.5cm} 
 
 
Математическое ожидание $\E(X)$ равно
 


 \end{block} 
\begin{enumerate} 
\item[] \hyperlink{19-Yes}{\beamergotobutton{} $2/3$}
\item[] \hyperlink{19-No}{\beamergotobutton{} $3/4$}
\item[] \hyperlink{19-No}{\beamergotobutton{} $2$
}
\item[] \hyperlink{19-No}{\beamergotobutton{} $1/2$}
\item[] \hyperlink{19-No}{\beamergotobutton{} $1/4$}
\end{enumerate} 
\end{frame} 


 \begin{frame} \label{20} 
\begin{block}{20} 

Совместная функция плотности пары $X$ и $Y$ имеет вид
\[
f(x,y)=\begin{cases}
cx^2y^2, \; \text{ если } x\in[0;1], y\in [0;1] \\
0, \; \text{ иначе}
\end{cases}
\]

\vspace{0.5cm} 
 
 
Константа $c$ равна
 


 \end{block} 
\begin{enumerate} 
\item[] \hyperlink{20-No}{\beamergotobutton{} $2$
}
\item[] \hyperlink{20-No}{\beamergotobutton{} $1$}
\item[] \hyperlink{20-No}{\beamergotobutton{} $1/4$}
\item[] \hyperlink{20-Yes}{\beamergotobutton{} $9$}
\item[] \hyperlink{20-No}{\beamergotobutton{} $1/2$}
\end{enumerate} 
\end{frame} 


 \begin{frame} \label{21} 
\begin{block}{21} 

Совместная функция плотности пары $X$ и $Y$ имеет вид
\[
f(x,y)=\begin{cases}
cx^2y^2, \; \text{ если } x\in[0;1], y\in [0;1] \\
0, \; \text{ иначе}
\end{cases}
\]

\vspace{0.5cm} 
 
 
Вероятность $\P(X<0.5, Y<0.5)$ равна
 


 \end{block} 
\begin{enumerate} 
\item[] \hyperlink{21-No}{\beamergotobutton{} $9/16$
}
\item[] \hyperlink{21-Yes}{\beamergotobutton{} $1/64$}
\item[] \hyperlink{21-No}{\beamergotobutton{} $1/16$}
\item[] \hyperlink{21-No}{\beamergotobutton{} $1/8$}
\item[] \hyperlink{21-No}{\beamergotobutton{} $1/4$}
\end{enumerate} 
\end{frame} 


 \begin{frame} \label{22} 
\begin{block}{22} 

Совместная функция плотности пары $X$ и $Y$ имеет вид
\[
f(x,y)=\begin{cases}
cx^2y^2, \; \text{ если } x\in[0;1], y\in [0;1] \\
0, \; \text{ иначе}
\end{cases}
\]
 
Условная функция плотности  $f_{X|Y=2}(x)$ равна

 \end{block} 

\begin{enumerate} 
\item[] \hyperlink{22-No}{\beamergotobutton{} $f_X|Y=2(x)=\begin{cases} 3x^2,\, \text{ если }  x\in [0;1] \\ 0, \text{ иначе}     \end{cases}$}
\item[] \hyperlink{22-No}{\beamergotobutton{} $f_X|Y=2(x)=\begin{cases} 36x^2,\, \text{ если }  x\in [0;1] \\ 0, \text{ иначе}     \end{cases}$}
\item[] \hyperlink{22-No}{\beamergotobutton{} $f_X|Y=2(x)=\begin{cases} 9x^2,\, \text{ если }  x\in [0;1] \\ 0, \text{ иначе}     \end{cases}$}
\item[] \hyperlink{22-Yes}{\beamergotobutton{} не определена}
\item[] \hyperlink{22-No}{\beamergotobutton{} $f_X|Y=2(x)=\begin{cases} x^2,\, \text{ если }  x\in [0;1] \\ 0, \text{ иначе}     \end{cases}$}
\end{enumerate} 
\end{frame} 


 \begin{frame} \label{23} 
\begin{block}{23} 

Совместная функция плотности пары $X$ и $Y$ имеет вид
\[
f(x,y)=\begin{cases}
cx^2y^2, \; \text{ если } x\in[0;1], y\in [0;1] \\
0, \; \text{ иначе}
\end{cases}
\]

\vspace{0.5cm} 
 
 
Математическое ожидание $\E(X/Y)$ равно
 


 \end{block} 
\begin{enumerate} 
\item[] \hyperlink{23-No}{\beamergotobutton{} $1/2$}
\item[] \hyperlink{23-No}{\beamergotobutton{} $2$
}
\item[] \hyperlink{23-Yes}{\beamergotobutton{} $9/8$}
\item[] \hyperlink{23-No}{\beamergotobutton{} $3$}
\item[] \hyperlink{23-No}{\beamergotobutton{} $1$}
\end{enumerate} 
\end{frame} 


 \begin{frame} \label{24} 
\begin{block}{24} 

Известно, что $\E(X)=1$, $\Var(X)=1$, $\E(Y)=4$, $\Var(Y)=9$, $\Cov(X,Y)=-3$

\vspace{0.5cm} 
 
 
Ковариация $\Cov(2X-Y,X+3Y)$ равна
 


 \end{block} 
\begin{enumerate} 
\item[] \hyperlink{24-No}{\beamergotobutton{} $40$}
\item[] \hyperlink{24-No}{\beamergotobutton{} $22$}
\item[] \hyperlink{24-No}{\beamergotobutton{} $18$
}
\item[] \hyperlink{24-No}{\beamergotobutton{} $-18$}
\item[] \hyperlink{24-Yes}{\beamergotobutton{} $-40$}
\end{enumerate} 
\end{frame} 


 \begin{frame} \label{25} 
\begin{block}{25} 

Известно, что $\E(X)=1$, $\Var(X)=1$, $\E(Y)=4$, $\Var(Y)=9$, $\Cov(X,Y)=-3$

\vspace{0.5cm} 
 
 
Корреляция $\Corr(2X+3,4Y-5)$ равна
 


 \end{block} 
\begin{enumerate} 
\item[] \hyperlink{25-No}{\beamergotobutton{} $-1/8$}
\item[] \hyperlink{25-No}{\beamergotobutton{} $1$}
\item[] \hyperlink{25-No}{\beamergotobutton{} $1/6$
}
\item[] \hyperlink{25-No}{\beamergotobutton{} $1/3$}
\item[] \hyperlink{25-Yes}{\beamergotobutton{} $-1$}
\end{enumerate} 
\end{frame} 


 \begin{frame} \label{26} 
\begin{block}{26} 

Пусть случайные величины $X$ и $Y$ — независимы, тогда \textbf{НЕ ВЕРНЫМ} является утверждение
 


 \end{block} 
\begin{enumerate} 
\item[] \hyperlink{26-No}{\beamergotobutton{} $\P(X<a, Y<b)=\P(X<a)\P(Y<b)$}
\item[] \hyperlink{26-No}{\beamergotobutton{} $\Cov(X,Y) = 0$}
\item[] \hyperlink{26-Yes}{\beamergotobutton{} $\Var(X-Y)<\Var(X)+\Var(Y)$ }
\item[] \hyperlink{26-No}{\beamergotobutton{} $\E(XY)=\E(X)\E(Y)$}
\item[] \hyperlink{26-No}{\beamergotobutton{} $\E(X|Y)=\E(X)$}
\item[] \hyperlink{26-No}{\beamergotobutton{} $\P(X<a | Y<b)=\P(X<a)$}
\end{enumerate} 
\end{frame} 


 \begin{frame} \label{27} 
\begin{block}{27} 

Если $\E(X)=0$, то, согласно неравенству Чебышева, $\P(|X| \leq 5 \sqrt{\Var(X)})$ лежит в интервале
 


 \end{block} 
\begin{enumerate} 
\item[] \hyperlink{27-Yes}{\beamergotobutton{} $[0.96;1]$ }
\item[] \hyperlink{27-No}{\beamergotobutton{} $[0;0.04]$}
\item[] \hyperlink{27-No}{\beamergotobutton{} $[0.8;1]$}
\item[] \hyperlink{27-No}{\beamergotobutton{} $[0;0.2]$}
\item[] \hyperlink{27-No}{\beamergotobutton{} $[0.5;1]$
}
\end{enumerate} 
\end{frame} 


 \begin{frame} \label{28} 
\begin{block}{28} 

Пусть $X_1$, $X_2$, \ldots, $X_n$ — последовательность независимых одинаково распределенных случайных величин, $\E(X_i)=3$ и $\Var(X_i)=9$. Следующая величина имеет асимптотически стандартное нормальное распределение
 


 \end{block} 
\begin{enumerate} 
\item[] \hyperlink{28-No}{\beamergotobutton{} $\frac{\bar{X}_n-3}{3}$}
\item[] \hyperlink{28-Yes}{\beamergotobutton{} $\sqrt{n}\frac{\bar{X}-3}{3}$}
\item[] \hyperlink{28-No}{\beamergotobutton{} $\frac{\bar{X}_n-3}{3\sqrt{n}}$}
\item[] \hyperlink{28-No}{\beamergotobutton{} $\frac{X_n-3}{3}$ }
\item[] \hyperlink{28-No}{\beamergotobutton{} $\sqrt{n}(\bar{X}-3)$ }
\end{enumerate} 
\end{frame} 


 \begin{frame} \label{29} 
\begin{block}{29} 

Случайная величина $X$ имеет функцию плотности $f(x)=\frac{1}{3\sqrt{2\pi}} \exp\left(-\frac{(x-1)^2}{18} \right)$. Следующее утверждение \textbf{НЕ ВЕРНО}
 


 \end{block} 
\begin{enumerate} 
\item[] \hyperlink{29-No}{\beamergotobutton{} $\E(X)=1$}
\item[] \hyperlink{29-No}{\beamergotobutton{} $\Var(X)=9$}
\item[] \hyperlink{29-No}{\beamergotobutton{} $\P(X>1)=0.5$}
\item[] \hyperlink{29-No}{\beamergotobutton{} $\P(X=0)=0$}
\item[] \hyperlink{29-Yes}{\beamergotobutton{} Случайная величина $X$ дискретна}
\end{enumerate} 
\end{frame} 


 \begin{frame} \label{30} 
\begin{block}{30} 

Пусть $X_1$, $X_2$, \ldots, $X_n$ — последовательность независимых одинаково распределенных случайных величин, $\E(X_i)=\mu$ и $\Var(X_i)=\sigma^2$. Следующее утверждение в общем случае \textbf{НЕ ВЕРНО}:
 


 \end{block} 
\begin{enumerate} 
\item[] \hyperlink{30-Yes}{\beamergotobutton{} $\frac{X_n-\mu}{\sigma} \overset{F}{\to} N(0;1)$ при $n\to\infty$}
\item[] \hyperlink{30-No}{\beamergotobutton{} $\lim_{n\to\infty} \Var(\bar{X}_n)=0$}
\item[] \hyperlink{30-No}{\beamergotobutton{} $\bar{X}_n \overset{P}{\to} \mu$ при $n\to\infty$}
\item[] \hyperlink{30-No}{\beamergotobutton{}$\frac{\bar{X}_n-\mu}{\sigma /\sqrt{n}} \overset{F}{\to } N(0,1) $ при $n\to\infty$}
\item[] \hyperlink{30-No}{\beamergotobutton{} $\frac{\bar{X}_n-\mu}{\sqrt{n} \sigma } \overset{P}{\to } 0 $ при $n\to\infty$}
\end{enumerate} 
\end{frame} 


 \begin{frame} \label{1-Yes} 
\begin{block}{1} 

Случайным образом выбирается семья с двумя детьми. Событие $A$ — в семье старший ребенок — мальчик,  событие $B$ — в семье только один из детей — мальчик, событие $C$ — в семье хотя бы один из детей — мальчик.  Вероятность $\P(C)$ равна
 


 \end{block} 
\begin{enumerate} 
\item[] \hyperlink{1-Yes}{\beamergotobutton{} $3/4$}
\item[] \hyperlink{1-No}{\beamergotobutton{} $1/4$}
\item[] \hyperlink{1-No}{\beamergotobutton{} $1/2$}
\item[] \hyperlink{1-No}{\beamergotobutton{} $2/3$
}
\item[] \hyperlink{1-No}{\beamergotobutton{} $1$}
\end{enumerate} 

 \textbf{Да!} 
 \hyperlink{2}{\beamerbutton{Следующий вопрос}}\end{frame} 


 \begin{frame} \label{2-Yes} 
\begin{block}{2} 

Случайным образом выбирается семья с двумя детьми. Событие $A$ — в семье старший ребенок — мальчик,  событие $B$ — в семье только один из детей — мальчик, событие $C$ — в семье хотя бы один из детей — мальчик.  Вероятность $\P(A \cup C)$ равна
    


 \end{block} 
\begin{enumerate} 
\item[] \hyperlink{2-No}{\beamergotobutton{} $2/3$}
\item[] \hyperlink{2-No}{\beamergotobutton{} $1/2$
}
\item[] \hyperlink{2-No}{\beamergotobutton{} $1$}
\item[] \hyperlink{2-Yes}{\beamergotobutton{} $3/4$}
\item[] \hyperlink{2-No}{\beamergotobutton{} $3/8$}
\end{enumerate} 

 \textbf{Да!} 
 \hyperlink{3}{\beamerbutton{Следующий вопрос}}\end{frame} 


 \begin{frame} \label{3-Yes} 
\begin{block}{3} 

Случайным образом выбирается семья с двумя детьми. Событие $A$ — в семье старший ребенок — мальчик,  событие $B$ — в семье только один из детей — мальчик, событие $C$ — в семье хотя бы один из детей — мальчик.  Вероятность $\P(A | C)$ равна
   


 \end{block} 
\begin{enumerate} 
\item[] \hyperlink{3-No}{\beamergotobutton{} $1/4$}
\item[] \hyperlink{3-No}{\beamergotobutton{} $3/4$
}
\item[] \hyperlink{3-Yes}{\beamergotobutton{} $2/3$}
\item[] \hyperlink{3-No}{\beamergotobutton{} $1$}
\item[] \hyperlink{3-No}{\beamergotobutton{} $1/2$}
\end{enumerate} 

 \textbf{Да!} 
 \hyperlink{4}{\beamerbutton{Следующий вопрос}}\end{frame} 


 \begin{frame} \label{4-Yes} 
\begin{block}{4} 

Случайным образом выбирается семья с двумя детьми. Событие $A$ — в семье старший ребенок — мальчик,  событие $B$ — в семье только один из детей — мальчик, событие $C$ — в семье хотя бы один из детей — мальчик.


 \end{block} 
\begin{enumerate} 
\item[] \hyperlink{4-Yes}{\beamergotobutton{} $A$ и $B$ — независимы, $A$ и $C$ — зависимы, $B$ и $C$ — зависимы}
\item[] \hyperlink{4-No}{\beamergotobutton{} Любые два события из $A$, $B$, $C$ — зависимы}
\item[] \hyperlink{4-No}{\beamergotobutton{} $\P(A\cap B\cap C)=\P(A)\P(B)\P(C)$
}
\item[] \hyperlink{4-No}{\beamergotobutton{} События $A$, $B$, $C$ — независимы в совокупности}
\item[] \hyperlink{4-No}{\beamergotobutton{} События $A$, $B$, $C$ — независимы попарно, но зависимы в совокупности}
\end{enumerate} 

 \textbf{Да!} 
 \hyperlink{5}{\beamerbutton{Следующий вопрос}}\end{frame} 


 \begin{frame} \label{5-Yes} 
\begin{block}{5} 

Имеется три монетки. Две «правильных» и одна — с «орлами» по обеим сторонам. Вася выбирает одну монетку наугад и подкидывает ее один раз. Вероятность того, что выпадет орел равна
     


 \end{block} 
\begin{enumerate} 
\item[] \hyperlink{5-No}{\beamergotobutton{} $1/3$}
\item[] \hyperlink{5-No}{\beamergotobutton{} $1/2$}
\item[] \hyperlink{5-No}{\beamergotobutton{} $3/5$}
\item[] \hyperlink{5-Yes}{\beamergotobutton{} $2/3$}
\item[] \hyperlink{5-No}{\beamergotobutton{} $2/5$
}
\end{enumerate} 

 \textbf{Да!} 
 \hyperlink{6}{\beamerbutton{Следующий вопрос}}\end{frame} 


 \begin{frame} \label{6-Yes} 
\begin{block}{6} 

Имеется три монетки. Две «правильных» и одна — с «орлами» по обеим сторонам. Вася выбирает одну монетку наугад и подкидывает ее один раз. Вероятность того, что была выбрана неправильная монетка, если выпал орел, равна
   


 \end{block} 
\begin{enumerate} 
\item[] \hyperlink{6-No}{\beamergotobutton{} $2/3$}
\item[] \hyperlink{6-No}{\beamergotobutton{} $3/5$}
\item[] \hyperlink{6-No}{\beamergotobutton{} $1/3$}
\item[] \hyperlink{6-Yes}{\beamergotobutton{} $1/2$}
\item[] \hyperlink{6-No}{\beamergotobutton{} $3/2$
}
\end{enumerate} 

 \textbf{Да!} 
 \hyperlink{7}{\beamerbutton{Следующий вопрос}}\end{frame} 


 \begin{frame} \label{7-Yes} 
\begin{block}{7} 

Вася бросает 7 правильных игральных кубиков. Наиболее вероятное количество выпавших шестёрок равно
     


 \end{block} 
\begin{enumerate} 
\item[] \hyperlink{7-No}{\beamergotobutton{} $0$}
\item[] \hyperlink{7-No}{\beamergotobutton{} $2$
}
\item[] \hyperlink{7-Yes}{\beamergotobutton{} $1$}
\item[] \hyperlink{7-No}{\beamergotobutton{} $7/6$}
\item[] \hyperlink{7-No}{\beamergotobutton{} $6/7$}
\end{enumerate} 

 \textbf{Да!} 
 \hyperlink{8}{\beamerbutton{Следующий вопрос}}\end{frame} 


 \begin{frame} \label{8-Yes} 
\begin{block}{8} 

Вася бросает 7 правильных игральных кубиков. Вероятность того, что ровно на пяти из кубиков выпадет шестёрка равна
     


 \end{block} 
\begin{enumerate} 
\item[] \hyperlink{8-No}{\beamergotobutton{} $\left(\frac{1}{6}\right)^7$}
\item[] \hyperlink{8-No}{\beamergotobutton{} $\left(\frac{1}{6}\right)^5$}
\item[] \hyperlink{8-No}{\beamergotobutton{} $\frac{525}{12}\left(\frac{1}{6}\right)^7$}
\item[] \hyperlink{8-No}{\beamergotobutton{} $\frac{7}{12}\left(\frac{1}{6}\right)^5$}
\item[] \hyperlink{8-Yes}{\beamergotobutton{} $525\left(\frac{1}{6}\right)^7$}
\end{enumerate} 

 \textbf{Да!} 
 \hyperlink{9}{\beamerbutton{Следующий вопрос}}\end{frame} 


 \begin{frame} \label{9-Yes} 
\begin{block}{9} 

Вася бросает 7 правильных игральных кубиков. Математическое ожидание суммы выпавших очков равно
     


 \end{block} 
\begin{enumerate} 
\item[] \hyperlink{9-No}{\beamergotobutton{} $7/6$}
\item[] \hyperlink{9-No}{\beamergotobutton{} $42$
}
\item[] \hyperlink{9-No}{\beamergotobutton{} $21$}
\item[] \hyperlink{9-No}{\beamergotobutton{} $30$}
\item[] \hyperlink{9-Yes}{\beamergotobutton{} $24.5$}
\end{enumerate} 

 \textbf{Да!} 
 \hyperlink{10}{\beamerbutton{Следующий вопрос}}\end{frame} 


 \begin{frame} \label{10-Yes} 
\begin{block}{10} 

Вася бросает 7 правильных игральных кубиков. Дисперсия суммы выпавших очков равна
 


 \end{block} 
\begin{enumerate} 
\item[] \hyperlink{10-No}{\beamergotobutton{} $35/36$}
\item[] \hyperlink{10-No}{\beamergotobutton{} $7/6$}
\item[] \hyperlink{10-Yes}{\beamergotobutton{} $7\cdot\frac{35}{12}$}
\item[] \hyperlink{10-No}{\beamergotobutton{} $7\cdot \frac{35}{36}$}
\item[] \hyperlink{10-No}{\beamergotobutton{} $7$}
\end{enumerate} 

 \textbf{Да!} 
 \hyperlink{11}{\beamerbutton{Следующий вопрос}}\end{frame} 


 \begin{frame} \label{11-Yes} 
\begin{block}{11} 

Вася бросает 7 правильных игральных кубиков. Пусть величина  $X$ — сумма очков, выпавших на первых двух кубиках, а величина  $Y$ — сумма очков, выпавших на следующих пяти кубиках. Ковариация $\Cov(X,Y)$ равна
 


 \end{block} 
\begin{enumerate} 
\item[] \hyperlink{11-Yes}{\beamergotobutton{} $0$}
\item[] \hyperlink{11-No}{\beamergotobutton{} $-2/5$
}
\item[] \hyperlink{11-No}{\beamergotobutton{} $0.5$}
\item[] \hyperlink{11-No}{\beamergotobutton{} $1$}
\item[] \hyperlink{11-No}{\beamergotobutton{} $2/5$}
\end{enumerate} 

 \textbf{Да!} 
 \hyperlink{12}{\beamerbutton{Следующий вопрос}}\end{frame} 


 \begin{frame} \label{12-Yes} 
\begin{block}{12} 

Число изюминок в булочке — случайная величина, имеющая распределение Пуассона. Известно, что в среднем каждая булочка содержит 13 изюминок. Вероятность того, что в случайно выбранной булочке окажется только одна изюминка равна:
 


 \end{block} 
\begin{enumerate} 
\item[] \hyperlink{12-No}{\beamergotobutton{} $1/13$}
\item[] \hyperlink{12-Yes}{\beamergotobutton{} $13e^{-13}$}
\item[] \hyperlink{12-No}{\beamergotobutton{} $e^{-13}/13$}
\item[] \hyperlink{12-No}{\beamergotobutton{} $e^{13}/13!$}
\item[] \hyperlink{12-No}{\beamergotobutton{} $e^{-13}$}
\end{enumerate} 

 \textbf{Да!} 
 \hyperlink{13}{\beamerbutton{Следующий вопрос}}\end{frame} 


 \begin{frame} \label{13-Yes} 
\begin{block}{13} 

Совместное распределение пары величин $X$ и $Y$ задано таблицей:
\begin{center}
\begin{tabular}{@{}c|ccc@{}}
\toprule
       & $Y=-1$ & $Y=0$ & $Y=1$ \\ \midrule
$X=-1$ & $1/4$  & $0$   & $1/4$ \\
$X=1$  & $1/6$  & $1/6$ & $1/6$ \\ \bottomrule
\end{tabular}
\end{center}
\vspace{0.5cm} 
 
 
Математическое ожидание случайной величины $X$ при условии, что $Y=-1$ равно
 


 \end{block} 
\begin{enumerate} 
\item[] \hyperlink{13-Yes}{\beamergotobutton{} $-1/5$}
\item[] \hyperlink{13-No}{\beamergotobutton{} $-1/3$}
\item[] \hyperlink{13-No}{\beamergotobutton{} $1/10$
}
\item[] \hyperlink{13-No}{\beamergotobutton{} $-1/12$}
\item[] \hyperlink{13-No}{\beamergotobutton{} $0$}
\end{enumerate} 

 \textbf{Да!} 
 \hyperlink{14}{\beamerbutton{Следующий вопрос}}\end{frame} 


 \begin{frame} \label{14-Yes} 
\begin{block}{14} 

Совместное распределение пары величин $X$ и $Y$ задано таблицей:
\begin{center}
\begin{tabular}{@{}c|ccc@{}}
\toprule
       & $Y=-1$ & $Y=0$ & $Y=1$ \\ \midrule
$X=-1$ & $1/4$  & $0$   & $1/4$ \\
$X=1$  & $1/6$  & $1/6$ & $1/6$ \\ \bottomrule
\end{tabular}
\end{center}

\vspace{0.5cm} 
 
 
Вероятность того, что $X=1$ при условии, что $Y<0$ равна
 


 \end{block} 
\begin{enumerate} 
\item[] \hyperlink{14-No}{\beamergotobutton{} $1/12$}
\item[] \hyperlink{14-Yes}{\beamergotobutton{} $2/5$}
\item[] \hyperlink{14-No}{\beamergotobutton{} $1/6$}
\item[] \hyperlink{14-No}{\beamergotobutton{} $1/3$
}
\item[] \hyperlink{14-No}{\beamergotobutton{} $5/12$}
\end{enumerate} 

 \textbf{Да!} 
 \hyperlink{15}{\beamerbutton{Следующий вопрос}}\end{frame} 


 \begin{frame} \label{15-Yes} 
\begin{block}{15} 

Совместное распределение пары величин $X$ и $Y$ задано таблицей:

\begin{center}
	\begin{tabular}{@{}c|ccc@{}}
		\toprule
		& $Y=-1$ & $Y=0$ & $Y=1$ \\ \midrule
		$X=-1$ & $1/4$  & $0$   & $1/4$ \\
		$X=1$  & $1/6$  & $1/6$ & $1/6$ \\ \bottomrule
	\end{tabular}
\end{center}

\vspace{0.5cm} 
 
 
Дисперсия случайной величины $Y$  равна
 


 \end{block} 
\begin{enumerate} 
\item[] \hyperlink{15-No}{\beamergotobutton{} $1/3$}
\item[] \hyperlink{15-No}{\beamergotobutton{} $5/12$}
\item[] \hyperlink{15-No}{\beamergotobutton{} $1/2$}
\item[] \hyperlink{15-Yes}{\beamergotobutton{} $5/6$}
\item[] \hyperlink{15-No}{\beamergotobutton{} $12/5$
}
\end{enumerate} 

 \textbf{Да!} 
 \hyperlink{16}{\beamerbutton{Следующий вопрос}}\end{frame} 


 \begin{frame} \label{16-Yes} 
\begin{block}{16} 

Совместное распределение пары величин $X$ и $Y$ задано таблицей:

\begin{center}
	\begin{tabular}{@{}c|ccc@{}}
		\toprule
		& $Y=-1$ & $Y=0$ & $Y=1$ \\ \midrule
		$X=-1$ & $1/4$  & $0$   & $1/4$ \\
		$X=1$  & $1/6$  & $1/6$ & $1/6$ \\ \bottomrule
	\end{tabular}
\end{center}

\vspace{0.5cm} 
 
 
Ковариация, $\Cov(X,Y)$, равна
 


 \end{block} 
\begin{enumerate} 
\item[] \hyperlink{16-Yes}{\beamergotobutton{} $0$}
\item[] \hyperlink{16-No}{\beamergotobutton{} $0.5$}
\item[] \hyperlink{16-No}{\beamergotobutton{} $1$}
\item[] \hyperlink{16-No}{\beamergotobutton{} $-1$
}
\item[] \hyperlink{16-No}{\beamergotobutton{} $-0.5$}
\end{enumerate} 

 \textbf{Да!} 
 \hyperlink{17}{\beamerbutton{Следующий вопрос}}\end{frame} 


 \begin{frame} \label{17-Yes} 
\begin{block}{17} 

Функция распределения случайной величины $X$ имеет вид
\[
F(x)=\begin{cases}
0, \; \text{ если } x<0 \\
cx^2, \; \text{ если } x\in [0;1] \\
1, \; \text{ если } x>1
\end{cases}
\]

\vspace{0.5cm} 
 
 
Константа $c$ равна
 


 \end{block} 
\begin{enumerate} 
\item[] \hyperlink{17-Yes}{\beamergotobutton{} $1$}
\item[] \hyperlink{17-No}{\beamergotobutton{} $0.5$}
\item[] \hyperlink{17-No}{\beamergotobutton{} $2/3$
}
\item[] \hyperlink{17-No}{\beamergotobutton{} $2$}
\item[] \hyperlink{17-No}{\beamergotobutton{} $1.5$}
\end{enumerate} 

 \textbf{Да!} 
 \hyperlink{18}{\beamerbutton{Следующий вопрос}}\end{frame} 


 \begin{frame} \label{18-Yes} 
\begin{block}{18} 

Функция распределения случайной величины $X$ имеет вид
\[
F(x)=\begin{cases}
0, \; \text{ если } x<0 \\
cx^2, \; \text{ если } x\in [0;1] \\
1, \; \text{ если } x>1
\end{cases}
\]

\vspace{0.5cm} 
 
 
Вероятность того, что величина $X$ примет значение из интервала  $[0.5, 1.5]$ равна
 


 \end{block} 
\begin{enumerate} 
\item[] \hyperlink{18-No}{\beamergotobutton{} $1/2$}
\item[] \hyperlink{18-No}{\beamergotobutton{} $3/2$
}
\item[] \hyperlink{18-No}{\beamergotobutton{} $1$}
\item[] \hyperlink{18-No}{\beamergotobutton{} $2/3$}
\item[] \hyperlink{18-Yes}{\beamergotobutton{} $3/4$}
\end{enumerate} 

 \textbf{Да!} 
 \hyperlink{19}{\beamerbutton{Следующий вопрос}}\end{frame} 


 \begin{frame} \label{19-Yes} 
\begin{block}{19} 

Функция распределения случайной величины $X$ имеет вид
\[
F(x)=\begin{cases}
0, \; \text{ если } x<0 \\
cx^2, \; \text{ если } x\in [0;1] \\
1, \; \text{ если } x>1
\end{cases}
\]

\vspace{0.5cm} 
 
 
Математическое ожидание $\E(X)$ равно
 


 \end{block} 
\begin{enumerate} 
\item[] \hyperlink{19-Yes}{\beamergotobutton{} $2/3$}
\item[] \hyperlink{19-No}{\beamergotobutton{} $3/4$}
\item[] \hyperlink{19-No}{\beamergotobutton{} $2$
}
\item[] \hyperlink{19-No}{\beamergotobutton{} $1/2$}
\item[] \hyperlink{19-No}{\beamergotobutton{} $1/4$}
\end{enumerate} 

 \textbf{Да!} 
 \hyperlink{20}{\beamerbutton{Следующий вопрос}}\end{frame} 


 \begin{frame} \label{20-Yes} 
\begin{block}{20} 

Совместная функция плотности пары $X$ и $Y$ имеет вид
\[
f(x,y)=\begin{cases}
cx^2y^2, \; \text{ если } x\in[0;1], y\in [0;1] \\
0, \; \text{ иначе}
\end{cases}
\]

\vspace{0.5cm} 
 
 
Константа $c$ равна
 


 \end{block} 
\begin{enumerate} 
\item[] \hyperlink{20-No}{\beamergotobutton{} $2$
}
\item[] \hyperlink{20-No}{\beamergotobutton{} $1$}
\item[] \hyperlink{20-No}{\beamergotobutton{} $1/4$}
\item[] \hyperlink{20-Yes}{\beamergotobutton{} $9$}
\item[] \hyperlink{20-No}{\beamergotobutton{} $1/2$}
\end{enumerate} 

 \textbf{Да!} 
 \hyperlink{21}{\beamerbutton{Следующий вопрос}}\end{frame} 


 \begin{frame} \label{21-Yes} 
\begin{block}{21} 

Совместная функция плотности пары $X$ и $Y$ имеет вид
\[
f(x,y)=\begin{cases}
cx^2y^2, \; \text{ если } x\in[0;1], y\in [0;1] \\
0, \; \text{ иначе}
\end{cases}
\]

\vspace{0.5cm} 
 
 
Вероятность $\P(X<0.5, Y<0.5)$ равна
 


 \end{block} 
\begin{enumerate} 
\item[] \hyperlink{21-No}{\beamergotobutton{} $9/16$
}
\item[] \hyperlink{21-Yes}{\beamergotobutton{} $1/64$}
\item[] \hyperlink{21-No}{\beamergotobutton{} $1/16$}
\item[] \hyperlink{21-No}{\beamergotobutton{} $1/8$}
\item[] \hyperlink{21-No}{\beamergotobutton{} $1/4$}
\end{enumerate} 

 \textbf{Да!} 
 \hyperlink{22}{\beamerbutton{Следующий вопрос}}\end{frame} 


 \begin{frame} \label{22-Yes} 
\begin{block}{22  \textbf{Да!} 
		\hyperlink{23}{\beamerbutton{Следующий вопрос}}} 

Совместная функция плотности пары $X$ и $Y$ имеет вид
\[
f(x,y)=\begin{cases}
cx^2y^2, \; \text{ если } x\in[0;1], y\in [0;1] \\
0, \; \text{ иначе}
\end{cases}
\]
 
Условная функция плотности  $f_{X|Y=2}(x)$ равна

 \end{block} 
\begin{enumerate} 
\item[] \hyperlink{22-No}{\beamergotobutton{} $f_X|Y=2(x)=\begin{cases} 3x^2,\, \text{ если }  x\in [0;1] \\ 0, \text{ иначе}     \end{cases}$}
\item[] \hyperlink{22-No}{\beamergotobutton{} $f_X|Y=2(x)=\begin{cases} 36x^2,\, \text{ если }  x\in [0;1] \\ 0, \text{ иначе}     \end{cases}$}
\item[] \hyperlink{22-No}{\beamergotobutton{} $f_X|Y=2(x)=\begin{cases} 9x^2,\, \text{ если }  x\in [0;1] \\ 0, \text{ иначе}     \end{cases}$}
\item[] \hyperlink{22-Yes}{\beamergotobutton{} не определена}
\item[] \hyperlink{22-No}{\beamergotobutton{} $f_X|Y=2(x)=\begin{cases} x^2,\, \text{ если }  x\in [0;1] \\ 0, \text{ иначе}     \end{cases}$}
\end{enumerate} 
\end{frame} 


 \begin{frame} \label{23-Yes} 
\begin{block}{23} 

Совместная функция плотности пары $X$ и $Y$ имеет вид
\[
f(x,y)=\begin{cases}
cx^2y^2, \; \text{ если } x\in[0;1], y\in [0;1] \\
0, \; \text{ иначе}
\end{cases}
\]

\vspace{0.5cm} 
 
 
Математическое ожидание $\E(X/Y)$ равно
 


 \end{block} 
\begin{enumerate} 
\item[] \hyperlink{23-No}{\beamergotobutton{} $1/2$}
\item[] \hyperlink{23-No}{\beamergotobutton{} $2$
}
\item[] \hyperlink{23-Yes}{\beamergotobutton{} $9/8$}
\item[] \hyperlink{23-No}{\beamergotobutton{} $3$}
\item[] \hyperlink{23-No}{\beamergotobutton{} $1$}
\end{enumerate} 

 \textbf{Да!} 
 \hyperlink{24}{\beamerbutton{Следующий вопрос}}\end{frame} 


 \begin{frame} \label{24-Yes} 
\begin{block}{24} 

Известно, что $\E(X)=1$, $\Var(X)=1$, $\E(Y)=4$, $\Var(Y)=9$, $\Cov(X,Y)=-3$

\vspace{0.5cm} 
 
 
Ковариация $\Cov(2X-Y,X+3Y)$ равна
 


 \end{block} 
\begin{enumerate} 
\item[] \hyperlink{24-No}{\beamergotobutton{} $40$}
\item[] \hyperlink{24-No}{\beamergotobutton{} $22$}
\item[] \hyperlink{24-No}{\beamergotobutton{} $18$
}
\item[] \hyperlink{24-No}{\beamergotobutton{} $-18$}
\item[] \hyperlink{24-Yes}{\beamergotobutton{} $-40$}
\end{enumerate} 

 \textbf{Да!} 
 \hyperlink{25}{\beamerbutton{Следующий вопрос}}\end{frame} 


 \begin{frame} \label{25-Yes} 
\begin{block}{25} 

Известно, что $\E(X)=1$, $\Var(X)=1$, $\E(Y)=4$, $\Var(Y)=9$, $\Cov(X,Y)=-3$

\vspace{0.5cm} 
 
 
Корреляция $\Corr(2X+3,4Y-5)$ равна
 


 \end{block} 
\begin{enumerate} 
\item[] \hyperlink{25-No}{\beamergotobutton{} $-1/8$}
\item[] \hyperlink{25-No}{\beamergotobutton{} $1$}
\item[] \hyperlink{25-No}{\beamergotobutton{} $1/6$
}
\item[] \hyperlink{25-No}{\beamergotobutton{} $1/3$}
\item[] \hyperlink{25-Yes}{\beamergotobutton{} $-1$}
\end{enumerate} 

 \textbf{Да!} 
 \hyperlink{26}{\beamerbutton{Следующий вопрос}}\end{frame} 


 \begin{frame} \label{26-Yes} 
\begin{block}{26} 

Пусть случайные величины $X$ и $Y$ — независимы, тогда \textbf{НЕ ВЕРНЫМ} является утверждение
 


 \end{block} 
\begin{enumerate} 
\item[] \hyperlink{26-No}{\beamergotobutton{} $\P(X<a, Y<b)=\P(X<a)\P(Y<b)$}
\item[] \hyperlink{26-No}{\beamergotobutton{} $\Cov(X,Y) = 0$}
\item[] \hyperlink{26-Yes}{\beamergotobutton{} $\Var(X-Y)<\Var(X)+\Var(Y)$ }
\item[] \hyperlink{26-No}{\beamergotobutton{} $\E(XY)=\E(X)\E(Y)$}
\item[] \hyperlink{26-No}{\beamergotobutton{} $\E(X|Y)=\E(X)$}
\item[] \hyperlink{26-No}{\beamergotobutton{} $\P(X<a | Y<b)=\P(X<a)$}
\end{enumerate} 

 \textbf{Да!} 
 \hyperlink{27}{\beamerbutton{Следующий вопрос}}\end{frame} 


 \begin{frame} \label{27-Yes} 
\begin{block}{27} 

Если $\E(X)=0$, то, согласно неравенству Чебышева, $\P(|X| \leq 5 \sqrt{\Var(X)})$ лежит в интервале
 


 \end{block} 
\begin{enumerate} 
\item[] \hyperlink{27-Yes}{\beamergotobutton{} $[0.96;1]$ }
\item[] \hyperlink{27-No}{\beamergotobutton{} $[0;0.04]$}
\item[] \hyperlink{27-No}{\beamergotobutton{} $[0.8;1]$}
\item[] \hyperlink{27-No}{\beamergotobutton{} $[0;0.2]$}
\item[] \hyperlink{27-No}{\beamergotobutton{} $[0.5;1]$
}
\end{enumerate} 

 \textbf{Да!} 
 \hyperlink{28}{\beamerbutton{Следующий вопрос}}\end{frame} 


 \begin{frame} \label{28-Yes} 
\begin{block}{28} 

Пусть $X_1$, $X_2$, \ldots, $X_n$ — последовательность независимых одинаково распределенных случайных величин, $\E(X_i)=3$ и $\Var(X_i)=9$. Следующая величина имеет асимптотически стандартное нормальное распределение
 


 \end{block} 
\begin{enumerate} 
\item[] \hyperlink{28-No}{\beamergotobutton{} $\frac{\bar{X}_n-3}{3}$}
\item[] \hyperlink{28-Yes}{\beamergotobutton{} $\sqrt{n}\frac{\bar{X}-3}{3}$}
\item[] \hyperlink{28-No}{\beamergotobutton{} $\frac{\bar{X}_n-3}{3\sqrt{n}}$}
\item[] \hyperlink{28-No}{\beamergotobutton{} $\frac{X_n-3}{3}$ }
\item[] \hyperlink{28-No}{\beamergotobutton{} $\sqrt{n}(\bar{X}-3)$ }
\end{enumerate} 

 \textbf{Да!} 
 \hyperlink{29}{\beamerbutton{Следующий вопрос}}\end{frame} 


 \begin{frame} \label{29-Yes} 
\begin{block}{29} 

Случайная величина $X$ имеет функцию плотности $f(x)=\frac{1}{3\sqrt{2\pi}} \exp\left(-\frac{(x-1)^2}{18} \right)$. Следующее утверждение \textbf{НЕ ВЕРНО}
 


 \end{block} 
\begin{enumerate} 
\item[] \hyperlink{29-No}{\beamergotobutton{} $\E(X)=1$}
\item[] \hyperlink{29-No}{\beamergotobutton{} $\Var(X)=9$}
\item[] \hyperlink{29-No}{\beamergotobutton{} $\P(X>1)=0.5$}
\item[] \hyperlink{29-No}{\beamergotobutton{} $\P(X=0)=0$}
\item[] \hyperlink{29-Yes}{\beamergotobutton{} Случайная величина $X$ дискретна}
\end{enumerate} 

 \textbf{Да!} 
 \hyperlink{30}{\beamerbutton{Следующий вопрос}}\end{frame} 


 \begin{frame} \label{30-Yes} 
\begin{block}{30} 

Пусть $X_1$, $X_2$, \ldots, $X_n$ — последовательность независимых одинаково распределенных случайных величин, $\E(X_i)=\mu$ и $\Var(X_i)=\sigma^2$. Следующее утверждение в общем случае \textbf{НЕ ВЕРНО}:
 


 \end{block} 
\begin{enumerate} 
\item[] \hyperlink{30-Yes}{\beamergotobutton{} $\frac{X_n-\mu}{\sigma} \overset{F}{\to} N(0;1)$ при $n\to\infty$}
\item[] \hyperlink{30-No}{\beamergotobutton{} $\lim_{n\to\infty} \Var(\bar{X}_n)=0$}
\item[] \hyperlink{30-No}{\beamergotobutton{} $\bar{X}_n \overset{P}{\to} \mu$ при $n\to\infty$}
\item[] \hyperlink{30-No}{\beamergotobutton{}$\frac{\bar{X}_n-\mu}{\sigma /\sqrt{n}} \overset{F}{\to } N(0,1) $ при $n\to\infty$}
\item[] \hyperlink{30-No}{\beamergotobutton{} $\frac{\bar{X}_n-\mu}{\sqrt{n} \sigma } \overset{P}{\to } 0 $ при $n\to\infty$}
\end{enumerate} 

 \textbf{Да!} 
 \hyperlink{31}{\beamerbutton{Следующий вопрос}}\end{frame} 


 \begin{frame} \label{1-No} 
\begin{block}{1} 

Случайным образом выбирается семья с двумя детьми. Событие $A$ — в семье старший ребенок — мальчик,  событие $B$ — в семье только один из детей — мальчик, событие $C$ — в семье хотя бы один из детей — мальчик.  Вероятность $\P(C)$ равна
 


 \end{block} 
\begin{enumerate} 
\item[] \hyperlink{1-Yes}{\beamergotobutton{} $3/4$}
\item[] \hyperlink{1-No}{\beamergotobutton{} $1/4$}
\item[] \hyperlink{1-No}{\beamergotobutton{} $1/2$}
\item[] \hyperlink{1-No}{\beamergotobutton{} $2/3$
}
\item[] \hyperlink{1-No}{\beamergotobutton{} $1$}
\end{enumerate} 

 \alert{Нет!} 
\end{frame} 


 \begin{frame} \label{2-No} 
\begin{block}{2} 

Случайным образом выбирается семья с двумя детьми. Событие $A$ — в семье старший ребенок — мальчик,  событие $B$ — в семье только один из детей — мальчик, событие $C$ — в семье хотя бы один из детей — мальчик.  Вероятность $\P(A \cup C)$ равна
    


 \end{block} 
\begin{enumerate} 
\item[] \hyperlink{2-No}{\beamergotobutton{} $2/3$}
\item[] \hyperlink{2-No}{\beamergotobutton{} $1/2$
}
\item[] \hyperlink{2-No}{\beamergotobutton{} $1$}
\item[] \hyperlink{2-Yes}{\beamergotobutton{} $3/4$}
\item[] \hyperlink{2-No}{\beamergotobutton{} $3/8$}
\end{enumerate} 

 \alert{Нет!} 
\end{frame} 


 \begin{frame} \label{3-No} 
\begin{block}{3} 

Случайным образом выбирается семья с двумя детьми. Событие $A$ — в семье старший ребенок — мальчик,  событие $B$ — в семье только один из детей — мальчик, событие $C$ — в семье хотя бы один из детей — мальчик.  Вероятность $\P(A | C)$ равна
   


 \end{block} 
\begin{enumerate} 
\item[] \hyperlink{3-No}{\beamergotobutton{} $1/4$}
\item[] \hyperlink{3-No}{\beamergotobutton{} $3/4$
}
\item[] \hyperlink{3-Yes}{\beamergotobutton{} $2/3$}
\item[] \hyperlink{3-No}{\beamergotobutton{} $1$}
\item[] \hyperlink{3-No}{\beamergotobutton{} $1/2$}
\end{enumerate} 

 \alert{Нет!} 
\end{frame} 


 \begin{frame} \label{4-No} 
\begin{block}{4} 

Случайным образом выбирается семья с двумя детьми. Событие $A$ — в семье старший ребенок — мальчик,  событие $B$ — в семье только один из детей — мальчик, событие $C$ — в семье хотя бы один из детей — мальчик.


 \end{block} 
\begin{enumerate} 
\item[] \hyperlink{4-Yes}{\beamergotobutton{} $A$ и $B$ — независимы, $A$ и $C$ — зависимы, $B$ и $C$ — зависимы}
\item[] \hyperlink{4-No}{\beamergotobutton{} Любые два события из $A$, $B$, $C$ — зависимы}
\item[] \hyperlink{4-No}{\beamergotobutton{} $\P(A\cap B\cap C)=\P(A)\P(B)\P(C)$
}
\item[] \hyperlink{4-No}{\beamergotobutton{} События $A$, $B$, $C$ — независимы в совокупности}
\item[] \hyperlink{4-No}{\beamergotobutton{} События $A$, $B$, $C$ — независимы попарно, но зависимы в совокупности}
\end{enumerate} 

 \alert{Нет!} 
\end{frame} 


 \begin{frame} \label{5-No} 
\begin{block}{5} 

Имеется три монетки. Две «правильных» и одна — с «орлами» по обеим сторонам. Вася выбирает одну монетку наугад и подкидывает ее один раз. Вероятность того, что выпадет орел равна
     


 \end{block} 
\begin{enumerate} 
\item[] \hyperlink{5-No}{\beamergotobutton{} $1/3$}
\item[] \hyperlink{5-No}{\beamergotobutton{} $1/2$}
\item[] \hyperlink{5-No}{\beamergotobutton{} $3/5$}
\item[] \hyperlink{5-Yes}{\beamergotobutton{} $2/3$}
\item[] \hyperlink{5-No}{\beamergotobutton{} $2/5$
}
\end{enumerate} 

 \alert{Нет!} 
\end{frame} 


 \begin{frame} \label{6-No} 
\begin{block}{6} 

Имеется три монетки. Две «правильных» и одна — с «орлами» по обеим сторонам. Вася выбирает одну монетку наугад и подкидывает ее один раз. Вероятность того, что была выбрана неправильная монетка, если выпал орел, равна
   


 \end{block} 
\begin{enumerate} 
\item[] \hyperlink{6-No}{\beamergotobutton{} $2/3$}
\item[] \hyperlink{6-No}{\beamergotobutton{} $3/5$}
\item[] \hyperlink{6-No}{\beamergotobutton{} $1/3$}
\item[] \hyperlink{6-Yes}{\beamergotobutton{} $1/2$}
\item[] \hyperlink{6-No}{\beamergotobutton{} $3/2$
}
\end{enumerate} 

 \alert{Нет!} 
\end{frame} 


 \begin{frame} \label{7-No} 
\begin{block}{7} 

Вася бросает 7 правильных игральных кубиков. Наиболее вероятное количество выпавших шестёрок равно
     


 \end{block} 
\begin{enumerate} 
\item[] \hyperlink{7-No}{\beamergotobutton{} $0$}
\item[] \hyperlink{7-No}{\beamergotobutton{} $2$
}
\item[] \hyperlink{7-Yes}{\beamergotobutton{} $1$}
\item[] \hyperlink{7-No}{\beamergotobutton{} $7/6$}
\item[] \hyperlink{7-No}{\beamergotobutton{} $6/7$}
\end{enumerate} 

 \alert{Нет!} 
\end{frame} 


 \begin{frame} \label{8-No} 
\begin{block}{8} 

Вася бросает 7 правильных игральных кубиков. Вероятность того, что ровно на пяти из кубиков выпадет шестёрка равна
     


 \end{block} 
\begin{enumerate} 
\item[] \hyperlink{8-No}{\beamergotobutton{} $\left(\frac{1}{6}\right)^7$}
\item[] \hyperlink{8-No}{\beamergotobutton{} $\left(\frac{1}{6}\right)^5$}
\item[] \hyperlink{8-No}{\beamergotobutton{} $\frac{525}{12}\left(\frac{1}{6}\right)^7$}
\item[] \hyperlink{8-No}{\beamergotobutton{} $\frac{7}{12}\left(\frac{1}{6}\right)^5$}
\item[] \hyperlink{8-Yes}{\beamergotobutton{} $525\left(\frac{1}{6}\right)^7$}
\end{enumerate} 

 \alert{Нет!} 
\end{frame} 


 \begin{frame} \label{9-No} 
\begin{block}{9} 

Вася бросает 7 правильных игральных кубиков. Математическое ожидание суммы выпавших очков равно
     


 \end{block} 
\begin{enumerate} 
\item[] \hyperlink{9-No}{\beamergotobutton{} $7/6$}
\item[] \hyperlink{9-No}{\beamergotobutton{} $42$
}
\item[] \hyperlink{9-No}{\beamergotobutton{} $21$}
\item[] \hyperlink{9-No}{\beamergotobutton{} $30$}
\item[] \hyperlink{9-Yes}{\beamergotobutton{} $24.5$}
\end{enumerate} 

 \alert{Нет!} 
\end{frame} 


 \begin{frame} \label{10-No} 
\begin{block}{10} 

Вася бросает 7 правильных игральных кубиков. Дисперсия суммы выпавших очков равна
 


 \end{block} 
\begin{enumerate} 
\item[] \hyperlink{10-No}{\beamergotobutton{} $35/36$}
\item[] \hyperlink{10-No}{\beamergotobutton{} $7/6$}
\item[] \hyperlink{10-Yes}{\beamergotobutton{} $7\cdot\frac{35}{12}$}
\item[] \hyperlink{10-No}{\beamergotobutton{} $7\cdot \frac{35}{36}$}
\item[] \hyperlink{10-No}{\beamergotobutton{} $7$}
\end{enumerate} 

 \alert{Нет!} 
\end{frame} 


 \begin{frame} \label{11-No} 
\begin{block}{11} 

Вася бросает 7 правильных игральных кубиков. Пусть величина  $X$ — сумма очков, выпавших на первых двух кубиках, а величина  $Y$ — сумма очков, выпавших на следующих пяти кубиках. Ковариация $\Cov(X,Y)$ равна
 


 \end{block} 
\begin{enumerate} 
\item[] \hyperlink{11-Yes}{\beamergotobutton{} $0$}
\item[] \hyperlink{11-No}{\beamergotobutton{} $-2/5$
}
\item[] \hyperlink{11-No}{\beamergotobutton{} $0.5$}
\item[] \hyperlink{11-No}{\beamergotobutton{} $1$}
\item[] \hyperlink{11-No}{\beamergotobutton{} $2/5$}
\end{enumerate} 

 \alert{Нет!} 
\end{frame} 


 \begin{frame} \label{12-No} 
\begin{block}{12} 

Число изюминок в булочке — случайная величина, имеющая распределение Пуассона. Известно, что в среднем каждая булочка содержит 13 изюминок. Вероятность того, что в случайно выбранной булочке окажется только одна изюминка равна:
 


 \end{block} 
\begin{enumerate} 
\item[] \hyperlink{12-No}{\beamergotobutton{} $1/13$}
\item[] \hyperlink{12-Yes}{\beamergotobutton{} $13e^{-13}$}
\item[] \hyperlink{12-No}{\beamergotobutton{} $e^{-13}/13$}
\item[] \hyperlink{12-No}{\beamergotobutton{} $e^{13}/13!$}
\item[] \hyperlink{12-No}{\beamergotobutton{} $e^{-13}$}
\end{enumerate} 

 \alert{Нет!} 
\end{frame} 


 \begin{frame} \label{13-No} 
\begin{block}{13} 

Совместное распределение пары величин $X$ и $Y$ задано таблицей:
\begin{center}
\begin{tabular}{@{}c|ccc@{}}
\toprule
       & $Y=-1$ & $Y=0$ & $Y=1$ \\ \midrule
$X=-1$ & $1/4$  & $0$   & $1/4$ \\
$X=1$  & $1/6$  & $1/6$ & $1/6$ \\ \bottomrule
\end{tabular}
\end{center}
\vspace{0.5cm} 
 
 
Математическое ожидание случайной величины $X$ при условии, что $Y=-1$ равно
 


 \end{block} 
\begin{enumerate} 
\item[] \hyperlink{13-Yes}{\beamergotobutton{} $-1/5$}
\item[] \hyperlink{13-No}{\beamergotobutton{} $-1/3$}
\item[] \hyperlink{13-No}{\beamergotobutton{} $1/10$
}
\item[] \hyperlink{13-No}{\beamergotobutton{} $-1/12$}
\item[] \hyperlink{13-No}{\beamergotobutton{} $0$}
\end{enumerate} 

 \alert{Нет!} 
\end{frame} 


 \begin{frame} \label{14-No} 
\begin{block}{14} 

Совместное распределение пары величин $X$ и $Y$ задано таблицей:
\begin{center}
\begin{tabular}{@{}c|ccc@{}}
\toprule
       & $Y=-1$ & $Y=0$ & $Y=1$ \\ \midrule
$X=-1$ & $1/4$  & $0$   & $1/4$ \\
$X=1$  & $1/6$  & $1/6$ & $1/6$ \\ \bottomrule
\end{tabular}
\end{center}

\vspace{0.5cm} 
 
 
Вероятность того, что $X=1$ при условии, что $Y<0$ равна
 


 \end{block} 
\begin{enumerate} 
\item[] \hyperlink{14-No}{\beamergotobutton{} $1/12$}
\item[] \hyperlink{14-Yes}{\beamergotobutton{} $2/5$}
\item[] \hyperlink{14-No}{\beamergotobutton{} $1/6$}
\item[] \hyperlink{14-No}{\beamergotobutton{} $1/3$
}
\item[] \hyperlink{14-No}{\beamergotobutton{} $5/12$}
\end{enumerate} 

 \alert{Нет!} 
\end{frame} 


 \begin{frame} \label{15-No} 
\begin{block}{15} 

Совместное распределение пары величин $X$ и $Y$ задано таблицей:

\begin{center}
	\begin{tabular}{@{}c|ccc@{}}
		\toprule
		& $Y=-1$ & $Y=0$ & $Y=1$ \\ \midrule
		$X=-1$ & $1/4$  & $0$   & $1/4$ \\
		$X=1$  & $1/6$  & $1/6$ & $1/6$ \\ \bottomrule
	\end{tabular}
\end{center}

\vspace{0.5cm} 
 
 
Дисперсия случайной величины $Y$  равна
 


 \end{block} 
\begin{enumerate} 
\item[] \hyperlink{15-No}{\beamergotobutton{} $1/3$}
\item[] \hyperlink{15-No}{\beamergotobutton{} $5/12$}
\item[] \hyperlink{15-No}{\beamergotobutton{} $1/2$}
\item[] \hyperlink{15-Yes}{\beamergotobutton{} $5/6$}
\item[] \hyperlink{15-No}{\beamergotobutton{} $12/5$
}
\end{enumerate} 

 \alert{Нет!} 
\end{frame} 


 \begin{frame} \label{16-No} 
\begin{block}{16} 

Совместное распределение пары величин $X$ и $Y$ задано таблицей:

\begin{center}
	\begin{tabular}{@{}c|ccc@{}}
		\toprule
		& $Y=-1$ & $Y=0$ & $Y=1$ \\ \midrule
		$X=-1$ & $1/4$  & $0$   & $1/4$ \\
		$X=1$  & $1/6$  & $1/6$ & $1/6$ \\ \bottomrule
	\end{tabular}
\end{center}

\vspace{0.5cm} 
 
 
Ковариация, $\Cov(X,Y)$, равна
 


 \end{block} 
\begin{enumerate} 
\item[] \hyperlink{16-Yes}{\beamergotobutton{} $0$}
\item[] \hyperlink{16-No}{\beamergotobutton{} $0.5$}
\item[] \hyperlink{16-No}{\beamergotobutton{} $1$}
\item[] \hyperlink{16-No}{\beamergotobutton{} $-1$
}
\item[] \hyperlink{16-No}{\beamergotobutton{} $-0.5$}
\end{enumerate} 

 \alert{Нет!} 
\end{frame} 


 \begin{frame} \label{17-No} 
\begin{block}{17} 

Функция распределения случайной величины $X$ имеет вид
\[
F(x)=\begin{cases}
0, \; \text{ если } x<0 \\
cx^2, \; \text{ если } x\in [0;1] \\
1, \; \text{ если } x>1
\end{cases}
\]

\vspace{0.5cm} 
 
 
Константа $c$ равна
 


 \end{block} 
\begin{enumerate} 
\item[] \hyperlink{17-Yes}{\beamergotobutton{} $1$}
\item[] \hyperlink{17-No}{\beamergotobutton{} $0.5$}
\item[] \hyperlink{17-No}{\beamergotobutton{} $2/3$
}
\item[] \hyperlink{17-No}{\beamergotobutton{} $2$}
\item[] \hyperlink{17-No}{\beamergotobutton{} $1.5$}
\end{enumerate} 

 \alert{Нет!} 
\end{frame} 


 \begin{frame} \label{18-No} 
\begin{block}{18} 

Функция распределения случайной величины $X$ имеет вид
\[
F(x)=\begin{cases}
0, \; \text{ если } x<0 \\
cx^2, \; \text{ если } x\in [0;1] \\
1, \; \text{ если } x>1
\end{cases}
\]

\vspace{0.5cm} 
 
 
Вероятность того, что величина $X$ примет значение из интервала  $[0.5, 1.5]$ равна
 


 \end{block} 
\begin{enumerate} 
\item[] \hyperlink{18-No}{\beamergotobutton{} $1/2$}
\item[] \hyperlink{18-No}{\beamergotobutton{} $3/2$
}
\item[] \hyperlink{18-No}{\beamergotobutton{} $1$}
\item[] \hyperlink{18-No}{\beamergotobutton{} $2/3$}
\item[] \hyperlink{18-Yes}{\beamergotobutton{} $3/4$}
\end{enumerate} 

 \alert{Нет!} 
\end{frame} 


 \begin{frame} \label{19-No} 
\begin{block}{19} 

Функция распределения случайной величины $X$ имеет вид
\[
F(x)=\begin{cases}
0, \; \text{ если } x<0 \\
cx^2, \; \text{ если } x\in [0;1] \\
1, \; \text{ если } x>1
\end{cases}
\]

\vspace{0.5cm} 
 
 
Математическое ожидание $\E(X)$ равно
 


 \end{block} 
\begin{enumerate} 
\item[] \hyperlink{19-Yes}{\beamergotobutton{} $2/3$}
\item[] \hyperlink{19-No}{\beamergotobutton{} $3/4$}
\item[] \hyperlink{19-No}{\beamergotobutton{} $2$
}
\item[] \hyperlink{19-No}{\beamergotobutton{} $1/2$}
\item[] \hyperlink{19-No}{\beamergotobutton{} $1/4$}
\end{enumerate} 

 \alert{Нет!} 
\end{frame} 


 \begin{frame} \label{20-No} 
\begin{block}{20} 

Совместная функция плотности пары $X$ и $Y$ имеет вид
\[
f(x,y)=\begin{cases}
cx^2y^2, \; \text{ если } x\in[0;1], y\in [0;1] \\
0, \; \text{ иначе}
\end{cases}
\]

\vspace{0.5cm} 
 
 
Константа $c$ равна
 


 \end{block} 
\begin{enumerate} 
\item[] \hyperlink{20-No}{\beamergotobutton{} $2$
}
\item[] \hyperlink{20-No}{\beamergotobutton{} $1$}
\item[] \hyperlink{20-No}{\beamergotobutton{} $1/4$}
\item[] \hyperlink{20-Yes}{\beamergotobutton{} $9$}
\item[] \hyperlink{20-No}{\beamergotobutton{} $1/2$}
\end{enumerate} 

 \alert{Нет!} 
\end{frame} 


 \begin{frame} \label{21-No} 
\begin{block}{21} 

Совместная функция плотности пары $X$ и $Y$ имеет вид
\[
f(x,y)=\begin{cases}
cx^2y^2, \; \text{ если } x\in[0;1], y\in [0;1] \\
0, \; \text{ иначе}
\end{cases}
\]

\vspace{0.5cm} 
 
 
Вероятность $\P(X<0.5, Y<0.5)$ равна
 


 \end{block} 
\begin{enumerate} 
\item[] \hyperlink{21-No}{\beamergotobutton{} $9/16$
}
\item[] \hyperlink{21-Yes}{\beamergotobutton{} $1/64$}
\item[] \hyperlink{21-No}{\beamergotobutton{} $1/16$}
\item[] \hyperlink{21-No}{\beamergotobutton{} $1/8$}
\item[] \hyperlink{21-No}{\beamergotobutton{} $1/4$}
\end{enumerate} 

 \alert{Нет!} 
\end{frame} 


 \begin{frame} \label{22-No} 
\begin{block}{22 \alert{Нет!}} 

Совместная функция плотности пары $X$ и $Y$ имеет вид
\[
f(x,y)=\begin{cases}
cx^2y^2, \; \text{ если } x\in[0;1], y\in [0;1] \\
0, \; \text{ иначе}
\end{cases}
\]
 
Условная функция плотности  $f_{X|Y=2}(x)$ равна
 
 \end{block} 
\begin{enumerate} 
\item[] \hyperlink{22-No}{\beamergotobutton{} $f_X|Y=2(x)=\begin{cases} 3x^2,\, \text{ если }  x\in [0;1] \\ 0, \text{ иначе}     \end{cases}$}
\item[] \hyperlink{22-No}{\beamergotobutton{} $f_X|Y=2(x)=\begin{cases} 36x^2,\, \text{ если }  x\in [0;1] \\ 0, \text{ иначе}     \end{cases}$}
\item[] \hyperlink{22-No}{\beamergotobutton{} $f_X|Y=2(x)=\begin{cases} 9x^2,\, \text{ если }  x\in [0;1] \\ 0, \text{ иначе}     \end{cases}$}
\item[] \hyperlink{22-Yes}{\beamergotobutton{} не определена}
\item[] \hyperlink{22-No}{\beamergotobutton{} $f_X|Y=2(x)=\begin{cases} x^2,\, \text{ если }  x\in [0;1] \\ 0, \text{ иначе}     \end{cases}$}
\end{enumerate} 
\end{frame} 


 \begin{frame} \label{23-No} 
\begin{block}{23} 

Совместная функция плотности пары $X$ и $Y$ имеет вид
\[
f(x,y)=\begin{cases}
cx^2y^2, \; \text{ если } x\in[0;1], y\in [0;1] \\
0, \; \text{ иначе}
\end{cases}
\]

\vspace{0.5cm} 
 
 
Математическое ожидание $\E(X/Y)$ равно
 


 \end{block} 
\begin{enumerate} 
\item[] \hyperlink{23-No}{\beamergotobutton{} $1/2$}
\item[] \hyperlink{23-No}{\beamergotobutton{} $2$
}
\item[] \hyperlink{23-Yes}{\beamergotobutton{} $9/8$}
\item[] \hyperlink{23-No}{\beamergotobutton{} $3$}
\item[] \hyperlink{23-No}{\beamergotobutton{} $1$}
\end{enumerate} 

 \alert{Нет!} 
\end{frame} 


 \begin{frame} \label{24-No} 
\begin{block}{24} 

Известно, что $\E(X)=1$, $\Var(X)=1$, $\E(Y)=4$, $\Var(Y)=9$, $\Cov(X,Y)=-3$

\vspace{0.5cm} 
 
 
Ковариация $\Cov(2X-Y,X+3Y)$ равна
 


 \end{block} 
\begin{enumerate} 
\item[] \hyperlink{24-No}{\beamergotobutton{} $40$}
\item[] \hyperlink{24-No}{\beamergotobutton{} $22$}
\item[] \hyperlink{24-No}{\beamergotobutton{} $18$
}
\item[] \hyperlink{24-No}{\beamergotobutton{} $-18$}
\item[] \hyperlink{24-Yes}{\beamergotobutton{} $-40$}
\end{enumerate} 

 \alert{Нет!} 
\end{frame} 


 \begin{frame} \label{25-No} 
\begin{block}{25} 

Известно, что $\E(X)=1$, $\Var(X)=1$, $\E(Y)=4$, $\Var(Y)=9$, $\Cov(X,Y)=-3$

\vspace{0.5cm} 
 
 
Корреляция $\Corr(2X+3,4Y-5)$ равна
 


 \end{block} 
\begin{enumerate} 
\item[] \hyperlink{25-No}{\beamergotobutton{} $-1/8$}
\item[] \hyperlink{25-No}{\beamergotobutton{} $1$}
\item[] \hyperlink{25-No}{\beamergotobutton{} $1/6$
}
\item[] \hyperlink{25-No}{\beamergotobutton{} $1/3$}
\item[] \hyperlink{25-Yes}{\beamergotobutton{} $-1$}
\end{enumerate} 

 \alert{Нет!} 
\end{frame} 


 \begin{frame} \label{26-No} 
\begin{block}{26} 

Пусть случайные величины $X$ и $Y$ — независимы, тогда \textbf{НЕ ВЕРНЫМ} является утверждение
 


 \end{block} 
\begin{enumerate} 
\item[] \hyperlink{26-No}{\beamergotobutton{} $\P(X<a, Y<b)=\P(X<a)\P(Y<b)$}
\item[] \hyperlink{26-No}{\beamergotobutton{} $\Cov(X,Y) = 0$}
\item[] \hyperlink{26-Yes}{\beamergotobutton{} $\Var(X-Y)<\Var(X)+\Var(Y)$ }
\item[] \hyperlink{26-No}{\beamergotobutton{} $\E(XY)=\E(X)\E(Y)$}
\item[] \hyperlink{26-No}{\beamergotobutton{} $\E(X|Y)=\E(X)$}
\item[] \hyperlink{26-No}{\beamergotobutton{} $\P(X<a | Y<b)=\P(X<a)$}
\end{enumerate} 

 \alert{Нет!} 
\end{frame} 


 \begin{frame} \label{27-No} 
\begin{block}{27} 

Если $\E(X)=0$, то, согласно неравенству Чебышева, $\P(|X| \leq 5 \sqrt{\Var(X)})$ лежит в интервале
 


 \end{block} 
\begin{enumerate} 
\item[] \hyperlink{27-Yes}{\beamergotobutton{} $[0.96;1]$ }
\item[] \hyperlink{27-No}{\beamergotobutton{} $[0;0.04]$}
\item[] \hyperlink{27-No}{\beamergotobutton{} $[0.8;1]$}
\item[] \hyperlink{27-No}{\beamergotobutton{} $[0;0.2]$}
\item[] \hyperlink{27-No}{\beamergotobutton{} $[0.5;1]$
}
\end{enumerate} 

 \alert{Нет!} 
\end{frame} 


 \begin{frame} \label{28-No} 
\begin{block}{28} 

Пусть $X_1$, $X_2$, \ldots, $X_n$ — последовательность независимых одинаково распределенных случайных величин, $\E(X_i)=3$ и $\Var(X_i)=9$. Следующая величина имеет асимптотически стандартное нормальное распределение
 


 \end{block} 
\begin{enumerate} 
\item[] \hyperlink{28-No}{\beamergotobutton{} $\frac{\bar{X}_n-3}{3}$}
\item[] \hyperlink{28-Yes}{\beamergotobutton{} $\sqrt{n}\frac{\bar{X}-3}{3}$}
\item[] \hyperlink{28-No}{\beamergotobutton{} $\frac{\bar{X}_n-3}{3\sqrt{n}}$}
\item[] \hyperlink{28-No}{\beamergotobutton{} $\frac{X_n-3}{3}$ }
\item[] \hyperlink{28-No}{\beamergotobutton{} $\sqrt{n}(\bar{X}-3)$ }
\end{enumerate} 

 \alert{Нет!} 
\end{frame} 


 \begin{frame} \label{29-No} 
\begin{block}{29} 

Случайная величина $X$ имеет функцию плотности $f(x)=\frac{1}{3\sqrt{2\pi}} \exp\left(-\frac{(x-1)^2}{18} \right)$. Следующее утверждение \textbf{НЕ ВЕРНО}
 


 \end{block} 
\begin{enumerate} 
\item[] \hyperlink{29-No}{\beamergotobutton{} $\E(X)=1$}
\item[] \hyperlink{29-No}{\beamergotobutton{} $\Var(X)=9$}
\item[] \hyperlink{29-No}{\beamergotobutton{} $\P(X>1)=0.5$}
\item[] \hyperlink{29-No}{\beamergotobutton{} $\P(X=0)=0$}
\item[] \hyperlink{29-Yes}{\beamergotobutton{} Случайная величина $X$ дискретна}
\end{enumerate} 

 \alert{Нет!} 
\end{frame} 


 \begin{frame} \label{30-No} 
\begin{block}{30} 

Пусть $X_1$, $X_2$, \ldots, $X_n$ — последовательность независимых одинаково распределенных случайных величин, $\E(X_i)=\mu$ и $\Var(X_i)=\sigma^2$. Следующее утверждение в общем случае \textbf{НЕ ВЕРНО}:
 


 \end{block} 
\begin{enumerate} 
\item[] \hyperlink{30-Yes}{\beamergotobutton{} $\frac{X_n-\mu}{\sigma} \overset{F}{\to} N(0;1)$ при $n\to\infty$}
\item[] \hyperlink{30-No}{\beamergotobutton{} $\lim_{n\to\infty} \Var(\bar{X}_n)=0$}
\item[] \hyperlink{30-No}{\beamergotobutton{} $\bar{X}_n \overset{P}{\to} \mu$ при $n\to\infty$}
\item[] \hyperlink{30-No}{\beamergotobutton{}$\frac{\bar{X}_n-\mu}{\sigma /\sqrt{n}} \overset{F}{\to } N(0,1) $ при $n\to\infty$}
\item[] \hyperlink{30-No}{\beamergotobutton{} $\frac{\bar{X}_n-\mu}{\sqrt{n} \sigma } \overset{P}{\to } 0 $ при $n\to\infty$}
\end{enumerate} 

 \alert{Нет!} 
\end{frame} 

\end{document}
