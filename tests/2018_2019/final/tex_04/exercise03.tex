
\begin{question}
Вася считает, что контрольные по макроэкономике и статистике нравятся
студентам с одинаковой вероятностью. Чтобы проверить эту гипотезу, он
опросил по 100 случайных однокурсников после каждой контрольной и
выяснил, что макроэкономика понравилась 30 студентам, а статистика ---
50. При проверке этой гипотезы, тестовая статистика может иметь
распределение
\begin{answerlist}
  \item \(t_{99}\)
  \item \(t_{100}\)
  \item \(t_{198}\)
  \item \(t_{98}\)
  \item \(\cN(0,1)\)
\end{answerlist}
\end{question}

\begin{solution}
\begin{answerlist}
  \item Bad answer :(
  \item Bad answer :(
  \item Bad answer :(
  \item Bad answer :(
  \item Good answer :)
\end{answerlist}
\end{solution}

