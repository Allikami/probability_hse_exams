
\begin{question}
Величина \(X\) принимает три значения \(1\), \(2\) и \(3\). По случайной
выборке из ста наблюдений оказалось, что \(1\) выпало 40 раз, \(2\) ---
40 раз и \(3\) --- 20 раз. Карл хочет проверить гипотезу о том, что все
три вероятности одинаковые. При верной \(H_0\) критерий Пирсона имеет
распределение
\begin{answerlist}
  \item \(\chi^2_1\)
  \item \(\chi^2_3\)
  \item \(\cN(0;1)\)
  \item \(\chi^2_{99}\)
  \item \(\chi^2_2\)
\end{answerlist}
\end{question}

\begin{solution}
\begin{answerlist}
  \item Bad answer :(
  \item Bad answer :(
  \item Bad answer :(
  \item Bad answer :(
  \item Good answer :)
\end{answerlist}
\end{solution}

