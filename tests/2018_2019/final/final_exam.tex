\documentclass[12pt]{article}

\usepackage{tikz} % картинки в tikz
\usepackage{microtype} % свешивание пунктуации
\usepackage{array} % для столбцов фиксированной ширины
\usepackage{comment} % для комментирования целых окружений
\usepackage{indentfirst} % отступ в первом параграфе

\usepackage{sectsty} % для центрирования названий частей
\allsectionsfont{\centering}

\usepackage{amsmath, amssymb, amsthm, amsfonts} % куча стандартных математических плюшек

\usepackage[top=2cm, left=1cm, right=1cm, bottom=2cm]{geometry} % размер текста на странице
\usepackage{lastpage} % чтобы узнать номер последней страницы
 
\usepackage{enumitem} % дополнительные плюшки для списков
%  например \begin{enumerate}[resume] позволяет продолжить нумерацию в новом списке

\usepackage{caption} % подписи к рисункам
\usepackage{hyperref} % гиперссылки
\usepackage{multicol} % текст в несколько столбцов


\usepackage{fancyhdr} % весёлые колонтитулы
\pagestyle{fancy}
\lhead{Теория вероятностей и математическая статистика, ВШЭ}
\chead{}
\rhead{2019-06-18}
\lfoot{Вариант $\mu$}
\cfoot{Паниковать запрещается!}
% \rfoot{Тест}
\renewcommand{\headrulewidth}{0.4pt}
\renewcommand{\footrulewidth}{0.4pt}

\usepackage{ifthen} % для написания условий

\usepackage{todonotes} % для вставки в документ заметок о том, что осталось сделать
% \todo{Здесь надо коэффициенты исправить}
% \missingfigure{Здесь будет Последний день Помпеи}
% \listoftodos --- печатает все поставленные \todo'шки


% более красивые таблицы
\usepackage{booktabs}
% заповеди из докупентации:
% 1. Не используйте вертикальные линни
% 2. Не используйте двойные линии
% 3. Единицы измерения - в шапку таблицы
% 4. Не сокращайте .1 вместо 0.1
% 5. Повторяющееся значение повторяйте, а не говорите "то же"


\usepackage{fontspec}
\usepackage{polyglossia}

\setmainlanguage{russian}
\setotherlanguages{english}

% download "Linux Libertine" fonts:
% http://www.linuxlibertine.org/index.php?id=91&L=1
\setmainfont{Linux Libertine O} % or Helvetica, Arial, Cambria
% why do we need \newfontfamily:
% http://tex.stackexchange.com/questions/91507/
\newfontfamily{\cyrillicfonttt}{Linux Libertine O}

\AddEnumerateCounter{\asbuk}{\russian@alph}{щ} % для списков с русскими буквами
\setlist[enumerate, 2]{label=\asbuk*),ref=\asbuk*}

%% эконометрические сокращения
\DeclareMathOperator{\Cov}{Cov}
\DeclareMathOperator{\Corr}{Corr}
\DeclareMathOperator{\Var}{Var}
\DeclareMathOperator{\E}{E}
\def \hb{\hat{\beta}}
\def \hs{\hat{\sigma}}
\def \htheta{\hat{\theta}}
\def \s{\sigma}
\def \hy{\hat{y}}
\def \hY{\hat{Y}}
\def \v1{\vec{1}}
\def \e{\varepsilon}
\def \he{\hat{\e}}
\def \z{z}
\def \hVar{\widehat{\Var}}
\def \hCorr{\widehat{\Corr}}
\def \hCov{\widehat{\Cov}}
\def \cN{\mathcal{N}}
\def \P{\mathbb{P}}


\def \putyourname{\fbox{
    \begin{minipage}{42em}
      Фамилия, имя, номер группы:\vspace*{3ex}\par
      \noindent\dotfill\vspace{2mm}
    \end{minipage}
  }
}

\def \checktable{

	\vspace{5pt}
	Табличка для проверяющих работу:

\vspace{5pt}

	\begin{tabular}{|m{2cm}|m{1cm}|m{1cm}|m{1cm}|m{1cm}|m{1cm}|m{2cm}|}
\toprule
		Тест & 1 &  2 & 3 & 4 & 5 & Итого \\
\midrule
		&  &  & & & & \\
		&  &  & & & & \\
 \bottomrule
\end{tabular}
}



\def \testtable{

	\vspace{5pt}
	Внесите сюда ответы на тест:

\vspace{5pt}

\begin{tabular}{|m{2cm}|m{0.6cm}|m{0.6cm}|m{0.6cm}|m{0.6cm}|m{0.6cm}|m{0.6cm}|m{0.6cm}|m{0.6cm}|m{0.6cm}|m{0.6cm}|}
\toprule
		Вопрос & 1 &  2 & 3 & 4 & 5 & 6 & 7 & 8 & 9 & 10 \\
\midrule
		Ответ &  &  & & & & & & & & \\
 \bottomrule
\end{tabular}
}




% [1][3] 1 = one argument, 3 = value if missing
% эта магия создаёт окружение answerlist
% именно в окружении answerlist записаны варианты ответов в подключаемых exerciseXX
% просто \begin{answerlist} сделает ответы в три столбца
% если ответы длинные, то надо в них руками сделать
% \begin{answerlist}[1] чтобы они шли в один столбец
\newenvironment{answerlist}[1][3]{
\begin{multicols}{#1}

\begin{enumerate}[label=\fbox{\emph{\Alph*}},ref=\emph{\alph*}]
}
{
\item Нет верного ответа.
\end{enumerate}
\end{multicols}
}

% BB: unicol version. don't know why \ifthenelse fails in second part of new-env
\newenvironment{answerlistu}{
\begin{enumerate}[label=\fbox{\emph{\Alph*}},ref=\emph{\alph*}]
}
{
\item Нет верного ответа.
\end{enumerate}
}



\excludecomment{solution} % without solutions

\theoremstyle{definition}
\newtheorem{question}{Вопрос}



\begin{document}

\putyourname


%\testtable

%\checktable


\begin{question}
Вероятность ошибки первого рода, \(\alpha\), и вероятность ошибки
второго рода, \(\beta\), всегда связаны соотношением
\begin{answerlist}
  \item \(\alpha\geq \beta\)
  \item \(\alpha\leq \beta\)
  \item \(\alpha+\beta=1\)
  \item \(\alpha+\beta \leq 1\)
  \item \(\alpha+\beta \geq 1\)
\end{answerlist}
\end{question}

\begin{solution}
\begin{answerlist}
  \item Неверно
  \item Неверно
  \item Неверно
  \item Неверно
  \item Неверно
\end{answerlist}
\end{solution}



\begin{question}
Пусть \(X = (X_1, \, \ldots, \, X_n)\) — случайная выборка из
равномерного распределения на отрезке \([0; \, \theta]\), где
\(\theta > 0\) — неизвестный параметр. Несмещённой является оценка
\begin{answerlist}
  \item \(2 \bar{X}\)
  \item \(\bar{X} / 2\)
  \item \(X_{(1)}\)
  \item \(X_{1}\)
  \item \(\bar{X}\)
\end{answerlist}
\end{question}

\begin{solution}
\begin{answerlist}
  \item Good answer :)
  \item Bad answer :(
  \item Bad answer :(
  \item Bad answer :(
  \item Bad answer :(
\end{answerlist}
\end{solution}



\begin{question}
Случайная выборка состоит из одного наблюдения \(X_1\), которое имеет
плотность распределения \[
f(x; \, \theta) = \begin{cases}
    \tfrac{1}{\theta^2} x e^{-x/\theta} & \text{при } x > 0,  \\
    0 & \text{при }x\leq 0,
  \end{cases}
\] где \(\theta > 0\). Чему равна оценка неизвестного параметра
\(\theta\), найденная с помощью метода максимального правдоподобия?
\begin{answerlist}
  \item \(X_1\)
  \item \(\ln X_1\)
  \item \(X_1 / 2\)
  \item \(\frac{X_1}{\ln X_1}\)
  \item \(1 / \ln X_1\)
\end{answerlist}
\end{question}

\begin{solution}
\begin{answerlist}
  \item Bad answer :(
  \item Bad answer :(
  \item Good answer :)
  \item Bad answer :(
  \item Bad answer :(
\end{answerlist}
\end{solution}



\begin{question}
Имеется выборка из одного наблюдения \(X_1\). На основе этой выборки
тестируется гипотеза \(H_0\): \(X_1 \sim U[0;2]\) против альтернативной
гипотезы \(X_1 \sim U[1,3]\). Используется критерий следующего вида:
если \(X_1>a\), то \(H_0\) отвергается. Если \(a=1.5\), то с ростом
\(a\)
\begin{answerlist}
  \item вероятности ошибок первого и второго рода от \(a\) не зависят
  \item вероятность ошибки второго рода падает, первого — растёт
  \item вероятности ошибок первого и второго рода растут
  \item вероятности ошибок первого и второго рода падают
  \item вероятность ошибки первого рода падает, второго — растёт
\end{answerlist}
\end{question}

\begin{solution}
\begin{answerlist}
  \item Не туда!
  \item Тоже ересь
  \item Неверно
  \item Не угадал
  \item Ураа!!!
\end{answerlist}
\end{solution}


\newpage

\begin{question}
Время подготовки студента к экзаменам и по статистике, и макроэкономике,
имеет нормальное распределение с неизвестными математическими ожиданиями
и дисперсиями. По \(10\) наблюдениям Вениамин получил оценку
стандартного отклонения времени подготовки к статистике равную \(5\)
часам. Оценка стандартного отклонения времени подготовки к
макроэкономике, рассчитанная по \(20\) наблюдениям, оказалась равной
\(2\). Тестовая статистика при проверке гипотезы о равенстве дисперсий
может быть равна
\begin{answerlist}
  \item \(0.4\)
  \item \(0.16\)
  \item \(12.5\)
  \item \(0.8\)
  \item \(2.5\)
\end{answerlist}
\end{question}

\begin{solution}
\begin{answerlist}
  \item Bad answer :(
  \item Good answer :)
  \item Bad answer :(
  \item Bad answer :(
  \item Bad answer :(
\end{answerlist}
\end{solution}



\begin{question}
При построения доверительного интервала для разности математических
ожиданий в двух нормальных независимых выборках размером \(m\) и \(n\) в
случае равных известных дисперсий используется распределение
\begin{answerlist}
  \item \(F_{m-1, n-1}\)
  \item \(F_{m,n}\)
  \item \(t_{m+n}\)
  \item \(t_{m+n-2}\)
  \item \(\cN(0, 1)\)
\end{answerlist}
\end{question}

\begin{solution}
\begin{answerlist}
  \item Bad answer :(
  \item Bad answer :(
  \item Bad answer :(
  \item Bad answer :(
  \item Good answer :)
\end{answerlist}
\end{solution}



\begin{question}
Случайная выборка состоит из одного наблюдения \(X_1\), которое имеет
плотность распределения \[
    f(x; \, \theta) = \begin{cases}
                          \frac{1}{\theta}x^{-1 + \frac{1}{\theta}} & \text{при } x \in (0;\,1),  \\
                          0 & \text{при }x \not\in (0;\,1). \\
                        \end{cases}
\] Оценка параметра \(\theta\), найденная с помощью метода максимального
правдоподобия, равна
\begin{answerlist}
  \item \(X_1\)
  \item \(-\ln X_1\)
  \item \(\ln X_1\)
  \item \(-X_1\)
  \item \(\frac{1}{\ln X_1}\)
\end{answerlist}
\end{question}

\begin{solution}
\begin{answerlist}
  \item Тоже ересь
  \item Ураа!!!
  \item Неверно
  \item Не туда!
  \item Не угадал
\end{answerlist}
\end{solution}



\begin{question}
Совместная функция плотности пары случайных величин \(X\) и \(Y\) имеет
вид \[
f(x,y)=\begin{cases}
c (2x+y), \; \text{ если } x\in[0;2], y\in [0;2] \\
0, \; \text{ иначе}
\end{cases}
\]

Константа \(c\) равна
\begin{answerlist}
  \item \(1/8\)
  \item \(8\)
  \item \(12\)
  \item \(1/12\)
  \item \(1/6\)
\end{answerlist}
\end{question}

\begin{solution}
\begin{answerlist}
  \item Bad answer :(
  \item Bad answer :(
  \item Bad answer :(
  \item Good answer :)
  \item Bad answer :(
\end{answerlist}
\end{solution}



\begin{question}
Функция правдоподобия, построенная по случайной выборке \(X_1\), \ldots,
\(X_n\) из распределения с функцией плотности
\(f(x)=(\theta+1)x^{\theta}\) при \(x\in [0;1]\) имеет вид
\begin{answerlist}
  \item \((\theta+1)^{\sum x_i}\)
  \item \((\theta+1)^n\prod x_i^{\theta}\)
  \item \(\sum (\theta+1)x_i^{\theta}\)
  \item \((\sum x_i)^{\theta}\)
  \item \((\theta+1)x^{n\theta}\)
\end{answerlist}
\end{question}

\begin{solution}
\begin{answerlist}
  \item Неверно
  \item Неверно
  \item Неверно
  \item Отлично
  \item Неверно
\end{answerlist}
\end{solution}



\begin{question}
События A, B и C независимы в совокупности, если
\begin{answerlist}
  \item \(\P(A|B) = \P(A), \P(A|C) = \P(A)\)
  \item \(\P(ABC) = \P(A) \P(B) \P(C)\)
  \item \(\P(A|B) = \P(A), \P(A|C) = \P(A), \P(B|C) = \P(B)\)
  \item \(\P(A\cap B) = \P(A)\P(B), \P(A\cap C) = \P(A)\P(C), \P(B\cap C) = \P(B)\P(C)\)
  \item \(\P(A \cap B \cap C) = 0\)
\end{answerlist}
\end{question}

\begin{solution}
\begin{answerlist}
  \item Bad answer :(
  \item Bad answer :(
  \item Bad answer :(
  \item Bad answer :(
  \item Bad answer :(
\end{answerlist}
\end{solution}


\newpage

\begin{question}
Компоненты вектора \(X=(X_{1},X_{2},X_{3})\) имеют совместное нормальное
распределение: \[
X\sim\cN \left(
\begin{bmatrix}1\\1\\1\end{bmatrix},
\begin{bmatrix}3 & 0.5 & 0.5\\0.5 & 2 & 0.5\\0.5 & 0.5 & 3\end{bmatrix}
\right).
\] Вероятность \(\P(X_{1}>X_{2}+X_{3} )\) равна
\begin{answerlist}
  \item \(0.688\)
  \item \(0.369\)
  \item \(0.215\)
  \item \(0.701\)
  \item \(0.593\)
\end{answerlist}
\end{question}

\begin{solution}
\begin{answerlist}
  \item Bad answer :(
  \item Good answer :)
  \item Bad answer :(
  \item Bad answer :(
  \item Bad answer :(
\end{answerlist}
\end{solution}



\begin{question}
Время подготовки студента к экзаменам и по статистике, и макроэкономике,
имеет нормальное распределение с неизвестными математическими ожиданиями
и дисперсиями. По \(10\) наблюдениям Вениамин получил оценку
стандартного отклонения времени подготовки к статистике равную \(5\)
часам. Оценка стандартного отклонения времени подготовки к
макроэкономике, рассчитанная по \(20\) наблюдениям, оказалась равной
\(2\). Тестовая статистика при проверке гипотезы о равенстве дисперсий
может быть равна
\begin{answerlist}
  \item \(0.16\)
  \item \(0.4\)
  \item \(12.5\)
  \item \(2.5\)
  \item \(0.8\)
\end{answerlist}
\end{question}

\begin{solution}
\begin{answerlist}
  \item Good answer :)
  \item Bad answer :(
  \item Bad answer :(
  \item Bad answer :(
  \item Bad answer :(
\end{answerlist}
\end{solution}



\begin{question}
Известно, что \(\E(X)=-1\), \(\E(Y)=2\), \(\Var(X)=4\), \(\Var(Y)=9\),
\(\Cov(X,Y)=-3\). Ковариация \(\Cov(aX, (1-a)Y)\) минимальна при \(a\)
равном
\begin{answerlist}
  \item \(3/12\)
  \item \(0\)
  \item \(1/2\)
  \item \(-1/4\)
  \item \(2/3\)
\end{answerlist}
\end{question}

\begin{solution}
\begin{answerlist}
  \item Bad answer :(
  \item Bad answer :(
  \item Good answer :)
  \item Bad answer :(
  \item Bad answer :(
\end{answerlist}
\end{solution}



\begin{question}
Величины \(X_1, \, \ldots, \, X_n\) — случайная выборка из
распределения Бернулли с параметром \(p \in (0;\,1)\). Информация Фишера
о параметре \(p\), заключенная в одном наблюдении, равна
\begin{answerlist}
  \item \(\frac{1}{p(1-p)}\)
  \item \(\frac{1}{p}\)
  \item \(p(1-p)\)
  \item \(1 - p\)
  \item \(p\)
\end{answerlist}
\end{question}

\begin{solution}
\begin{answerlist}
  \item Ураа!!!
  \item Не угадал
  \item Неверно
  \item Тоже ересь
  \item Не туда!
\end{answerlist}
\end{solution}


\newpage

\begin{question}
Требуется проверить гипотезу о равенстве математических ожиданий по
независимым нормальным выборкам размером 33 и 16 наблюдений. Истинные
дисперсии по обеим выборкам известны, совпадают и равны 196. Разница
выборочных средних равна 1. Тестовая статистика может быть равна
\begin{answerlist}
  \item \(1/2\)
  \item \(1/7\)
  \item \(1/4\)
  \item \(1/14\)
  \item \(1/49\)
\end{answerlist}
\end{question}

\begin{solution}
\begin{answerlist}
  \item Bad answer :(
  \item Bad answer :(
  \item Good answer :)
  \item Bad answer :(
  \item Bad answer :(
\end{answerlist}
\end{solution}



\begin{question}
Оценка \(\hat \theta_n\) называется эффективной оценкой параметра
\(\theta\) в классе оценок \(K\), если
\begin{answerlist}[2]
  \item \(\Var(\hat \theta_n)=(\theta)^2/n\)
  \item \(\E((\hat \theta_n - \theta)^2) \stackrel{}{\to} 0\) при
\(n\stackrel{}{\to} \infty\)
  \item \(\hat \theta_n \stackrel{\P}{\to} \theta\) при
\(n\stackrel{}{\to} \infty\)
  \item \(\E(\hat \theta_n)=\theta\)
  \item \(\E((\hat \theta_n - \theta)^2) \leq \E((\tilde \theta - \theta)^2)\)
для всех \(\tilde \theta \in K\)
\end{answerlist}
\end{question}

\begin{solution}
\begin{answerlist}
  \item Bad answer :(
  \item Bad answer :(
  \item Bad answer :(
  \item Bad answer :(
  \item Good answer :)
\end{answerlist}
\end{solution}



\begin{question}
Случайная величина \(X\) равномерно распределена на отрезке от \(-2\) до
\(2\). Вероятность \(\P(X^2>0.64)\) равна
\begin{answerlist}
  \item \(0.8\)
  \item \(0.2\)
  \item \(0.1\)
  \item \(0.6\)
  \item \(\frac{1}{2\sqrt{2}}\)
\end{answerlist}
\end{question}

\begin{solution}
\begin{answerlist}
  \item Bad answer :(
  \item Bad answer :(
  \item Bad answer :(
  \item Good answer :)
  \item Bad answer :(
\end{answerlist}
\end{solution}



\begin{question}
Пусть \(X = (X_1, \ldots , X_n)\) — случайная выборка из
экспоненциального распределения с плотностью \[
f(x; \theta) =
\begin{cases}
\frac{1}{\theta}\exp(-\frac{x}{\theta}) \text{ при } x \geq 0,  \\
0 \text{ при } x < 0.
\end{cases}
\] Информация Фишера \(I_n(p)\) равна:
\begin{answerlist}
  \item \(\frac{\theta}{n}\)
  \item \(\frac{n}{\theta^2}\)
  \item \(n \theta^2\)
  \item \(\frac{\theta^2}{n}\)
  \item \(\frac{n}{\theta}\)
\end{answerlist}
\end{question}

\begin{solution}
\begin{answerlist}
  \item Неверно
  \item Отлично
  \item Неверно
  \item Неверно
  \item Неверно
\end{answerlist}
\end{solution}



\begin{question}
Среди 100 случайно выбранных ацтеков 20 платят дань Кулуакану, а 80 ---
Аскапоцалько. Соответственно, оценка доли ацтеков, платящих дань
Кулуакану, равна \(\hat{p}=0.2\). Разумная оценка стандартного
отклонения случайной величины \(\hat{p}\) равна
\begin{answerlist}
  \item \(0.16\)
  \item \(1.6\)
  \item \(0.04\)
  \item \(0.016\)
  \item \(0.4\)
\end{answerlist}
\end{question}

\begin{solution}
\begin{answerlist}
  \item Неверно
  \item Неверно
  \item Отлично
  \item Неверно
  \item Неверно
\end{answerlist}
\end{solution}



\begin{question}
Известно, что \(\E(X)=-1\), \(\E(Y)=2\), \(\Var(X)=4\), \(\Var(Y)=9\),
\(\Cov(X,Y)=-3\). Ожидание \(\E(X^2-Y^2)\) равно
\begin{answerlist}
  \item \(-4\)
  \item 8
  \item 4
  \item \(-8\)
  \item 0
\end{answerlist}
\end{question}

\begin{solution}
\begin{answerlist}
  \item Bad answer :(
  \item Bad answer :(
  \item Bad answer :(
  \item Good answer :)
  \item Bad answer :(
\end{answerlist}
\end{solution}


\newpage

\begin{question}
По 10 наблюдениям проверяется гипотеза \(H_0: \; \mu=10\) против
\(H_a: \; \mu \neq 10\) на выборке из нормального распределения с
неизвестной дисперсией. Величина
\(\sqrt{n}\cdot (\bar{X}-\mu)/\hat{\sigma}\) оказалась равной \(1\).
P-значение примерно равно
\begin{answerlist}
  \item \(0.17\)
  \item \(0.34\)
  \item \(0.32\)
  \item \(0.16\)
  \item \(0.83\)
\end{answerlist}
\end{question}

\begin{solution}
\begin{answerlist}
  \item Неверно
  \item Неверно
  \item Неверно
  \item Отлично
  \item Неверно
\end{answerlist}
\end{solution}



\begin{question}
Известно, что \(\E(X)=-1\), \(\E(Y)=2\), \(\Var(X)=4\), \(\Var(Y)=9\),
\(\Cov(X,Y)=-3\). Ковариация \(\Cov(X+2Y, 2X+3)\) равна
\begin{answerlist}
  \item \(-4\)
  \item \(1\)
  \item \(4\)
  \item \(0\)
  \item \(-1\)
\end{answerlist}
\end{question}

\begin{solution}
\begin{answerlist}
  \item Good answer :)
  \item Bad answer :(
  \item Bad answer :(
  \item Bad answer :(
  \item Bad answer :(
\end{answerlist}
\end{solution}



\begin{question}
Пусть \(X_1\), \ldots, \(X_n\) --- выборка объема \(n\) из равномерного
на \([0, \theta]\) распределения. Состоятельной оценкой параметра
\(\theta\) является:
\begin{answerlist}
  \item \(X_(n-1)\)
  \item \(\frac{n}{n+1} X_{(n-1)}\)
  \item \(\frac{n^2}{n^2-n+3} X_{(n-3)}\)
  \item все перечисленные случайные величины
  \item \(X_(n)\)
\end{answerlist}
\end{question}

\begin{solution}
\begin{answerlist}
  \item Неверно
  \item Неверно
  \item Неверно
  \item Отлично
  \item Неверно
\end{answerlist}
\end{solution}



\begin{question}
Величина \(X\) имеет биномиальное распределение с параметрами \(n\) и
\(p\). Дисперсия величины \(X\) максимальна при \(p\) равном
\begin{answerlist}
  \item \(0.9\)
  \item \(0.5\)
  \item \(0.75\)
  \item \(0.2\)
  \item \(0.25\)
\end{answerlist}
\end{question}

\begin{solution}
\begin{answerlist}
  \item Bad answer :(
  \item Good answer :)
  \item Bad answer :(
  \item Bad answer :(
  \item Bad answer :(
\end{answerlist}
\end{solution}



\begin{question}
Дисперсия случайной величины \(Y\) равна
\begin{answerlist}
  \item \(-1\)
  \item \(2/3\)
  \item \(1\)
  \item \(1/3\)
  \item \(0\)
\end{answerlist}
\end{question}

\begin{solution}
\begin{answerlist}
  \item Bad answer :(
  \item Good answer :)
  \item Bad answer :(
  \item Bad answer :(
  \item Bad answer :(
\end{answerlist}
\end{solution}


\newpage

\begin{question}
Николай Коперник подбросил бутерброд 200 раз. Бутерброд упал маслом вниз
95 раз, а маслом вверх — 105 раз. Значение критерия \(\chi^2\) Пирсона
для проверки гипотезы о равной вероятности данных событий равно
\begin{answerlist}
  \item 7.5
  \item 0.5
  \item 2.5
  \item 0.75
  \item 0.25
\end{answerlist}
\end{question}

\begin{solution}
\begin{answerlist}
  \item Неверно
  \item Отлично
  \item Неверно
  \item Неверно
  \item Неверно
\end{answerlist}
\end{solution}



\begin{question}
Имеются две случайных выборки \(X_1\), \ldots, \(X_{31}\) и \(Y_1\),
\ldots, \(Y_{41}\) из нормальных распределений. Известно, что
\(\sum_{i=1}^{31}(X_i - \bar X)^2 = 120\) и
\(\sum_{i=1}^{41}(Y_i - \bar Y)^2 = 400\). При проверке гипотезы о
равенстве дисперсий этих распределений значение тестовой статистики
может быть равно
\begin{answerlist}
  \item \(3.33\)
  \item \(2.5\)
  \item \(2\)
  \item \(2.52\)
  \item \(0.3\)
\end{answerlist}
\end{question}

\begin{solution}
\begin{answerlist}
  \item Тоже ересь
  \item Ураа!!!
  \item Не угадал
  \item Неверно
  \item Не туда!
\end{answerlist}
\end{solution}



\begin{question}
Имеется выборка из одного наблюдения \(X_1\). На основе этой выборки
тестируется гипотеза \(H_0\): \(X_1 \sim U[0;2]\) против альтернативной
гипотезы \(X_1 \sim U[1,3]\). Используется критерий следующего вида:
если \(X_1>a\), то \(H_0\) отвергается. Минимальная вероятность ошибки
первого рода достигается при \(a\) равном
\begin{answerlist}
  \item \(0.5\)
  \item \(1.5\)
  \item \(2\)
  \item \(1.9\)
  \item \(1\)
\end{answerlist}
\end{question}

\begin{solution}
\begin{answerlist}
  \item Неверно
  \item Тоже ересь
  \item Ураа!!!
  \item Не угадал
  \item Не туда!
\end{answerlist}
\end{solution}



\begin{question}
В школе три девятых класса: 9А, 9Б и 9В. В 9А классе — 50\% отличники,
в 9Б — 30\%, в 9В — 40\%. Если сначала равновероятно выбрать один из
трёх классов, а затем внутри класса равновероятно выбрать школьника, то
вероятность выбрать отличника равна
\begin{answerlist}
  \item \((3+4+5)/3\)
  \item \(0.3\)
  \item \(0.5\)
  \item \(0.27\)
  \item \(0.4\)
\end{answerlist}
\end{question}

\begin{solution}
\begin{answerlist}
  \item Bad answer :(
  \item Bad answer :(
  \item Bad answer :(
  \item Bad answer :(
  \item Good answer :)
\end{answerlist}
\end{solution}



\begin{question}
Величины \(X\) и \(Y\) одинаково распределены и равновероятно принимают
только два значения, \(-1\) и \(1\), при этом \(\P(Y=1|X=1)=0.4\).
Вероятность \(\P(Y=-1,X=1)\) равна
\begin{answerlist}
  \item \(1\)
  \item \(0.4\)
  \item \(0.6\)
  \item \(0.3\)
  \item \(0.5\)
\end{answerlist}
\end{question}

\begin{solution}
\begin{answerlist}
  \item Bad answer :(
  \item Bad answer :(
  \item Bad answer :(
  \item Good answer :)
  \item Bad answer :(
\end{answerlist}
\end{solution}


 






\newpage
\lfoot{Вариант $\kappa$}
\setcounter{question}{0}


\putyourname

%\testtable

%\checktable


\begin{question}
Вероятность ошибки первого рода, \(\alpha\), и вероятность ошибки
второго рода, \(\beta\), всегда связаны соотношением
\begin{answerlist}
  \item \(\alpha\geq \beta\)
  \item \(\alpha\leq \beta\)
  \item \(\alpha+\beta=1\)
  \item \(\alpha+\beta \leq 1\)
  \item \(\alpha+\beta \geq 1\)
\end{answerlist}
\end{question}

\begin{solution}
\begin{answerlist}
  \item Неверно
  \item Неверно
  \item Неверно
  \item Неверно
  \item Неверно
\end{answerlist}
\end{solution}



\begin{question}
Пусть \(X = (X_1, \, \ldots, \, X_n)\) — случайная выборка из
равномерного распределения на отрезке \([0; \, \theta]\), где
\(\theta > 0\) — неизвестный параметр. Несмещённой является оценка
\begin{answerlist}
  \item \(2 \bar{X}\)
  \item \(\bar{X} / 2\)
  \item \(X_{(1)}\)
  \item \(X_{1}\)
  \item \(\bar{X}\)
\end{answerlist}
\end{question}

\begin{solution}
\begin{answerlist}
  \item Good answer :)
  \item Bad answer :(
  \item Bad answer :(
  \item Bad answer :(
  \item Bad answer :(
\end{answerlist}
\end{solution}



\begin{question}
Случайная выборка состоит из одного наблюдения \(X_1\), которое имеет
плотность распределения \[
f(x; \, \theta) = \begin{cases}
    \tfrac{1}{\theta^2} x e^{-x/\theta} & \text{при } x > 0,  \\
    0 & \text{при }x\leq 0,
  \end{cases}
\] где \(\theta > 0\). Чему равна оценка неизвестного параметра
\(\theta\), найденная с помощью метода максимального правдоподобия?
\begin{answerlist}
  \item \(X_1\)
  \item \(\ln X_1\)
  \item \(X_1 / 2\)
  \item \(\frac{X_1}{\ln X_1}\)
  \item \(1 / \ln X_1\)
\end{answerlist}
\end{question}

\begin{solution}
\begin{answerlist}
  \item Bad answer :(
  \item Bad answer :(
  \item Good answer :)
  \item Bad answer :(
  \item Bad answer :(
\end{answerlist}
\end{solution}



\begin{question}
Имеется выборка из одного наблюдения \(X_1\). На основе этой выборки
тестируется гипотеза \(H_0\): \(X_1 \sim U[0;2]\) против альтернативной
гипотезы \(X_1 \sim U[1,3]\). Используется критерий следующего вида:
если \(X_1>a\), то \(H_0\) отвергается. Если \(a=1.5\), то с ростом
\(a\)
\begin{answerlist}
  \item вероятности ошибок первого и второго рода от \(a\) не зависят
  \item вероятность ошибки второго рода падает, первого — растёт
  \item вероятности ошибок первого и второго рода растут
  \item вероятности ошибок первого и второго рода падают
  \item вероятность ошибки первого рода падает, второго — растёт
\end{answerlist}
\end{question}

\begin{solution}
\begin{answerlist}
  \item Не туда!
  \item Тоже ересь
  \item Неверно
  \item Не угадал
  \item Ураа!!!
\end{answerlist}
\end{solution}



\begin{question}
Время подготовки студента к экзаменам и по статистике, и макроэкономике,
имеет нормальное распределение с неизвестными математическими ожиданиями
и дисперсиями. По \(10\) наблюдениям Вениамин получил оценку
стандартного отклонения времени подготовки к статистике равную \(5\)
часам. Оценка стандартного отклонения времени подготовки к
макроэкономике, рассчитанная по \(20\) наблюдениям, оказалась равной
\(2\). Тестовая статистика при проверке гипотезы о равенстве дисперсий
может быть равна
\begin{answerlist}
  \item \(0.4\)
  \item \(0.16\)
  \item \(12.5\)
  \item \(0.8\)
  \item \(2.5\)
\end{answerlist}
\end{question}

\begin{solution}
\begin{answerlist}
  \item Bad answer :(
  \item Good answer :)
  \item Bad answer :(
  \item Bad answer :(
  \item Bad answer :(
\end{answerlist}
\end{solution}


\newpage

\begin{question}
При построения доверительного интервала для разности математических
ожиданий в двух нормальных независимых выборках размером \(m\) и \(n\) в
случае равных известных дисперсий используется распределение
\begin{answerlist}
  \item \(F_{m-1, n-1}\)
  \item \(F_{m,n}\)
  \item \(t_{m+n}\)
  \item \(t_{m+n-2}\)
  \item \(\cN(0, 1)\)
\end{answerlist}
\end{question}

\begin{solution}
\begin{answerlist}
  \item Bad answer :(
  \item Bad answer :(
  \item Bad answer :(
  \item Bad answer :(
  \item Good answer :)
\end{answerlist}
\end{solution}



\begin{question}
Случайная выборка состоит из одного наблюдения \(X_1\), которое имеет
плотность распределения \[
    f(x; \, \theta) = \begin{cases}
                          \frac{1}{\theta}x^{-1 + \frac{1}{\theta}} & \text{при } x \in (0;\,1),  \\
                          0 & \text{при }x \not\in (0;\,1). \\
                        \end{cases}
\] Оценка параметра \(\theta\), найденная с помощью метода максимального
правдоподобия, равна
\begin{answerlist}
  \item \(X_1\)
  \item \(-\ln X_1\)
  \item \(\ln X_1\)
  \item \(-X_1\)
  \item \(\frac{1}{\ln X_1}\)
\end{answerlist}
\end{question}

\begin{solution}
\begin{answerlist}
  \item Тоже ересь
  \item Ураа!!!
  \item Неверно
  \item Не туда!
  \item Не угадал
\end{answerlist}
\end{solution}



\begin{question}
Совместная функция плотности пары случайных величин \(X\) и \(Y\) имеет
вид \[
f(x,y)=\begin{cases}
c (2x+y), \; \text{ если } x\in[0;2], y\in [0;2] \\
0, \; \text{ иначе}
\end{cases}
\]

Константа \(c\) равна
\begin{answerlist}
  \item \(1/8\)
  \item \(8\)
  \item \(12\)
  \item \(1/12\)
  \item \(1/6\)
\end{answerlist}
\end{question}

\begin{solution}
\begin{answerlist}
  \item Bad answer :(
  \item Bad answer :(
  \item Bad answer :(
  \item Good answer :)
  \item Bad answer :(
\end{answerlist}
\end{solution}



\begin{question}
Функция правдоподобия, построенная по случайной выборке \(X_1\), \ldots,
\(X_n\) из распределения с функцией плотности
\(f(x)=(\theta+1)x^{\theta}\) при \(x\in [0;1]\) имеет вид
\begin{answerlist}
  \item \((\theta+1)^{\sum x_i}\)
  \item \((\theta+1)^n\prod x_i^{\theta}\)
  \item \(\sum (\theta+1)x_i^{\theta}\)
  \item \((\sum x_i)^{\theta}\)
  \item \((\theta+1)x^{n\theta}\)
\end{answerlist}
\end{question}

\begin{solution}
\begin{answerlist}
  \item Неверно
  \item Неверно
  \item Неверно
  \item Отлично
  \item Неверно
\end{answerlist}
\end{solution}



\begin{question}
События A, B и C независимы в совокупности, если
\begin{answerlist}
  \item \(\P(A|B) = \P(A), \P(A|C) = \P(A)\)
  \item \(\P(ABC) = \P(A) \P(B) \P(C)\)
  \item \(\P(A|B) = \P(A), \P(A|C) = \P(A), \P(B|C) = \P(B)\)
  \item \(\P(A\cap B) = \P(A)\P(B), \P(A\cap C) = \P(A)\P(C), \P(B\cap C) = \P(B)\P(C)\)
  \item \(\P(A \cap B \cap C) = 0\)
\end{answerlist}
\end{question}

\begin{solution}
\begin{answerlist}
  \item Bad answer :(
  \item Bad answer :(
  \item Bad answer :(
  \item Bad answer :(
  \item Bad answer :(
\end{answerlist}
\end{solution}


\newpage

\begin{question}
Компоненты вектора \(X=(X_{1},X_{2},X_{3})\) имеют совместное нормальное
распределение: \[
X\sim\cN \left(
\begin{bmatrix}1\\1\\1\end{bmatrix},
\begin{bmatrix}3 & 0.5 & 0.5\\0.5 & 2 & 0.5\\0.5 & 0.5 & 3\end{bmatrix}
\right).
\] Вероятность \(\P(X_{1}>X_{2}+X_{3} )\) равна
\begin{answerlist}
  \item \(0.688\)
  \item \(0.369\)
  \item \(0.215\)
  \item \(0.701\)
  \item \(0.593\)
\end{answerlist}
\end{question}

\begin{solution}
\begin{answerlist}
  \item Bad answer :(
  \item Good answer :)
  \item Bad answer :(
  \item Bad answer :(
  \item Bad answer :(
\end{answerlist}
\end{solution}



\begin{question}
Время подготовки студента к экзаменам и по статистике, и макроэкономике,
имеет нормальное распределение с неизвестными математическими ожиданиями
и дисперсиями. По \(10\) наблюдениям Вениамин получил оценку
стандартного отклонения времени подготовки к статистике равную \(5\)
часам. Оценка стандартного отклонения времени подготовки к
макроэкономике, рассчитанная по \(20\) наблюдениям, оказалась равной
\(2\). Тестовая статистика при проверке гипотезы о равенстве дисперсий
может быть равна
\begin{answerlist}
  \item \(0.16\)
  \item \(0.4\)
  \item \(12.5\)
  \item \(2.5\)
  \item \(0.8\)
\end{answerlist}
\end{question}

\begin{solution}
\begin{answerlist}
  \item Good answer :)
  \item Bad answer :(
  \item Bad answer :(
  \item Bad answer :(
  \item Bad answer :(
\end{answerlist}
\end{solution}



\begin{question}
Известно, что \(\E(X)=-1\), \(\E(Y)=2\), \(\Var(X)=4\), \(\Var(Y)=9\),
\(\Cov(X,Y)=-3\). Ковариация \(\Cov(aX, (1-a)Y)\) минимальна при \(a\)
равном
\begin{answerlist}
  \item \(3/12\)
  \item \(0\)
  \item \(1/2\)
  \item \(-1/4\)
  \item \(2/3\)
\end{answerlist}
\end{question}

\begin{solution}
\begin{answerlist}
  \item Bad answer :(
  \item Bad answer :(
  \item Good answer :)
  \item Bad answer :(
  \item Bad answer :(
\end{answerlist}
\end{solution}



\begin{question}
Величины \(X_1, \, \ldots, \, X_n\) — случайная выборка из
распределения Бернулли с параметром \(p \in (0;\,1)\). Информация Фишера
о параметре \(p\), заключенная в одном наблюдении, равна
\begin{answerlist}
  \item \(\frac{1}{p(1-p)}\)
  \item \(\frac{1}{p}\)
  \item \(p(1-p)\)
  \item \(1 - p\)
  \item \(p\)
\end{answerlist}
\end{question}

\begin{solution}
\begin{answerlist}
  \item Ураа!!!
  \item Не угадал
  \item Неверно
  \item Тоже ересь
  \item Не туда!
\end{answerlist}
\end{solution}



\begin{question}
Требуется проверить гипотезу о равенстве математических ожиданий по
независимым нормальным выборкам размером 33 и 16 наблюдений. Истинные
дисперсии по обеим выборкам известны, совпадают и равны 196. Разница
выборочных средних равна 1. Тестовая статистика может быть равна
\begin{answerlist}
  \item \(1/2\)
  \item \(1/7\)
  \item \(1/4\)
  \item \(1/14\)
  \item \(1/49\)
\end{answerlist}
\end{question}

\begin{solution}
\begin{answerlist}
  \item Bad answer :(
  \item Bad answer :(
  \item Good answer :)
  \item Bad answer :(
  \item Bad answer :(
\end{answerlist}
\end{solution}


\newpage

\begin{question}
Оценка \(\hat \theta_n\) называется эффективной оценкой параметра
\(\theta\) в классе оценок \(K\), если
\begin{answerlist}[2]
  \item \(\Var(\hat \theta_n)=(\theta)^2/n\)
  \item \(\E((\hat \theta_n - \theta)^2) \stackrel{}{\to} 0\) при
\(n\stackrel{}{\to} \infty\)
  \item \(\hat \theta_n \stackrel{\P}{\to} \theta\) при
\(n\stackrel{}{\to} \infty\)
  \item \(\E(\hat \theta_n)=\theta\)
  \item \(\E((\hat \theta_n - \theta)^2) \leq \E((\tilde \theta - \theta)^2)\)
для всех \(\tilde \theta \in K\)
\end{answerlist}
\end{question}

\begin{solution}
\begin{answerlist}
  \item Bad answer :(
  \item Bad answer :(
  \item Bad answer :(
  \item Bad answer :(
  \item Good answer :)
\end{answerlist}
\end{solution}



\begin{question}
Случайная величина \(X\) равномерно распределена на отрезке от \(-2\) до
\(2\). Вероятность \(\P(X^2>0.64)\) равна
\begin{answerlist}
  \item \(0.8\)
  \item \(0.2\)
  \item \(0.1\)
  \item \(0.6\)
  \item \(\frac{1}{2\sqrt{2}}\)
\end{answerlist}
\end{question}

\begin{solution}
\begin{answerlist}
  \item Bad answer :(
  \item Bad answer :(
  \item Bad answer :(
  \item Good answer :)
  \item Bad answer :(
\end{answerlist}
\end{solution}



\begin{question}
Пусть \(X = (X_1, \ldots , X_n)\) — случайная выборка из
экспоненциального распределения с плотностью \[
f(x; \theta) =
\begin{cases}
\frac{1}{\theta}\exp(-\frac{x}{\theta}) \text{ при } x \geq 0,  \\
0 \text{ при } x < 0.
\end{cases}
\] Информация Фишера \(I_n(p)\) равна:
\begin{answerlist}
  \item \(\frac{\theta}{n}\)
  \item \(\frac{n}{\theta^2}\)
  \item \(n \theta^2\)
  \item \(\frac{\theta^2}{n}\)
  \item \(\frac{n}{\theta}\)
\end{answerlist}
\end{question}

\begin{solution}
\begin{answerlist}
  \item Неверно
  \item Отлично
  \item Неверно
  \item Неверно
  \item Неверно
\end{answerlist}
\end{solution}



\begin{question}
Среди 100 случайно выбранных ацтеков 20 платят дань Кулуакану, а 80 ---
Аскапоцалько. Соответственно, оценка доли ацтеков, платящих дань
Кулуакану, равна \(\hat{p}=0.2\). Разумная оценка стандартного
отклонения случайной величины \(\hat{p}\) равна
\begin{answerlist}
  \item \(0.16\)
  \item \(1.6\)
  \item \(0.04\)
  \item \(0.016\)
  \item \(0.4\)
\end{answerlist}
\end{question}

\begin{solution}
\begin{answerlist}
  \item Неверно
  \item Неверно
  \item Отлично
  \item Неверно
  \item Неверно
\end{answerlist}
\end{solution}



\begin{question}
Известно, что \(\E(X)=-1\), \(\E(Y)=2\), \(\Var(X)=4\), \(\Var(Y)=9\),
\(\Cov(X,Y)=-3\). Ожидание \(\E(X^2-Y^2)\) равно
\begin{answerlist}
  \item \(-4\)
  \item 8
  \item 4
  \item \(-8\)
  \item 0
\end{answerlist}
\end{question}

\begin{solution}
\begin{answerlist}
  \item Bad answer :(
  \item Bad answer :(
  \item Bad answer :(
  \item Good answer :)
  \item Bad answer :(
\end{answerlist}
\end{solution}


\newpage

\begin{question}
По 10 наблюдениям проверяется гипотеза \(H_0: \; \mu=10\) против
\(H_a: \; \mu \neq 10\) на выборке из нормального распределения с
неизвестной дисперсией. Величина
\(\sqrt{n}\cdot (\bar{X}-\mu)/\hat{\sigma}\) оказалась равной \(1\).
P-значение примерно равно
\begin{answerlist}
  \item \(0.17\)
  \item \(0.34\)
  \item \(0.32\)
  \item \(0.16\)
  \item \(0.83\)
\end{answerlist}
\end{question}

\begin{solution}
\begin{answerlist}
  \item Неверно
  \item Неверно
  \item Неверно
  \item Отлично
  \item Неверно
\end{answerlist}
\end{solution}



\begin{question}
Известно, что \(\E(X)=-1\), \(\E(Y)=2\), \(\Var(X)=4\), \(\Var(Y)=9\),
\(\Cov(X,Y)=-3\). Ковариация \(\Cov(X+2Y, 2X+3)\) равна
\begin{answerlist}
  \item \(-4\)
  \item \(1\)
  \item \(4\)
  \item \(0\)
  \item \(-1\)
\end{answerlist}
\end{question}

\begin{solution}
\begin{answerlist}
  \item Good answer :)
  \item Bad answer :(
  \item Bad answer :(
  \item Bad answer :(
  \item Bad answer :(
\end{answerlist}
\end{solution}



\begin{question}
Пусть \(X_1\), \ldots, \(X_n\) --- выборка объема \(n\) из равномерного
на \([0, \theta]\) распределения. Состоятельной оценкой параметра
\(\theta\) является:
\begin{answerlist}
  \item \(X_(n-1)\)
  \item \(\frac{n}{n+1} X_{(n-1)}\)
  \item \(\frac{n^2}{n^2-n+3} X_{(n-3)}\)
  \item все перечисленные случайные величины
  \item \(X_(n)\)
\end{answerlist}
\end{question}

\begin{solution}
\begin{answerlist}
  \item Неверно
  \item Неверно
  \item Неверно
  \item Отлично
  \item Неверно
\end{answerlist}
\end{solution}



\begin{question}
Величина \(X\) имеет биномиальное распределение с параметрами \(n\) и
\(p\). Дисперсия величины \(X\) максимальна при \(p\) равном
\begin{answerlist}
  \item \(0.9\)
  \item \(0.5\)
  \item \(0.75\)
  \item \(0.2\)
  \item \(0.25\)
\end{answerlist}
\end{question}

\begin{solution}
\begin{answerlist}
  \item Bad answer :(
  \item Good answer :)
  \item Bad answer :(
  \item Bad answer :(
  \item Bad answer :(
\end{answerlist}
\end{solution}



\begin{question}
Дисперсия случайной величины \(Y\) равна
\begin{answerlist}
  \item \(-1\)
  \item \(2/3\)
  \item \(1\)
  \item \(1/3\)
  \item \(0\)
\end{answerlist}
\end{question}

\begin{solution}
\begin{answerlist}
  \item Bad answer :(
  \item Good answer :)
  \item Bad answer :(
  \item Bad answer :(
  \item Bad answer :(
\end{answerlist}
\end{solution}


\newpage

\begin{question}
Николай Коперник подбросил бутерброд 200 раз. Бутерброд упал маслом вниз
95 раз, а маслом вверх — 105 раз. Значение критерия \(\chi^2\) Пирсона
для проверки гипотезы о равной вероятности данных событий равно
\begin{answerlist}
  \item 7.5
  \item 0.5
  \item 2.5
  \item 0.75
  \item 0.25
\end{answerlist}
\end{question}

\begin{solution}
\begin{answerlist}
  \item Неверно
  \item Отлично
  \item Неверно
  \item Неверно
  \item Неверно
\end{answerlist}
\end{solution}



\begin{question}
Имеются две случайных выборки \(X_1\), \ldots, \(X_{31}\) и \(Y_1\),
\ldots, \(Y_{41}\) из нормальных распределений. Известно, что
\(\sum_{i=1}^{31}(X_i - \bar X)^2 = 120\) и
\(\sum_{i=1}^{41}(Y_i - \bar Y)^2 = 400\). При проверке гипотезы о
равенстве дисперсий этих распределений значение тестовой статистики
может быть равно
\begin{answerlist}
  \item \(3.33\)
  \item \(2.5\)
  \item \(2\)
  \item \(2.52\)
  \item \(0.3\)
\end{answerlist}
\end{question}

\begin{solution}
\begin{answerlist}
  \item Тоже ересь
  \item Ураа!!!
  \item Не угадал
  \item Неверно
  \item Не туда!
\end{answerlist}
\end{solution}



\begin{question}
Имеется выборка из одного наблюдения \(X_1\). На основе этой выборки
тестируется гипотеза \(H_0\): \(X_1 \sim U[0;2]\) против альтернативной
гипотезы \(X_1 \sim U[1,3]\). Используется критерий следующего вида:
если \(X_1>a\), то \(H_0\) отвергается. Минимальная вероятность ошибки
первого рода достигается при \(a\) равном
\begin{answerlist}
  \item \(0.5\)
  \item \(1.5\)
  \item \(2\)
  \item \(1.9\)
  \item \(1\)
\end{answerlist}
\end{question}

\begin{solution}
\begin{answerlist}
  \item Неверно
  \item Тоже ересь
  \item Ураа!!!
  \item Не угадал
  \item Не туда!
\end{answerlist}
\end{solution}



\begin{question}
В школе три девятых класса: 9А, 9Б и 9В. В 9А классе — 50\% отличники,
в 9Б — 30\%, в 9В — 40\%. Если сначала равновероятно выбрать один из
трёх классов, а затем внутри класса равновероятно выбрать школьника, то
вероятность выбрать отличника равна
\begin{answerlist}
  \item \((3+4+5)/3\)
  \item \(0.3\)
  \item \(0.5\)
  \item \(0.27\)
  \item \(0.4\)
\end{answerlist}
\end{question}

\begin{solution}
\begin{answerlist}
  \item Bad answer :(
  \item Bad answer :(
  \item Bad answer :(
  \item Bad answer :(
  \item Good answer :)
\end{answerlist}
\end{solution}



\begin{question}
Величины \(X\) и \(Y\) одинаково распределены и равновероятно принимают
только два значения, \(-1\) и \(1\), при этом \(\P(Y=1|X=1)=0.4\).
Вероятность \(\P(Y=-1,X=1)\) равна
\begin{answerlist}
  \item \(1\)
  \item \(0.4\)
  \item \(0.6\)
  \item \(0.3\)
  \item \(0.5\)
\end{answerlist}
\end{question}

\begin{solution}
\begin{answerlist}
  \item Bad answer :(
  \item Bad answer :(
  \item Bad answer :(
  \item Good answer :)
  \item Bad answer :(
\end{answerlist}
\end{solution}









\newpage
\lfoot{Вариант $\delta$}
\setcounter{question}{0}


\putyourname

%\testtable

%\checktable



\begin{question}
Вероятность ошибки первого рода, \(\alpha\), и вероятность ошибки
второго рода, \(\beta\), всегда связаны соотношением
\begin{answerlist}
  \item \(\alpha\geq \beta\)
  \item \(\alpha\leq \beta\)
  \item \(\alpha+\beta=1\)
  \item \(\alpha+\beta \leq 1\)
  \item \(\alpha+\beta \geq 1\)
\end{answerlist}
\end{question}

\begin{solution}
\begin{answerlist}
  \item Неверно
  \item Неверно
  \item Неверно
  \item Неверно
  \item Неверно
\end{answerlist}
\end{solution}



\begin{question}
Пусть \(X = (X_1, \, \ldots, \, X_n)\) — случайная выборка из
равномерного распределения на отрезке \([0; \, \theta]\), где
\(\theta > 0\) — неизвестный параметр. Несмещённой является оценка
\begin{answerlist}
  \item \(2 \bar{X}\)
  \item \(\bar{X} / 2\)
  \item \(X_{(1)}\)
  \item \(X_{1}\)
  \item \(\bar{X}\)
\end{answerlist}
\end{question}

\begin{solution}
\begin{answerlist}
  \item Good answer :)
  \item Bad answer :(
  \item Bad answer :(
  \item Bad answer :(
  \item Bad answer :(
\end{answerlist}
\end{solution}



\begin{question}
Случайная выборка состоит из одного наблюдения \(X_1\), которое имеет
плотность распределения \[
f(x; \, \theta) = \begin{cases}
    \tfrac{1}{\theta^2} x e^{-x/\theta} & \text{при } x > 0,  \\
    0 & \text{при }x\leq 0,
  \end{cases}
\] где \(\theta > 0\). Чему равна оценка неизвестного параметра
\(\theta\), найденная с помощью метода максимального правдоподобия?
\begin{answerlist}
  \item \(X_1\)
  \item \(\ln X_1\)
  \item \(X_1 / 2\)
  \item \(\frac{X_1}{\ln X_1}\)
  \item \(1 / \ln X_1\)
\end{answerlist}
\end{question}

\begin{solution}
\begin{answerlist}
  \item Bad answer :(
  \item Bad answer :(
  \item Good answer :)
  \item Bad answer :(
  \item Bad answer :(
\end{answerlist}
\end{solution}



\begin{question}
Имеется выборка из одного наблюдения \(X_1\). На основе этой выборки
тестируется гипотеза \(H_0\): \(X_1 \sim U[0;2]\) против альтернативной
гипотезы \(X_1 \sim U[1,3]\). Используется критерий следующего вида:
если \(X_1>a\), то \(H_0\) отвергается. Если \(a=1.5\), то с ростом
\(a\)
\begin{answerlist}
  \item вероятности ошибок первого и второго рода от \(a\) не зависят
  \item вероятность ошибки второго рода падает, первого — растёт
  \item вероятности ошибок первого и второго рода растут
  \item вероятности ошибок первого и второго рода падают
  \item вероятность ошибки первого рода падает, второго — растёт
\end{answerlist}
\end{question}

\begin{solution}
\begin{answerlist}
  \item Не туда!
  \item Тоже ересь
  \item Неверно
  \item Не угадал
  \item Ураа!!!
\end{answerlist}
\end{solution}



\begin{question}
Время подготовки студента к экзаменам и по статистике, и макроэкономике,
имеет нормальное распределение с неизвестными математическими ожиданиями
и дисперсиями. По \(10\) наблюдениям Вениамин получил оценку
стандартного отклонения времени подготовки к статистике равную \(5\)
часам. Оценка стандартного отклонения времени подготовки к
макроэкономике, рассчитанная по \(20\) наблюдениям, оказалась равной
\(2\). Тестовая статистика при проверке гипотезы о равенстве дисперсий
может быть равна
\begin{answerlist}
  \item \(0.4\)
  \item \(0.16\)
  \item \(12.5\)
  \item \(0.8\)
  \item \(2.5\)
\end{answerlist}
\end{question}

\begin{solution}
\begin{answerlist}
  \item Bad answer :(
  \item Good answer :)
  \item Bad answer :(
  \item Bad answer :(
  \item Bad answer :(
\end{answerlist}
\end{solution}


\newpage

\begin{question}
При построения доверительного интервала для разности математических
ожиданий в двух нормальных независимых выборках размером \(m\) и \(n\) в
случае равных известных дисперсий используется распределение
\begin{answerlist}
  \item \(F_{m-1, n-1}\)
  \item \(F_{m,n}\)
  \item \(t_{m+n}\)
  \item \(t_{m+n-2}\)
  \item \(\cN(0, 1)\)
\end{answerlist}
\end{question}

\begin{solution}
\begin{answerlist}
  \item Bad answer :(
  \item Bad answer :(
  \item Bad answer :(
  \item Bad answer :(
  \item Good answer :)
\end{answerlist}
\end{solution}



\begin{question}
Случайная выборка состоит из одного наблюдения \(X_1\), которое имеет
плотность распределения \[
    f(x; \, \theta) = \begin{cases}
                          \frac{1}{\theta}x^{-1 + \frac{1}{\theta}} & \text{при } x \in (0;\,1),  \\
                          0 & \text{при }x \not\in (0;\,1). \\
                        \end{cases}
\] Оценка параметра \(\theta\), найденная с помощью метода максимального
правдоподобия, равна
\begin{answerlist}
  \item \(X_1\)
  \item \(-\ln X_1\)
  \item \(\ln X_1\)
  \item \(-X_1\)
  \item \(\frac{1}{\ln X_1}\)
\end{answerlist}
\end{question}

\begin{solution}
\begin{answerlist}
  \item Тоже ересь
  \item Ураа!!!
  \item Неверно
  \item Не туда!
  \item Не угадал
\end{answerlist}
\end{solution}



\begin{question}
Совместная функция плотности пары случайных величин \(X\) и \(Y\) имеет
вид \[
f(x,y)=\begin{cases}
c (2x+y), \; \text{ если } x\in[0;2], y\in [0;2] \\
0, \; \text{ иначе}
\end{cases}
\]

Константа \(c\) равна
\begin{answerlist}
  \item \(1/8\)
  \item \(8\)
  \item \(12\)
  \item \(1/12\)
  \item \(1/6\)
\end{answerlist}
\end{question}

\begin{solution}
\begin{answerlist}
  \item Bad answer :(
  \item Bad answer :(
  \item Bad answer :(
  \item Good answer :)
  \item Bad answer :(
\end{answerlist}
\end{solution}



\begin{question}
Функция правдоподобия, построенная по случайной выборке \(X_1\), \ldots,
\(X_n\) из распределения с функцией плотности
\(f(x)=(\theta+1)x^{\theta}\) при \(x\in [0;1]\) имеет вид
\begin{answerlist}
  \item \((\theta+1)^{\sum x_i}\)
  \item \((\theta+1)^n\prod x_i^{\theta}\)
  \item \(\sum (\theta+1)x_i^{\theta}\)
  \item \((\sum x_i)^{\theta}\)
  \item \((\theta+1)x^{n\theta}\)
\end{answerlist}
\end{question}

\begin{solution}
\begin{answerlist}
  \item Неверно
  \item Неверно
  \item Неверно
  \item Отлично
  \item Неверно
\end{answerlist}
\end{solution}



\begin{question}
События A, B и C независимы в совокупности, если
\begin{answerlist}
  \item \(\P(A|B) = \P(A), \P(A|C) = \P(A)\)
  \item \(\P(ABC) = \P(A) \P(B) \P(C)\)
  \item \(\P(A|B) = \P(A), \P(A|C) = \P(A), \P(B|C) = \P(B)\)
  \item \(\P(A\cap B) = \P(A)\P(B), \P(A\cap C) = \P(A)\P(C), \P(B\cap C) = \P(B)\P(C)\)
  \item \(\P(A \cap B \cap C) = 0\)
\end{answerlist}
\end{question}

\begin{solution}
\begin{answerlist}
  \item Bad answer :(
  \item Bad answer :(
  \item Bad answer :(
  \item Bad answer :(
  \item Bad answer :(
\end{answerlist}
\end{solution}


\newpage

\begin{question}
Компоненты вектора \(X=(X_{1},X_{2},X_{3})\) имеют совместное нормальное
распределение: \[
X\sim\cN \left(
\begin{bmatrix}1\\1\\1\end{bmatrix},
\begin{bmatrix}3 & 0.5 & 0.5\\0.5 & 2 & 0.5\\0.5 & 0.5 & 3\end{bmatrix}
\right).
\] Вероятность \(\P(X_{1}>X_{2}+X_{3} )\) равна
\begin{answerlist}
  \item \(0.688\)
  \item \(0.369\)
  \item \(0.215\)
  \item \(0.701\)
  \item \(0.593\)
\end{answerlist}
\end{question}

\begin{solution}
\begin{answerlist}
  \item Bad answer :(
  \item Good answer :)
  \item Bad answer :(
  \item Bad answer :(
  \item Bad answer :(
\end{answerlist}
\end{solution}



\begin{question}
Время подготовки студента к экзаменам и по статистике, и макроэкономике,
имеет нормальное распределение с неизвестными математическими ожиданиями
и дисперсиями. По \(10\) наблюдениям Вениамин получил оценку
стандартного отклонения времени подготовки к статистике равную \(5\)
часам. Оценка стандартного отклонения времени подготовки к
макроэкономике, рассчитанная по \(20\) наблюдениям, оказалась равной
\(2\). Тестовая статистика при проверке гипотезы о равенстве дисперсий
может быть равна
\begin{answerlist}
  \item \(0.16\)
  \item \(0.4\)
  \item \(12.5\)
  \item \(2.5\)
  \item \(0.8\)
\end{answerlist}
\end{question}

\begin{solution}
\begin{answerlist}
  \item Good answer :)
  \item Bad answer :(
  \item Bad answer :(
  \item Bad answer :(
  \item Bad answer :(
\end{answerlist}
\end{solution}



\begin{question}
Известно, что \(\E(X)=-1\), \(\E(Y)=2\), \(\Var(X)=4\), \(\Var(Y)=9\),
\(\Cov(X,Y)=-3\). Ковариация \(\Cov(aX, (1-a)Y)\) минимальна при \(a\)
равном
\begin{answerlist}
  \item \(3/12\)
  \item \(0\)
  \item \(1/2\)
  \item \(-1/4\)
  \item \(2/3\)
\end{answerlist}
\end{question}

\begin{solution}
\begin{answerlist}
  \item Bad answer :(
  \item Bad answer :(
  \item Good answer :)
  \item Bad answer :(
  \item Bad answer :(
\end{answerlist}
\end{solution}



\begin{question}
Величины \(X_1, \, \ldots, \, X_n\) — случайная выборка из
распределения Бернулли с параметром \(p \in (0;\,1)\). Информация Фишера
о параметре \(p\), заключенная в одном наблюдении, равна
\begin{answerlist}
  \item \(\frac{1}{p(1-p)}\)
  \item \(\frac{1}{p}\)
  \item \(p(1-p)\)
  \item \(1 - p\)
  \item \(p\)
\end{answerlist}
\end{question}

\begin{solution}
\begin{answerlist}
  \item Ураа!!!
  \item Не угадал
  \item Неверно
  \item Тоже ересь
  \item Не туда!
\end{answerlist}
\end{solution}



\begin{question}
Требуется проверить гипотезу о равенстве математических ожиданий по
независимым нормальным выборкам размером 33 и 16 наблюдений. Истинные
дисперсии по обеим выборкам известны, совпадают и равны 196. Разница
выборочных средних равна 1. Тестовая статистика может быть равна
\begin{answerlist}
  \item \(1/2\)
  \item \(1/7\)
  \item \(1/4\)
  \item \(1/14\)
  \item \(1/49\)
\end{answerlist}
\end{question}

\begin{solution}
\begin{answerlist}
  \item Bad answer :(
  \item Bad answer :(
  \item Good answer :)
  \item Bad answer :(
  \item Bad answer :(
\end{answerlist}
\end{solution}


\newpage

\begin{question}
Оценка \(\hat \theta_n\) называется эффективной оценкой параметра
\(\theta\) в классе оценок \(K\), если
\begin{answerlist}[2]
  \item \(\Var(\hat \theta_n)=(\theta)^2/n\)
  \item \(\E((\hat \theta_n - \theta)^2) \stackrel{}{\to} 0\) при
\(n\stackrel{}{\to} \infty\)
  \item \(\hat \theta_n \stackrel{\P}{\to} \theta\) при
\(n\stackrel{}{\to} \infty\)
  \item \(\E(\hat \theta_n)=\theta\)
  \item \(\E((\hat \theta_n - \theta)^2) \leq \E((\tilde \theta - \theta)^2)\)
для всех \(\tilde \theta \in K\)
\end{answerlist}
\end{question}

\begin{solution}
\begin{answerlist}
  \item Bad answer :(
  \item Bad answer :(
  \item Bad answer :(
  \item Bad answer :(
  \item Good answer :)
\end{answerlist}
\end{solution}



\begin{question}
Случайная величина \(X\) равномерно распределена на отрезке от \(-2\) до
\(2\). Вероятность \(\P(X^2>0.64)\) равна
\begin{answerlist}
  \item \(0.8\)
  \item \(0.2\)
  \item \(0.1\)
  \item \(0.6\)
  \item \(\frac{1}{2\sqrt{2}}\)
\end{answerlist}
\end{question}

\begin{solution}
\begin{answerlist}
  \item Bad answer :(
  \item Bad answer :(
  \item Bad answer :(
  \item Good answer :)
  \item Bad answer :(
\end{answerlist}
\end{solution}



\begin{question}
Пусть \(X = (X_1, \ldots , X_n)\) — случайная выборка из
экспоненциального распределения с плотностью \[
f(x; \theta) =
\begin{cases}
\frac{1}{\theta}\exp(-\frac{x}{\theta}) \text{ при } x \geq 0,  \\
0 \text{ при } x < 0.
\end{cases}
\] Информация Фишера \(I_n(p)\) равна:
\begin{answerlist}
  \item \(\frac{\theta}{n}\)
  \item \(\frac{n}{\theta^2}\)
  \item \(n \theta^2\)
  \item \(\frac{\theta^2}{n}\)
  \item \(\frac{n}{\theta}\)
\end{answerlist}
\end{question}

\begin{solution}
\begin{answerlist}
  \item Неверно
  \item Отлично
  \item Неверно
  \item Неверно
  \item Неверно
\end{answerlist}
\end{solution}



\begin{question}
Среди 100 случайно выбранных ацтеков 20 платят дань Кулуакану, а 80 ---
Аскапоцалько. Соответственно, оценка доли ацтеков, платящих дань
Кулуакану, равна \(\hat{p}=0.2\). Разумная оценка стандартного
отклонения случайной величины \(\hat{p}\) равна
\begin{answerlist}
  \item \(0.16\)
  \item \(1.6\)
  \item \(0.04\)
  \item \(0.016\)
  \item \(0.4\)
\end{answerlist}
\end{question}

\begin{solution}
\begin{answerlist}
  \item Неверно
  \item Неверно
  \item Отлично
  \item Неверно
  \item Неверно
\end{answerlist}
\end{solution}



\begin{question}
Известно, что \(\E(X)=-1\), \(\E(Y)=2\), \(\Var(X)=4\), \(\Var(Y)=9\),
\(\Cov(X,Y)=-3\). Ожидание \(\E(X^2-Y^2)\) равно
\begin{answerlist}
  \item \(-4\)
  \item 8
  \item 4
  \item \(-8\)
  \item 0
\end{answerlist}
\end{question}

\begin{solution}
\begin{answerlist}
  \item Bad answer :(
  \item Bad answer :(
  \item Bad answer :(
  \item Good answer :)
  \item Bad answer :(
\end{answerlist}
\end{solution}


\newpage

\begin{question}
По 10 наблюдениям проверяется гипотеза \(H_0: \; \mu=10\) против
\(H_a: \; \mu \neq 10\) на выборке из нормального распределения с
неизвестной дисперсией. Величина
\(\sqrt{n}\cdot (\bar{X}-\mu)/\hat{\sigma}\) оказалась равной \(1\).
P-значение примерно равно
\begin{answerlist}
  \item \(0.17\)
  \item \(0.34\)
  \item \(0.32\)
  \item \(0.16\)
  \item \(0.83\)
\end{answerlist}
\end{question}

\begin{solution}
\begin{answerlist}
  \item Неверно
  \item Неверно
  \item Неверно
  \item Отлично
  \item Неверно
\end{answerlist}
\end{solution}



\begin{question}
Известно, что \(\E(X)=-1\), \(\E(Y)=2\), \(\Var(X)=4\), \(\Var(Y)=9\),
\(\Cov(X,Y)=-3\). Ковариация \(\Cov(X+2Y, 2X+3)\) равна
\begin{answerlist}
  \item \(-4\)
  \item \(1\)
  \item \(4\)
  \item \(0\)
  \item \(-1\)
\end{answerlist}
\end{question}

\begin{solution}
\begin{answerlist}
  \item Good answer :)
  \item Bad answer :(
  \item Bad answer :(
  \item Bad answer :(
  \item Bad answer :(
\end{answerlist}
\end{solution}



\begin{question}
Пусть \(X_1\), \ldots, \(X_n\) --- выборка объема \(n\) из равномерного
на \([0, \theta]\) распределения. Состоятельной оценкой параметра
\(\theta\) является:
\begin{answerlist}
  \item \(X_(n-1)\)
  \item \(\frac{n}{n+1} X_{(n-1)}\)
  \item \(\frac{n^2}{n^2-n+3} X_{(n-3)}\)
  \item все перечисленные случайные величины
  \item \(X_(n)\)
\end{answerlist}
\end{question}

\begin{solution}
\begin{answerlist}
  \item Неверно
  \item Неверно
  \item Неверно
  \item Отлично
  \item Неверно
\end{answerlist}
\end{solution}



\begin{question}
Величина \(X\) имеет биномиальное распределение с параметрами \(n\) и
\(p\). Дисперсия величины \(X\) максимальна при \(p\) равном
\begin{answerlist}
  \item \(0.9\)
  \item \(0.5\)
  \item \(0.75\)
  \item \(0.2\)
  \item \(0.25\)
\end{answerlist}
\end{question}

\begin{solution}
\begin{answerlist}
  \item Bad answer :(
  \item Good answer :)
  \item Bad answer :(
  \item Bad answer :(
  \item Bad answer :(
\end{answerlist}
\end{solution}



\begin{question}
Дисперсия случайной величины \(Y\) равна
\begin{answerlist}
  \item \(-1\)
  \item \(2/3\)
  \item \(1\)
  \item \(1/3\)
  \item \(0\)
\end{answerlist}
\end{question}

\begin{solution}
\begin{answerlist}
  \item Bad answer :(
  \item Good answer :)
  \item Bad answer :(
  \item Bad answer :(
  \item Bad answer :(
\end{answerlist}
\end{solution}


\newpage

\begin{question}
Николай Коперник подбросил бутерброд 200 раз. Бутерброд упал маслом вниз
95 раз, а маслом вверх — 105 раз. Значение критерия \(\chi^2\) Пирсона
для проверки гипотезы о равной вероятности данных событий равно
\begin{answerlist}
  \item 7.5
  \item 0.5
  \item 2.5
  \item 0.75
  \item 0.25
\end{answerlist}
\end{question}

\begin{solution}
\begin{answerlist}
  \item Неверно
  \item Отлично
  \item Неверно
  \item Неверно
  \item Неверно
\end{answerlist}
\end{solution}



\begin{question}
Имеются две случайных выборки \(X_1\), \ldots, \(X_{31}\) и \(Y_1\),
\ldots, \(Y_{41}\) из нормальных распределений. Известно, что
\(\sum_{i=1}^{31}(X_i - \bar X)^2 = 120\) и
\(\sum_{i=1}^{41}(Y_i - \bar Y)^2 = 400\). При проверке гипотезы о
равенстве дисперсий этих распределений значение тестовой статистики
может быть равно
\begin{answerlist}
  \item \(3.33\)
  \item \(2.5\)
  \item \(2\)
  \item \(2.52\)
  \item \(0.3\)
\end{answerlist}
\end{question}

\begin{solution}
\begin{answerlist}
  \item Тоже ересь
  \item Ураа!!!
  \item Не угадал
  \item Неверно
  \item Не туда!
\end{answerlist}
\end{solution}



\begin{question}
Имеется выборка из одного наблюдения \(X_1\). На основе этой выборки
тестируется гипотеза \(H_0\): \(X_1 \sim U[0;2]\) против альтернативной
гипотезы \(X_1 \sim U[1,3]\). Используется критерий следующего вида:
если \(X_1>a\), то \(H_0\) отвергается. Минимальная вероятность ошибки
первого рода достигается при \(a\) равном
\begin{answerlist}
  \item \(0.5\)
  \item \(1.5\)
  \item \(2\)
  \item \(1.9\)
  \item \(1\)
\end{answerlist}
\end{question}

\begin{solution}
\begin{answerlist}
  \item Неверно
  \item Тоже ересь
  \item Ураа!!!
  \item Не угадал
  \item Не туда!
\end{answerlist}
\end{solution}



\begin{question}
В школе три девятых класса: 9А, 9Б и 9В. В 9А классе — 50\% отличники,
в 9Б — 30\%, в 9В — 40\%. Если сначала равновероятно выбрать один из
трёх классов, а затем внутри класса равновероятно выбрать школьника, то
вероятность выбрать отличника равна
\begin{answerlist}
  \item \((3+4+5)/3\)
  \item \(0.3\)
  \item \(0.5\)
  \item \(0.27\)
  \item \(0.4\)
\end{answerlist}
\end{question}

\begin{solution}
\begin{answerlist}
  \item Bad answer :(
  \item Bad answer :(
  \item Bad answer :(
  \item Bad answer :(
  \item Good answer :)
\end{answerlist}
\end{solution}



\begin{question}
Величины \(X\) и \(Y\) одинаково распределены и равновероятно принимают
только два значения, \(-1\) и \(1\), при этом \(\P(Y=1|X=1)=0.4\).
Вероятность \(\P(Y=-1,X=1)\) равна
\begin{answerlist}
  \item \(1\)
  \item \(0.4\)
  \item \(0.6\)
  \item \(0.3\)
  \item \(0.5\)
\end{answerlist}
\end{question}

\begin{solution}
\begin{answerlist}
  \item Bad answer :(
  \item Bad answer :(
  \item Bad answer :(
  \item Good answer :)
  \item Bad answer :(
\end{answerlist}
\end{solution}







\newpage
\lfoot{Вариант $\rho$}
\setcounter{question}{0}


\putyourname
%\testtable

%\checktable


\begin{question}
Вероятность ошибки первого рода, \(\alpha\), и вероятность ошибки
второго рода, \(\beta\), всегда связаны соотношением
\begin{answerlist}
  \item \(\alpha\geq \beta\)
  \item \(\alpha\leq \beta\)
  \item \(\alpha+\beta=1\)
  \item \(\alpha+\beta \leq 1\)
  \item \(\alpha+\beta \geq 1\)
\end{answerlist}
\end{question}

\begin{solution}
\begin{answerlist}
  \item Неверно
  \item Неверно
  \item Неверно
  \item Неверно
  \item Неверно
\end{answerlist}
\end{solution}



\begin{question}
Пусть \(X = (X_1, \, \ldots, \, X_n)\) — случайная выборка из
равномерного распределения на отрезке \([0; \, \theta]\), где
\(\theta > 0\) — неизвестный параметр. Несмещённой является оценка
\begin{answerlist}
  \item \(2 \bar{X}\)
  \item \(\bar{X} / 2\)
  \item \(X_{(1)}\)
  \item \(X_{1}\)
  \item \(\bar{X}\)
\end{answerlist}
\end{question}

\begin{solution}
\begin{answerlist}
  \item Good answer :)
  \item Bad answer :(
  \item Bad answer :(
  \item Bad answer :(
  \item Bad answer :(
\end{answerlist}
\end{solution}



\begin{question}
Случайная выборка состоит из одного наблюдения \(X_1\), которое имеет
плотность распределения \[
f(x; \, \theta) = \begin{cases}
    \tfrac{1}{\theta^2} x e^{-x/\theta} & \text{при } x > 0,  \\
    0 & \text{при }x\leq 0,
  \end{cases}
\] где \(\theta > 0\). Чему равна оценка неизвестного параметра
\(\theta\), найденная с помощью метода максимального правдоподобия?
\begin{answerlist}
  \item \(X_1\)
  \item \(\ln X_1\)
  \item \(X_1 / 2\)
  \item \(\frac{X_1}{\ln X_1}\)
  \item \(1 / \ln X_1\)
\end{answerlist}
\end{question}

\begin{solution}
\begin{answerlist}
  \item Bad answer :(
  \item Bad answer :(
  \item Good answer :)
  \item Bad answer :(
  \item Bad answer :(
\end{answerlist}
\end{solution}



\begin{question}
Имеется выборка из одного наблюдения \(X_1\). На основе этой выборки
тестируется гипотеза \(H_0\): \(X_1 \sim U[0;2]\) против альтернативной
гипотезы \(X_1 \sim U[1,3]\). Используется критерий следующего вида:
если \(X_1>a\), то \(H_0\) отвергается. Если \(a=1.5\), то с ростом
\(a\)
\begin{answerlist}
  \item вероятности ошибок первого и второго рода от \(a\) не зависят
  \item вероятность ошибки второго рода падает, первого — растёт
  \item вероятности ошибок первого и второго рода растут
  \item вероятности ошибок первого и второго рода падают
  \item вероятность ошибки первого рода падает, второго — растёт
\end{answerlist}
\end{question}

\begin{solution}
\begin{answerlist}
  \item Не туда!
  \item Тоже ересь
  \item Неверно
  \item Не угадал
  \item Ураа!!!
\end{answerlist}
\end{solution}


\newpage

\begin{question}
Время подготовки студента к экзаменам и по статистике, и макроэкономике,
имеет нормальное распределение с неизвестными математическими ожиданиями
и дисперсиями. По \(10\) наблюдениям Вениамин получил оценку
стандартного отклонения времени подготовки к статистике равную \(5\)
часам. Оценка стандартного отклонения времени подготовки к
макроэкономике, рассчитанная по \(20\) наблюдениям, оказалась равной
\(2\). Тестовая статистика при проверке гипотезы о равенстве дисперсий
может быть равна
\begin{answerlist}
  \item \(0.4\)
  \item \(0.16\)
  \item \(12.5\)
  \item \(0.8\)
  \item \(2.5\)
\end{answerlist}
\end{question}

\begin{solution}
\begin{answerlist}
  \item Bad answer :(
  \item Good answer :)
  \item Bad answer :(
  \item Bad answer :(
  \item Bad answer :(
\end{answerlist}
\end{solution}



\begin{question}
При построения доверительного интервала для разности математических
ожиданий в двух нормальных независимых выборках размером \(m\) и \(n\) в
случае равных известных дисперсий используется распределение
\begin{answerlist}
  \item \(F_{m-1, n-1}\)
  \item \(F_{m,n}\)
  \item \(t_{m+n}\)
  \item \(t_{m+n-2}\)
  \item \(\cN(0, 1)\)
\end{answerlist}
\end{question}

\begin{solution}
\begin{answerlist}
  \item Bad answer :(
  \item Bad answer :(
  \item Bad answer :(
  \item Bad answer :(
  \item Good answer :)
\end{answerlist}
\end{solution}



\begin{question}
Случайная выборка состоит из одного наблюдения \(X_1\), которое имеет
плотность распределения \[
    f(x; \, \theta) = \begin{cases}
                          \frac{1}{\theta}x^{-1 + \frac{1}{\theta}} & \text{при } x \in (0;\,1),  \\
                          0 & \text{при }x \not\in (0;\,1). \\
                        \end{cases}
\] Оценка параметра \(\theta\), найденная с помощью метода максимального
правдоподобия, равна
\begin{answerlist}
  \item \(X_1\)
  \item \(-\ln X_1\)
  \item \(\ln X_1\)
  \item \(-X_1\)
  \item \(\frac{1}{\ln X_1}\)
\end{answerlist}
\end{question}

\begin{solution}
\begin{answerlist}
  \item Тоже ересь
  \item Ураа!!!
  \item Неверно
  \item Не туда!
  \item Не угадал
\end{answerlist}
\end{solution}



\begin{question}
Совместная функция плотности пары случайных величин \(X\) и \(Y\) имеет
вид \[
f(x,y)=\begin{cases}
c (2x+y), \; \text{ если } x\in[0;2], y\in [0;2] \\
0, \; \text{ иначе}
\end{cases}
\]

Константа \(c\) равна
\begin{answerlist}
  \item \(1/8\)
  \item \(8\)
  \item \(12\)
  \item \(1/12\)
  \item \(1/6\)
\end{answerlist}
\end{question}

\begin{solution}
\begin{answerlist}
  \item Bad answer :(
  \item Bad answer :(
  \item Bad answer :(
  \item Good answer :)
  \item Bad answer :(
\end{answerlist}
\end{solution}



\begin{question}
Функция правдоподобия, построенная по случайной выборке \(X_1\), \ldots,
\(X_n\) из распределения с функцией плотности
\(f(x)=(\theta+1)x^{\theta}\) при \(x\in [0;1]\) имеет вид
\begin{answerlist}
  \item \((\theta+1)^{\sum x_i}\)
  \item \((\theta+1)^n\prod x_i^{\theta}\)
  \item \(\sum (\theta+1)x_i^{\theta}\)
  \item \((\sum x_i)^{\theta}\)
  \item \((\theta+1)x^{n\theta}\)
\end{answerlist}
\end{question}

\begin{solution}
\begin{answerlist}
  \item Неверно
  \item Неверно
  \item Неверно
  \item Отлично
  \item Неверно
\end{answerlist}
\end{solution}



\begin{question}
События A, B и C независимы в совокупности, если
\begin{answerlist}
  \item \(\P(A|B) = \P(A), \P(A|C) = \P(A)\)
  \item \(\P(ABC) = \P(A) \P(B) \P(C)\)
  \item \(\P(A|B) = \P(A), \P(A|C) = \P(A), \P(B|C) = \P(B)\)
  \item \(\P(A\cap B) = \P(A)\P(B), \P(A\cap C) = \P(A)\P(C), \P(B\cap C) = \P(B)\P(C)\)
  \item \(\P(A \cap B \cap C) = 0\)
\end{answerlist}
\end{question}

\begin{solution}
\begin{answerlist}
  \item Bad answer :(
  \item Bad answer :(
  \item Bad answer :(
  \item Bad answer :(
  \item Bad answer :(
\end{answerlist}
\end{solution}


\newpage

\begin{question}
Компоненты вектора \(X=(X_{1},X_{2},X_{3})\) имеют совместное нормальное
распределение: \[
X\sim\cN \left(
\begin{bmatrix}1\\1\\1\end{bmatrix},
\begin{bmatrix}3 & 0.5 & 0.5\\0.5 & 2 & 0.5\\0.5 & 0.5 & 3\end{bmatrix}
\right).
\] Вероятность \(\P(X_{1}>X_{2}+X_{3} )\) равна
\begin{answerlist}
  \item \(0.688\)
  \item \(0.369\)
  \item \(0.215\)
  \item \(0.701\)
  \item \(0.593\)
\end{answerlist}
\end{question}

\begin{solution}
\begin{answerlist}
  \item Bad answer :(
  \item Good answer :)
  \item Bad answer :(
  \item Bad answer :(
  \item Bad answer :(
\end{answerlist}
\end{solution}



\begin{question}
Время подготовки студента к экзаменам и по статистике, и макроэкономике,
имеет нормальное распределение с неизвестными математическими ожиданиями
и дисперсиями. По \(10\) наблюдениям Вениамин получил оценку
стандартного отклонения времени подготовки к статистике равную \(5\)
часам. Оценка стандартного отклонения времени подготовки к
макроэкономике, рассчитанная по \(20\) наблюдениям, оказалась равной
\(2\). Тестовая статистика при проверке гипотезы о равенстве дисперсий
может быть равна
\begin{answerlist}
  \item \(0.16\)
  \item \(0.4\)
  \item \(12.5\)
  \item \(2.5\)
  \item \(0.8\)
\end{answerlist}
\end{question}

\begin{solution}
\begin{answerlist}
  \item Good answer :)
  \item Bad answer :(
  \item Bad answer :(
  \item Bad answer :(
  \item Bad answer :(
\end{answerlist}
\end{solution}



\begin{question}
Известно, что \(\E(X)=-1\), \(\E(Y)=2\), \(\Var(X)=4\), \(\Var(Y)=9\),
\(\Cov(X,Y)=-3\). Ковариация \(\Cov(aX, (1-a)Y)\) минимальна при \(a\)
равном
\begin{answerlist}
  \item \(3/12\)
  \item \(0\)
  \item \(1/2\)
  \item \(-1/4\)
  \item \(2/3\)
\end{answerlist}
\end{question}

\begin{solution}
\begin{answerlist}
  \item Bad answer :(
  \item Bad answer :(
  \item Good answer :)
  \item Bad answer :(
  \item Bad answer :(
\end{answerlist}
\end{solution}



\begin{question}
Величины \(X_1, \, \ldots, \, X_n\) — случайная выборка из
распределения Бернулли с параметром \(p \in (0;\,1)\). Информация Фишера
о параметре \(p\), заключенная в одном наблюдении, равна
\begin{answerlist}
  \item \(\frac{1}{p(1-p)}\)
  \item \(\frac{1}{p}\)
  \item \(p(1-p)\)
  \item \(1 - p\)
  \item \(p\)
\end{answerlist}
\end{question}

\begin{solution}
\begin{answerlist}
  \item Ураа!!!
  \item Не угадал
  \item Неверно
  \item Тоже ересь
  \item Не туда!
\end{answerlist}
\end{solution}



\begin{question}
Требуется проверить гипотезу о равенстве математических ожиданий по
независимым нормальным выборкам размером 33 и 16 наблюдений. Истинные
дисперсии по обеим выборкам известны, совпадают и равны 196. Разница
выборочных средних равна 1. Тестовая статистика может быть равна
\begin{answerlist}
  \item \(1/2\)
  \item \(1/7\)
  \item \(1/4\)
  \item \(1/14\)
  \item \(1/49\)
\end{answerlist}
\end{question}

\begin{solution}
\begin{answerlist}
  \item Bad answer :(
  \item Bad answer :(
  \item Good answer :)
  \item Bad answer :(
  \item Bad answer :(
\end{answerlist}
\end{solution}


\newpage

\begin{question}
Оценка \(\hat \theta_n\) называется эффективной оценкой параметра
\(\theta\) в классе оценок \(K\), если
\begin{answerlist}[2]
  \item \(\Var(\hat \theta_n)=(\theta)^2/n\)
  \item \(\E((\hat \theta_n - \theta)^2) \stackrel{}{\to} 0\) при
\(n\stackrel{}{\to} \infty\)
  \item \(\hat \theta_n \stackrel{\P}{\to} \theta\) при
\(n\stackrel{}{\to} \infty\)
  \item \(\E(\hat \theta_n)=\theta\)
  \item \(\E((\hat \theta_n - \theta)^2) \leq \E((\tilde \theta - \theta)^2)\)
для всех \(\tilde \theta \in K\)
\end{answerlist}
\end{question}

\begin{solution}
\begin{answerlist}
  \item Bad answer :(
  \item Bad answer :(
  \item Bad answer :(
  \item Bad answer :(
  \item Good answer :)
\end{answerlist}
\end{solution}



\begin{question}
Случайная величина \(X\) равномерно распределена на отрезке от \(-2\) до
\(2\). Вероятность \(\P(X^2>0.64)\) равна
\begin{answerlist}
  \item \(0.8\)
  \item \(0.2\)
  \item \(0.1\)
  \item \(0.6\)
  \item \(\frac{1}{2\sqrt{2}}\)
\end{answerlist}
\end{question}

\begin{solution}
\begin{answerlist}
  \item Bad answer :(
  \item Bad answer :(
  \item Bad answer :(
  \item Good answer :)
  \item Bad answer :(
\end{answerlist}
\end{solution}



\begin{question}
Пусть \(X = (X_1, \ldots , X_n)\) — случайная выборка из
экспоненциального распределения с плотностью \[
f(x; \theta) =
\begin{cases}
\frac{1}{\theta}\exp(-\frac{x}{\theta}) \text{ при } x \geq 0,  \\
0 \text{ при } x < 0.
\end{cases}
\] Информация Фишера \(I_n(p)\) равна:
\begin{answerlist}
  \item \(\frac{\theta}{n}\)
  \item \(\frac{n}{\theta^2}\)
  \item \(n \theta^2\)
  \item \(\frac{\theta^2}{n}\)
  \item \(\frac{n}{\theta}\)
\end{answerlist}
\end{question}

\begin{solution}
\begin{answerlist}
  \item Неверно
  \item Отлично
  \item Неверно
  \item Неверно
  \item Неверно
\end{answerlist}
\end{solution}



\begin{question}
Среди 100 случайно выбранных ацтеков 20 платят дань Кулуакану, а 80 ---
Аскапоцалько. Соответственно, оценка доли ацтеков, платящих дань
Кулуакану, равна \(\hat{p}=0.2\). Разумная оценка стандартного
отклонения случайной величины \(\hat{p}\) равна
\begin{answerlist}
  \item \(0.16\)
  \item \(1.6\)
  \item \(0.04\)
  \item \(0.016\)
  \item \(0.4\)
\end{answerlist}
\end{question}

\begin{solution}
\begin{answerlist}
  \item Неверно
  \item Неверно
  \item Отлично
  \item Неверно
  \item Неверно
\end{answerlist}
\end{solution}



\begin{question}
Известно, что \(\E(X)=-1\), \(\E(Y)=2\), \(\Var(X)=4\), \(\Var(Y)=9\),
\(\Cov(X,Y)=-3\). Ожидание \(\E(X^2-Y^2)\) равно
\begin{answerlist}
  \item \(-4\)
  \item 8
  \item 4
  \item \(-8\)
  \item 0
\end{answerlist}
\end{question}

\begin{solution}
\begin{answerlist}
  \item Bad answer :(
  \item Bad answer :(
  \item Bad answer :(
  \item Good answer :)
  \item Bad answer :(
\end{answerlist}
\end{solution}


\newpage

\begin{question}
По 10 наблюдениям проверяется гипотеза \(H_0: \; \mu=10\) против
\(H_a: \; \mu \neq 10\) на выборке из нормального распределения с
неизвестной дисперсией. Величина
\(\sqrt{n}\cdot (\bar{X}-\mu)/\hat{\sigma}\) оказалась равной \(1\).
P-значение примерно равно
\begin{answerlist}
  \item \(0.17\)
  \item \(0.34\)
  \item \(0.32\)
  \item \(0.16\)
  \item \(0.83\)
\end{answerlist}
\end{question}

\begin{solution}
\begin{answerlist}
  \item Неверно
  \item Неверно
  \item Неверно
  \item Отлично
  \item Неверно
\end{answerlist}
\end{solution}



\begin{question}
Известно, что \(\E(X)=-1\), \(\E(Y)=2\), \(\Var(X)=4\), \(\Var(Y)=9\),
\(\Cov(X,Y)=-3\). Ковариация \(\Cov(X+2Y, 2X+3)\) равна
\begin{answerlist}
  \item \(-4\)
  \item \(1\)
  \item \(4\)
  \item \(0\)
  \item \(-1\)
\end{answerlist}
\end{question}

\begin{solution}
\begin{answerlist}
  \item Good answer :)
  \item Bad answer :(
  \item Bad answer :(
  \item Bad answer :(
  \item Bad answer :(
\end{answerlist}
\end{solution}



\begin{question}
Пусть \(X_1\), \ldots, \(X_n\) --- выборка объема \(n\) из равномерного
на \([0, \theta]\) распределения. Состоятельной оценкой параметра
\(\theta\) является:
\begin{answerlist}
  \item \(X_(n-1)\)
  \item \(\frac{n}{n+1} X_{(n-1)}\)
  \item \(\frac{n^2}{n^2-n+3} X_{(n-3)}\)
  \item все перечисленные случайные величины
  \item \(X_(n)\)
\end{answerlist}
\end{question}

\begin{solution}
\begin{answerlist}
  \item Неверно
  \item Неверно
  \item Неверно
  \item Отлично
  \item Неверно
\end{answerlist}
\end{solution}



\begin{question}
Величина \(X\) имеет биномиальное распределение с параметрами \(n\) и
\(p\). Дисперсия величины \(X\) максимальна при \(p\) равном
\begin{answerlist}
  \item \(0.9\)
  \item \(0.5\)
  \item \(0.75\)
  \item \(0.2\)
  \item \(0.25\)
\end{answerlist}
\end{question}

\begin{solution}
\begin{answerlist}
  \item Bad answer :(
  \item Good answer :)
  \item Bad answer :(
  \item Bad answer :(
  \item Bad answer :(
\end{answerlist}
\end{solution}


\newpage

\begin{question}
Дисперсия случайной величины \(Y\) равна
\begin{answerlist}
  \item \(-1\)
  \item \(2/3\)
  \item \(1\)
  \item \(1/3\)
  \item \(0\)
\end{answerlist}
\end{question}

\begin{solution}
\begin{answerlist}
  \item Bad answer :(
  \item Good answer :)
  \item Bad answer :(
  \item Bad answer :(
  \item Bad answer :(
\end{answerlist}
\end{solution}



\begin{question}
Николай Коперник подбросил бутерброд 200 раз. Бутерброд упал маслом вниз
95 раз, а маслом вверх — 105 раз. Значение критерия \(\chi^2\) Пирсона
для проверки гипотезы о равной вероятности данных событий равно
\begin{answerlist}
  \item 7.5
  \item 0.5
  \item 2.5
  \item 0.75
  \item 0.25
\end{answerlist}
\end{question}

\begin{solution}
\begin{answerlist}
  \item Неверно
  \item Отлично
  \item Неверно
  \item Неверно
  \item Неверно
\end{answerlist}
\end{solution}



\begin{question}
Имеются две случайных выборки \(X_1\), \ldots, \(X_{31}\) и \(Y_1\),
\ldots, \(Y_{41}\) из нормальных распределений. Известно, что
\(\sum_{i=1}^{31}(X_i - \bar X)^2 = 120\) и
\(\sum_{i=1}^{41}(Y_i - \bar Y)^2 = 400\). При проверке гипотезы о
равенстве дисперсий этих распределений значение тестовой статистики
может быть равно
\begin{answerlist}
  \item \(3.33\)
  \item \(2.5\)
  \item \(2\)
  \item \(2.52\)
  \item \(0.3\)
\end{answerlist}
\end{question}

\begin{solution}
\begin{answerlist}
  \item Тоже ересь
  \item Ураа!!!
  \item Не угадал
  \item Неверно
  \item Не туда!
\end{answerlist}
\end{solution}



\begin{question}
Имеется выборка из одного наблюдения \(X_1\). На основе этой выборки
тестируется гипотеза \(H_0\): \(X_1 \sim U[0;2]\) против альтернативной
гипотезы \(X_1 \sim U[1,3]\). Используется критерий следующего вида:
если \(X_1>a\), то \(H_0\) отвергается. Минимальная вероятность ошибки
первого рода достигается при \(a\) равном
\begin{answerlist}
  \item \(0.5\)
  \item \(1.5\)
  \item \(2\)
  \item \(1.9\)
  \item \(1\)
\end{answerlist}
\end{question}

\begin{solution}
\begin{answerlist}
  \item Неверно
  \item Тоже ересь
  \item Ураа!!!
  \item Не угадал
  \item Не туда!
\end{answerlist}
\end{solution}



\begin{question}
В школе три девятых класса: 9А, 9Б и 9В. В 9А классе — 50\% отличники,
в 9Б — 30\%, в 9В — 40\%. Если сначала равновероятно выбрать один из
трёх классов, а затем внутри класса равновероятно выбрать школьника, то
вероятность выбрать отличника равна
\begin{answerlist}
  \item \((3+4+5)/3\)
  \item \(0.3\)
  \item \(0.5\)
  \item \(0.27\)
  \item \(0.4\)
\end{answerlist}
\end{question}

\begin{solution}
\begin{answerlist}
  \item Bad answer :(
  \item Bad answer :(
  \item Bad answer :(
  \item Bad answer :(
  \item Good answer :)
\end{answerlist}
\end{solution}



\begin{question}
Величины \(X\) и \(Y\) одинаково распределены и равновероятно принимают
только два значения, \(-1\) и \(1\), при этом \(\P(Y=1|X=1)=0.4\).
Вероятность \(\P(Y=-1,X=1)\) равна
\begin{answerlist}
  \item \(1\)
  \item \(0.4\)
  \item \(0.6\)
  \item \(0.3\)
  \item \(0.5\)
\end{answerlist}
\end{question}

\begin{solution}
\begin{answerlist}
  \item Bad answer :(
  \item Bad answer :(
  \item Bad answer :(
  \item Good answer :)
  \item Bad answer :(
\end{answerlist}
\end{solution}



 

\end{document}
