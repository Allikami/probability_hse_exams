
\begin{question}
Имеется выборка из одного наблюдения \(X_1\). На основе этой выборки
тестируется гипотеза \(H_0\): \(X_1 \sim U[0;2]\) против альтернативной
гипотезы \(X_1 \sim U[1,3]\). Используется критерий следующего вида:
если \(X_1>a\), то \(H_0\) отвергается. Если \(a=1.5\), то с ростом
\(a\)
\begin{answerlist}
  \item вероятности ошибок первого и второго рода от \(a\) не зависят
  \item вероятность ошибки второго рода падает, первого — растёт
  \item вероятности ошибок первого и второго рода растут
  \item вероятности ошибок первого и второго рода падают
  \item вероятность ошибки первого рода падает, второго — растёт
\end{answerlist}
\end{question}

\begin{solution}
\begin{answerlist}
  \item Не туда!
  \item Тоже ересь
  \item Неверно
  \item Не угадал
  \item Ураа!!!
\end{answerlist}
\end{solution}

