
\begin{question}
В коробке 10 купюр — тысячные, пятисотенные и сотенные. Случайным
образом достаются две купюры. Случайная величина \(X_1\) принимает
значение, равное номиналу первой купюры, случайная величина \(X_2\) --
второй. Величины \(X_1\) и \(X_2\)
\begin{answerlist}
  \item положительно коррелированы и независимы
  \item некоррелированы и зависимы
  \item некоррелированы и независимы
  \item отрицательно коррелированы
  \item положительно коррелированы
\end{answerlist}
\end{question}

\begin{solution}
\begin{answerlist}
  \item Не туда!
  \item Неверно
  \item Тоже ересь
  \item Ураа!!!
  \item Не угадал
\end{answerlist}
\end{solution}

