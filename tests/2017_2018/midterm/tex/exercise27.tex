
\begin{question}
У Васи есть пять кнопок, генерирующих целые числа от 1 до 6. Три
работают как честные кубики, одна — с увеличенной вероятностью
выпадения 6 (она выпадает с веростностью 0.5, остальные ---
равновероятно), одна — с увеличенной вероятностью выпадения 1 (она
выпадает с вероятностью 0.5, остальные — равновероятно). Вася нажимает
на случайную кнопку. После нажатия на случайную кнопку выпала 6.
Условная вероятность того, что это была кнопка «честный кубик» равна
\begin{answerlist}
  \item 1/2
  \item 4/11
  \item 5/11
  \item 8/11
  \item 6/11
\end{answerlist}
\end{question}

\begin{solution}
\begin{answerlist}
  \item Bad answer :(
  \item Bad answer :(
  \item Good answer :)
  \item Bad answer :(
  \item Bad answer :(
\end{answerlist}
\end{solution}

