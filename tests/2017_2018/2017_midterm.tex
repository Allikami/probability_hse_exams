% вопросы от Иры:
\element{midterm_2017_rejected}{ % в фигурных скобках название группы вопросов
 \AMCcompleteMulti
  \begin{questionmult}{1ira} % тип вопроса (questionmult — множественный выбор) и в фигурных — номер вопроса
  Для случайной величины $X \sim \cN(\mu_X, \sigma^2_X)$ вероятность $\P(X - \mu_x > 5\sigma_X)$  примерно равна
 \begin{multicols}{3} % располагаем ответы в 3 колонки
   \begin{choices} % опция [o] не рандомизирует порядок ответов
      \correctchoice{$0$}
      \wrongchoice{$1/5$}
      \wrongchoice{$0.05$}
      \wrongchoice{$0.95$}
      \wrongchoice{$0.5$}
      \end{choices}
  \end{multicols}
  \end{questionmult}
}


\element{midterm_2017}{ % в фигурных скобках название группы вопросов
 \AMCcompleteMulti
  \begin{questionmult}{2ira} % тип вопроса (questionmult — множественный выбор) и в фигурных — номер вопроса
Двумерная случайная величина $(X, Y)$ равномерно распределена в треугольнике ограниченном линиями $x=0$, $y=0$ и $y+2x=4$. Значение функции плотности $f_{X,Y}(1,1)$ равно
 \begin{multicols}{3} % располагаем ответы в 3 колонки
   \begin{choices} % опция [o] не рандомизирует порядок ответов
     \correctchoice{$0.25$}
     \wrongchoice{$0.125$}
     \wrongchoice{$1$}
     \wrongchoice{$0.5$}
     \wrongchoice{$\frac{1}{\sqrt{2\pi}}\exp(-0.5)$}
      \end{choices}
  \end{multicols}
  \end{questionmult}
}


\element{midterm_2017_rejected}{ % в фигурных скобках название группы вопросов
 \AMCcompleteMulti
  \begin{questionmult}{3ira} % тип вопроса (questionmult — множественный выбор) и в фигурных — номер вопроса
  Двумерная функция распределения $F_{X,Y}(x,y)$ может \textbf{НЕ} удовлетворять свойству
 \begin{multicols}{3} % располагаем ответы в 3 колонки
   \begin{choices} % опция [o] не рандомизирует порядок ответов
      \correctchoice{$F_{X,Y}(x,y)$ не убывает по $x$}
      \wrongchoice{$\lim_{x,y \to +\infty} F_{X,Y}(x,y) = 1$}
      \wrongchoice{$0 \leq F_{X,Y}(x, y)\leq 1$}
      \wrongchoice{функция $F_{X,Y}(x, y)$ непрерывна}
      \wrongchoice{$\lim_{y \to +\infty} F_{X,Y}(x,y) = F_X(x)$}
      \end{choices}
  \end{multicols}
  \end{questionmult}
}


\element{midterm_2017}{ % в фигурных скобках название группы вопросов
 \AMCcompleteMulti
  \begin{questionmult}{4ira} % тип вопроса (questionmult — множественный выбор) и в фигурных — номер вопроса
  Случайные величины $X$ и $Y$ независимы и нормально распределены с параметрами $\E(X)=2$, $\Var(X)=3$, $\E(Y)=1$, $\Var(Y)=4$. Вероятность $\P(X+Y<3)$ равна
 \begin{multicols}{3} % располагаем ответы в 3 колонки
   \begin{choices} % опция [o] не рандомизирует порядок ответов
      \correctchoice{$0.5$}
      \wrongchoice{$0.05$}
      \wrongchoice{$3/7$}
      \wrongchoice{$2/7$}
      \wrongchoice{$0.995$}
      \end{choices}
  \end{multicols}
  \end{questionmult}
}



\element{midterm_2017}{ % в фигурных скобках название группы вопросов
 \AMCcompleteMulti
  \begin{questionmult}{5ira} % тип вопроса (questionmult — множественный выбор) и в фигурных — номер вопроса
  Ковариационной матрицей может являться матрица
 \begin{multicols}{3} % располагаем ответы в 3 колонки
   \begin{choices} % опция [o] не рандомизирует порядок ответов
      \correctchoice{$\begin{pmatrix} 9 & 7 \\ 7 & 6 \\ \end{pmatrix}$}
      \wrongchoice{$\begin{pmatrix} 1 & 2 \\ 1 & 2 \\ \end{pmatrix}$}
      \wrongchoice{$\begin{pmatrix} -1 & 2 \\ 2 & 10 \\ \end{pmatrix}$}
      \wrongchoice{$\begin{pmatrix} 1 & 2 \\ 2 & 1 \\ \end{pmatrix}$}
      \wrongchoice{$\begin{pmatrix} 1 & 4 \\ 4 & 9 \\ \end{pmatrix}$}
      \end{choices}
  \end{multicols}
  \end{questionmult}
}


\element{midterm_2017_rejected}{ % в фигурных скобках название группы вопросов
 \AMCcompleteMulti
  \begin{questionmult}{6ira} % тип вопроса (questionmult — множественный выбор) и в фигурных — номер вопроса
Совместное распределение дискретных случайных величин $X$ и $Y$ задано таблицей:

\begin{tabular}{cccc}
\toprule
 & $Y=-2$ & $Y=0$ & $Y=1$ \\
\midrule
$X=3$ & $0.3$ & $0.1$ & $0.2$  \\
$X=6$ & $0.1$ & $0.2$ & $0.1$ \\
\bottomrule
\end{tabular}

Условное ожидание $\E(X|Y=-2)$ равно

\begin{multicols}{3} % располагаем ответы в 3 колонки
   \begin{choices} % опция [o] не рандомизирует порядок ответов
      \correctchoice{$3.75$}
      \wrongchoice{$3.5$}
      \wrongchoice{$3.25$}
      \wrongchoice{$4.2$}
      \wrongchoice{$3.(3)$}
      \end{choices}
  \end{multicols}
  \end{questionmult}
}



\element{midterm_2017_rejected}{ % в фигурных скобках название группы вопросов
 \AMCcompleteMulti
  \begin{questionmult}{7ira} % тип вопроса (questionmult — множественный выбор) и в фигурных — номер вопроса
У пары случайных величин $X$, $Y$ существует совместная функция плотности $f(x,y)$ и условная функция плотности $f(x|y)$. Условную дисперсию $\Var(X|Y)$ можно найти по формуле
 \begin{multicols}{2} % располагаем ответы в 3 колонки
   \begin{choices} % опция [o] не рандомизирует порядок ответов
      \correctchoice{$\int_{-\infty}^{+\infty} x^2 f(x|Y) \, dx - (\E(X|Y))^2$}
      \wrongchoice{$\int_{-\infty}^{+\infty} (x - \E(X))^2 f(x|Y) \, dx$}
      \wrongchoice{$\int_{-\infty}^{+\infty} x^2 f(x|Y) \, dx$}
      \wrongchoice{$\int_{-\infty}^{+\infty} (x - \E(X|Y))^2 \, dx$}
      \wrongchoice{$\left(\int_{-\infty}^{+\infty} x f(x|Y) \, dx\right)^2 - (\E(X|Y))^2$}
      \end{choices}
  \end{multicols}
  \end{questionmult}
}



\element{midterm_2017}{ % в фигурных скобках название группы вопросов
 \AMCcompleteMulti
  \begin{questionmult}{8ira} % тип вопроса (questionmult — множественный выбор) и в фигурных — номер вопроса
  Случайная величина $X$ принимает равновероятно целые значение от $-5$ до $5$ включительно. Случайная величина $Y$ принимает равновероятно целые значение от $-1$ до $1$ включительно. Величины $X$ и $Y$ независимы. Вероятность $\P(X+Y^2=2)$ равна
 \begin{multicols}{3} % располагаем ответы в 3 колонки
   \begin{choices} % опция [o] не рандомизирует порядок ответов
      \correctchoice{$1/11$}
      \wrongchoice{$1/33$}
      \wrongchoice{$2/33$}
      \wrongchoice{$5/33$}
      \wrongchoice{$1/5$}
      \end{choices}
  \end{multicols}
  \end{questionmult}
}

% кончились вопросы Иры

\element{midterm_2017}{ % в фигурных скобках название группы вопросов
 \AMCcompleteMulti
  \begin{questionmult}{1vania} % тип вопроса (questionmult — множественный выбор) и в фигурных — номер вопроса

Круг разделён на секторы с углом $\frac{\pi}{3}$. Один из них закрашен красным, один сектор — синим, остальные сектора — белым. Вася кидает дротики и всегда попадает в круг, все точки круга равновероятны. Вероятность того, что Вася попадёт в красный сектор, равна

 \begin{multicols}{3} % располагаем ответы в 3 колонки
   \begin{choices} % опция [o] не рандомизирует порядок ответов
      \correctchoice{1/6}
      \wrongchoice{1/4}
      \wrongchoice{$\pi / 6$}
      \wrongchoice{$\pi / 3$}
      \wrongchoice{не хватает данных}
      \end{choices}
  \end{multicols}
  \end{questionmult}
}


\element{midterm_2017}{ % в фигурных скобках название группы вопросов
 \AMCcompleteMulti
  \begin{questionmult}{2vania} % тип вопроса (questionmult — множественный выбор) и в фигурных — номер вопроса

Круг разделён на секторы с углом $\frac{\pi}{3}$. Один из них закрашен красным, один — синим, остальные — белым. Вася кидает дротики и всегда попадает в круг, все точки круга равновероятны. Пусть событие $A$ — попадание в красный сектор, $B$ — попадание в синий сектор. Эти события

 \begin{multicols}{2} % располагаем ответы в 3 колонки
   \begin{choices} % опция [o] не рандомизирует порядок ответов
      \correctchoice{несовместны}
      \wrongchoice{независимы}
      \wrongchoice{случаются с вероятностями 1/4}
      \wrongchoice{случаются с разными вероятностями}
      \wrongchoice{образуют полную группу событий}
      \end{choices}
  \end{multicols}
  \end{questionmult}
}


\element{midterm_2017}{ % в фигурных скобках название группы вопросов
 \AMCcompleteMulti
  \begin{questionmult}{3vania} % тип вопроса (questionmult — множественный выбор) и в фигурных — номер вопроса

Известно, что $\P(A \cap B) = 0.2$, $\P(A \cup B) = 0.6$, $\P(A) = 0.3$. Вероятность $\P(B)$ равна

 \begin{multicols}{3} % располагаем ответы в 3 колонки
   \begin{choices} % опция [o] не рандомизирует порядок ответов
      \correctchoice{0.5}
      \wrongchoice{0.3}
      \wrongchoice{0.1}
      \wrongchoice{0.6}
      \wrongchoice{не хватает данных}
      \end{choices}
  \end{multicols}
  \end{questionmult}
}



\element{midterm_2017_rejected}{ % в фигурных скобках название группы вопросов
 \AMCcompleteMulti
  \begin{questionmult}{4vania} % тип вопроса (questionmult — множественный выбор) и в фигурных — номер вопроса

В каком из этих случаев события $A$ и $B$ будут независимы?

 \begin{multicols}{2} % располагаем ответы в 3 колонки
   \begin{choices} % опция [o] не рандомизирует порядок ответов
      \correctchoice{ $\P(A \cap B) = 0.1$, $\P (A) = 0.5$, $\P(B) = 0.2$ }
      \wrongchoice{ $\P(A \cup B) = 0.2$, $\P (A) = 0.5$, $\P(B) = 0.4$ }
      \wrongchoice{ $\P(A \cup B) = 0.6$, $\P (A) = 0.5$, $\P(B) = 0.2$ }
      \wrongchoice{ $\P(A \cap B) = 0$, $\P (A) = 0.8$, $\P(B) = 0.1$ }
      \wrongchoice{ $\P(A \cap B) = 0.1$, $\P (A) = 0.5$, $\P(B) = 0.9$ }
      \end{choices}
  \end{multicols}
  \end{questionmult}
}



\element{midterm_2017}{ % в фигурных скобках название группы вопросов
 \AMCcompleteMulti
  \begin{questionmult}{5vania} % тип вопроса (questionmult — множественный выбор) и в фигурных — номер вопроса

В самолёте 200 пассажиров. Четверть пассажиров летит без багажа, половина из них — с рюкзаками. Среди пассажиров с багажом 55 человек летит с рюкзаками. Вероятность того, что случайно выбранный человек летит без рюкзака, равна

 \begin{multicols}{3} % располагаем ответы в 3 колонки
   \begin{choices} % опция [o] не рандомизирует порядок ответов
      \correctchoice{ 0.6 }
      \wrongchoice{ 0.4 }
      \wrongchoice{ 0.5 }
      \wrongchoice{ 0.45 }
      \wrongchoice{ 0.65 }
      \end{choices}
  \end{multicols}
  \end{questionmult}
}




\element{midterm_2017_rejected}{ % в фигурных скобках название группы вопросов
 \AMCcompleteMulti
  \begin{questionmult}{6vania} % тип вопроса (questionmult — множественный выбор) и в фигурных — номер вопроса

У Васи есть пять кнопок, генерирующих целые числа от 1 до 6. Три работают как честные кубики, одна — с увеличенной вероятностью выпадения 6 (она выпадает с веростностью 0.5, остальные — равновероятно), одна — с увеличенной вероятностью выпадения 1 (она выпадает с вероятностью 0.5, остальные — равновероятно). Вася нажимает на случайную кнопку. Число 6 выпадет с вероятностью
 \begin{multicols}{3} % располагаем ответы в 3 колонки
   \begin{choices} % опция [o] не рандомизирует порядок ответов
      \correctchoice{ 0.22 }
      \wrongchoice{ 0.12 }
      \wrongchoice{ 1/6 }
      \wrongchoice{ 1/4 }
      \wrongchoice{ 0.11 }
      \end{choices}
  \end{multicols}
  \end{questionmult}
}



\element{midterm_2017}{ % в фигурных скобках название группы вопросов
 \AMCcompleteMulti
  \begin{questionmult}{7vania} % тип вопроса (questionmult — множественный выбор) и в фигурных — номер вопроса

У Васи есть пять кнопок, генерирующих целые числа от 1 до 6. Три работают как честные кубики, одна — с увеличенной вероятностью выпадения 6 (она выпадает с веростностью 0.5, остальные — равновероятно), одна — с увеличенной вероятностью выпадения 1 (она выпадает с вероятностью 0.5, остальные — равновероятно). Вася нажимает на случайную кнопку. После нажатия на случайную кнопку выпала 6. Условная вероятность того, что это была кнопка «честный кубик» равна


 \begin{multicols}{3} % располагаем ответы в 3 колонки
   \begin{choices} % опция [o] не рандомизирует порядок ответов
      \correctchoice{ 5/11 }
      \wrongchoice{ 8/11 }
      \wrongchoice{ 1/2 }
      \wrongchoice{ 6/11 }
      \wrongchoice{ 4/11 }
      \end{choices}
  \end{multicols}
  \end{questionmult}
}






\element{midterm_2017}{ % в фигурных скобках название группы вопросов
 \AMCcompleteMulti
  \begin{questionmult}{8vania} % тип вопроса (questionmult — множественный выбор) и в фигурных — номер вопроса

События A, B и C независимы в совокупности, если

 \begin{multicols}{2} % располагаем ответы в 3 колонки
   \begin{choices} % опция [o] не рандомизирует порядок ответов
      \wrongchoice{ $\P(ABC) = \P(A) \P(B) \P(C)$ }
      \wrongchoice{ $\P(A\cap B) = \P(A)\P(B), \P(A\cap C) = \P(A)\P(C), \P(B\cap C) = \P(B)\P(C)$ }
      \wrongchoice{ $\P(A|B) = \P(A), \P(A|C) = \P(A)$ }
      \wrongchoice{ $\P(A|B) = \P(A), \P(A|C) = \P(A), \P(B|C) = \P(B)$ }
      \wrongchoice{ $\P(A \cap B \cap C) = 0$ }
      \end{choices}
  \end{multicols}
  \end{questionmult}
}


\element{midterm_2017_rejected}{ % в фигурных скобках название группы вопросов
 \AMCcompleteMulti
  \begin{questionmult}{1kolya} % тип вопроса (questionmult — множественный выбор) и в фигурных — номер вопроса

Известно, что $\E(X)=-1$, $\E(Y)=2$, $\Var(X)=4$, $\Var(Y)=9$, $\Cov(X,Y)=-3$. Ожидание $\E(X^2-Y^2)$ равно

 \begin{multicols}{3} % располагаем ответы в 3 колонки
   \begin{choices} % опция [o] не рандомизирует порядок ответов
      \correctchoice{$-8$}
      \wrongchoice{$-4$}
      \wrongchoice{0}
      \wrongchoice{4}
      \wrongchoice{8}
      \end{choices}
  \end{multicols}
  \end{questionmult}
}


\element{midterm_2017}{ % в фигурных скобках название группы вопросов
 \AMCcompleteMulti
  \begin{questionmult}{2kolya} % тип вопроса (questionmult — множественный выбор) и в фигурных — номер вопроса

Известно, что $\E(X)=-1$, $\E(Y)=2$, $\Var(X)=4$, $\Var(Y)=9$, $\Cov(X,Y)=-3$. Ожидание $\E((X-1)Y)$ равно

 \begin{multicols}{3} % располагаем ответы в 3 колонки
   \begin{choices} % опция [o] не рандомизирует порядок ответов
      \correctchoice{$-7$}
      \wrongchoice{$-8$}
      \wrongchoice{$-9$}
      \wrongchoice{$-6$}
      \wrongchoice{$-5$}
      \end{choices}
  \end{multicols}
  \end{questionmult}
}


\element{midterm_2017}{ % в фигурных скобках название группы вопросов
 \AMCcompleteMulti
  \begin{questionmult}{3kolya} % тип вопроса (questionmult — множественный выбор) и в фигурных — номер вопроса

Известно, что $\E(X)=-1$, $\E(Y)=2$, $\Var(X)=4$, $\Var(Y)=9$, $\Cov(X,Y)=-3$. Дисперсия $\Var(2X-Y+1)$ равна

 \begin{multicols}{3} % располагаем ответы в 3 колонки
   \begin{choices} % опция [o] не рандомизирует порядок ответов
      \correctchoice{$37$}
      \wrongchoice{31}
      \wrongchoice{34}
      \wrongchoice{$-31$}
      \wrongchoice{$24$}
      \end{choices}
  \end{multicols}
  \end{questionmult}
}



\element{midterm_2017}{ % в фигурных скобках название группы вопросов
 \AMCcompleteMulti
  \begin{questionmult}{4kolya} % тип вопроса (questionmult — множественный выбор) и в фигурных — номер вопроса

Известно, что $\E(X)=-1$, $\E(Y)=2$, $\Var(X)=4$, $\Var(Y)=9$, $\Cov(X,Y)=-3$. Ковариация $\Cov(X+2Y, 2X+3)$ равна

 \begin{multicols}{3} % располагаем ответы в 3 колонки
   \begin{choices} % опция [o] не рандомизирует порядок ответов
      \correctchoice{ $-4$ }
      \wrongchoice{ $4$ }
      \wrongchoice{ $1$ }
      \wrongchoice{ $-1$ }
      \wrongchoice{ $0$ }
      \end{choices}
  \end{multicols}
  \end{questionmult}
}



\element{midterm_2017}{ % в фигурных скобках название группы вопросов
 \AMCcompleteMulti
  \begin{questionmult}{5kolya} % тип вопроса (questionmult — множественный выбор) и в фигурных — номер вопроса

Известно, что $\E(X)=-1$, $\E(Y)=2$, $\Var(X)=4$, $\Var(Y)=9$, $\Cov(X,Y)=-3$. Корреляция $\Corr(X+Y, Y)$ равна

 \begin{multicols}{3} % располагаем ответы в 3 колонки
   \begin{choices} % опция [o] не рандомизирует порядок ответов
      \correctchoice{ $2/\sqrt{7}$ }
      \wrongchoice{ $-1/\sqrt{7}$ }
      \wrongchoice{ $-2/\sqrt{6}$ }
      \wrongchoice{ $1/\sqrt{6}$  }
      \wrongchoice{ $-3/\sqrt{6}$ }
      \end{choices}
  \end{multicols}
  \end{questionmult}
}




\element{midterm_2017}{ % в фигурных скобках название группы вопросов
 \AMCcompleteMulti
  \begin{questionmult}{6kolya} % тип вопроса (questionmult — множественный выбор) и в фигурных — номер вопроса

Известно, что $\E(X)=-1$, $\E(Y)=2$, $\Var(X)=4$, $\Var(Y)=9$, $\Cov(X,Y)=-3$. Из условия  $\E(aX+(1-a)Y)=0$ следует, что $a$ равно

 \begin{multicols}{3} % располагаем ответы в 3 колонки
   \begin{choices} % опция [o] не рандомизирует порядок ответов
      \correctchoice{ 2/3 }
      \wrongchoice{ 1/3 }
      \wrongchoice{ 1/2 }
      \wrongchoice{ 0 }
      \wrongchoice{ 1 }
      \end{choices}
  \end{multicols}
  \end{questionmult}
}



\element{midterm_2017}{ % в фигурных скобках название группы вопросов
 \AMCcompleteMulti
  \begin{questionmult}{7kolya} % тип вопроса (questionmult — множественный выбор) и в фигурных — номер вопроса

Известно, что $\E(X)=-1$, $\E(Y)=2$, $\Var(X)=4$, $\Var(Y)=9$, $\Cov(X,Y)=-3$. Дисперсия $\Var(aX+(1-a)Y)$ минимальна при $a$ равном

 \begin{multicols}{3} % располагаем ответы в 3 колонки
   \begin{choices} % опция [o] не рандомизирует порядок ответов
      \correctchoice{ 11/12 }
      \wrongchoice{ 7/12 }
      \wrongchoice{ 3/24 }
      \wrongchoice{ $-1/4$ }
      \wrongchoice{ $3/12$ }
      \end{choices}
  \end{multicols}
  \end{questionmult}
}






\element{midterm_2017_rejected}{ % в фигурных скобках название группы вопросов
 \AMCcompleteMulti
  \begin{questionmult}{8kolya} % тип вопроса (questionmult — множественный выбор) и в фигурных — номер вопроса

Известно, что $\E(X)=-1$, $\E(Y)=2$, $\Var(X)=4$, $\Var(Y)=9$, $\Cov(X,Y)=-3$. Ковариация $\Cov(aX, (1-a)Y)$ минимальна при $a$ равном

 \begin{multicols}{3} % располагаем ответы в 3 колонки
   \begin{choices} % опция [o] не рандомизирует порядок ответов
      \correctchoice{ $1/2$ }
      \wrongchoice{ $2/3$ }
      \wrongchoice{ $0$ }
      \wrongchoice{ $-1/4$ }
      \wrongchoice{ $3/12$ }
      \end{choices}
  \end{multicols}
  \end{questionmult}
}




\element{midterm_2017}{ % в фигурных скобках название группы вопросов
 \AMCcompleteMulti
  \begin{questionmult}{1dima} % тип вопроса (questionmult — множественный выбор) и в фигурных — номер вопроса

Случайная величина $\xi$ имеет распределение Бернулли с параметром $p$. Математическое ожидание $\E[\xi^2]$ равно
 \begin{multicols}{3} % располагаем ответы в 3 колонки
   \begin{choices} % опция [o] не рандомизирует порядок ответов
      \correctchoice{$p$}
      \wrongchoice{$p^2$}
      \wrongchoice{$1-p$}
      \wrongchoice{$0$}
      \wrongchoice{$p(1-p)$}
      \end{choices}
  \end{multicols}
  \end{questionmult}
}


\element{midterm_2017}{ % в фигурных скобках название группы вопросов
 \AMCcompleteMulti
  \begin{questionmult}{2dima} % тип вопроса (questionmult — множественный выбор) и в фигурных — номер вопроса
Случайная величина $\xi$ имеет биномиальное распределение с параметрами $n = 2$ и $p = 3/4$. Вероятность $\P(\xi = 0)$ равна
 \begin{multicols}{3} % располагаем ответы в 3 колонки
   \begin{choices} % опция [o] не рандомизирует порядок ответов
      \correctchoice{$1/16$}
      \wrongchoice{$3/4$}
      \wrongchoice{$9/16$}
      \wrongchoice{$1/2$}
      \wrongchoice{$3/4$}
      \end{choices}
  \end{multicols}
  \end{questionmult}
}


\element{midterm_2017}{ % в фигурных скобках название группы вопросов
 \AMCcompleteMulti
  \begin{questionmult}{3dima} % тип вопроса (questionmult — множественный выбор) и в фигурных — номер вопроса

Случайная величина $\xi$ имеет распределение Пуассона с параметром $\lambda$. Математическое ожидание $\E[\xi^2]$ равно
 \begin{multicols}{3} % располагаем ответы в 3 колонки
   \begin{choices} % опция [o] не рандомизирует порядок ответов
      \correctchoice{$\lambda(\lambda+1)$}
      \wrongchoice{$\lambda^2$}
      \wrongchoice{$\lambda(1 - \lambda)$}
      \wrongchoice{$e^{-\lambda}$}
      \wrongchoice{$\lambda$}
      \end{choices}
  \end{multicols}
  \end{questionmult}
}



\element{midterm_2017}{ % в фигурных скобках название группы вопросов
 \AMCcompleteMulti
  \begin{questionmult}{4dima} % тип вопроса (questionmult — множественный выбор) и в фигурных — номер вопроса
Количество сбоев системы SkyNet за сутки имеет распределение Пуассона. Среднее количество сбоев за сутки равно 4. Вероятность того, что за сутки произойдет не менее одного сбоя, равна
 \begin{multicols}{3} % располагаем ответы в 3 колонки
   \begin{choices} % опция [o] не рандомизирует порядок ответов
      \correctchoice{ $1- e^{-4}$ }
      \wrongchoice{ $e^4$ }
      \wrongchoice{ $1-e^4$ }
      \wrongchoice{ $\tfrac{1}{4!}e^{-4}$ }
      \wrongchoice{ $e^{-4}$ }
      \end{choices}
  \end{multicols}
  \end{questionmult}
}



\element{midterm_2017}{ % в фигурных скобках название группы вопросов
 \AMCcompleteMulti
  \begin{questionmult}{5dima} % тип вопроса (questionmult — множественный выбор) и в фигурных — номер вопроса

Случайная величина $\xi$ имеет равномерное распределение на отрезке $[0;\,4]$. Вероятность $\P(\{\xi \in [3;\,6]\})$ равна
 \begin{multicols}{3} % располагаем ответы в 3 колонки
   \begin{choices} % опция [o] не рандомизирует порядок ответов
      \correctchoice{ $1/4$ }
      \wrongchoice{ $1/2$ }
      \wrongchoice{ $3/4$ }
      \wrongchoice{ $\Phi(4) - \Phi(3)$  }
      \wrongchoice{ $3/6$ }
      \end{choices}
  \end{multicols}
  \end{questionmult}
}




\element{midterm_2017_rejected}{ % в фигурных скобках название группы вопросов
 \AMCcompleteMulti
  \begin{questionmult}{6dima} % тип вопроса (questionmult — множественный выбор) и в фигурных — номер вопроса

Случайная величина $\xi$ имеет равномерное распределение на отрезке $[0;\,4]$. Математическое ожидание $\E[\xi^2]$ равно
 \begin{multicols}{3} % располагаем ответы в 3 колонки
   \begin{choices} % опция [o] не рандомизирует порядок ответов
      \correctchoice{ $64/12$ }
      \wrongchoice{ $2$ }
      \wrongchoice{ $4$ }
      \wrongchoice{ $16/12$ }
      \wrongchoice{ $52/12$ }
      \end{choices}
  \end{multicols}
  \end{questionmult}
}



\element{midterm_2017}{ % в фигурных скобках название группы вопросов
 \AMCcompleteMulti
  \begin{questionmult}{7dima} % тип вопроса (questionmult — множественный выбор) и в фигурных — номер вопроса

Случайная величина $\xi$ имеет показательное (экспоненциальное) распределение с параметром $\lambda$. Математическое ожидание $\E[\xi^2]$ равно
 \begin{multicols}{3} % располагаем ответы в 3 колонки
   \begin{choices} % опция [o] не рандомизирует порядок ответов
      \correctchoice{ $2/\lambda^2$ }
      \wrongchoice{  $1/\lambda$ }
      \wrongchoice{ $1/\lambda^2$ }
      \wrongchoice{ $\lambda^2$ }
      \wrongchoice{ $1/\lambda^2 - 1/ \lambda$ }
      \end{choices}
  \end{multicols}
  \end{questionmult}
}






\element{midterm_2017}{ % в фигурных скобках название группы вопросов
 \AMCcompleteMulti
  \begin{questionmult}{8dima} % тип вопроса (questionmult — множественный выбор) и в фигурных — номер вопроса

Случайная величина $\xi$ имеет стандартное нормальное распределение. Вероятность $\P(\{\xi \in [-1; \, 2]\})$ равна

 \begin{multicols}{3} % располагаем ответы в 3 колонки
   \begin{choices} % опция [o] не рандомизирует порядок ответов
      \correctchoice{ $\int_{-1}^{2}\tfrac{1}{\sqrt{2\pi}}e^{-x^2 / 2}\,dx$ }
      \wrongchoice{ $\int_{-1}^{2}\tfrac{1}{\sqrt{2\pi}}e^{x^2}\,dx$ }
      \wrongchoice{ $\int_{-1}^{2}\tfrac{1}{\sqrt{2\pi}}e^{-x^2}\,dx$ }
      \wrongchoice{ $\int_{-1}^{2}\tfrac{1}{\sqrt{2\pi}}e^{x^2 / 2}\,dx$ }
      \wrongchoice{ $\int_{-1}^{2}\tfrac{1}{2\pi}e^{-x^2 / 2}\,dx$ }
      \end{choices}
  \end{multicols}
  \end{questionmult}
}



\element{midterm_2017}{ % в фигурных скобках название группы вопросов
 \AMCcompleteMulti
  \begin{questionmult}{1borya} % тип вопроса (questionmult — множественный выбор) и в фигурных — номер вопроса
Математическое ожидание величины $X$ равно 2, а дисперсия равна 6. Вероятность $\P(|2-X|\leq 10)$ принадлежит диапазону
 \begin{multicols}{3} % располагаем ответы в 3 колонки
   \begin{choices} % опция [o] не рандомизирует порядок ответов
      \correctchoice{$[0.94; 1]$}
      \wrongchoice{$[0.6; 0.8]$}
      \wrongchoice{$[0; 0.06]$}
      \wrongchoice{$[0.2;0.4]$}
      \wrongchoice{$[0.99;1]$}
      \end{choices}
  \end{multicols}
  \end{questionmult}
}


\element{midterm_2017}{ % в фигурных скобках название группы вопросов
 \AMCcompleteMulti
  \begin{questionmult}{2borya} % тип вопроса (questionmult — множественный выбор) и в фигурных — номер вопроса
Математическое ожидание величины $X$ равно 2, а дисперсия равна 6. Вероятность $\P(X^2 \geq 100)$ лежит в диапазоне
 \begin{multicols}{3} % располагаем ответы в 3 колонки
   \begin{choices} % опция [o] не рандомизирует порядок ответов
      \correctchoice{$[0;0.1]$}
      \wrongchoice{$[0.1;0.2]$}
      \wrongchoice{$[0;0.01]$}
      \wrongchoice{$[0.9;1]$}
      \wrongchoice{$[0.99;1]$}
      \end{choices}
  \end{multicols}
  \end{questionmult}
}


\element{midterm_2017}{ % в фигурных скобках название группы вопросов
 \AMCcompleteMulti
  \begin{questionmult}{3borya} % тип вопроса (questionmult — множественный выбор) и в фигурных — номер вопроса
Величины $X_1$, $X_2$, \ldots, независимы и одинаково распределены $\cN(0;1)$. Предел по вероятности $\plim_{n\to\infty} \frac{X_1^2+ X_2^2 + \ldots + X_n^2}{n}$ равен
\begin{multicols}{3} % располагаем ответы в 3 колонки
   \begin{choices} % опция [o] не рандомизирует порядок ответов
      \correctchoice{$1$}
      \wrongchoice{$0$}
      \wrongchoice{$2$}
      \wrongchoice{$3$}
      \wrongchoice{$1/2$}
      \end{choices}
  \end{multicols}
  \end{questionmult}
}



\element{midterm_2017}{ % в фигурных скобках название группы вопросов
 \AMCcompleteMulti
  \begin{questionmult}{4borya} % тип вопроса (questionmult — множественный выбор) и в фигурных — номер вопроса
Величины $X_1$, $X_2$, \ldots, независимы и одинаково распределены с $\E(X_i) = 4$ и $\Var(X_i) = 100$. Вероятность $\P(\bar X_n \leq 5)$ примерно равна
 \begin{multicols}{3} % располагаем ответы в 3 колонки
   \begin{choices} % опция [o] не рандомизирует порядок ответов
      \correctchoice{ $0.84$ }
      \wrongchoice{ $0.67$ }
      \wrongchoice{ $0.95$ }
      \wrongchoice{ $0.28$ }
      \wrongchoice{ $0.50$ }
      \end{choices}
  \end{multicols}
  \end{questionmult}
}


\element{midterm_2017_rejected}{ % в фигурных скобках название группы вопросов
 \AMCcompleteMulti
  \begin{questionmult}{5borya} % тип вопроса (questionmult — множественный выбор) и в фигурных — номер вопроса

Величины $X_1$, $X_2$, \ldots, независимы и одинаково распределены с $\E(X_i) = 4$ и $\Var(X_i) = 100$, а $S_n = X_1 + X_2 + \ldots + X_n$. К нормальному стандартному распределению сходится последовательность
 \begin{multicols}{3} % располагаем ответы в 3 колонки
   \begin{choices} % опция [o] не рандомизирует порядок ответов
      \correctchoice{ $\frac{S_n - 4n}{10\sqrt{n}}$ }
      \wrongchoice{ $\sqrt{n}\frac{S_n - 4n}{10/\sqrt{n}}$ }
      \wrongchoice{ $\sqrt{n}\frac{S_n - 4}{10/\sqrt{n}}$ }
      \wrongchoice{ $\sqrt{n}\frac{S_n - 4}{10}$  }
      \wrongchoice{ $\sqrt{n}\frac{S_n - 4}{10}\sqrt{n}$ }
      \end{choices}
  \end{multicols}
  \end{questionmult}
}




\element{midterm_2017}{ % в фигурных скобках название группы вопросов
 \AMCcompleteMulti
  \begin{questionmult}{6borya} % тип вопроса (questionmult — множественный выбор) и в фигурных — номер вопроса

Совместная функция плотности величин $X$ и $Y$ имеет вид
\[
f(x,y) =
\begin{cases}
6xy^2, \text{ при } x, y \in [0;1] \\
0, \text{ иначе } \\
\end{cases}.
\]
При $Y=1/2$ величина $X$ имеет условное распределение
 \begin{multicols}{2} % располагаем ответы в 3 колонки
   \begin{choices} % опция [o] не рандомизирует порядок ответов
      \correctchoice{ с плотностью $f(x)=2x$ при $x\in[0;1]$ }
      \wrongchoice{ нормальное, $\cN(0;1)$ }
      \wrongchoice{ с плотностью $f(x)=1.5x$ при $x\in[0;1]$ }
      \wrongchoice{ с плотностью $f(x)=3x^2$ при $x\in[0;1]$ }
      \wrongchoice{ равномерное, $U[0;1]$ }
      \end{choices}
  \end{multicols}
  \end{questionmult}
}




\element{midterm_2017}{ % в фигурных скобках название группы вопросов
 \AMCcompleteMulti
  \begin{questionmult}{7borya} % тип вопроса (questionmult — множественный выбор) и в фигурных — номер вопроса

Совместная функция плотности величин $X$ и $Y$ имеет вид
\[
f(x,y) =
\begin{cases}
6xy^2, \text{ при } x, y \in [0;1] \\
0, \text{ иначе } \\
\end{cases}.
\]

Математическое ожидание $\E(XY)$ равно
 \begin{multicols}{3} % располагаем ответы в 3 колонки
   \begin{choices} % опция [o] не рандомизирует порядок ответов
      \correctchoice{ $1/2$ }
      \wrongchoice{  $2/3$ }
      \wrongchoice{ $3/4$ }
      \wrongchoice{ $4/5$ }
      \wrongchoice{ $1$ }
      \end{choices}
  \end{multicols}
  \end{questionmult}
}






\element{midterm_2017}{ % в фигурных скобках название группы вопросов
 \AMCcompleteMulti
  \begin{questionmult}{8borya} % тип вопроса (questionmult — множественный выбор) и в фигурных — номер вопроса

Правильный кубик подбрасывается два раза, величина $X_i$ равна 1, если в $i$-ый раз выпала шестёрка, и нулю иначе. Условный закон распределения $X_1$ при условии $X_1+X_2=1$ совпадает с распределением

 \begin{multicols}{2} % располагаем ответы в 3 колонки
   \begin{choices} % опция [o] не рандомизирует порядок ответов
      \correctchoice{ Бернулли с $p=1/2$ }
      \wrongchoice{ Бернулли с $p=1/6$ }
      \wrongchoice{ Биномиальным $Bin(n=2, p=1/6)$ }
      \wrongchoice{ Биномиальным $Bin(n=2, p=1/2)$ }
      \wrongchoice{ нормальным $\cN(0;1)$ }
      \end{choices}
  \end{multicols}
  \end{questionmult}
}


\begin{comment}

\end{comment}
