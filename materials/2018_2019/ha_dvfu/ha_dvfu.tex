\documentclass[12pt]{article}

\usepackage{tikz} % картинки в tikz
\usepackage{microtype} % свешивание пунктуации

\usepackage{array} % для столбцов фиксированной ширины

\usepackage{indentfirst} % отступ в первом параграфе

\usepackage{sectsty} % для центрирования названий частей
\allsectionsfont{\centering}

\usepackage{amsmath} % куча стандартных математических плюшек

\usepackage{comment}
\usepackage{amsfonts}

\usepackage[top=2cm, left=1cm, right=1cm, bottom=2cm]{geometry} % размер текста на странице

\usepackage{lastpage} % чтобы узнать номер последней страницы

\usepackage{enumitem} % дополнительные плюшки для списков
%  например \begin{enumerate}[resume] позволяет продолжить нумерацию в новом списке
\usepackage{caption}

\usepackage{hyperref} % гиперссылки

\usepackage{multicol} % текст в несколько столбцов


\usepackage{fancyhdr} % весёлые колонтитулы
\pagestyle{fancy}
\lhead{Теория вероятностей-ВШЭ-ДВФУ}
\chead{}
\rhead{Долгожданная домашка :)}
\lfoot{}
\cfoot{}
\rfoot{}
\renewcommand{\headrulewidth}{0.4pt}
\renewcommand{\footrulewidth}{0.4pt}



\usepackage{todonotes} % для вставки в документ заметок о том, что осталось сделать
% \todo{Здесь надо коэффициенты исправить}
% \missingfigure{Здесь будет Последний день Помпеи}
% \listoftodos --- печатает все поставленные \todo'шки


% более красивые таблицы
\usepackage{booktabs}
% заповеди из докупентации:
% 1. Не используйте вертикальные линни
% 2. Не используйте двойные линии
% 3. Единицы измерения - в шапку таблицы
% 4. Не сокращайте .1 вместо 0.1
% 5. Повторяющееся значение повторяйте, а не говорите "то же"


\usepackage{fontspec}
\usepackage{polyglossia}

\setmainlanguage{russian}
\setotherlanguages{english}

% download "Linux Libertine" fonts:
% http://www.linuxlibertine.org/index.php?id=91&L=1
\setmainfont{Linux Libertine O} % or Helvetica, Arial, Cambria
% why do we need \newfontfamily:
% http://tex.stackexchange.com/questions/91507/
\newfontfamily{\cyrillicfonttt}{Linux Libertine O}

\AddEnumerateCounter{\asbuk}{\russian@alph}{щ} % для списков с русскими буквами
\setlist[enumerate, 2]{label=\asbuk*),ref=\asbuk*}

%% эконометрические сокращения
\DeclareMathOperator{\Cov}{Cov}
\DeclareMathOperator{\Corr}{Corr}
\DeclareMathOperator{\Var}{Var}
\DeclareMathOperator{\E}{E}
\def \hb{\hat{\beta}}
\def \hs{\hat{\sigma}}
\def \htheta{\hat{\theta}}
\def \s{\sigma}
\def \hy{\hat{y}}
\def \hY{\hat{Y}}
\def \v1{\vec{1}}
\def \e{\varepsilon}
\def \he{\hat{\e}}
\def \z{z}
\def \hVar{\widehat{\Var}}
\def \hCorr{\widehat{\Corr}}
\def \hCov{\widehat{\Cov}}
\def \cN{\mathcal{N}}
\def \P{\mathbb{P}}


\begin{document}


Скан выполненной можно прислать на \url{boris.demeshev@gmail.com}.

Срок сдачи — до 24 декабря 2018 года :) Всем удачи!!!

\begin{enumerate}
  \item Я подбрасываю правильный кубик до выпадения первой единицы. Так вышло,
  что нечётные числа ни разу не выпали.

  Чему равно ожидаемое число подбрасываний, которые я сделал?

  \item Маша подкидывает монетку бесконечное количество раз.
  Пусть $N_{\text{ОРР}}$ — номер подбрасывания, когда впервые выпадет орёл-решка-решка,
  а $N_{\text{РОР}}$ — номер подбрасывания, когда впервые выпадет решка-орёл-решка.

  Найдите $\E(N_{\text{ОРР}})$, $\E(N_{\text{РОР}})$, $\P(N_{\text{ОРР}}>N_{\text{РОР}})$.

  \item
Дворняги благородных кровей, Шарик и Тузик, очень любят тусоваться вместе.
Изначально блоха Изабелла сидит на Шарике.
Вероятность перескока Изабеллы с одной собаки на другую за малый интервал времени
прямо пропорциональна длине этого интервала, то есть:
\[
\P(\text{перескок за отрезок времени }[t;t+\Delta]) = \lambda \Delta + o(\Delta),
\]

При этом для перескока с Тузика на Шарика $\lambda=1$, а для перескока
с Шарика на Тузика $\lambda=2$.

Какова точная вероятность того, что блоха Изабелла будет сидеть на Шарике
в момент времени $t$?
\item В столовую пришли 30 студентов и встали в очередь в случайном порядке.
Среди них есть Вовочка и Машенька. Пусть $V$ — это количество человек в очереди
перед Вовочкой, а $M\geq 0$ — количество человек между Вовочкой и Машенькой.

Найдите $\E(V)$, $\E(M)$, $\Var(M)$, $\Cov(V, M)$.

  \item
Немного упрощая реальность можно сказать, что ген карих глаз доминирует ген синих
\footnote{На самом деле цвет глаз кодируется несколькими генами.}.
Следовательно, у носителя пары bb глаза синие, а у носителя пар BB и Bb — карие.
У диплоидных организмов одна аллель наследуется от папы,
а одна — от мамы. В семье у кареглазых родителей два сына — кареглазый и
синеглазый. Кареглазый женился на синеглазой девушке.

Какова вероятность рождения у них синеглазого ребенка?

\item Злопамятный Джо очень любит играть в картишки. Перед Джо хорошо перемешанная
стандартная колода в 52 карты. Джо извлекает карты по одной.

\begin{enumerate}
\item На каком месте в среднем появляется первая Дама?
\item Какова вероятность того, что за первой Дамой сразу следует Туз Пик?
\item Какова вероятность того, что за первой Дамой сразу следует Дама Пик?
\end{enumerate}

  \item Расскажите «случай на охоте». Поделитесь интересной находкой или проблемой.
  Это может быть рассказ в вольной форме о том, как удалось очень понятно
  или нестандартно изложить какой-то сюжет. Очень красивая и поучительная задачка.
  Или наоборот, проблема, которую совершенно не ясно как решать.



\end{enumerate}

\end{document}
