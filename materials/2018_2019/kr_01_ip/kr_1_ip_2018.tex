\documentclass[12pt]{article}

\usepackage{tikz} % картинки в tikz
\usepackage{microtype} % свешивание пунктуации

\usepackage{array} % для столбцов фиксированной ширины

\usepackage{indentfirst} % отступ в первом параграфе

\usepackage{sectsty} % для центрирования названий частей
\allsectionsfont{\centering}

\usepackage{amsmath} % куча стандартных математических плюшек

\usepackage{comment}
\usepackage{amsfonts}

\usepackage[top=2cm, left=1cm, right=1cm, bottom=2cm]{geometry} % размер текста на странице

\usepackage{lastpage} % чтобы узнать номер последней страницы

\usepackage{enumitem} % дополнительные плюшки для списков
%  например \begin{enumerate}[resume] позволяет продолжить нумерацию в новом списке
\usepackage{caption}

\usepackage{hyperref} % гиперссылки

\usepackage{multicol} % текст в несколько столбцов


\usepackage{fancyhdr} % весёлые колонтитулы
\pagestyle{fancy}
\lhead{Теория вероятностей-ВШЭ}
\chead{2018-10-23}
\rhead{Контрольная 1. Максимум :)}
\lfoot{}
\cfoot{}
\rfoot{}
\renewcommand{\headrulewidth}{0.4pt}
\renewcommand{\footrulewidth}{0.4pt}



\usepackage{todonotes} % для вставки в документ заметок о том, что осталось сделать
% \todo{Здесь надо коэффициенты исправить}
% \missingfigure{Здесь будет Последний день Помпеи}
% \listoftodos --- печатает все поставленные \todo'шки


% более красивые таблицы
\usepackage{booktabs}
% заповеди из докупентации:
% 1. Не используйте вертикальные линни
% 2. Не используйте двойные линии
% 3. Единицы измерения - в шапку таблицы
% 4. Не сокращайте .1 вместо 0.1
% 5. Повторяющееся значение повторяйте, а не говорите "то же"


\usepackage{fontspec}
\usepackage{polyglossia}

\setmainlanguage{russian}
\setotherlanguages{english}

% download "Linux Libertine" fonts:
% http://www.linuxlibertine.org/index.php?id=91&L=1
\setmainfont{Linux Libertine O} % or Helvetica, Arial, Cambria
% why do we need \newfontfamily:
% http://tex.stackexchange.com/questions/91507/
\newfontfamily{\cyrillicfonttt}{Linux Libertine O}

\AddEnumerateCounter{\asbuk}{\russian@alph}{щ} % для списков с русскими буквами
\setlist[enumerate, 2]{label=\asbuk*),ref=\asbuk*}

%% эконометрические сокращения
\DeclareMathOperator{\Cov}{Cov}
\DeclareMathOperator{\Corr}{Corr}
\DeclareMathOperator{\Var}{Var}
\DeclareMathOperator{\E}{E}
\def \hb{\hat{\beta}}
\def \hs{\hat{\sigma}}
\def \htheta{\hat{\theta}}
\def \s{\sigma}
\def \hy{\hat{y}}
\def \hY{\hat{Y}}
\def \v1{\vec{1}}
\def \e{\varepsilon}
\def \he{\hat{\e}}
\def \z{z}
\def \hVar{\widehat{\Var}}
\def \hCorr{\widehat{\Corr}}
\def \hCov{\widehat{\Cov}}
\def \cN{\mathcal{N}}
\def \P{\mathbb{P}}


\begin{document}


\fbox{
  \begin{minipage}{42em}
    Имя, фамилия и номер группы:\vspace*{3ex}\par
    \noindent\dotfill\vspace{2mm}
  \end{minipage}
}


\begin{enumerate}
\item Пират Злопамятный Джо очень любит неразбавленный ром. Из-за того,
что он много пьёт, у него проблемы с памятью, и он помнит не больше, чем три последних
пинты. Хозяин таверны «Огненная зебра» с вероятностью $1/8$ разбавляет каждую подаваемую пинту рома.
Если по ощущением Джо половина выпитых пинт или больше была разбавлена, то он
разносит таверну к чертям собачьим. Только что Джо вошёл в таверну и закал первую пинту.

Сколько в среднем пинт выпьет Джо, прежде чем разнесёт таверну?

\item В таверне «Крутой ковбой» разбавленный ром подают с вероятностью $1/2$.
Джо немного сменил свой характер и теперь устраивает скандал,
если две пинты рома подряд разбавлены.

Какова вероятность того, что Джо сможет выпить 100 пинт подряд без скандалов?

\item Али-Баба хочет проникнуть в пещеру с сокровищами. Вход в
пещеру закрыт и его охраняет Джин с квадратным подносом.
В каждой вершине подноса — непрозрачный стаканчик. Под
каждым стаканчиком — монетка.
Если все четыре монетки окажутся в одинаковом положении, все —
орлом вверх, или все — решкой вверх, то вход откроется.
За одно действие Али-Баба может открыть любые два стаканчика и
положить открывшиеся монетки любой стороной вверх.
После действия Али-Бабы Джин накрывает монетки стаканчиками,
быстро-быстро вращает поднос и снова предоставляет поднос
Али-Бабе.
Углядеть за Джином или сделать пометки на подносе невозможно.

\begin{enumerate}
  \item Как надо действовать Али-Бабе, чтобы гарантировать себе вход в
  пещеру за наименьшее количество действий?
  \item Сколько действий
  потребуется в худшем случае?
\end{enumerate}

\item Злопамятный Джо очень любит играть в картишки. Перед Джо хорошо перемешанная
стандартная колода в 52 карты. Джо извлекает карты по одной.

На каком месте в среднем появляется первая Дама?

\item Вероятность того, что Ученик достигнет Просветления за малый интервал времени,
прямо пропорциональна длине этого интервала, а именно,
\[
\P(\text{достигнуть Просветления за отрезок времени }[t;t+\Delta]) = 0.2018 \Delta + o(\Delta)
\]

Какова точная вероятность того, что Ученик, начавший искать Просветление, так
и не достигнет его к моменту времени $t$?

\item Исследователь Василий выбирает равномерно и независимо друг от друга 10 точек на
отрезке $[0;1]$. Затем Василий записывает их координаты в порядке возрастания,
$Y_1 \leq Y_2 \leq \ldots \leq Y_{10}$.

Не производя вычислений, \textit{по определению},
выпишите функции плотности случайной величины $Y_4$.
\end{enumerate}

\end{document}
