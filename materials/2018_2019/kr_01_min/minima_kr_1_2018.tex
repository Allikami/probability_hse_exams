\documentclass[12pt]{article}

\usepackage{tikz} % картинки в tikz
\usepackage{microtype} % свешивание пунктуации

\usepackage{array} % для столбцов фиксированной ширины

\usepackage{indentfirst} % отступ в первом параграфе

\usepackage{sectsty} % для центрирования названий частей
\allsectionsfont{\centering}

\usepackage{amsmath} % куча стандартных математических плюшек

\usepackage{comment}
\usepackage{amsfonts}

\usepackage[top=2cm, left=1.2cm, right=1.2cm, bottom=2cm]{geometry} % размер текста на странице

\usepackage{lastpage} % чтобы узнать номер последней страницы

\usepackage{enumitem} % дополнительные плюшки для списков
%  например \begin{enumerate}[resume] позволяет продолжить нумерацию в новом списке
\usepackage{caption}

\usepackage{hyperref} % гиперссылки

\usepackage{multicol} % текст в несколько столбцов


\usepackage{fancyhdr} % весёлые колонтитулы
\pagestyle{fancy}
\lhead{Теория вероятностей-ВШЭ}
\chead{2018-10-23}
\rhead{Контрольная 1. Минимум.}
\lfoot{Вариант $\gamma$}
\cfoot{}
\rfoot{\thepage/2}
\renewcommand{\headrulewidth}{0.4pt}
\renewcommand{\footrulewidth}{0.4pt}



\usepackage{todonotes} % для вставки в документ заметок о том, что осталось сделать
% \todo{Здесь надо коэффициенты исправить}
% \missingfigure{Здесь будет Последний день Помпеи}
% \listoftodos --- печатает все поставленные \todo'шки


% более красивые таблицы
\usepackage{booktabs}
% заповеди из докупентации:
% 1. Не используйте вертикальные линни
% 2. Не используйте двойные линии
% 3. Единицы измерения - в шапку таблицы
% 4. Не сокращайте .1 вместо 0.1
% 5. Повторяющееся значение повторяйте, а не говорите "то же"


\usepackage{fontspec}
\usepackage{polyglossia}

\setmainlanguage{russian}
\setotherlanguages{english}

% download "Linux Libertine" fonts:
% http://www.linuxlibertine.org/index.php?id=91&L=1
\setmainfont{Linux Libertine O} % or Helvetica, Arial, Cambria
% why do we need \newfontfamily:
% http://tex.stackexchange.com/questions/91507/
\newfontfamily{\cyrillicfonttt}{Linux Libertine O}

\AddEnumerateCounter{\asbuk}{\russian@alph}{щ} % для списков с русскими буквами
\setlist[enumerate, 2]{label=\asbuk*),ref=\asbuk*}

%% эконометрические сокращения
\DeclareMathOperator{\Cov}{Cov}
\DeclareMathOperator{\Corr}{Corr}
\DeclareMathOperator{\Var}{Var}
\DeclareMathOperator{\E}{E}
\def \hb{\hat{\beta}}
\def \hs{\hat{\sigma}}
\def \htheta{\hat{\theta}}
\def \s{\sigma}
\def \hy{\hat{y}}
\def \hY{\hat{Y}}
\def \v1{\vec{1}}
\def \e{\varepsilon}
\def \he{\hat{\e}}
\def \z{z}
\def \hVar{\widehat{\Var}}
\def \hCorr{\widehat{\Corr}}
\def \hCov{\widehat{\Cov}}
\def \cN{\mathcal{N}}
\def \P{\mathbb{P}}


\begin{document}

Продолжительность: 30 минут.

\fbox{
  \begin{minipage}{42em}
    Имя, фамилия и номер группы:\vspace*{3ex}\par
    \noindent\dotfill\vspace{2mm}
  \end{minipage}
}


\begin{enumerate}
  \item Определение независимости двух случайных событий.
  \item Формула Байеса.
  \item  Карлсон выложил кубиками слово КОМБИНАТОРИКА. Малыш выбирает наугад четыре кубика и выкладывает их в случайном порядке.
  Найдите вероятность того, что при этом получится слово КРОТ.
  \item  Пусть случайная величина $X$ имеет распределение Пуассона с параметром $\lambda = 23$ .
  Найдите
  \begin{enumerate}
  	\item $\P(X = 0)$;
  	\item $\P(X > 0)$;
  	\item $\P(X < 0)$;
  	\item $\E(X)$
  \end{enumerate}
\end{enumerate}


\newpage
\setcounter{page}{2}
\mbox{}

\newpage
\lfoot{Вариант $\omega$}
\setcounter{page}{1}
Продолжительность: 30 минут.

\fbox{
  \begin{minipage}{42em}
    Имя, фамилия и номер группы:\vspace*{3ex}\par
    \noindent\dotfill\vspace{2mm}
  \end{minipage}
}


\begin{enumerate}
  \item Классическое определение вероятности.
  \item Распределение Пуассона: определение, математическое ожидание.
  \item  Пусть $\P(A) = 0.3, \P(B) = 0.5, \P(A\cap B) = 0.15 $.
  \begin{enumerate}
  	\item  Найдите $\P(A|B)$;
  	\item  Найдите $\P(A\cup B)$;
  	\item  Являются ли события $A$ и $B$ независимыми?
  \end{enumerate}
  \item  В операционном отделе банка работает 70\% опытных сотрудников и 30\%
  неопытных. Вероятность совершения ошибки при очередной банковской операции
  опытным сотрудником равна 0.01, а неопытным — 0.1.
  Известно, что при очередной банковской операции была допущена ошибка.

  Найдите вероятность того, что ошибку допустил неопытный сотрудник.

\end{enumerate}

\newpage
\setcounter{page}{2}
\mbox{}

\newpage
\lfoot{Вариант $\theta$}
\setcounter{page}{1}
Продолжительность: 30 минут.

\fbox{
  \begin{minipage}{42em}
    Имя, фамилия и номер группы:\vspace*{3ex}\par
    \noindent\dotfill\vspace{2mm}
  \end{minipage}
}


\begin{enumerate}
  \item Определение условной вероятности.
  \item Функция распределения случайной величины: определение и три свойства.
  \item  В первой урне 7 белых и 3 черных шара, во второй урне 8 белых и 2 черных
  шара, в третьей урне 5 белых и 5 черных шаров.

  Из этих урн наугад выбирается одна урна.
  Какова вероятность того, что шар, взятый наугад из выбранной урны, окажется белым?

  \item При работе некоторого устройства время от времени возникают сбои.
  Количество сбоев за сутки имеет распределение Пуассона.
  Среднее количество сбоев за сутки равно 5.

  Найти вероятность того, что за двое суток не произойдет ни одного сбоя.
\end{enumerate}

\newpage
\setcounter{page}{2}
\mbox{}

\end{document}
