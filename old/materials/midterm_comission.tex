\documentclass[12pt]{article}

\usepackage{tikz} % картинки в tikz
\usepackage{microtype} % свешивание пунктуации

\usepackage{array} % для столбцов фиксированной ширины

\usepackage{indentfirst} % отступ в первом параграфе

\usepackage{sectsty} % для центрирования названий частей
\allsectionsfont{\centering}

\usepackage{amsmath} % куча стандартных математических плюшек

\usepackage{comment}
\usepackage{amsfonts}

\usepackage[top=2cm, left=1.2cm, right=1.2cm, bottom=2cm]{geometry} % размер текста на странице

\usepackage{lastpage} % чтобы узнать номер последней страницы

\usepackage{enumitem} % дополнительные плюшки для списков
%  например \begin{enumerate}[resume] позволяет продолжить нумерацию в новом списке
\usepackage{caption}


\usepackage{fancyhdr} % весёлые колонтитулы
\pagestyle{fancy}
\lhead{Теория вероятностей}
\chead{}
\rhead{Комиссия, 2018-02-13}
\lfoot{}
\cfoot{}
\rfoot{\thepage/\pageref{LastPage}}
\renewcommand{\headrulewidth}{0.4pt}
\renewcommand{\footrulewidth}{0.4pt}



\usepackage{todonotes} % для вставки в документ заметок о том, что осталось сделать
% \todo{Здесь надо коэффициенты исправить}
% \missingfigure{Здесь будет Последний день Помпеи}
% \listoftodos --- печатает все поставленные \todo'шки


% более красивые таблицы
\usepackage{booktabs}
% заповеди из докупентации:
% 1. Не используйте вертикальные линни
% 2. Не используйте двойные линии
% 3. Единицы измерения - в шапку таблицы
% 4. Не сокращайте .1 вместо 0.1
% 5. Повторяющееся значение повторяйте, а не говорите "то же"



\usepackage{fontspec}
\usepackage{polyglossia}

\setmainlanguage{russian}
\setotherlanguages{english}

% download "Linux Libertine" fonts:
% http://www.linuxlibertine.org/index.php?id=91&L=1
\setmainfont{Linux Libertine O} % or Helvetica, Arial, Cambria
% why do we need \newfontfamily:
% http://tex.stackexchange.com/questions/91507/
\newfontfamily{\cyrillicfonttt}{Linux Libertine O}

\AddEnumerateCounter{\asbuk}{\russian@alph}{щ} % для списков с русскими буквами
\setlist[enumerate, 2]{label=\asbuk*),ref=\asbuk*}

%% эконометрические сокращения
\DeclareMathOperator{\Cov}{Cov}
\DeclareMathOperator{\Corr}{Corr}
\DeclareMathOperator{\Var}{Var}
\DeclareMathOperator{\E}{E}
\def \hb{\hat{\beta}}
\def \hs{\hat{\sigma}}
\def \htheta{\hat{\theta}}
\def \s{\sigma}
\def \hy{\hat{y}}
\def \hY{\hat{Y}}
\def \v1{\vec{1}}
\def \e{\varepsilon}
\def \he{\hat{\e}}
\def \z{z}
\def \hVar{\widehat{\Var}}
\def \hCorr{\widehat{\Corr}}
\def \hCov{\widehat{\Cov}}
\def \cN{\mathcal{N}}
\def \P{\mathbb{P}}


\begin{document}

Фамилия, имя, группа:

\vspace{40pt}

Задача 1. Приведите определение условной вероятности случайного события, формулу Байеса. Все обозначения должны быть пояснены.

\newpage
Фамилия, имя, группа:

\vspace{40pt}

Задача 2. Сформулируйте определение и свойства математического ожидания 
для абсолютно непрерывной случайной величины. 
Все обозначения должны быть пояснены.

Требуемые свойства: 
линейность, ожидание для произведения независимых величин,
ожидание для почти наверное неотрицательной величины, ожидание от функции случайной величины.

\newpage
Фамилия, имя, группа:

\vspace{40pt}

Задача 3. Сформулируйте теорему Муавра—Лапласа.
 Все обозначения должны быть пояснены.

\newpage
Фамилия, имя, группа:

\vspace{40pt}

Задача 4. Пусть задана таблица совместного распределения случайных величин $X$ и $Y$.

\begin{center}\begin{tabular}{lccc}
\toprule
   & $Y=-1$  & $Y=0$   & $Y=2$   \\ \midrule
$X=-1$                 & $0.2$ & $0.3$ & $0.2$ \\
 $X=2$                 & $0.1$ & $0.1$ & $0.1$ \\ \bottomrule
\end{tabular}\end{center}

Найдите
\begin{enumerate}
    \item $\E(X)$, $\E(X^{2})$, $\Var(X)$;
    \item $\E(Y)$, $\E(Y^{2})$, $\Var(Y)$;
    \item $\E(XY)$, $\Cov(X,Y)$, $\Corr(X,Y)$;
    \item Являются ли случайные величины $X$ и $Y$ некоррелированными?
\end{enumerate}


\newpage
Фамилия, имя, группа:

\vspace{40pt}

Задача 5. Пусть $\E(X)=-3$, $\E(Y)=4$, $\Var(X) = 5$, $\Var(Y) = 6$, $\Cov(X,Y) = -1$. Найдите
\begin{enumerate}
\item $\E(2X + Y - 4)$, $\Var(2Y + 3)$;
\item $\Var(X - Y)$, $\Var(2X - 3Y +1)$;
\item $\Cov(3X+ Y + 1,X - 2Y -1)$, $\Corr(X + Y, X - Y)$;
\item Ковариационную матрицу случайного вектора $Z = (X+Y, Y-X)$.
\end{enumerate}

\end{document}
