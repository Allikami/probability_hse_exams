\subsection{Контрольная работа №2, 22.12.04}

\begin{enumerate}
\item Вычислите вероятность $\P(|X-\E(X)|>2\sqrt{\Var(X)})$, если известно, что
случайная величина $X$ подчиняется нормальному закону распределения.
\item Определите значения математического ожидания и дисперсии случайной
величины, функция плотности которой имеет вид
\[
f(x)=\frac{1}{3\sqrt{2\pi}}e^{-\frac{(x+1)^2}{18}}
\]
\item Страховая компания «Ой» заключает договор страхования от «невыезда» (невыдачи визы) с туристами, покупающими туры в Европу. Из предыдущей практики известно,
что в среднем отказывают в визе одному из 130 человек. Найдите вероятность того, что из 200
застраховавшихся в «Ой» туристов, четверым потребуется страховое возмещение.
\item Считая вероятность рождения мальчика равной 0.52, вычислите вероятность того, что
из 24 новорожденных будет 15 мальчиков.
\item Для случайной величины $X$ с нулевым математическим ожиданием
дисперсией 16, оцените сверху вероятность $\P(|X| > 15)$.
\item Случайные величины $X$ и $Y$ независимы. Известно, что $\E(X)=0$, $\Var(X)=4$, $\E(Y)=5$. Определите значение дисперсии случайной величины $Y$, если известно, что случайная величина $Z=2X-Y$, принимает неотрицательные значения с вероятностью 0.9.
\item Вычислите вероятность $\P(| X - \E(X)| > 2\Var(X))$, если известно, что
случайная величина $X$ распределена по закону Пуассона с параметром $\lambda = 0.09$
\item Портфель страховой компании состоит из 1000 договоров, заключенных 1
января и действующих в течение года. При наступлении страхового случая по каждому из
договоров компания обязуется выплатить 1500 рублей. Вероятность наступления страхового
события по каждому из договоров предполагается равной 0.05 и не зависящей от наступления
страховых событий по другим контрактам. Каков должен быть совокупный размер резерва
страховой компании для того, чтобы с вероятностью 0.95 она могла бы удовлетворить
требования, возникающие по указанным договорам?
\item В коробке лежат три купюры, достоинством в 100, 10 и 50 рублей
соответственно. Они извлекаются в случайном порядке. Пусть $X_1$, $X_2$ и $X_3$ — достоинства
купюр в порядке их появления из коробки.
\begin{enumerate}
\item Верно ли, что $X_1$ и $X_3$ одинаково распределены?
\item Верно ли, что $X_1$ и $X_3$ независимы?
\item Найдите дисперсию $X_2$
\end{enumerate}
\item Когда Винни-Пуха не кусают пчелы, он сочиняет в среднем одну кричалку в день.
Верный друг и соратник Винни-Пуха Пятачок записал, сколько кричалок сочинялось в дни
укусов. Эта выборка из 36 наблюдений перед вами:

2, 0, 0, 2, 0, 0, 0, 2, 0, 2, 0, 2, 2, 0, 2, 0, 2, 2, 0, 0, 2, 2, 0, 0, 2, 0, 2, 2, 0, 2, 0, 2, 2, 2, 0, 2.

Верно ли, что укусы пчел положительно сказываются на творческом потенциале Винни-Пуха
(используйте нормальную аппроксимацию биномиального распределения)?
\item Пусть $X_t$ — количество бактерий, живущих в момент времени $t$. Известно, что $X_1 =1$ и $X_t = A_t \cdot X_{t-1}$, где случайные величины $A_t$ независимы и равномерно
распределены на отрезке $[0; 2a]$. Величина $A_t$ может интерпретироваться как среднее
количество потомков. Можно догадаться, что данная модель приводить к экспоненциальной динамике.
\begin{enumerate}
\item Определите долгосрочный темп роста бактерий, то есть найдите предел $\lim_{n\to\infty}\frac{\ln X_n}{n}$
\item При каком $a$ темп роста будет положительным?
\end{enumerate}
\end{enumerate}


\subsection{Контрольная работа №2, 22.12.04, решения}
\begin{enumerate}
\item $\P\left(|X-\E(X)|>2\sqrt{\Var(X)}\right) = \P\left(\frac{|X-\E(X)|}{\sqrt{\Var(X)}} > 2\right) = 2\P(\cN(0,1) >2) \approx 0.05$
\item $\mu = -1$, $\sigma^2 = 9$
\item $C_{200}^4 \left(\frac{1}{130}\right)^4 \left(\frac{129}{130}\right)^{196}$
\item $C_{20}^5 0.52^{15}0.48^9$
\item $\P(|X| > 15) \leq \frac{16}{15^2}$
\item Подготовимся: $\E(2X-Y) = -5$, $\Var(2X-Y) = 16 + \Var(Y)$

$ \P(Z>0) = 0.9 \Rightarrow \P\left(\frac{2X-Y+5}{\sqrt{16+\Var(Y)}} > \frac{5}{\sqrt{16+\Var(Y)}}\right) \Rightarrow \frac{5}{\sqrt{16+\Var(Y)}} = 1.28  $

$ \Var(Y) = 0.74$
\item $\E(X) = \Var(X) = \lambda = 0.09$
\[
\P(| X - 0.09| > 0.18) = 1 - \P(-0.18 < X - 0.09 < 0.18)  \stackrel{X\geq0}{=} 1 - \P(X=0) = 1 - e^{-0.09}
\]
\item Пусть $X$ — случайная величина, число страховых случаев, $X \sim Bin(n=1000, p=0.05)$. $S$ — размер резерва.

Тогда условие можно записать в виде: $\P(1500X \leq S) = 0.95$
\[
\P\left(\frac{X - 50}{\sqrt{1000 \cdot 0.05 \cdot 0.95}} \leq \frac{\frac{S}{1500}-50}{\sqrt{1000 \cdot 0.05 \cdot 0.95}} \right) =0.95 \Rightarrow \frac{\frac{S}{1500}-50}{\sqrt{1000 \cdot 0.05 \cdot 0.95}} \approx 1.65 \Rightarrow S \approx 92058
\]
\item
\begin{enumerate}
\item Да
\item Нет
\item $1356$
\end{enumerate}
\item Пусть $X$ — случайная величина, число сочинённых песенок в день, когда Винни-Пуха кусает пчела, $X \sim Bin(n,p)$.

Из данной в условии выборки находим $\overline{X} = 19/18$, поскольку число наблюдений достаточно велико, $\E(X) = np = 36p = 19/18$, откуда получаем $p=19/(18\cdot36)$ и $\Var(X) = np(1-p) \approx 1$.

Нормальная аппроксимация: $X \sim \cN(19/18, 1)$.
\[
\P(X>1) = \P\left(X - \frac{19}{18} > -\frac{1}{18} \right) \approx 0.52
\]
\item
\begin{enumerate}
\item Заметим, что величину $X_t$ можно представить в виде:
\[
X_t = A_t \cdot X_{t-1} = A_t \cdot A_{t-1} \cdot X_{t-1} = \ldots = A_t \cdot A_{t-1} \cdot \ldots \cdot A_{2} \cdot X_1
\]
Тогда и предел тоже можно переписать:
\[
\lim_{n\to\infty}\frac{\ln X_n}{n} = \lim_{n\to\infty} \frac{\ln A_n + \ldots + \ln A_2 + \ln X_1}{n} \stackrel{X_1 = 1}{=} \lim_{n\to\infty} \frac{\ln A_n + \ldots + \ln A_2}{n} \stackrel{\text{ЗБЧ}}{=} \E(\ln A_1)
\]
Осталось найти математическое ожидание $\ln A_1$:
\[
\E(\ln A_1) = \int_0^{2a} \frac{1}{2a} \cdot \ln x dx = \ln(2a)-1
\]
\item Из неравенста $\ln(2a)-1>0$ получаем, что темп роста будет положительным при $a>e/2$.
\end{enumerate}
\end{enumerate}
