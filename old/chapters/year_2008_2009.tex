\subsection{Контрольная работа №1, демо-версия, ??.11.2008}

\subsubsection*{Часть I.}

Обведите верный ответ:

\begin{enumerate}
\item Для любой случайной величины $\P(X>0)\ge \P(X+1>0)$. Да. Нет.
\item Для любой случайной величины с $\E(X)<2$, выполняется условие $\P(X<2)=1$. Да. Нет.
\item Если $A\subset B$, то $\P(A|B)\le \P(B|A)$. Да. Нет.
\item Если  $X$ — случайная величина, то $\E(X)+1=\E(X+1)$. Да. Нет.
\item Функция распределения случайной величины является неубывающей. Да. Нет.
\item Для любых событий $A$ и $B$, выполняется $\P(A|B)+\P(A|B^{c})=1$. Да. Нет.
\item Для любых событий  $A$  и  $B$  верно, что $\P(A|B)\ge \P(A\cap
B)$, если обе вероятности существуют. Да. Нет.
\item Функция плотности может быть периодической. Да. Нет.
\item Если случайная величина $X$ имеет функцию плотности, то $\P(X=0)=0$. Да. Нет.
\item Для неотрицательной случайной величины $\E(X)\ge \E(-X)$. Да.
Нет.
\item Вероятность бывает отрицательной. Да. Нет.
\end{enumerate}

\subsubsection*{Часть II.}

Стоимость задач 10 баллов.

\begin{enumerate}
% числа выверены
\item На день рождения к Васе пришли две Маши, два Саши, Петя и Коля. Все вместе с Васей сели за круглый стол. Какова вероятность, что Вася окажется между двумя тезками?

% числа выверены
\item Поезда метро идут регулярно с интервалом 3 минуты. Пассажир
приходит на платформу в случайный момент времени. Пусть $X$ —
время ожидания поезда в минутах.

Найдите $\P(X<1)$, $\E(X)$.

%\textbf{Задача 2} \\ % числа выверены
%На десяти карточках написаны числа от 1 до 9. Число 8 фигурирует
%два раза, остальные числа - по одному разу. Карточки извлекают в
%случайном порядке. \\
%Какова вероятность того, что девятка появится позже обеих
%восьмерок? \\

% числа выверены
\item Жители уездного города N независимо друг от друга говорят правду с вероятностью $\frac{1}{3}$. Вчера мэр города заявил, что в 2014 году в городе будет проведен межпланетный шахматный турнир. Затем заместитель мэра подтвердил эту информацию.
Какова вероятность того, что шахматный турнир действительно будет проведен?

% числа выверены
\item Время устного ответа на экзамене распределено по экспоненциальному закону, т.е. имеет функцию плотности $p(t)=c\cdot e^{-0.1t}$ при $t>0$.
\begin{enumerate}
\item Найдите значение параметра $c$
\item Какова вероятность того, что Иванов будет отвечать более получаса?
\item Какова вероятность того, что Иванов будет отвечать еще более получаса, если он уже отвечает 15 минут?
\item Сколько времени в среднем длится ответ одного студента?
\end{enumerate}

%\textbf{Задача 5} \\ % числа выверены
%Допустим, что вероятности рождения мальчика и девочки одинаковы. Сколько детей должно быть в семье, чтобы вероятность того, что имеется по крайней мере один ребенок каждого пола была больше
%0,95? \\



%\textbf{Задача 7} \\ % числа выверены
%Известно, что предварительно зарезервированный билет на автобус
%дальнего следования выкупается с вероятностью 0,9. В обычном
%автобусе 18 мест, в микроавтобусе 9 мест. Компания «Микро»,
%перевозящая людей в микроавтобусах, допускает резервирование 10
%билетов на один микроавтобус. Компания «Макро», перевозящая
%людей в обычных автобусах допускает резервирование 20
%мест на один автобус. \\
%У какой компании больше вероятность оказаться в ситуации нехватки
%мест? \\

% числа выверены
\item Студент решает тест (множественного выбора) проставлением
ответов наугад. В тесте 10 вопросов, на каждый из которых 4
варианта ответов. Зачет ставится в том случае, если правильных
ответов будет не менее 5.
\begin{enumerate}
\item Найдите вероятность того, что студент правильно ответит только
на один вопрос
\item Найдите наиболее вероятное число правильных ответов
\item Найдите математическое ожидание и дисперсию числа правильных
ответов
\item Найдите вероятность того, что студент получит зачет
\end{enumerate}


 % числа выверены
\item Совместный закон распределения случайных величин  $X$  и  $Y$
задан таблицей:

\begin{tabular}{@{}cccc@{}}
\toprule
    & $Y=-1$ & $Y=0$ & $Y=2$ \\ \midrule
$X=0$ & $0.2$  & $c$   & $0.2$ \\
$X=1$ & $0.1$  & $0.2$ & $0.1$ \\ \bottomrule
\end{tabular}


Найдите  $c$,  $\P\left(Y>-X\right)$,  $\E\left(X\cdot Y \right)$, $\Corr(X,Y)$, $\E\left(Y|X>0\right)$
%в) При каких $\theta$ дисперсия будет наибольшей? При каких - наименьшей? \\

% числа выверены
\item Вася пригласил трех друзей навестить его. Каждый из них появится
независимо от другого с вероятностью $0.9$, $0.7$ и $0.5$
соответственно. Пусть $N$ — количество пришедших гостей. Найдите $\E(N)$


% числа выверены
\item Охотник, имеющий 4 патрона, стреляет по дичи до первого
попадания или до израсходования всех патронов. Вероятность
попадания при первом выстреле равна 0.6, при каждом последующем —
уменьшается на 0.1. Найдите
\begin{enumerate}
\item Закон распределения числа патронов, израсходованных охотником
\item Математическое ожидание и дисперсию этой случайной величины
\end{enumerate}
\end{enumerate}

\subsubsection*{Часть III.}

Стоимость задачи 20 баллов.

Требуется решить \textbf{\underbar{одну}} из двух задач (9-А или 9-Б) по
выбору!

\begin{enumerate}
\item[9-А.] У Мистера Х есть $n$ зонтиков. Зонтики мистер Х хранит дома и на работе. Каждый день утром мистер Х едет на работу, а каждый день вечером - возвращается домой. При этом каждый раз дождь идет с вероятностью 0.8 независимо от прошлого, (т.е. утром дождь идет с вероятностью 0.8 и вечером дождь идет с вероятностью 0.8 вне зависимости от того, что было утром). Если идет дождь и есть доступный зонтик, то мистер Х обязательно возьмет его в дорогу. Если дождя нет, то мистер Х поедет без зонтика.

Какой процент поездок окажется для мистера Х неудачными (т.е. будет идти дождь, а зонта не будет) в долгосрочном периоде?

\item[9-Б.] Начинающая певица дает концерты каждый день. Каждый ее концерт приносит продюсеру 0.75 тысяч евро. После каждого концерта певица может впасть в депрессию с вероятностью 0.5. Самостоятельно выйти из депрессии певица не может. В депрессии она не в состоянии проводить концерты. Помочь ей могут только цветы от продюсера. Если подарить цветы на сумму $0\le x\le 1$ тысяч евро, то она выйдет из депрессии с вероятностью $\sqrt{x}$.

Какова оптимальная стратегия продюсера?
\end{enumerate}

\subsection{Контрольная работа №1, демо-версия, ??.11.2008, решения}
\begin{enumerate}
\item $\P(A)=2/15$
\item $\P(X<1)=1/3$, $\E(X)=1.5$
\item
\item $c=0.1$, $\P(X>30)=e^{-3}$, $\P(X>45\mid X>15)=e^{-3}$, $\E(X)=10$
\item $k^*=2$, $\Var(X)=1.875$, $\E(X)=2.5$
\item $c=0.2$, $\P(Y>-X)=0.5$, $\E(XY)=0$, $\Corr(X,Y)=-0.155$, $\E(Y|X>0)=1/4$
\item $\E(N)=\E(X_1)+\E(X_2)+\E(X_3)=0.9+0.7+0.5=2.1$
\item $\E(X)\approx 1.7$, $\Var(X)\approx 1.08$
\item[9-А.]
\item[9-Б.] Рассмотрим совершенно конкурентный невольничий рынок начинающих певиц. Певицы в хорошем настроении продаются по $V_1$, в депрессии — по $V_2$. Получаем систему уравнений:
\[
\begin{cases}
  V_1 = 0.75 + (0.5 V_1 + 0.5 V_2) \\
  V_2 = \max_x \sqrt{x}V_1 + (1 - \sqrt{x})V_2 - x
\end{cases}
\]
Оптимизируем и получаем, $x^* = (V_1 - V_2)^2/4$. Из первого уравнения находим $(V_1 - V_2)/2=0.75$.
\end{enumerate}



\subsection{Контрольная работа №1, ??.11.2008}

\subsubsection*{Часть I.}

Верны ли следующие утверждения? Обведите ваш выбор.

\begin{enumerate}
\item Пуассоновская случайная величина является непрерывной. Нет.
\item Не существует случайной величины с $\E(X)=2008$ и $\Var(X)=2008$. Неверно.
\item $\P(A|B)=\P(A\cap B|B)$ для любых событий $A$ и $B$ . Да.
\item $\E(X/Y)=\E(X)/\E(Y)$ для любых случайных величин $X$ и $Y$. Нет.
\item При увеличении $t$ величина $\P(X\le t)$ не убывает. Да.
\item Для любых событий $A$ и $B$, выполняется $\P(A|B)+\P(A|B^{c})=1$. Нет.
\item События $A$ и $B$ независимы, если они не могут наступить одновременно. Нет.
\item Функция плотности может принимать значения больше 2008. Да.
\item Если $\P(A)=0.7$ и $\P(B)=0.5$, то события $A$ и $B$ могут быть несовместными. Нет.
\item Если $X$ — неотрицательная случайная величина, то $\P(X\le 0)=0$. Нет.
\end{enumerate}

\subsubsection*{Часть II.}

Стоимость задач 10 баллов.

%1. Простой эксперимент - вероятность
%2. Простой эксперимент (или изв. распределение) - вероятность и ожидание
%3. Условная вероятность
%4. Экспоненциальное распределение (или про функцию плотности), вер, увер, ожид
%5. Биномиальное и Пуассон
%6. Математическое ожидание с параметром (при каком параметре...)
%7. Разложение в сумму или муторные вычисления
%8. Сложный эксперимент - вер, ожидание, макс. веро-сть
% прочее - свойства Е, Вар, Ков


\begin{enumerate}
% числа выверены
\item Вася купил два арбуза у торговки тети Маши и один арбуз у торговки тети Оли. Арбузы у тети Маши спелые с вероятностью 90\% (независимо друг от друга), арбузы у тети Оли спелые с вероятностью 80\%.
\begin{enumerate}
\item Какова вероятность того, что все три Васиных арбуза будут спелыми?
\item Какова вероятность того, что хотя бы два арбуза из Васиных будут спелыми?
\item Каково ожидаемое количество спелых арбузов у Васи?
\end{enumerate}

% easy
\item Случайная величина $X$ может принимать только значения 5 и 9, с неизвестными вероятностями
\begin{enumerate}
\item Каково наибольшее возможное математическое ожидание величины $X$?
\item Какова наибольшая возможная дисперсия величины $X$?
\end{enumerate}

%\textbf{Задача 2} \\ % числа выверены
%Поезда метро идут регулярно с интервалом 3 минуты. Пассажир
%приходит на платформу в случайный момент времени. Пусть $X$ -
%время ожидания поезда в минутах. \\
%Найдите $\P(X<1)$, $\E(X)$ \\

%\textbf{Задача 2} \\ % числа выверены
%На десяти карточках написаны числа от 1 до 9. Число 8 фигурирует
%два раза, остальные числа - по одному разу. Карточки извлекают в
%случайном порядке. \\
%Какова вероятность того, что девятка появится позже обеих
%восьмерок? \\

% числа выверены
\item Предположим, что социологическим опросам доверяют 70\% жителей. Те, кто доверяют, опросам всегда отвечают искренне; те, кто не доверяют, отвечают наугад. Социолог Петя  в анкету очередного опроса включил вопрос «Доверяете ли Вы социологическим опросам?»
\begin{enumerate}
\item Какова вероятность, что случайно выбранный респондент ответит «Да»?
\item Какова вероятность того, что он действительно доверяет, если известно, что он ответил «Да»?
\end{enumerate}

\item Случайные величины $X$ и $Y$ независимы и имеют функции плотности $f(x)=\frac{1}{4\sqrt{2\pi } } e^{-\frac{1}{32} (x-1)^{2} }$ и $g(y)=\frac{1}{3\sqrt{2\pi } } e^{-\frac{1}{18} y^{2} }$ соответственно.
Найдите:
\begin{enumerate}
\item $\E(X)$, $\Var(X)$
\item $\E(X-Y)$, $\Var(X-Y)$
\end{enumerate}

%Пете и Васе предложили одну и ту же задачу. Они могут правильно решить ее с %вероятностями 0.7 и 0.8, соответственно. К задаче предлагается 5 ответов на выбор, %поэтому будем считать, что выбор каждого из пяти ответов равновероятен, если задача %решена неправильно. \\
%а) Какова вероятность несовпадения ответов Пети и Васи? \\
%б) Какова вероятность того, что Петя ошибся, если ответы совпали? \\
%в) Каково ожидаемое количество правильных решений, если ответы совпали? \\


\item Закон распределения пары случайных величин $X$ и $Y$ задан табличкой:

\begin{tabular}{@{}cccc@{}}
\toprule
    & $X=-1$ & $X=0$ & $X=2$ \\ \midrule
$Y=1$ & $0.2$  & $0.1$   & $0.2$ \\
$Y=2$ & $0.1$  & $0.2$ & $0.2$ \\ \bottomrule
\end{tabular}

Найдите: $\E(X)$, $\E(Y)$, $\Var(X)$, $\Cov(X,Y)$, $\Cov(2X+3,-3Y+1)$


% числа выверены
\item Время устного ответа на экзамене распределено по экспоненциальному закону, т.е. имеет функцию плотности $p(t)=c\cdot e^{-0.2t}$ при $t>0$.
\begin{enumerate}
\item Найдите значение параметра $c$
\item Какова вероятность того, что Иванов будет отвечать более двадцати минут?
\item Какова вероятность того, что Иванов будет отвечать еще более двадцати минут, если он уже отвечает 10 минут?
\item Сколько времени в среднем длится ответ одного студента?
\end{enumerate}

 % числа выверены
\item Полугодовой договор страховой компании со спортсменом-теннисистом, предусматривает выплату страхового возмещения  в случае травмы специального вида. Из предыдущей практики известно, что вероятность получения теннисистом такой травмы  в любой фиксированный день равна 0.00037. Для периода действия договора вычислите
\begin{enumerate}
\item Математическое ожидание числа страховых случаев
\item Вероятность того, что не произойдет ни одного страхового случая
\item Вероятность того, что произойдет ровно 2 страховых случая
\end{enumerate}

\item Большой Адронный Коллайдер запускают ровно в полночь. Оставшееся время до Конца Света — случайная величина $X$ распределенная равномерно от 0 до 16 часов. Когда произойдет Конец Света, механические часы остановятся и будут показывать время $Y$.
\begin{enumerate}
\item Найдите $\P(Y<2)$
\item Постройте функцию плотности для величины $Y$
\item Найдите $\E(Y)$, $\Var(Y)$
\item Найдите $\Cov(X,Y)$
\end{enumerate}
Комментарий: по остановившимся механическим часам, к примеру, невозможно отличить, прошло ли от пуска Коллайдера 2.7 часа или 14.7 часа, т.к. $Y$ принимает значения только на отрезке от 0 до 12 часов.
\end{enumerate}

\subsubsection*{Часть III.}

Стоимость задачи 20 баллов.

Требуется решить \textbf{\underbar{одну}} из двух задач (9-A или 9-B) по
выбору!

\begin{enumerate}
\item[9-А.] На даче у мистера А две входных двери. Сейчас у каждой двери стоит две пары ботинок. Перед каждой прогулкой он выбирает наугад одну из дверей для выхода из дома и надевает пару ботинок, стоящую у выбранной двери. Возвращаясь с прогулки мистер А случайным образом выбирает дверь, через которую он попадет в дом и снимает ботинки рядом с этой дверью. Сколько прогулок мистер А в среднем совершит, прежде чем обнаружит, что у выбранной им для выхода из дома двери не осталось ботинок?
% ответ: 12 \\

Источник: American Mathematical Monthly, problem E3043, (1984, p.310; 1987, p.79)

\item[9-Б.] Если смотреть на корпус Ж здания Вышки с Дурасовского переулка, то видно 70 окон расположенных прямоугольником $7\times 10$ (7 этажей, т.к. первый не видно, и 10 окон на каждом этаже). Допустим, что каждое из них освещено вечером независимо от других с вероятностью одна вторая. Назовем «уголком» комбинацию из 4-х окон, расположенных квадратом, в которой освещено ровно три окна (не важно, какие). Пусть $X$ - число «уголков», возможно пересекающихся, видимых с Дурасовского переулка.
Найдите  $\E(X)$ и $\Var(X)$

Примечание — для наглядности:
\begin{tabular}{|c|c|}
  \hline
  X & X\\
  \hline
    & X \\
  \hline
\end{tabular},
\begin{tabular}{|c|c|}
  \hline
  X & \\
  \hline
  X & X \\
  \hline
\end{tabular},
\begin{tabular}{|c|c|}
  \hline
   & X\\
  \hline
  X & X \\
  \hline
\end{tabular},
\begin{tabular}{|c|c|}
  \hline
  X & X\\
  \hline
  X &  \\
  \hline
\end{tabular} - это «уголки». \\
\begin{tabular}{|c|c|c|}
  \hline
  X & X & X\\
  \hline
    & X & \\
  \hline
  X & X & \\
  \hline

\end{tabular} — в этой конфигурации три «уголка»;
\begin{tabular}{|c|c|c|}
  \hline
  X &  & X\\
  \hline
    & X & \\
  \hline
  X &  & X\\
  \hline

\end{tabular} — а здесь — ни одного «уголка».
\end{enumerate}



\subsection{Контрольная работа №1, ??.11.2008, решения}

\begin{enumerate}
\item
\begin{enumerate}
\item $0.9\cdot 0.9\cdot 0.8$
\item $2\cdot 0.1\cdot 0.9\cdot 0.8+0.9\cdot 0.9\cdot 1=0.9(0.16+0.9)=0.9\cdot 1.06=0.954$
\item $0.9+0.9+0.8=2.6$
\end{enumerate}
\item
\begin{enumerate}
\item 9 (если взять 9 с вероятностью один)
\item 4 (если взять 5 и 9 равновероятно)
\end{enumerate}
\item
\begin{enumerate}
\item $0.7+0.3\cdot0.5=0.85$
\item $\frac{0.7}{0.85}=\frac{14}{17}\approx 0.82$
\end{enumerate}
\item Нормальная случайная величина имеют функцию плотности $p(t)=c\cdot \exp(-\frac{1}{2\sigma^{2}}(x-\mu)^{2})$

Отсюда: $\E(X)=1$, $\E(Y)=0$, $\Var(X)=16$, $\Var(Y)=9$
\item $\E(X)=0.5$, $\E(Y)=1.5$, $\Var(X)=1.65$, $\Cov(X,Y)=0.05$, $\Cov(2X+3,-3Y+1)=-0.3$
\item 0.2, $\frac{1}{e^4}$, $\frac{1}{e^4}$, 5
\item Любое разумное понимание «полугодовой» принимается. Т.е. подходят 182, 183, и если посчитаны только рабочие дни, и если взят пример марсианского теннисиста с указанием кол-ва дней в марсианском году и пр.

И биномиальные и пуассоновские ответы принимаются.

Для 182:
\begin{enumerate}
\item $182\cdot 0.00037=0.06734$
\item $(1-0.00037)^182\approx \exp(-0.06734)$
\item $C_{182}^{2}p^{2}(1-p)^{180}\approx 0.5\exp(-0.06734)0.06734^2$
\end{enumerate}
\item
\begin{enumerate}
\item $\P(Y<2)=1/4$
\item два отрезка: на высоте 2/16 (от 0 до 4) и 1/16 (от 4 до 12)
\item $\E(Y)=5$, $\Var(Y)=12.(3)$
\item $\Cov(X,Y)=3.(3)$
\end{enumerate}
\item[9-А.] Составляется граф по которому «блуждает» мистер А. Пишутся рекуррентные соотношения.
Получается 12 или 13 в зависимости от того, считать ли прогулку «босиком» или нет.
Оба ответа считать правильными.
\item[9-Б.] X раскладывается в сумму индикаторов.

Имеется $6\cdot9$ позиций для потенциального «уголка».

$\E(X)=6\cdot9\cdot1/4=13.5$

Имеется $6\cdot5+5\cdot9$ «боковых» пересечений потенциальных позиций.

Имеется $5\cdot8$ «угловых» пересечений потенциальных позиций.

Только они и могут дать ковариацию.

$\Var(X)=54\cdot1/4\cdot3/4+2\cdot(6\cdot8+5\cdot9)\cdot3/32+2\cdot5\cdot8\cdot5/64=541/16$
\end{enumerate}



\subsection{Контрольная работа №2, демо-версия, 26.12.2008}

\subsubsection*{Часть I.}

Обведите верный ответ:

\begin{enumerate}
\item Сумма двух нормальных независимых случайных величин нормальна.
Да. Нет.
\item Нормальная случайная величина может принимать отрицательные
значения. Да. Нет.
\item Пуассоновская случайная величина является непрерывной. Да. Нет.
\item Дисперсия суммы зависимых величин всегда не меньше суммы
дисперсий. Да. Нет.
\item Теорема Муавра-Лапласа является частным случаем центральной
предельной. Да. Нет.
\item Пусть $X$ — длина наугад выловленного удава в сантиметрах, а
$Y$ — в дециметрах. Коэффициент корреляции между этими
величинами равен $\frac{1}{10}$. Да. Нет.
\item Математическое ожидание выборочного среднего не зависит от
объема выборки, если $X_{i}$ одинаково распределены. Да. Нет.
\item Зная закон распределения $X$ и закон распределения $Y$
можно восстановить совместный закон распределения пары $(X,Y)$. Да. Нет.
\item Если  $X$ — непрерывная случайная величиа,  $\E\left(X\right)=6$  и
$\Var\left(X\right)=9$ , то  $Y=\frac{X-6}{3} \sim
\cN\left(0;1\right)$.  Да. Нет.
\item Если ты отвечать на первые 10 вопросов этого теста наугад, то
число правильных ответов — случайная величина, имеющая
биномиальное распределение. Да. Нет.
\end{enumerate}

Обозначения:

$\E(X)$ — математическое ожидание

$\Var(X)$ — дисперсия

\subsubsection*{Часть II.}

Стоимость задач 10 баллов.

%На что? \\
%1. совместная функция плотности \\+
%2. несмещенность (эффективность?) \\+
%3. хи-хи распределение \\+
%4. неравенство чебышева (добавить и при конкретном распределении) \\+
%5. совместное нормальное \\+
%6. цпт \\
%7. пуассоновский поток \\
%8. про лог-нормальное распределение \\+

\begin{enumerate}
% числа выверены
\item Совместная функция плотности имеет вид
\[
p_{X,Y} \left(x,y\right)=
\begin{cases}
x+y, & \text{ если } x\in \left[0;1\right],\, y\in \left[0;1\right] \\
0, & \text{ иначе}
\end{cases}
\]
Найдите  $\P\left(Y>2X\right)$ ,  $\E\left(X\right)$.
Являются ли величины $X$ и $Y$ независимыми?
%Решение: \\
%$\P(Y>2X)=\int_{0}^{1}\int_{0}^{y/2}(x+y)dxdy=\frac{5}{24}$ $[5]$\\
%$\E(X)=\int_{0}^{1}\int_{0}^{1}x(x+y)dxdy=\frac{7}{12}$ $[5]$\\
%(если интеграл выписан верно, но не взят, то $[3]$ вместо $[5]$)
%\\
% числа выверены
\item Случайный вектор  $\left(\begin{array}{c}
{X_{1} } \\ {X_{2} }
\end{array}\right)$  имеет нормальное распределение с
математическим ожиданием  $\left(\begin{array}{c} {2} \\ {-1}
\end{array}\right)$  и ковариационной матрицей
$\left(\begin{array}{cc} {9} & {-4.5} \\ {-4.5} & {25}
\end{array}\right)$.
\begin{enumerate}
\item Найдите  $\P\left(X_{1} +3X_{2} >20\right)$.
\item Какое условное распределение имеет $X_{1}$ при условии, что $X_{2}=0$?
\end{enumerate}

% числа выверены
\item Компания заключила 1000 однотипных договоров. Выплаты по каждому договору возникают независимо друг от друга с вероятностью 0.1. В случае наступления выплат их размер распределен экспоненциально со средним значением 1000 рублей.
\begin{enumerate}
\item Найдите дисперсию и среднее значение размера выплат по одному контракту.
\item Какова вероятность того, что компании потребуется более 110 тысяч рублей на выплаты по всем контрактам?
\end{enumerate}

% числа выверены
\item Определите, в каких границах может лежать $\P\left(\frac{(X-30)^{2}}{\Var(X)}<3\right)$, если известно, что $\E(X)=30$. Можно ли уточнить ответ, если дополнительно известно, что $X$ — экспоненциально распределена.

% числа выверены
\item Предположим, что величины $X_{1}$, $X_{2}$, ..., $X_{13}$ - независимы и распределены нормально $\cN(\mu,\sigma^{2})$. Найдите число $a$, если известно, что $\P( \sum (X_{i}-\bar{X})^{2}>a\sigma^{2})=0.1$.
\item Предположим, что оценки студентов на экзамене распределены равномерно на отрезке $[0;a]$. Вася хочет оценить вероятность того, что отдельно взятый студент наберет больше 30 баллов. Васе известно, что экзамен сдавали 100 человек и 15 из них набрали более 60 баллов. Помогите Васе построить несмещенную оценку!

Коля напрямую узнал у наугад выбранных 50 студентов, получили ли они больше 30 баллов. Какая оценка вероятности имеет меньшую дисперсию, Васина или Колина?

% цель: (a-30)/a
% дано: (a-60)/a

\item К продавцу мороженого подходят покупатели: мамы, папы и дети. Предположим, что это независимые Пуассоновские потоки с интесивностями 12, 10 и 16 чел/час.
\begin{enumerate}
\item Какова вероятность того, что за час будет всего 30 покупателей?
\item Какова вероятность того, что подошло одинаковое количество мам, пап детей, если за некий промежуток времени подошло ровно 30 покупателей?
\end{enumerate}

% числа выверены
\item Известно, что $X\sim \cN(\mu,\sigma^{2})$ и $Y=\exp(X)$. В таком случае говорят, что $Y$ имеет лог-нормальное распределение. Найдите $\E(Y)$.
\end{enumerate}


\subsubsection*{Часть III.}

Стоимость задачи 20 баллов.

Требуется решить \textbf{\underbar{одну}} из двух 9-х задач по
выбору!

\begin{enumerate}
\item[9-А.] There are two unfair coins. One coin has 0.7 probability head-up; the other has 0.3 probability head-up. To begin with, you have no information on which is which. Now, you will toss the coin 10 times. Each time, if the coin is head-up, you will receive \$1; otherwise you will receive \$0. You can select one of the two coins before each toss. What is your best strategy to earn more money?
\item[9-Б.] Дед Мороз развешивает новогодние гирлянды на аллее. Вдоль аллеи высажено 2008 елок. Каждой гирляндой Дед Мороз соединяет две елки (не обязательно соседние). В результате Дед Мороз повесил 1004 гирлянды и все елки оказались украшенными. Какова вероятность того, что существует хотя бы одна гирлянда, пересекающаяся с каждой из других? \\
Например, гирлянда 5-123 (гирлянда соединяющая 5-ую и 123-ю елки) пересекает гирлянду 37-78 и гирлянду 110-318.
\emph{Подсказка}: Думайте!
\end{enumerate}


\subsection{Контрольная работа №2, демо-версия, 26.12.2008, решения}

\begin{enumerate}
\item
$\P(Y>2X)=\int_0^{0.5} \int_{2x}^{1} (x+y) dydx = 5/24$

$f_X(x) = \int_0^1 (x+y) dy = x + \frac{1}{2} \Rightarrow \E(X) = \int_0^1 \left(x+\frac{1}{2}\right)x dx = \frac{7}{12}$
\item
\begin{enumerate}
\item $\E(X_1 + 3X_2) = -1$, $\Var(X_1 + 3X_2) = 207$

$\P(X_1 + 3X_2 > 20) = \P\left(\frac{X_1 + 3X_2 +1}{\sqrt{207}} > \frac{20+1}{\sqrt{207}} \right) = \P(\cN(0,1)>1.46) \approx 0.07$
\item $\E(X_1|X_2 = 0) = 2 - 4.5\cdot\frac{1}{25}(0+1) = 1.82$, $\Var(X_1 | X_2 = 0) = 9 - (-4.5) \cdot \frac{1}{25} \cdot(-4.5) = 8.19$

$X_1 | X_2 = 0 \sim \cN(1.82, 8.19)$
\end{enumerate}
\item
\begin{enumerate}
\item $\E(X)=1000$, $\Var(X) = 10^6$
\item $X$ — случайная величина, сумма выплат по одному контракту, $X \sim \exp(0.001)$

$\P(1000X>110000) = \P(X>100) = \int_{110}^{+\infty} 0.001 \exp(-0.001x) dx \approx 0.9$
\end{enumerate}
\item $\P\left(\frac{(X-30)^{2}}{\Var(X)}<3\right) = \P(|X-30|< \sqrt{3\Var(X)}) \geq 1 - \frac{\Var(X)}{ \sqrt{3\Var(X)}} = \frac{2}{3}$

$\P(|X-30|< \sqrt{3\Var(X)}) = \P(X< \sqrt{3\cdot900}+30) = \int_{0}^{82}\frac{1}{30} \exp\left(-\frac{1}{30}x\right)dx\approx 0.94$
\item $a \approx 1.28$
\item[9-Б.]
Подразумевая под точками концы гирлянды, докажем следующее утверждение.

Бросим $2n \geq 4$ точек $X_1, X_2, \ldots, X_{2n}$ случайным образом на отрезок $[0;1]$. Пусть для $1 \leq i \leq n$ $J_i$ — это отрезок с концами $X_{2i-1}$ и $X_{2i}$.
Тогда вероятность того, что найдётся такой отрезок $J_i$, который пересекает все другие отрезки, равна $2/3$ и не зависит от $n$.

Доказательство. Бросим $2n+1$ точек на окружность, тогда $2n$ точек образуют пары, а оставшуюся обозначим $X$ и будем считать её и началом, и концом отрезка.
Каждому получившемуся отрезку присвоим уникальное имя.
Далее, будем двигаться от точки $X$ по часовой стрелке до тех пор, пока не встретим одно и то же имя дважды, например «а».
После этого пойдём в обратную сторону, и будем идти, пока не встретим какое-нибудь другое имя дважды, например, «б».
Теперь посмотрим на получившуюся последовательность между «б» и «а» на концах пути, читая её по часовой стрелке от «б» до «а».
Нас интересует взаимное расположение $X$, второй «а» и второй «б».
Зная, что «а» стоит после $X$, выпишем все возможные случаи, где может стоять «б»:
\begin{enumerate}
\item перед $X$
\item между $X$ и «а»
\item после «а»
\end{enumerate}
Покажем, что во втором и третьем случае отрезок «б» пересекает все остальные, а в первом такого отрезка вообще нет. Попутно заметим, что появление каждого и случаев равновероятно.

Действительно, если «б» стоит после $X$, и отрезок соответствующий этому имени, не пересекает какой-нибудь другой отрезок «в», то последовательность выглядела бы как «бвв$X$б» или «б$X$ввб», что противоречит описанному построению.
Если «б» стоит перед $X$ и отрезок «в» пересекает оба отрезка «а» и «б», то мы снова приходим в противоречие с построением.
В итоге, получаем, что искомая вероятность равна $2/3$.
\end{enumerate}

\subsection{Контрольная работа №2, 26.12.2008}

\subsubsection*{Часть I.}

Обведите верный ответ:
\begin{enumerate}
\item Если пара величин $(X,Y)$ имеет совместное нормальное распределение, то каждая случайная величина по отдельности также имеет нормальное распределение. Верно.
\item Неравенство Чебышева неприменимо к дискретным случайным величинам. Нет.
\item Нормальная случайная величины является дискретной. Нет.
\item Дисперсия любой несмещенной оценки не превосходит дисперсию любой смещенной. Нет.
\item При большом количестве степеней свободы хи-квадрат распределение похоже на нормальное. Верно.
\item Сумма ста независимых равномерных на $[0;1]$ величин является равномерной случайной величиной на $[0;100]$. Нет.
\item Ковариация всегда больше корреляции по модулю.  Нет.
\item Если величины $X$ и $Y$ одинаково распределены и $\P(X=Y)=0.9999$, то корреляция $X$ и $Y$ близка к единице. Нет.

Комментарий: корреляция показывает насколько согласованно величины изменяются. Например, взяв X c законом распределения:
\begin{center}
\begin{tabular}{@{}cccc@{}}
\toprule
$x$         & $1/2$     & $1$      & $2$       \\ \midrule
$\P(X=x)$ & $0.00005$ & $0.9999$ & $0.00005$ \\ \bottomrule
\end{tabular}
\end{center}
и $Y=1/X$ получим отрицательную корреляцию между $X$ и $Y$.
\item Нормально распределенная величина $X$ и биномиально распределенная величина $Y$ могут быть зависимы. Запросто.
\item Дисперсия суммы положительных величин всегда больше суммы дисперсий. Нет.
\item Раз уж выпал свежий снег, то вместо контрольной можно
было бы покататься на лыжах! Неплохо бы.
\end{enumerate}

\subsubsection*{Часть II.}

Стоимость задач 10 баллов.

\begin{enumerate}
% числа выверены
\item Совместная функция плотности имеет вид
\[
p_{X,Y} \left(x,y\right)=
\begin{cases}
\frac{3}{2}x+\frac{1}{2}y, & \text{ если } x\in \left[0;1\right],\, y\in \left[0;1\right] \\
0, & \text{ иначе}
\end{cases}
\]
\begin{enumerate}
\item Найдите  $\P\left(Y>X\right)$,  $\E\left(Y\right)$
\item Являются ли величины $X$ и $Y$ независимыми?
\end{enumerate}

% числа выверены
\item Пусть $X_{i}$ — независимы и одинаково распределены, причем $\E(X_{i})=0$, $\Var(X_{i})=1$ и $\Var(X_{i}^{2})=2$
\begin{enumerate}
\item С помощью неравенства Чебышева оцените $\P(|X_{1}+X_{2}+...+X_{7}|>14)$ и $\P(X_{1}^{2}+X_{2}^{2}+...+X_{7}^{2}>14)$
\item Найдите эти вероятности, если дополнительно известно, что $X_{i}$ - нормально распределены.
\end{enumerate}

% числа выверены
\item Случайный вектор  $\left(\begin{array}{c}
{X_{1} } \\ {X_{2} }
\end{array}\right)$  имеет нормальное распределение с
математическим ожиданием  $\left(\begin{array}{c} {1} \\ {2}
\end{array}\right)$  и ковариационной матрицей
$\left(\begin{array}{cc} {9} & {-5} \\ {-5} & {25}
\end{array}\right)$.
\begin{enumerate}
\item Найдите  $\P\left(X_{1} +2X_{2} >20\right)$.
\item Какое условное распределение имеет $X_{1}$ при условии, что $X_{2}=0$?
\end{enumerate}

%Решение: \\

%(если интеграл выписан верно, но не взят, то $[3]$ вместо $[5]$)
%\\

\item Случайные величины $X$ и $Y$ независимы и равномерно распределены: $X$ — на отрезке $[0;a]$, а $Y$ — на отрезке $[0;3a]$.
Вася знает значение $XY$ и хочет оценить неизвестный параметр $\beta=\E(X^{2})$.
Петя знает значение $Y^2$ и хочет оценить тот же параметр $\beta$
\begin{enumerate}
\item Какую несмещенную оценку может построить Вася?
\item Какую несмещенную оценку может построить Петя?
\item У какой оценки дисперсия меньше?
\end{enumerate}

%Предположим, что ежемесячный уровень инфляции - случайная величина, равномерно %распределенная на отрезке от 0\% до $a$\%. Банк предлагает срочные двухмесячные вклады %на следующих условиях: месячная процентная ставка по вкладу равна уровню инфляции в %прошлом месяце и не может быть изменена во время действия договора. Т.е. если в %прошлом месяце инфляция была в 2\%, то

\item Вася играет в компьютерную игру, где нужно убить 80 однотипных монстров, чтобы пройти уровень. Количество патронов, которое Вася тратит на одного монстра имеет Пуассоновское распределение со средним значение 2 патрона.
\begin{enumerate}
\item Какова вероятность того, что на трех первых монстров придется потратить 6 патронов?
\item Какова вероятность того, на всех монстров уровня придется потратить более 200 патронов?
\end{enumerate}

\item Допустим, что срок службы пылесоса имеет экспоненциальное распределение. В среднем один пылесос бесперебойно работает 10 лет. Завод предоставляет гарантию 7 лет на свои изделия. Предположим для простоты, что все потребители соблюдают условия гарантии.
\begin{enumerate}
\item Какой процент потребителей в среднем обращается за гарантийным ремонтом?
\item Какова вероятность того, что из 1000 потребителей за гарантийным ремонтом обратится более 55\% покупателей?
\end{enumerate}
Подсказка: $\ln(2)\approx 0.7$

% числа выверены
\item Вася и Петя решают тест из 10 вопросов по теории вероятностей (на каждый вопрос есть два варианта ответа). Петя кое-что знает по первым пяти вопросам, поэтому вероятность правильного ответа на каждый равняется 0.9 независимо от других. Остальные пять вопросов Пете непонятны и он отвечает на них наугад. Вася списывает у Пети вопросы с 3-го по 7-ой, а остальные отвечает наугад.
Пусть $X$ - число правильных ответов Пети, а $Y$ - число правильных ответов Васи.
Найдите $\Var(X)$, $\Var(Y)$, $\Var(X-Y)$.

% числа выверены
\item Стоимость выборочного исследования генеральной совокупности, состоящей из 3 страт, определяется по формуле: $TC=n_{1}\cdot c_{1}+n_{2}\cdot c_{2}+n_{3}\cdot c_{3}$,
Где $c_{i}$ — стоимость наблюдения из $i$-ой страты, $n_{i}$ — число наблюдений в выборке, относящихся к страте $i$.
Предполагая, что стоимость исследования $TC$ фиксирована и равна 7000, определите значения $n_{i}$, при которых дисперсия соответствующего выборочного стратифицированного среднего достигает наименьшего значения, если:

\begin{tabular}{@{}lrrr@{}}
\toprule
Страта             & 1     & 2     & 3     \\ \midrule
Среднее значение   & $40$  & $80$  & $150$ \\
Стандартная ошибка & $10$  & $20$  & $60$  \\
Вес                & $20\%$ & $20\%$ & $60\%$ \\
Цена наблюдения    & $4$   & $16$  & $25$  \\ \bottomrule
\end{tabular}

Примечание: Округлите полученные значения до ближайших целых.
\end{enumerate}



\subsubsection*{Часть III.}

Стоимость задачи 20 баллов.

Требуется решить \textbf{\underbar{одну}} из двух 9-х задач по выбору!

\begin{enumerate}
\item[9-А.] Усама бен Ладен хочет сделать запас в 1000 тротиловых шашек в пещере А. Тротиловые шашки производят на секретном заводе бесплатно. При транспортировке от завода до пещеры каждая шашка взрывается с небольшой вероятностью $p$. Если взрывается одна шашка, то взрываются и все остальные, перевозимые вместе с ней. Сам Усама при взрыве всегда чудом остается жив. Какими партиями нужно переносить шашки, чтобы минимизировать среднее число переносок?

$[$В стартовой пещере бесконечный запас шашек$]$.

\item[9-Б.] У Пети нет денег, но он может сыграть 100 игр следующего типа.

В каждой игре Петя может по своему желанию:

- либо без риска получить $1$ рубль,

- либо назвать натуральное число $n>1$ и выиграть $n$ рублей с вероятностью $\frac{2}{n+1}$ или проиграть $1$ рубль с вероятностью $\frac{n-1}{n+1}$.

Чтобы выбирать вторую альтернативу Петя должен иметь как минимум рубль. Пете позарез нужно 200 рублей. Как выглядит Петина оптимальная стратегия?
\emph{Подсказка}: Думайте!
\end{enumerate}



\subsection{Контрольная работа №2, 26.12.2008, решения}

\begin{enumerate}
\item
\begin{enumerate}
\item $\int_{0}^{1}\int_{x}^{1}p(x,y)dydx=5/12$

$\int_{0}^{1}\int_{0}^{1}y\cdot p(x,y)dydx=13/24$
\item Нет, т.к. совместная функция плотности не разлагается в произведение индивидуальных
\end{enumerate}
\item
\begin{enumerate}
\item $\P(|X_{1}+X_{2}+...+X_{7}|>14)\leq \frac{7}{14^2}=\frac{1}{28}$

$\P(X_{1}^{2}+...+X_{7}^{2}>14)=\P(X_{1}^{2}+...+X_{7}^{2}-7>7)=\P(|X_{1}^{2}+...+X_{7}^{2}-7|>7)\leq \frac{2\cdot 7}{7^2}=\frac{2}{7}$
\item $\P(|X_{1}+...+X_{7}|>14)=\P(|N(0;1)|>14/\sqrt{7})=\P(|N(0;1)|>5.29)\approx 0$

$\P(X_{1}^{2}+X_{2}^{2}+...+X_{7}^{2}>14)\approx 0.05$
\end{enumerate}
\item
\begin{enumerate}
\item $X_{1}+2X_{2}\sim \cN(5;89)$, $\P(Z>1.59)=0.056$
\[
\Var(X_1+2X_2)=\Var(X_1)+4\Var(X_2)+4\Cov(X_1,X_2)=89
\]
\item Нормальное, причем $\cN(1.4;8)$, корреляция равна $-1/3$
\end{enumerate}
\item $\beta=\frac{1}{3}a^{2}$

$\E(XY)=\frac{3}{4}a^{2}$

$\E(Y^{2})=3a^{2}$

$\hat{\beta}_{1}=\frac{4}{9}XY$

$\hat{\beta}_{2}=\frac{1}{9}Y^{2}$

Так как обе оценки несмещенные вместо сравнения дисперсий можно сравнить квадраты ожиданий

$\frac{16}{81}\E(X^{2}Y^{2})$ vs $\frac{1}{81}\E(Y^{4})$

\ldots

$16 a^4$ vs $\frac{81}{5} a^{4}$

Дисперсия васиной оценки меньше.
\item Заметим, что Пуассоновская величина с положительной вероятностью принимает значение ноль, значит бывает, что монстры дохнут от одного устрашающего взгляда Васи :)
\begin{enumerate}
\item Сумма трех независимых пуассоновских величин - пуассоновская с параметром: $3\lambda=6$.

$\P(X=6)=e^{-6}\frac{6^6}{6!}\approx 0.16$

Ответ с факториалам считается полным.
\item Сумма 80 величин имеет пуассоновское распределение, но при большом количестве слагаемых пуассоновское мало отличается от нормального.

$\E(S)=160$, $\Var(S)=160$

$\P(S>200)=\P\left(\frac{S-160}{\sqrt{160}}>3.16\right)\approx 0$
\end{enumerate}
\item
\begin{enumerate}
\item $\lambda=1/10$, $\P(X<7)=0.5$
\item $\P(\bar{X}>0.55)=\P(N(0;1)>\frac{0.05\sqrt{1000}}{0.5})=\P(N(0;1)>3.16)\approx 0$
\end{enumerate}
\item $\Var(X)=5\cdot 0.1\cdot 0.9+5\cdot 0.5\cdot 0.5=1.7$

$\Var(Y)=3\cdot 0.1\cdot 0.9+7\cdot 0.5\cdot 0.5=2.02$

Пусть $Z$ — число правильных ответов на вопросы с 3-го по 7-ой (у Пети и у Васи)
\begin{multline*}
\Cov(X,Y)=\Cov(Z+(X-Z),Z+(Y-Z))=\Var(Z)+ \\
+ \Cov(X-Z,Z)+\Cov(Z,Y-Z)+\Cov(X-Z,Y-Z)=\Var(Z)
\end{multline*}
$Y-Z$ — это сколько правильных ответов дал лично Вася и оно не зависит от числа $Z$ правильных списанных ответов, значит $\Cov(Y-Z,Z)=0$

Аналогично все остальные ковариации равны нулю.

$\Var(Z)=3\cdot 0.1\cdot 0.9+2\cdot 0.5\cdot 0.5=0.77$
\item Любые совпадения с курсом экономической и социальный статистики случайны и непреднамеренны.

Чтобы оценка среднего по всем трем стратам была несмещена, она должны строиться по формуле:

$\bar{X}=w_{1}\bar{X}_{1}+w_{2}\bar{X}_{2}+w_{3}\bar{X}_{3}$ (здесь $\bar{X}_{i}$ — среднее арифметическое по $i$-ой страте)

Поэтому $\Var(\bar{X})$ (минимизируемая функция) равняется:

$\Var(\bar{X})=\sum \frac{w^{2}_{i}\sigma^{2}_{i}}{n_{i}}$

Принцип кота Матроскина\footnote{«Чтобы продать что-нибудь ненужное, нужно сначала купить что нибудь ненужное. А у нас денег нет!»} (aka бюджетное ограничение):  $4n_{1}+16n_{2}+25n_{3}=7000$

Решаем Лагранжем и получаем ответ: 35, 35, 252.

Некоторые маньяки наизусть знают:

$n_{i}=\frac{C}{\sum w_{i}\cdot \sigma_{i}\cdot\sqrt{c_{i}}}\frac{w_{i}\cdot \sigma_{i}}{\sqrt{c_{i}}}$
\item[9-А.] Замечание: неудачные переноски считаются, так как иначе решение тривиально — пробовать нести по 1000 шашек.
\begin{enumerate}
\item  Так как $p$ небольшая будем считать, что $\ln(1-p)\approx -p$. Уже страшно, да?
\item Допустим, что $s(n)$ оптимальная стратегия, указывающая, сколько нужно брать сейчас шашек, если осталось перенести $n$ шашек. Возможно, что $s$ зависит от $n$.
Обозначим $e(n)$ ожидаемое количество переносов при использовании оптимальной стратегии.
\item Начинаем:

$s(1)=1$, $e(1)=1/(1-p)$

$s(n)=\argmin_{a}(1/(1-p)^{a}+e(n-a))$, $e(n)=\min_{a}(1/(1-p)^{a}+e(n-a))$

Замечаем, что поначалу (где-то до $1/p$ шашек) все идет хорошо, а затем плохо...
\item Ищем упрощенное решение вида $s(n)=s$.

Ожидаемое число переносок равно $\frac{1000}{s}\frac{1}{(1-p)^{s}}$

Минимизируем по $s$. Получаем: $s=-1/\ln(1-p)\approx 1/p$.
\item Для тех кому интересно, точный график (10000 шашек, p=0.01):
\end{enumerate}

$[$Здесь оставлено место для картины Усама-Бен-Ладен будь он не ладен таскает шашки.$]$
%\begin{figure}[h]
%    \includegraphics{usama.eps}
%\end{figure}

ps. В оригинале мы сканировали ксерокопию учебника Микоша. Сканер был очень умный: в него нужно положить стопку листов, а на выходе он выдавал готовый pdf файл. Проблема была в том, что он иногда жевал бумагу. В этом случае, он обрывал сканирование и нужно было начинать все заново. Возник вопрос, какого размера должна быть партия, чтобы минимизировать число подходов к ксероксу.

\item[9-Б.]
\begin{enumerate}
\item Если сейчас 0 долларов, то брать 1 доллар.

Назовем ситуацию, «шоколадной» если можно выиграть без риска. То есть если игр осталось больше, чем недостающее количество денег.
\item Если игрок не в шоколаде, то оптимальным будет рисковать на первом ходе.

Почему? Получение одного доллара можно перенести на попозже.
\item В любой оптимальной стратегии достаточно одного успеха для выигрыша.

Почему? Допустим стратегии необходимо два успеха в двух рискованных играх. Заменим их  на одну рискованную игру. Получим большую вероятность.

Оптимальная стратегия:

Если сейчас 0 долларов, то брать доллар.

Пусть $d$ — дефицит в долларах, а $k$ — число оставшихся попыток.

Если $d\le k$, то брать по доллару.

Если $d>k$, то с риском попробовать захапнуть $1+d-k$ долларов.
\end{enumerate}
\end{enumerate}



\subsection{Контрольная работа №3, 02.03.2009}

\subsubsection*{Часть I.}

Обведите верный ответ:

\begin{enumerate}
\item Если $X\sim \cN(0;1)$, то $X^{2}\sim \chi^{2}_{1}$. Верно. Нет.
\item Если $X\sim t_{n}$ и $Y\sim t_{m}$, то $\frac{X/n}{Y/m}\sim F_{n,m}$. Верно. Нет.
\item Если основная гипотеза отвергается  при 1\% уровне значимости, то она будет отвергаться и при 5\% уровне значимости. Верно. Нет.
\item Неравенство Рао-Крамера справедливо только для оценок максимального правдоподобия. Верно. Нет.
\item Оценки метода максимального правдоподобия всегда несмещенные. Верно. Нет.
\item Ошибка второго рода происходит при отвержении основной гипотезы, когда она верна. Верно. Нет.
\item Из несмещенности оценки следует её состоятельность. Верно. Нет.
\item Длина доверительного интервала увеличивается при увеличении уровня доверия (доверительной вероятности) Верно. Нет.
\item Выборочное среднее независимых одинаково распределенных случайных величин с конечной дисперсией имеет асимптотически нормальное распределение. Верно. Нет.
\item Теорема Муавра-Лапласа  является частным случаем ЦПТ. Верно. Нет.
\item Оценка, получаемая за эту контрольную, является несмещенной. Верно. Нет.
\end{enumerate}

Любой ответ на 11 считается правильным.

\subsubsection*{Часть II-A.}

Стоимость задач 10 баллов. Теория вероятностей.

Нужно решить любые \textbf{\underbar{3 (три)}} задачи из части II-A.

\begin{enumerate}
% числа выверены
\item При контроле правдивости показаний подозреваемого на «детекторе лжи» вероятность признать ложью ответ, не соответствующий действительности, равна 0.99, вероятность ошибочно признать ложью правдивый ответ равна 0.01. Известно, что ответы, не соответствующие действительности, составляют 1\% всех ответов подозреваемого.
Какова вероятность того, что ответ, признанный ложью, и в самом деле не соответствует действительности?
\item Предположим, что вероятность того, что среднегодовой доход наугад выбранного жителя некоторого города не превосходит уровень $t$, равна $\P(I\le t)=a+be^{-t/300}$ при $t\ge 0$.
Найдите:
\begin{enumerate}
\item Числа $a$ и $b$
\item Средний доход жителей этого города (математическое ожидание, моду и медиану распределения). Какую из данных характеристик следует использовать для рапорта о высоком уровне жизни?
\end{enumerate}

\item Доходности акций двух компаний являются случайными величинами $X$ и $Y$ с одинаковым математическим ожиданием и ковариационной матрицей $\left( \begin{array}{cc}
   4 & -2  \\
   -2 & 9  \\
\end{array}\right)$
\begin{enumerate}
\item Найдите $\Corr(X,Y)$
\item В какой пропорции нужно приобрести акции этих двух компаний, чтобы дисперсия доходности получившегося портфеля была наименьшей?

Подсказка: Если $R$ - доходность портфеля, то $R=\alpha X+(1-\alpha)Y$
\item Можно ли утверждать, что величины $X+Y$ и $7X-2Y$ независимы?
\end{enumerate}

\item Волшебный Сундук

Если присесть на Волшебный Сундук, то сумма денег, лежащих в нем, увеличится в два раза. Изначально в Сундуке был один рубль. Предположим, что «посадки» на Сундук - Пуассоновский процесс с интенсивностью $\lambda$. Каково ожидаемое количество денег в Сундуке к моменту времени $t$?

\item На окружности единичной длины случайным образом равномерно и независимо друг от друга выбирают две дуги: длины 0.3 и длины 0.4.
\begin{enumerate}
\item Найдите функцию распределения длины пересечения этих отрезков.
\item Найдите среднюю длину пересечения.
\end{enumerate}
\end{enumerate}

\subsubsection*{Часть II-B.}

Стоимость задач 10 баллов. Построение и свойства оценок.

Нужно решить любые \textbf{\underbar{2 (две)}} задачи из части II-B.

\begin{enumerate}
\item[6.] Асимметричная монета подбрасывается $n$ раз. При этом $X$ раз выпал «орел».
\begin{enumerate}
\item Методом максимального правдоподобия найдите оценку вероятности «орла»
\item Проверьте является ли полученная оценка состоятельной, несмещенной и эффективной.
\item Считая, что $n$ велико, укажите, в каких пределах с вероятностью 0,95 должно находиться значение оценки, если монета симметрична.
\end{enumerate}

\item[7.] Вася попадает по мишени с вероятностью $p$ при каждом выстреле независимо от других. Он стрелял до 3-х промахов (не обязательно подряд). При этом у него получилось $X$ попаданий.
\begin{enumerate}
\item Постройте оценку $p$ с помощью метода максимального правдоподобия.
\item Является ли полученная оценка несмещенной?
\end{enumerate}

\item[8.] Известно, что величины $X_{1}$, ..., $X_{n}$ независимы и равномерны на $[0;b]$. Пусть $Y$ — это минимум этих $n$ величин. Вася знает $n$ и $Y$.
\begin{enumerate}
\item Найдите оценку $b$ методом моментов.
\item Является ли полученная оценка несмещенной?
\end{enumerate}
\end{enumerate}

\subsubsection*{Часть II-C.}

Стоимость задач 10 баллов. Проверка гипотез и доверительные интервалы.

Нужно решить любые \textbf{\underbar{3 (три)}} задачи из части II-C.

\begin{enumerate}
\item[9.] Вес выпускаемого заводом кирпича распределен по нормальному закону. По выборке из 16 штук средний вес кирпича составил 2.9 кг, выборочное стандартное отклонение 0.3. Постройте 80\% доверительные интервалы для истинного значения веса кирпича и стандартного отклонения.

Примечание: можно строить односторонний интервал для стандартного отклонения, если таблицы не хватает, чтобы построить двусторонний.

\item[10.] В городе N за год родились 520 мальчиков и 500 девочек. Считая вероятность рождения мальчика неизменной:
\begin{enumerate}
\item Проверить гипотезу о том, что мальчики и девочки рождаются одинаково чаще против гипотезы о том, что мальчиков рождается больше, чем девочек
\item Вычислить p-значение (минимальный уровень значимости, при котором основная гипотеза отвергается)
\item Каким должен быть размер выборки, чтобы с вероятностью 0.95 можно было утверждать, что выборочная доля отличается от теоретической не более, чем на 0.02?
\end{enumerate}

\item[11.] Проверьте гипотезу о независимости пола респондента и предпочитаемого им сока.

\begin{tabular}{@{}cccc@{}}
\toprule
  & Апельсиновый & Томатный & Вишнёвый \\ \midrule
Мужчины & $70$         & $40$     & $25$     \\
Женщины & $75$         & $60$     & $35$     \\ \bottomrule
\end{tabular}

\item[12.] Даны независимые выборки доходов выпускников двух ведущих экономических вузов A и B, по 50 выпускников каждого вуза: $\bar{X}_{A}=650$, $\bar{X}_{B}=690$, $\hat{\sigma}_{A}=50$, $\hat{\sigma}_{B}=70$. \\ Предполагая, что распределение доходов подчиняется нормальному закону, проверьте гипотезу об отсутствии преимуществ выпускников вуза B (уровень значимости 0.05).

\item[13.] Величины $X_{1}$, $X_{2}$, ..., $X_{100}$ независимы и распределены $\cN(10,16)$. Вася знает дисперсию, но не знает среднего. Поэтому он строит 60\% доверительный интервал для истинного среднего значения. Какова вероятность того, что:
\begin{enumerate}
\item Доверительный интервал накрывает настоящее среднее?
\item Доверительный интервал накрывает число 9?
\end{enumerate}
\end{enumerate}

\subsubsection*{Часть III.}

Стоимость задачи 20 баллов.

Нужно решить любую \textbf{\underbar{1 (одну)}} задачу из части III.

\begin{enumerate}
\item[14-A.] Набранную книгу независимо друг от друга вычитывают два корректора. Первый корректор обнаружил $m_{1}$ опечаток, второй заметил $m_{2}$ опечаток. При этом $m$ опечаток оказались обнаруженными и первым, и вторым корректорами.
\begin{enumerate}
\item Постройте любым методом состоятельную оценку для общего числа опечаток (замеченных и незамеченных).
\item Является ли построенная оценка несмещенной?
\end{enumerate}

\item[14-B.] Вася хочет купить чудо-швабру! Магазинов, где продается чудо-швабра, бесконечно много. Любое посещение магазина связано с издержками равными $c>0$. Цена чудо-швабры в каждом магазине имеет равномерное распределение на отрезке $[0;M]$. Цены в магазинах не меняются, то есть при желании Вася может вернуться в уже посещенный им магазин для совершения покупки.
\begin{enumerate}
\item Как выглядит оптимальная стратегия Васи? (Вася нейтрален к риску).
\item Каковы ожидаемые Васины затраты при использовании оптимальной стратегии?
\item Сколько магазинов в среднем будет посещено?
\end{enumerate}
\emph{Подсказка}: Думайте!
\end{enumerate}
