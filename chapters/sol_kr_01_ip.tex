\section{Решения контрольной номер 1. ИП}

\subsection[2017-2018]{\hyperref[sec:kr_01_ip_2017_2018]{2017-2018}}
\label{sec:sol_kr_01_ip_2017_2018}



\begin{enumerate}

\item Обозначим вероятность того, что сыр достанется Белому за $b$, если игра начинается с его броска.

\begin{enumerate}
\item Получаем уравнение
\[
	b = \frac{1}{12} + \frac{11}{12} \cdot \frac{11}{12} b
\]

Пояснение: Как Белый может победить в исходной игре? Либо сразу выкинуть 6 с вероятностью $1/12$.
Либо передать ход Серому ($11/12$), получить ход снова ($11/12$) и выиграть в продолжении игры.
Продолжение игры по сути совпадает с исходной игрой.

\item Игра продолжается до тех пор, пока кто-то не выкинет «6».
Для нахождения среднего количества бросков воспользуемся методом первого шага.

Обозначим среднее количество бросков нашей игры за $S$.
Когда Белый бросает кубик, с вероятностью $\frac{1}{12}$ игра закончится за один бросок,
а с вероятностью $\frac{11}{12}$ игра продолжится и ход перейдёт к Серому.
Но та игра, которая начнётся, когда бросать будет Серый, ничем не отличается от предыдущей,
поэтому среднее количество бросков в ней будет равно $S$.
Однако мы попадём в эту игру, «потратив» один бросок. Таким образом мы получаем:

\[
S = \frac{1}{12} \cdot 1 + \frac{11}{12}(S +1)
\]

Получается, что $S = 12$, значит игра длится в среднем 12 бросков.
\end{enumerate}

\item

\item Для того, чтобы выжить, мышам нужно ещё до начала игры договориться о стратегии,
которая позволит им с наибольшей вероятностью открыть нужные сундуки.
Если хотя бы две мыши выберут одинаковый сундук, то их в любом случае съедят.
Поэтому одной из оптимальных стратегий будет ещё до начала игры мышам договориться и назвать левый сундук золотым,
сундук посередине серебряным, а правый — платиновым.
Каждый мышонок должен открыть тот сундук, в честь которого назван необходимый ему металл.
Если внутри он обнаруживает свой металл, то он выбирает этот сундук,
если внутри находится не тот металл, мышонок открывает тот сундук,
на который указывает лежащий внутри предмет.

Например, первым заходит Микки Маус. Он открывает золотой (левый) ящик.
Если внутри лежит золото, то он выходит из комнаты. Если же внутри лежит, например, серебро,
то Микки Маус открывает сундук посередине.
Путём перебора можно посчитать, что в 4 случаях из 6 мыши смогут найти нужный металл,
поэтому вероятность выигрыша при данной стратегии равна $\frac{2}{3}$.

\item

\item Благосостояние кота Василия, положившего один гурд на вклад,
равно $m_t = 1\cdot e^{rt}$, где $r$ — процентная ставка, а $t$ — прошедшее время.
Момент закрытия вклада $T$ равномерно распределён на отрезке от 0 до $a$,
поэтому сумма, которую получит Василий, представима в виде $Z = e^{Y}$, где $Y \sim U[0; ra]$.
По условию, $a$ очень велико, поэтому $ra$ тоже очень велико.

Вероятность того, что первая цифра будет равна 1, равна вероятности того,
что доход Василия будет лежать в пределах от 1 до 2 гурдов, плюс вероятность того,
что он лежит в пределах от 10 до 20 гурдов и т.д.
Таким образом, можно представить эту вероятность, как:
\[
\P(N=1) = \P(e^Y \in [1;2) ) + \P(e^Y \in [10; 20) ) + \ldots
\]

Это выражение можно преобразовать таким образом:
\[
\P(N=1) = \P(Y \in [\ln 1; \ln2) ) + \P(Y \in [\ln 10; \ln 20) ) + \ldots
\]

Так как Y — равномерно распределённая величина,
то $\P(Y \in [\ln 1; \ln2) ) = \frac{\ln 2 - \ln 1}{ra}$.
Для последующих слагаемых вероятность рассчитывается таким же образом.
Воспользовавшись свойством логарифма, можно заметить,
что $\frac{\ln 20 - \ln 10}{ra} = \frac{\ln 2}{ra}$.
Поэтому вероятность того, что на первом месте суммы вклада стоит единица,
равна $n\cdot \frac{\ln 2}{ra}$, где $n$ — количество слагаемых.
Путём аналогичных рассуждений получаем, что вероятность того,
что на первом месте стоит двойка, равна $n\cdot \frac{\ln 3- \ln 2}{ra}$.
Из-за того, что $a$ велико, можно считать, что число слагаемых одинаково.

На первом месте обязательно будет находиться какая-то цифра,
поэтому сумма вероятностей будет равна 1. Получаем:
\[
\dfrac{n}{ra}\left(\ln \frac{2}{1} + \ln \frac{3}{2} + \ldots + \ln \frac{10}{9}\right) = 1
\]

Таким образом $\frac{n}{ra} = \frac{1}{\ln 10}$.
Получается, что вероятность того, что на первом месте стоит единица, равна:
\[
\P (N=1) = \dfrac{\ln 2}{\ln 10}
\]

Закон распределения первой цифры выводится сложением соответствующих вероятностей.

\end{enumerate}
