\section{Контрольная работа 1}


% \subsection[что идет в оглавление]{\hyperref[на что ссылка]{текст ссылки}}
\subsection[2017-2018]{\hyperref[sec:sol_kr_01_2017_2018]{2017-2018}}
\label{sec:kr_01_2017_2018} % \label{ссылка сюда}

% * — не идёт в оглавление
\subsubsection*{Минимум}

\begin{enumerate}
\item Функция распределения случайной величины: определения и свойства.
\item Экспоненциальное распределение: определение, математическое ожидание и дисперсия.
\item В операционном отделе банка работает 80\% опытных сотрудников и 20\% неопытных. Вероятность совершения ошибки при очередной банковской операции опытным сотрудником равна $0.01$, а неопытным — $0.1$. Известно, что при очередной банковской операции была допущена ошибка. Найдите вероятность того, что ошибку допустил неопытный сотрудник.
\item При работе некоторого устройства время от времени возникают сбои. Количество сбоев за сутки имеет распределение Пуассона. Среднее количество сбоев за сутки равно 3. Найдите вероятность того, что за двое суток не произойдет ни одного сбоя.

\end{enumerate}

\subsubsection*{Задачи}

\begin{enumerate}

\item Правильный кубик подбрасывают один раз. Событие $A$ — выпало чётное число, событие $B$ — выпало число кратное трём, событие $C$ — выпало число, большее трёх.

\begin{enumerate}
\item Сформулируйте определение независимости двух событий;
\item Определите, какие из пар событий $A$, $B$ и $C$ будут независимыми.
\end{enumerate}


\item Теоретический минимум (ТМ) состоит из 10 вопросов, задачный (ЗМ) — из 24 задач.
Каждый вариант контрольной содержит два вопроса из ТМ и две задачи из ЗМ.
Чтобы получить за контрольную работу оценку 4 и выше, необходимо и достаточно правильно ответить на каждый вопрос ТМ и задачу ЗМ доставшегося варианта. Студент Вася принципиально выучил только $k$ вопросов ТМ и две трети ЗМ.
\begin{enumerate}
\item Сколько всего можно составить вариантов, отличающихся хотя бы одним заданием в ТМ или ЗМ части? Порядок заданий внутри варианта не важен.
\item Найдите вероятность того, что Вася правильно решит задачи ЗМ;
\item Дополнительно известно, что Васина вероятность правильно ответить на вопросы ТМ, составляет $1/15$. Сколько вопросов ТМ выучил Вася?
\end{enumerate}

\item Производитель молочных продуктов выпустил новый низкокалорийный йогурт Fit и утверждает, что он вкуснее его более калорийного аналога Fat.
Четырем независимым экспертам предлагают выбрать наиболее вкусный йогурт из трёх, предлагая им в одинаковых стаканчиках в случайном порядке два Fat и один Fit.
Предположим, что йогурты одинаково привлекательны.
Величина $\xi$ — число экспертов, отдавших предпочтение Fit.
\begin{enumerate}
\item Какова вероятность, что большинство экспертов выберут Fit?
\item Постройте функцию распределения величины $\xi$;
\item Каково наиболее вероятное число экспертов, отдавших предпочтение йогорту Fit?
\item Вычислите математическое ожидание и дисперсию $\xi$.
\end{enumerate}

\item Дядя Фёдор каждую субботу закупает в магазине продукты по списку, составленному котом Матроскином. Список не изменяется, и в него всегда входит 1 кг сметаны, цена которого является равномерно распределённой величиной $\alpha$, принимающей значения от 250 до 1000 рублей. Стоимость остальных продуктов из списка в тысячах рублей является случайной величиной $\xi$ с функцией распределения

\[
F(x)=\begin{cases}
1-\exp(-x^2 ), \text{ если } x \geq 0 \\
0, \text{ иначе.}\\
\end{cases}
\]

\begin{enumerate}
\item Какую сумму должен выделить кот Матроскин дяде Фёдору, чтобы её достоверно хватало на покупку сметаны?
\item Какую сумму должен выделить кот Матроскин дяде Фёдору, чтобы Дядя Фёдор с вероятностью 0.9 мог оплатить продукты без сметаны?
\item Найдите математическое ожидание стоимости продуктов без сметаны;
\item Найдите математическое ожидание стоимости всего списка.
\item Какова вероятность того, что общие расходы будут в точности равны их математическому ожиданию?
\end{enumerate}

Подсказка: $\int_0^{\infty} \exp(-x^2) \, dx = \sqrt{\pi} / 2$.

\item Эксперт с помощью детектора лжи пытается определить, говорит ли подозреваемый правду. Если подозреваемый говорит правду, то эксперт ошибочно выявляет ложь с вероятностью 0.1. Если подозреваемый обманывает, то эксперт выявляет ложь с вероятностью 0.95.

В деле об одиночном нападении подозревают десять человек, один из которых виновен и будет лгать, остальные невиновны и говорят правду.

\begin{enumerate}
\item Какова вероятность того, что детектор покажет, что конкретный подозреваемый лжёт?
\item Какова вероятность того, что подозреваемый невиновен, если детектор показал, что он лжёт?
\item Какова вероятность того, что эксперт точно выявит преступника?
\item Какова вероятность того, что эксперт ошибочно выявит  преступника, то есть покажет, что лжёт невиновный, а все остальные говорят правду?
\end{enumerate}
\end{enumerate}


\newpage
\subsection[2016-2017]{\hyperref[sec:sol_kr_01_2016_2017]{2016-2017}}
\label{sec:kr_01_2016_2017}

\begin{enumerate}
\item Из семей, имеющих двоих разновозрастных детей, случайным образом выбирается одна семья.
Известно, что в семье есть девочка (событие $A$).

\begin{enumerate}
\item	Какова вероятность того, что в семье есть мальчик (событие $B$)?
\item	Сформулируйте определение независимости событий и проверьте,
являются ли события $A$ и $B$ независимыми?
\end{enumerate}

\item Система состоит из $N$ независимых узлов.
При выходе из строя хотя бы одного узла, система дает сбой.
Вероятность выхода из строя любого из узлов равна $0.000001$.
Вычислите максимально возможное число узлов системы,
при котором вероятность её сбоя не превышает $0.01$.

\item Исследование состояния здоровья населения в шахтерском регионе
«Велико-кротовск» за пятилетний период показало,
что из всех людей с диагностированным заболеванием легких, 22\% работало на шахтах.
Из тех, у кого не было диагностировано заболевание легких, только 14\% работало на шахтах.
Заболевание легких было диагностировано у 4\% населения региона.

\begin{enumerate}
\item	Какой процент людей среди тех, кто работал в шахте,
составляют люди с диагностированным заболеванием легких?
\item	Какой процент людей среди тех, кто НЕ работал в шахте,
составляют люди с диагностированным заболеванием легких?
\end{enumerate}

\item  Студент Петя выполняет тест (множественного выбора) проставлением ответов наугад.
В тесте 17 вопросов, в каждом из которых пять вариантов ответов и только один из них правильный.
Оценка по десятибалльной шкале формируется следующим образом:
\[
    \text{Оценка} = \left\{
                      \begin{array}{ll}
                        \text{ЧПО} - 7, & \text{если $\text{ЧПО}\in [8;\,17]$,} \\
                        1,              & \text{если $\text{ЧПО}\in [0;\,7]$,}
                      \end{array}
                    \right.
\]
где ЧПО означает число правильных ответов.

\begin{enumerate}
\item	Найдите наиболее вероятное число правильных ответов.
\item	Найдите математическое ожидание и дисперсию числа правильных ответов.
\item	Найдите вероятность того, что Петя получит «отлично»
(по десятибалльной шкале получит 8, 9 или 10 баллов).

Студент Вася также выполняет тест проставлением ответов наугад.

\item	Найдите вероятность того, что все ответы Пети и Васи совпадут.
\end{enumerate}

\item  Продавец высокотехнологичного оборудования контактирует с одним или двумя
потенциальными покупателями в день с вероятностями $1/3$ и $2/3$ соответственно.
Каждый контакт заканчивается «ничем» с вероятностью $0.9$ и покупкой оборудования
на сумму в 50\,000 у.\,е. с вероятностью $0.1$.
Пусть $\xi$ — случайная величина, означающая объем дневных продаж в у.\,е.

\begin{enumerate}
\item	Вычислите  $\P(\xi = 0)$.
\item	Сформулируйте определение функции распределения и постройте функцию распределения
случайной величины $\xi$.
\item	Вычислите математическое ожидание и дисперсию случайной величины $\xi$.
\end{enumerate}

\item Интервал движения поездов метро фиксирован и равен $b$ минут,
т.е. каждый следующий поезд появляется после предыдущего ровно через $b$ минут.
Пассажир приходит на станцию в случайный момент времени.
Пусть случайная величина $\xi$, означающая время ожидания поезда,
имеет равномерное распределение на отрезке $[0; b]$.

\begin{enumerate}
\item Запишите плотность распределения случайной величины $\xi$.
\item	Найдите константу $b$, если известно, что в среднем пассажиру приходится
ждать поезда одну минуту, т.\,е. $\E(\xi) = 1$.
\item	Вычислите дисперсию случайной величины $\xi$.
\item	Найдите вероятность того, что пассажир будет ждать поезд менее одной минуты.
\item	Найдите квантиль порядка $0.25$ распределения случайной величины $\xi$.
\item	Найдите центральный момент порядка 2017 случайной величины $\xi$.
\item	Постройте функцию распределения случайной величины $\xi$.

Марья Ивановна из суеверия всегда пропускает два поезда и садится в третий.

\item	Найдите математическое ожидание и дисперсию времени,
затрачиваемого Марьей Ивановной на ожидание «своего» поезда.

Глафира Петровна не садится в поезд, если видит в нем подозрительного человека.
Подозрительные люди встречаются в каждом поезде с вероятностью $3/4$.

\item	Найдите вероятность того, что Глафире Петровне придется ждать не менее пяти минут,
чтобы уехать со станции.
\item	Найдите математическое ожидание времени ожидания «своего» поезда для Глафиры Петровны.
\end{enumerate}

\item (Бонусная задача)
На первом этаже десятиэтажного дома в лифт заходят 9 человек.
Найдите математическое ожидание числа остановок лифта, если люди выходят из лифта независимо друг от друга.
\end{enumerate}
