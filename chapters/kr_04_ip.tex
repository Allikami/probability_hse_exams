\clearpage
\thispagestyle{empty}
\section{Контрольная работа 4. ИП}



\subsection[2017-2018]{\hyperref[sec:sol_kr_04_ip_2017_2018]{2017-2018}}
\label{sec:kr_04_ip_2017_2018}


Напутствие в добрый путь:

\begin{enumerate}
\item Работа сдаётся только в виде запроса pull-request на гитхаб-репозиторий.

\item Имя файла должно быть вида \verb|ivanov_ivan_161_kr_4.Rmd|.

\item Также фамилию и имя нужно указать в шапке документа в поле \verb|author| :)

\item Если нужно, то установите пакеты \verb|tidyverse|, \verb|maxLik|, \verb|nycflights13|.
\end{enumerate}


\begin{enumerate}
\item Симулируем бурную деятельность!
В качестве параметра $k$ в задаче используй число букв в своей фамилии в именительном падеже :)

Каждый день Василий съедает случайное количество булочек, которое распределено по Пуассону с параметром $10$. Логарифм затрат в рублях на каждую булочку распределён нормально $N(2, 1)$.
Андрей каждый день съедает биномиальное количество булочек $Bin(2k, 0.5)$. Затраты Андрей на каждую булочку распределены равномерно на отрезке $[2;20]$.

\begin{enumerate}
\item Сколько в среднем тратит Василий на булочки за день?
\item Чему равна дисперсия дневных расходов Василия?
\item Какова вероятность того, что за один день Василий потратит больше денег,
чем Андрей?
\item Какова условная вероятность того,
что Василий за день съел больше булочек, чем Андрей,
если известно, что Василий потратил больше денег?
\end{enumerate}


\item Сражаемся с реальностью!
В пакете \verb|nycflights13| встроен набор данных \verb|weather| о погоде в разные дни в разных аэропортах.
\begin{enumerate}
	\item Постройте гистограмму переменной влажность, \verb|humid|.
	У графика подпишите оси!

\item Постройте диаграмму рассеяния переменных влажность и количество осадков,
precip. У графика подпишите оси!

Посчитайте выборочное среднее и выборочную дисперсию влажности и количества осадков.

\item С помощью максимального правдоподобия оцените параметр $\mu$,
предположив, что наблюдения за влажностью имеют нормальное $N\left(\mu, 370\right)$-распределение и независимы.
Постройте 95\%-ый доверительный интервал для $\mu$.

\item С помощью максимального правдоподобия оцените параметр $\sigma^2$,
предположив, что наблюдения за влажностью имеют нормальное $N\left(60, \sigma^2\right)$-распределение и независимы.
Постройте 95\%-ый доверительный интервал для $\sigma^2$.

Если при численной оптимизации параметр $\sigma^2$ становится отрицательным,
можно задать параметры по-другому, например, $\sigma^2 = \exp(\gamma)$.
\end{enumerate}
\end{enumerate}
