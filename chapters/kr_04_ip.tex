% !TEX root = ../probability_hse_exams.tex
\newpage
\thispagestyle{empty}
\section{Контрольная работа 4. ИП}


\subsection[2018-2019]{\hyperref[sec:sol_kr_04_ip_2018_2019]{2018-2019}}
\label{sec:kr_04_ip_2018_2019}

Ровно 189 лет назад, 1 июня 1830 британский учёный Джон Росс открыл северный магнитный полюс :)


\begin{enumerate}
\item Пусть $y$ — стандартный нормальный $n$-мерный вектор. 
Он случайный, просто Джону Россу лень писать заглавные буквы для векторов :) 
Вектор $a$ — неслучайный, но тоже гордый $n$-мерный.

Пусть $H$ — матрица, проецирующая любой вектор на $(n-1)$-мерное подпространство $a^{\perp}$, 
являющееся ортогональным дополнением к вектору $a$. 
То есть, для любого вектора $v$ вектор $Hv$ перпендикулярен вектору $a$.

\begin{enumerate}
    \item Найдите матрицу $H$, если $n=3$ и $a=(1,2,2)'$.
    \item Найдите матрицу $H$ для произвольного $n$ и $a$;
    \item Найдите $\E(y)$ и $\Var(y)$;
    \item Найдите $\E(Hy)$ и $\Var(Hy)$;
    \item Укажите закон распределения $y'Hy$, где $y'$ — это транспонированный вектор $y$.
\end{enumerate}

\item Рассмотрим формулу, здорово упрощающую подсчёт критерия Пирсона:
\[
 \sum_{j=1}^m \frac{(X_j - np_j)^2}{np_j} + n = \sum_{j=1}^m \frac{X_j^2}{np_j}
\]

\begin{enumerate}
    \item Докажите формулу.
    \item Нарисуйте картинку к этой формуле. На картинке подпишите прямой угол, катеты и гипотенузу. 
    Явно запишите каждый вектор. Объясните, почему треугольник, действительно, прямоугольный. 
\end{enumerate}


\item На Земле короля Уильяма Джон Росс нашёл странную монетку. 
Он подбрасывает её $n$ раз и обнаруживает, что она выпадает на орла, решку и ребро. 
Джон Росс проверяет гипотезу $H_0$ о том, что все три вероятности равны.

Пусть $y = (y_1, y_2, y_3)'$ — количество выпадений орла, решки и ребра. Рассмотрим так же вектор
$z = (z_1, z_2, z_3)'$, такой, что $z_i = (y_i - \E(y_i)) / \sqrt{\E(y_i)}$. 
Джон Росс сознательно перепутал ожидание и дисперсию в классической формуле!

Предположим, что гипотеза $H_0$ верна.
\begin{enumerate}
    \item Укажите закон распределения каждой величины $y_i$;
    \item Найдите вектор $\E(y)$ и матрицу $\Var(y)$;
    \item Найдите вектор $\E(z)$ и матрицу $\Var(z)$;
    \item Докажите, что матрица $H=\Var(z)$ является проектором на ортогональное дополнение к некоторому вектору $a$. 
  Явно выпишите вектор $a$.
  \item Объясните, почему критерий Пирсона имеет хи-квадрат распределение с нужным числом степеней свободы.
  % \item Обобщите решение на случай прозвольных вероятностей или произвольного количества граней у монетки.
\end{enumerate}

\newpage

\item На Земле короля Уильяма Джон Росс нашёл странную монетку. 
Он подбрасывает её $n$ раз и обнаруживает, что она выпадает на орла, решку и ребро. 
Джон Росс проверяет гипотезу о том, что все три вероятности равны с помощью двух статистики: 
$LR$, отношения правдоподобия, и $CP$, критерия Пирсона. 

\begin{enumerate}
\item Найдите $\plim_{n\to\infty} \frac{LR}{CP}$;
\item Обобщите решение на случай произвольного количества равновероятных граней у монетки.    
% \item Обобщите решение на случай произвольного количества граней и произвольных вероятностей в гипотезе $H_0$.
\end{enumerate}

\item Идея доказательства состоятельности ML оценки :)

Пусть наблюдения $y_1$, \ldots, $y_n$ независимы и одинаково распределены с функцией плотности, зависящей от параметра $a$.
Истинное значение параметра обозначим буквой $a_0$. Оценку максимального правдоподобия обозначим $\hat a$.

Рассмотрим отмасштабированную логарифмическую функцию правдоподобия $\ell_n(a)=\ell(a) / n$, и
ожидаемую логарифмическую функцию правдоподобия\footnote{Внимание:
ожидание считается с помощью истинного $a_0$ от функции, в которую входит константа $a$.},
$\tilde \ell(a)=\E(\ell(a))$.
\begin{enumerate}
\item Что больше, $\ln x$ или $x-1$? Докажите!
\item В какой точке находится максимум функции $\ell_n(a)$?
\item В какой точке находится максимум функции $\tilde \ell(a)$?

Подсказка: рассмотрите выражение $\tilde \ell(a) - \tilde \ell(a_0)$ и примените доказанное неравество :)
\item К чему сходится $\ell_n(a)$ по вероятности?
%\item К чему сходится $\ell^{\prime\prime}_n(a)$?

\end{enumerate}


\end{enumerate}



\subsection[2017-2018]{\hyperref[sec:sol_kr_04_ip_2017_2018]{2017-2018}}
\label{sec:kr_04_ip_2017_2018}


Напутствие в добрый путь:

\begin{enumerate}
\item Работа сдаётся только в виде запроса pull-request на гитхаб-репозиторий.

\item Имя файла должно быть вида \verb|ivanov_ivan_161_kr_4.Rmd|.

\item Также фамилию и имя нужно указать в шапке документа в поле \verb|author| :)

\item Если нужно, то установите пакеты \verb|tidyverse|, \verb|maxLik|, \verb|nycflights13|.
\end{enumerate}


\begin{enumerate}
\item Симулируем бурную деятельность!
В качестве параметра $k$ в задаче используй число букв в своей фамилии в именительном падеже :)

Каждый день Василий съедает случайное количество булочек, которое распределено по Пуассону с параметром $10$. Логарифм затрат в рублях на каждую булочку распределён нормально $N(2, 1)$.
Андрей каждый день съедает биномиальное количество булочек $Bin(2k, 0.5)$. Затраты Андрей на каждую булочку распределены равномерно на отрезке $[2;20]$.

\begin{enumerate}
\item Сколько в среднем тратит Василий на булочки за день?
\item Чему равна дисперсия дневных расходов Василия?
\item Какова вероятность того, что за один день Василий потратит больше денег,
чем Андрей?
\item Какова условная вероятность того,
что Василий за день съел больше булочек, чем Андрей,
если известно, что Василий потратил больше денег?
\end{enumerate}


\item Сражаемся с реальностью!
В пакете \verb|nycflights13| встроен набор данных \verb|weather| о погоде в разные дни в разных аэропортах.
\begin{enumerate}
	\item Постройте гистограмму переменной влажность, \verb|humid|.
	У графика подпишите оси!

\item Постройте диаграмму рассеяния переменных влажность и количество осадков,
precip. У графика подпишите оси!

Посчитайте выборочное среднее и выборочную дисперсию влажности и количества осадков.

\item С помощью максимального правдоподобия оцените параметр $\mu$,
предположив, что наблюдения за влажностью имеют нормальное $N\left(\mu, 370\right)$-распределение и независимы.
Постройте 95\%-ый доверительный интервал для $\mu$.

\item С помощью максимального правдоподобия оцените параметр $\sigma^2$,
предположив, что наблюдения за влажностью имеют нормальное $N\left(60, \sigma^2\right)$-распределение и независимы.
Постройте 95\%-ый доверительный интервал для $\sigma^2$.

Если при численной оптимизации параметр $\sigma^2$ становится отрицательным,
можно задать параметры по-другому, например, $\sigma^2 = \exp(\gamma)$.
\end{enumerate}
\end{enumerate}
