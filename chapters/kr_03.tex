\section{Контрольная работа 3}



\subsection[2017-2018]{\hyperref[sec:sol_kr_03_2017_2018]{2017-2018}}
\label{sec:kr_03_2017_2018}



	\subsubsection{Минимум}

	\begin{enumerate}
	\item Дайте определение выборочной функции распределения.
	\item Предположим, что величины $X_1$, $X_2$, \ldots, $X_n$ независимы и нормальны $\cN(\mu;\sigma^2)$. Укажите закон распределения выборочного среднего, величины $\frac{\bar X - \mu}{\sigma/\sqrt{n}}$, величины $\frac{\bar X - \mu}{\hat\sigma/\sqrt{n}}$, величины $\frac{\hat\sigma^2(n-1)}{\sigma^2}$.
	\item Рост в сантиметрах, случайная величина $X$, и вес в килограммах, случайная величина $Y$, взрослого мужчины является нормальным случайным вектором $Z = (X, Y)$ с математическим ожиданием $\E(Z) = (175, 75)$ и ковариационной матрицей

	\[
	\Var(Z) =
	\begin{pmatrix}
	49 & 28 \\
	28 & 36
	\end{pmatrix}
	\]

	\begin{enumerate}
		\item Найдите средний вес мужчины при условии, что его рост составляет $172$ см.
		\item Выпишите условную плотность распределения веса мужчины при условии, что его рост составляет $172$ см.
		\item Найдите условную вероятность того, что человек будет иметь вес, больший $92$ кг, при условии, что его рост составляет $172$ см.
	\end{enumerate}

	\item Стоимость выборочного исследования генеральной совокупности, состоящей из трех страт, определяется по формуле $TC = c_1n_1 + c_2n_2 + c_3n_3$, где $c_i$ — цена одного наблюдения в $i$-ой страте, a $n_i$ — число наблюдений, которые приходятся на $i$-ую страту. Найдите $n_1$, $n_2$ и $n_3$, при которых дисперсия стратифицированного среднего достигает наименьшего значения, если бюджет исследования 8000 и имеется следующая информация:

	\begin{center}
		\begin{tabular}{cccc}
			\toprule
			Страта & $1$ & $2$ & $3$  \\
			\midrule
			Среднее значение & $30$ & $40$ & $50$ \\
			Стандартная ошибка  & $5$ & $10$ & $20$ \\
			Вес & $25\%$ & $25\%$ & $50\%$ \\
			Цена наблюдения & $2$ & $5$ & $8$ \\
			\bottomrule
		\end{tabular}
	\end{center}
	\end{enumerate}



\subsubsection{Задачи}

\begin{enumerate}[resume]

	%Задача 1
	\item Пусть $X_{1}, \ldots, X_{n}$ выборка из нормального распределения $N(\mu,1)$.
	\begin{enumerate}
		\item Выпишите функцию правдоподобия;
		\item Методом максимального правдоподобия найдите оценку $\hat{\mu}$ математического ожидания $\mu$;

		\item Проверьте состоятельность и несмещённость оценки $\hat{\mu}$;
		\item Вычислите информацию Фишера о параметре $\mu$, содержащуюся во всей выборке;
		\item Для произвольной несмещённой оценки $\mu$ выпишите неравенство Рао-Крамера-Фреше;
		\item Проверьте свойство эффективности оценки $\hat{\mu}$;
		\item Найдите оценку максимального правдоподобия $\hat{\theta}$ для второго начального момента;
	\item Проверьте свойства несмещенности и асимптотической несмещенности оценки $\hat{\theta}$;
	\item С помощью дельта-метода вычислите, примерно, дисперсию оценки $\hat{\theta}$;
	\item Проверьте состоятельность оценки $\hat{\theta}$.
\end{enumerate}

 	%Задача 2
 	\item Пусть $X_{1}$, \ldots, $X_{n}$ выборка из распределения с функцией плотности:

\[
f(x)=\begin{cases}
 		\frac{2}{\theta^2}(\theta-x),&\text{при }x\in[0,\theta]\\
 		0,&\text{при }x\notin[0,\theta]
 		\end{cases}
\]


 \begin{enumerate}
 	\item Методом моментов найдите оценку параметра $\theta$;
 	\item Приведите определение состоятельности оценки и проверьте, будет ли найденная оценка состоятельной.
 \end{enumerate}

	%Задача 3
	\item В прихожей лежат четыре карты «тройка». На двух из них нет денег, на двух других 30 и 500 рублей. Вовочка не помнит, на какой из карт есть деньги, поэтому берет три карточки.

	 \begin{enumerate}
		\item Найдите математическое ожидание и дисперсию средней по выбранным карточкам суммы денег;
		\item Определите, какова вероятность того, что Вовочке удастся войти в метро, если стоимость проезда по тройке составляет 35 рублей.
	\end{enumerate}

	%Задача 4
	\item По выборочному опросу студенческих семейных пар о расходах на ланч были получены следующие результаты:

	\begin{center}
		\begin{tabular}{ccccc}
			\toprule
			Номер семьи & 1 & 2 & 3 & 4\\
			Расходы мужа & 450 & 370 & 170 & 200\\
			Расходы жены & 210 & 350 & 250 & 180\\
			\bottomrule
		\end{tabular}
	\end{center}

	Считая, что разница в расходах мужа и жены хорошо описываются нормальным распределением, постройте 95\%-ый доверительный интервал для разницы математических ожиданий расходов супругов. Есть ли основания утверждать, что расходы одинаковы?

	%Задание №5
	\item Наблюдатель Алексей Недопускальный решил проверить честность выборов. Ему удалось подглядеть, как проголосовали 60 избирателей. Из них 42 выбрали действующего президента.

	\begin{enumerate}
		\item Постройте 95\%-ый доверительный интервал для истинной доли избирателей, проголосовавших «за» действующего президента.
		\item По результатам ЦентрИзберКома «за» действующего президента проголосовало 76.67\% населения. Согласуются ли эти данные с данными Алексея?
		\item Сколько бюллетеней нужно подглядеть Алексею, чтобы с вероятностью 0.95 отклонение от выборочной доли проголосовавших «за» действующего президента от истинной не превышало 0.01?
	\end{enumerate}

\end{enumerate}
