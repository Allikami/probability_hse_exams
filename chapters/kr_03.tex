\newpage
\thispagestyle{empty}
\section{Контрольная работа 3}



\subsection[2018-2019]{\hyperref[sec:sol_kr_03_2018_2019]{2018-2019}}
\label{sec:kr_03_2018_2019}

Примечание: минимум писали 30 минут без чит-листа, потом перерыв 10 минут, 
потом задачи писали час сорок с чит-листом. 

\subsubsection*{Минимум}

\begin{enumerate}
\item У случайно выбираемого взрослого мужчины рост в сантиметрах, $X$, и вес
в килограммах, $Y$,
являются нормальным случайным вектором $Z=(X,Y)$ с математическим ожиданием $E(Z)=(180,90)$ и ковариационной матрицей:

\[
\Var(Z)=\begin{bmatrix}100 & 35\\35 & 25\end{bmatrix}
\]		

Рассмотрим величину $U=X-Y$. Считается, что человек страдает избыточным весом, если $U<80$.
\begin{enumerate}
\item Укажите распределение случайной величины $U$. Выпишите её плотность распределения.
\item Найдите вероятность того, что случайно выбранный мужчина страдает избыточным весом.
\item Найдите условную вероятность того, что случайно выбранный мужчина страдает избыточным весом, при условии, что его рост равен 185 см.
\end{enumerate}
\item Стоимость выборочного исследования генеральной совокупности, состоящей из двух страт, определяется по формуле $TC=c_{1}n_{1}+c_{2}n_{2}$, где $c_{i}$ — цена одного наблюдения в $i$-ой страте, а $n_{i}$ — число наблюдений, которое приходится на $i$-ую страту. Найдите $n_{1}$ и $n_{2}$, при которых дисперсия стратифицированного среднего достигает наименьшего значения, если бюджет исследования 10000 и имеется следующая информация:
		
			\begin{tabular}{c|c|c|c}
			Страта & 1 & 2 \\
			\hline
			Среднее значение & 10 & 20  \\ \hline
			Стандартная ошибка & 20 & 10  \\ \hline
			Вес & 10\% & 90\% \\
			\hline
			Цена наблюдения & 1 & 4 \\
			\end{tabular}
		
\item Для независимых нормальных $\cN(\mu,\sigma^2)$ случайных величин $(X_{1}, \ldots, X_{n})$ укажите формулу доверительного интервала с уровнем доверия $(1-\alpha)$ для неизвестного математического ожидания $\mu$ при \textbf{известной} дисперсии $\sigma^2$.
		
\item Дайте определение распределения Стьюдента с помощью нормальных
случайных величин. Укажите диапазон возможных значений. Нарисуйте функцию плотности распределения Стьюдента при разных степенях свободы на фоне нормальной стандартной функции плотности.
		
Дополнительная задача для пропустивших мини-контрольную на лекции по уважительной причине
		
\item Для реализации случайной выборки $x=(2,2,-1,2,1)$:
\begin{enumerate}
\item Найдите вариационный ряд;
\item Найдите выборочный второй начальный момент;
\item Постройте график выборочной функции распределения.
\end{enumerate}
		
\end{enumerate}


\subsubsection*{Задачи}



\begin{enumerate}[resume]
		
%Задача №1
\item Василиса Премудрая в рамках борьбы с гендерным дресс-кодом в Тридевятом Царстве устроила распродажу всех своих пяти кокошников. Ожидаемая цена случайно выбираемого кокошника составляет $3500$ у.е., а стандартное отклонение – $500$ у.е.. За неделю было продано три кокошника. 
		
Найдите математическое ожидание и дисперсию вырученных Василисой денег, если она продаёт кокошники по себестоимости и вероятность покупки любого из них одна и та же. \textbf{(5 баллов)}
		
%Задача №2
\item К Весеннему слёту по обмену премудростями
независимо готовятся $n$ Василис Премудрых.
Время подготовки каждой Василисы в часах, $X_{i}$,
имеет функцию плотности:
		

\[
f(x;\theta)=\begin{cases}\frac{2x}{\theta^2}\text{, при }x\in[0;\theta]\\0\text{, при }x\notin[0;\theta]\end{cases},
\]
		
где $\theta>0$ — неизвестный параметр. 
Найдите оценку $\theta$ методом моментов. \textbf{(5 баллов)}
		
%Задача №3
\item Пусть $X_{1}, \ldots, X_{n}$ — случайная выборка из распределения с плотностью:
		
\[
f(x;\theta)=\begin{cases}\frac{2x}{\theta^2}\text{, при }x\in[0;\theta]\\0\text{, при }x\notin[0;\theta]\end{cases},
\]

		
где $\theta>0$ — неизвестный параметр. Для параметра $\theta$ предлагаются две оценки: $\hat{\theta}_{n}=\frac{3}{2}\overline{X}_{n}$ и $T_{n}=\max(X_{1}, \ldots, X_{n})$:
		
\begin{enumerate}
\item Является ли оценка $\hat{\theta}_{n}$ несмещенной оценкой неизвестного параметра $\theta$? \textbf{(3 балла)}
\item Найдите $D(\hat{\theta}_{n})$ \textbf{(3 балла)}
\item Проверьте состоятельность оценки $\hat{\theta}_{n}$ 
\textbf{(4 балла)}
\item Проверьте несмещенность оценки $T_{n}$ и вычислите величину смещения \textbf{(7 баллов)}
\item Какая из оценок $\hat{\theta}_{n}$ или $T_{n}$ является более эффективной согласно критерию MSE? \textbf{(8 баллов)}
\end{enumerate}
		
%Задача №4
\item Пусть $X_{1}, \ldots, X_{n}$ — случайная выборка из нормального распределения с нулевым математическим ожиданием и дисперсией $\theta$.
		
\begin{enumerate}
\item С помощью метода максимального правдоподобия найдите оценку $\hat{\theta}_{n}$ параметра $\theta$ \textbf{(6 баллов)}
\item Проверьте несмещенность найденной оценки \textbf{(3 балла)}
\item Вычислите информацию Фишера о параметре $\theta$, заключенную в выборке \textbf{(2 балла)}
\item Проверьте, является ли найденная оценка эффективной \textbf{(4 балла)}
			
\textbf{Подсказка}: четвёртый момент стандартной нормальной случайной величины равен 3.
\end{enumerate}
		
\item Пусть $X_{1}, \ldots, X_{n}$ — случайная выборка из равномерного распределения с плотностью


\[
f(x;\theta)=\begin{cases}\frac{1}{\theta}\text{, если }x\in[0,\theta]\\ 0\text{, если }x\notin[0,\theta]\end{cases},
\]
		
где $\theta>0$. Постройте оценку параметра $\theta$ методом максимального правдоподобия. \textbf{(10 баллов)}
		
\end{enumerate}



\subsection[2017-2018]{\hyperref[sec:sol_kr_03_2017_2018]{2017-2018}}
\label{sec:kr_03_2017_2018}

\subsubsection*{Минимум}

\begin{enumerate}
\item Дайте определение выборочной функции распределения.
\item Предположим, что величины $X_1$, $X_2$, \ldots, $X_n$ независимы и нормальны
$\cN(\mu;\sigma^2)$. Укажите закон распределения выборочного среднего, величины
$\frac{\bar X - \mu}{\sigma/\sqrt{n}}$, величины $\frac{\bar X - \mu}{\hat\sigma/\sqrt{n}}$,
величины $\frac{\hat\sigma^2(n-1)}{\sigma^2}$.
\item Рост в сантиметрах, случайная величина $X$, и вес в килограммах, случайная
величина $Y$, взрослого мужчины является нормальным случайным вектором $Z = (X, Y)$
с математическим ожиданием $\E(Z) = (175, 75)$ и ковариационной матрицей

\[
\Var(Z) =
\begin{pmatrix}
49 & 28 \\
28 & 36
\end{pmatrix}
\]

\begin{enumerate}
\item Найдите средний вес мужчины при условии, что его рост составляет $172$ см.
\item Выпишите условную плотность распределения веса мужчины при условии, что его
рост составляет $172$ см.
\item Найдите условную вероятность того, что человек будет иметь вес, больший $92$ кг,
при условии, что его рост составляет $172$ см.
\end{enumerate}

\item Стоимость выборочного исследования генеральной совокупности, состоящей из трёх
страт, определяется по формуле $TC = c_1n_1 + c_2n_2 + c_3n_3$, где $c_i$ — цена
одного наблюдения в $i$-ой страте, a $n_i$ — число наблюдений, которые приходятся
на $i$-ую страту. Найдите $n_1$, $n_2$ и $n_3$, при которых дисперсия стратифицированного
среднего достигает наименьшего значения, если бюджет исследования 8000 и имеется
следующая информация:

\begin{center}
\begin{tabular}{cccc}
\toprule
Страта & $1$ & $2$ & $3$  \\
\midrule
Среднее значение & $30$ & $40$ & $50$ \\
Стандартная ошибка  & $5$ & $10$ & $20$ \\
Вес & $25\%$ & $25\%$ & $50\%$ \\
Цена наблюдения & $2$ & $5$ & $8$ \\
\bottomrule
\end{tabular}
\end{center}
\end{enumerate}


\subsubsection*{Задачи}

\begin{enumerate}[resume]
	%Задача 1
\item Пусть $X_{1}, \ldots, X_{n}$ выборка из нормального распределения $N(\mu,1)$.
\begin{enumerate}
\item Выпишите функцию правдоподобия;
\item Методом максимального правдоподобия найдите оценку $\hat{\mu}$ математического
ожидания $\mu$;
\item Проверьте состоятельность и несмещённость оценки $\hat{\mu}$;
\item Вычислите информацию Фишера о параметре $\mu$, содержащуюся во всей выборке;
\item Для произвольной несмещённой оценки $\mu$ выпишите неравенство Рао-Крамера-Фреше;
\item Проверьте свойство эффективности оценки $\hat{\mu}$;
\item Найдите оценку максимального правдоподобия $\hat{\theta}$ для второго начального
момента;
\item Проверьте свойства несмещенности и асимптотической несмещенности оценки $\hat{\theta}$;
\item С помощью дельта-метода вычислите, примерно, дисперсию оценки $\hat{\theta}$;
\item Проверьте состоятельность оценки $\hat{\theta}$.
\end{enumerate}

 %Задача 2
 \item Пусть $X_{1}$, \ldots, $X_{n}$ выборка из распределения с функцией плотности:

\[
f(x)=\begin{cases}
\frac{2}{\theta^2}(\theta-x),&\text{при }x\in[0,\theta]\\
0,&\text{при }x\notin[0,\theta]
\end{cases}
\]

\begin{enumerate}
\item Методом моментов найдите оценку параметра $\theta$;
\item Приведите определение состоятельности оценки и проверьте, будет ли найденная
оценка состоятельной.
\end{enumerate}

%Задача 3
\item В прихожей лежат четыре карты «тройка». На двух из них нет денег, на двух
других 30 и 500 рублей. Вовочка не помнит, на какой из карт есть деньги, поэтому
берёт три карточки.

\begin{enumerate}
\item Найдите математическое ожидание и дисперсию средней по выбранным карточкам
суммы денег;
\item Определите, какова вероятность того, что Вовочке удастся войти в метро,
если стоимость проезда по тройке составляет 35 рублей.
\end{enumerate}

%Задача 4
\item По выборочному опросу студенческих семейных пар о расходах на ланч были
получены следующие результаты:

\begin{center}
\begin{tabular}{ccccc}
\toprule
Номер семьи & 1 & 2 & 3 & 4\\
Расходы мужа & 450 & 370 & 170 & 200\\
Расходы жены & 210 & 350 & 250 & 180\\
\bottomrule
\end{tabular}
\end{center}

Считая, что разница в расходах мужа и жены хорошо описываются нормальным распределением,
постройте 95\%-ый доверительный интервал для разницы математических ожиданий расходов
супругов. Есть ли основания утверждать, что расходы одинаковы?

	%Задание №5
\item Наблюдатель Алексей Недопускальный решил проверить честность выборов.
Ему удалось подглядеть, как проголосовали 60 избирателей. Из них 42 выбрали
действующего президента.

\begin{enumerate}
\item Постройте 95\%-ый доверительный интервал для истинной доли избирателей,
проголосовавших «за» действующего президента.
\item По результатам ЦентрИзберКома «за» действующего президента проголосовало
76.67\% населения. Согласуются ли эти данные с данными Алексея?
\item Сколько бюллетеней нужно подглядеть Алексею, чтобы с вероятностью
$0.95$ отклонение от выборочной доли проголосовавших «за» действующего
президента от истинной не превышало $0.01$?
\end{enumerate}
\end{enumerate}



\newpage
\subsection[2016-2017]{\hyperref[sec:sol_kr_03_2016_2017]{2016-2017}}
\label{sec:kr_03_2016_2017}

\begin{enumerate}

\item Дана реализация случайной выборки: $1$, $10$, $7$, $4$, $-2$. Выпишите
определения и найдите значения следующих характеристик:
\begin{enumerate}
  \item вариационного ряда,
  \item выборочного среднего,
  \item выборочной дисперсии,
  \item несмещенной оценки дисперсии,
  \item выборочного второго начального момента.
  \item Постройте выборочную функцию распределения.
\end{enumerate}


\item
Мама дяди Фёдора каждое лето приезжает в Простоквашино с тремя вечерними платьями.
Если выбирать одно платье из трёх случайно, то ожидание стоимости равно 11 тысяч рублей,
а дисперсия стоимости равна 3 тысячи квадратных рублей.
Рачительный кот Матроскин случайным образом выбирает одно из платьев и продаёт его как ненужное. Вычислите математическое ожидание и дисперсию стоимости двух оставшихся платьев.

\item
Ресторанный критик ходит по трём типам ресторанов (дешевых, бюджетных и дорогих)
города N для того, чтобы оценить среднюю стоимость бизнес-ланча. В городе 40\%
дешевых ресторанов, 50\% — бюджетных и 10\% — дорогих. Стандартное отклонение
цены бизнес-ланча составляет 10, 30 и 60 рублей соответственно. В ресторане
критик заказывает только кофе. Стоимость кофе в дешевых/бюджетных/дорогих ресторанах
составляет 150, 300 и 600 рублей соответственно, а бюджет исследования — 30\,000 рублей.
\begin{enumerate}
  \item Какое количество ресторанов каждого типа нужно посетить критику, чтобы
	как можно точнее оценить среднюю стоимость бизнес-ланча при заданном бюджетном
	ограничении (округлите полученные значения до ближайших целых)?
  \item Вычислите дисперсию соответствующего стратифицированного среднего.
\end{enumerate}

\item
В «акции протеста против коррупции» в Москве 26.03.2017 по данным МВД приняло
участие 8\,000 человек. Считая, что население Москвы составляет 12\,300\,000 человек,
постройте 95\% доверительный интервал для истинной доли желающих участвовать в
подобных акциях жителей России. Можно ли утверждать, что эта доля статистически
не отличается от нуля?

\item
Для некоторой отрасли проведено исследование об оплате труда мужчин и женщин. Их зарплаты (тыс. руб. в месяц) приведены ниже:
\begin{center}
\begin{tabular}{cccccc}
  \toprule
  \text{мужчины}         &$50$    &$40$    &$45$   &$45$   &$35$   \\
  \text{женщины}         &$60$    &$30$    &$30$   &$35$   &$30$   \\ \bottomrule
\end{tabular}
\end{center}

\begin{enumerate}
  \item Считая, что распределение заработных плат мужчин хорошо описывается
	нормальным распределением, постройте
  \begin{enumerate}
    \item 99\%-ый доверительный интервал для математического ожидания заработной
		платы мужчин,
    \item 90\%-ый доверительный интервал для стандартного отклонения заработной
		платы мужчин.
  \end{enumerate}
  \item
  \begin{enumerate}
    \item Сформулируйте предпосылки, необходимые для построения доверительно интервала
		для разности математических ожиданий заработных плат мужчин и женщин.
    \item Считая предпосылки выполненными, постройте 90\%-ый доверительный интервал
		для разности математических ожиданий заработных плат мужчин и женщин.
    \item Можно ли считать зарплаты мужчин и женщин одинаковыми?
  \end{enumerate}
\end{enumerate}

\item
Пусть $X = (X_1, \, \ldots, \, X_n)$ — случайная выборка из нормального распределения
с нулевым математическим ожиданием и дисперсией $\theta$.
\begin{enumerate}
  \item Используя второй начальный момент, найдите оценку параметра $\theta$
	методом моментов.
  \item Сформулируйте определение несмещённости оценки и проверьте выполнение
	данного свойства для оценки, найденной в пункте а).
  \item Сформулируйте определение состоятельности оценки и проверьте выполнение
	данного свойства для оценки, найденной в пункте а).
  \item Найдите оценку параметра $\theta$ методом максимального правдоподобия.
  \item Вычислите информацию Фишера о параметре $\theta$, заключенную в $n$
	наблюдениях случайной выборки.
  \item Сформулируйте неравенство Рао-Крамера-Фреше.
  \item Сформулируйте определение эффективности оценки и проверьте выполнение
	данного свойства для оценки, найденной в пункте г).
\end{enumerate}

\item
Аэрофлот утверждает, что 10\% пассажиров, купивших билет, не являются на рейс.
В случайной выборке из шести рейсов аэробуса А320, имеющего 180 посадочных мест,
число не явившихся оказалось: $5$, $10$, $25$, $0$, $17$, $30$. Пусть число пассажиров
$X$, не явившихся на рейс, хорошо описывается распределением Пуассона $\P(X = k)
= \tfrac{\lambda^{k}}{k!}e^{-\lambda}$, $k \in \{0,\, 1,\, 2,\, \ldots\}$. При
помощи метода максимального правдоподобия найдите:
\begin{enumerate}
  \item оценку $\E(X)$ и её числовое значение по выборке,
  \item оценку дисперсии $X$ и её числовое значение по выборке,
  \item оценку стандартного отклонения $X$ и её числовое значение по выборке,
  \item оценку вероятности того, что на рейс явятся все пассажиры, а также найдите
	её числовое значение по выборке.
  \item Используя асимптотические свойства оценок максимального правдоподобия,
	постройте 95\% доверительный интервал для $\E(X)$.
  \item С помощью дельта-метода найдите 95\% доверительный интервал для вероятности
	полной загруженности самолёта.
\end{enumerate}
\end{enumerate}


\newpage
\subsection[2015-2016]{\hyperref[sec:sol_kr_03_2015_2016]{2015-2016}}
\label{sec:kr_03_2015_2016}

\epigraph{Ищите и обрящете, толцыте и отверзется вам}{Лука 11:9}

\begin{enumerate}
\item В студенческом буфете осталось только три булочки одинаковой привлекательности
и цены, но разной калорийности: 250, 400 и 550 ккал. Голодные Маша и Саша, не глядя
на калорийность, покупают по булочке. Найдите математическое ожидание и дисперсию
суммы поглощенных студентами калорий.
\item Дана реализация случайной выборки  независимых одинаково распределенных
случайных величин: 11, 4, 6.
\begin{enumerate}
  \item Выпишите вариационный ряд;
  \item Постройте выборочную функцию распределения;
  \item Найдите выборочную медиану распределения;
  \item Вычислите выборочное среднее и несмещенную оценку дисперсии.
\end{enumerate}

\item Найдите математическое ожидание, дисперсию и коэффициент корреляции случайных
величин $X$ и $Y$, совместное распределение которых имеет функцию плотности
\[
f(x, y) = \frac{5}{4\pi \sqrt{6}} \exp\left(
-\frac{25}{48}\left( (x-1)^2 -0.4(x-1)y + y^2 \right)
\right)
\]

\item Рост и размер обуви $(X, Y)$ взрослого мужчины хорошо описывается двумерным
нормальным распределением с математическим ожиданием $(178, 42)$ и ковариационной
матрицей
\[
C = \begin{pmatrix}
49 & 5.6 \\
5.6 & 1 \\
\end{pmatrix}
\]
\begin{enumerate}
  \item Какой процент мужчин обладает ростом выше 185 см?
  \item Являются ли рост и размер обуви случайно выбранного мужчины независимыми?
	Обоснуйте ответ.
  \item Среди мужчин с ростом 185 см, каков процент тех, кто имеет размер обуви,
	меньший сорок второго  $\P(Y < 42 \mid X=185)$?
\end{enumerate}


\item Дана случайная выборка $X_1$, \ldots, $X_n$ из равномерного распределения
$U[0; 2\theta]$.
\begin{enumerate}
  \item С помощью первого момента найдите оценку параметра  $\theta$ методом моментов;
  \item Сформулируйте определения несмещенности, состоятельности и эффективности оценок;
  \item Проверьте, будет ли найденная в пункте (а) оценка несмещенной и состоятельной.
  \item С помощью статистики $X_{(n)}= \max\{ X_1,\ldots, X_n \}$ постройте несмещенную
	оценку параметра $\theta$  вида $cX_{(n)}$. Укажите значение $c$.
  \item Проверьте, будет ли данная оценка состоятельной;
  \item Какая из двух оценок является более эффективной? Обоснуйте ответ.
\end{enumerate}

\item Вовочка хочет проверить утверждение организаторов юбилейной лотереи
«Метро-80 лет в ритме столицы», что почти треть всех билетов выигрышные.
Для этого он попросил $n$ своих друзей купить по 10 лотерейных билетов.
Пусть  $X_i$ — число выигрышных билетов друга $i$ и $p$ — вероятность выигрыша
одного билета.
\begin{enumerate}
  \item  Какое распределение имеет величина $X_i$?
  \item Запишите функцию правдоподобия $L(p)$  для выборки $X_1$, \ldots, $X_n$;
  \item Методом максимального правдоподобия найдите оценку $p$;
  \item Найдите информацию Фишера для одного наблюдения $i(p)$;
  \item Для произвольной несмещенной оценки $T(X_1, \ldots, X_n)$ запишите
	неравенство Рао-Крамера-Фреше;
  \item Будет оценка $\hat p_{ML}$ эффективной?
  \item Найдите оценку максимального правдоподобия математического ожидания и
	дисперсии выигранных произвольным другом билетов;
  \item Дана реализация случайной выборки 5 Вовочкиных друзей. Число выигрышных
	билетов  оказалось равно (3, 4, 0, 2, 6). Найдите значение точечной оценки
	вероятности выигрыша $p$. Как Вы думаете, похоже ли утверждение организаторов на правду?
\end{enumerate}

\item  Дана выборка $X_1$, $X_2$, \ldots, $X_n$ независимых одинаково распределенных
величин из распределения с функцией плотности
\[
f(x)=\begin{cases}
(1+\theta)x^\theta, \text{ если } 0<y<1, \theta+1>0 \\
0, \text{ иначе}
\end{cases}.
\]

Методом максимального правдоподобия найдите оценку параметра $\theta$.

\item Пробег (в 1000 км) автомобиля «Лада Калина» до капитального ремонта двигателя
является нормальной случайной величиной с неизвестным математическим ожиданием $\mu$
и известной дисперсией 49. По выборке из 20 автомобилей найдите значение доверительного
интервала для математического ожидания пробега с уровнем доверия $0.95$.
\end{enumerate}


\newpage
\subsection[2014-2015]{\hyperref[sec:sol_kr_03_2014_2015]{2014-2015}}
\label{sec:kr_03_2014_2015}

\begin{enumerate}

\item В студенческом буфете осталось только три булочки одинаковой привлекательности
и цены, но разной калорийности: 250, 400 и 550 ккал. Голодные Маша и Саша, не глядя
на калорийность, покупают по булочке. Найдите математическое ожидание и дисперсию
суммы поглощённых студентами калорий.

\item Ресторанный критик ходит по трём типам ресторанов (дешёвых, бюджетных и дорогих)
города N для того, чтобы оценить среднюю стоимость бизнес-ланча. В городе N 30\%
дешёвых ресторанов, 60\% бюджетных  и 10\% дорогих. Стандартное отклонение цены
бизнес-ланча составляет 10, 30 и 60 рублей соответственно. В ресторане критик
заказывает только кофе.  Стоимость кофе в дешёвых/бюджетных/дорогих ресторанах
составляет 150, 300 и 600 рублей соответственно, а бюджет  исследования — 15 000
рублей. Какое количество ресторанов каждого типа нужно посетить критику, чтобы
как можно точнее оценить среднюю стоимость бизнес-ланча при заданном бюджетном
ограничении (округлите полученные значения до ближайших целых)? Вычислите дисперсию
соответствующего стратифицированного среднего.

\item Дана случайная выборка $X_1$, \ldots, $X_n$  из некоторого распределения
с математическим ожиданием $\mu$ и дисперсией $\sigma^2$. Даны три оценки $\mu$:
 \[
\hat\mu_1 = (X_1 + X_2)/2, \quad \hat\mu_2 = X_1/4 + (X_2 + \ldots + X_{n-1})/(2n-4)
+ X_n/4, \quad \hat\mu_3 = \bar X
 \]
\begin{enumerate}
\item Какая из оценок является несмещённой?
\item Какая из оценок является более эффективной, чем остальные?
\end{enumerate}

\item Случайный вектор $(X, Y)^T$ имеет двумерное нормальное распределение
с математическим ожиданием  $(1, 2)^T$ и ковариационной матрицей
\[
C=\begin{pmatrix}
1 & -1 \\
-1 & 4
\end{pmatrix}
\]
\begin{enumerate}
\item $\P(X>1)$
\item $\P(2X+Y>2)$
\item $\E(2X+Y|X=2)$, $\Var(2X+Y|X=2)$, $\P(2X+Y>2|X=2)$
\item Сравните вероятности двух предыдущих пунктов, объясните, почему они отличаются.
Являются ли компоненты случайного вектора независимыми?
\end{enumerate}


\item Величины $X_1$, $X_2$ и $X_3$  независимы и стандартно нормально распределены.
Вычислите:
\begin{enumerate}
\item $\P(X_1^2 + X_2^2 > 6)$
\item $\P(X_1^2 / (X_2^2 + X_3^2) > 9.25 )$
\end{enumerate}

\item Дана случайная выборка $X_1$, \ldots, $X_n$ из равномерного распределения
$U[0, \theta]$.
\begin{enumerate}
\item С помощью статистики $X_{(n)}=\max\{X_1, \ldots, X_n \}$ постройте несмещённую
оценку параметра $\theta$  вида $cX_{(n)}$ (укажите значение $c$).
\item Будет ли данная оценка состоятельной?
\item Найдите оценку параметра $\theta$ методом моментов.
\item Какая из двух оценок является более эффективной?
\end{enumerate}

\item Каждый из $n$ биатлонистов одинакового уровня подготовки стреляет по мишеням
до первого промаха.  Пусть $X_i$ — число выстрелов $i$-го биатлониста,
$\P(X_i = x_i)=p^{x_i-1}(1-p)$, где $p$ — вероятность попадания при одном выстреле.
\begin{enumerate}
\item Методом максимального правдоподобия найдите оценку $p$.
\item Методом максимального правдоподобия найдите оценку математического ожидания
числа выстрелов.
\item Сформулируйте определения несмещенности, состоятельности и эффективности
оценок, и проверьте выполнение данных свойств для найденной в предыдущем пункте
оценки математического ожидания.
\end{enumerate}
\end{enumerate}


\newpage
\subsection[2013-2014]{\hyperref[sec:sol_kr_03_2013_2014]{2013-2014}}
\label{sec:kr_03_2013_2014}


Вычислите константы $B_1=\{\text{Цифра, соответствующая первой букве}$
Вашей\\ фамилии$\}$ и $B_2=\{\text{Цифра, соответствующая первой букве}$
 Вашего имени$\}$.\\
Уровень значимости для всех проверяемых гипотез $0.0\alpha$, уровень доверия
для всех доверительных интервалов $(1-0.0\alpha)$, где  $\alpha = 1+
\{\text{остаток от деления } B_1 \text{ на }  5\}$.\\

\begin{center}
\begin{tabular}{|c|c|c|c|c|c|c|c|c|c|c|c|c|c|}
\hline  А & Б & В & Г & Д & Е & Ж & З & И & К & Л & М & Н & О \\
\hline 1 & 2 & 3 & 4 & 5 & 6 & 7 & 8 & 9 & 10 & 11 & 12 & 13 & 14 \\
\hline  П & Р & С & Т & У & Ф & Х & Ц & Ч & Ш & Щ & Э & Ю & Я \\
\hline 15& 16  &  17 &  18&  19&  20&  21& 22 & 23 &  24& 25 & 26  &  27 & 28 \\
\hline
\end{tabular}
\end{center}

\begin{enumerate}
\item Вес упаковки с лекарством является нормальной случайной величиной с
неизвестными математическим ожиданием  $\mu$ и дисперсией $\sigma^2$. Контрольное
взвешивание $(10+B_1)$ упаковок показало, что выборочное среднее  $\bar{X} =
(50+B_2)$, а  несмещенная оценка дисперсии равна $B_1\cdot B_2$. Постройте
доверительные интервалы для математического ожидания и дисперсии веса упаковки
(для дисперсии односторонний с нижней границей).

\item Экзамен принимают два преподавателя, случайным образом выбирая студентов.
По выборкам из 85 и 100 наблюдений, выборочные доли не сдавших экзамен студентов
составили соответственно $\frac{1}{B_1+1}$ и $\frac{1}{B_2+1}$ . Можно ли утверждать,
что преподаватели предъявляют к студентам одинаковый уровень требований? Вычислите
минимальный уровень значимости, при котором основная гипотеза (уровень требований одинаков)
отвергается (p-value).

\item Даны независимые выборки доходов выпускников двух ведущих экономических вузов
A и B, по $(10+B_1)$ и $(10+B_2)$ выпускников соответственно: $\bar{X}_A=45$,
$\hat{\sigma}_A=5$, $\bar{X}_B=50$, $\hat{\sigma}_B=6$.
Предполагая, что распределение доходов подчиняется нормальному закону, проверьте
гипотезу об отсутствии преимуществ выпускников вуза B.

\item 	По выборке независимых одинаково распределенных случайных величин
$X_1,\dots,X_n$ с функцией плотности $f(x)=\frac{1}{\theta} x^{-1+\frac{1}{\theta}}$,
$x\in(0, 1)$, найдите оценки максимального правдоподобия параметра $\theta$.
Сформулируйте определения свойств несмещенности, состоятельности и эффективности
и проверьте, выполняются ли эти свойства для найденной оценки.
\end{enumerate}
\underline{Примечание.} В помощь несчастным, забывшим формулу интегрирования по
частям и таблицу неопределенных интегралов, или просто ленивым студентам:
\[
\int\limits_{0}^1 t^\alpha \ln (t) dt = -\frac{1}{(\alpha+1)^2}
\]
