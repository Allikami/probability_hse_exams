\section{Решения контрольной номер 3}

\subsection[2017-2018]{\hyperref[sec:kr_03_2017_2018]{2017-2018}}
\label{sec:sol_kr_03_2017_2018}


\begin{enumerate}
\item[5.]
\begin{enumerate}
\item $L(X_1, \ldots, X_n, \mu) = \prod_{i=1}^n \frac{1}{\sqrt{2\pi}} e^{-\frac{1}{2}\sum_{i=1}^n (X_i - \mu)^2}$
\item $\hat\mu_{ML} = \bar X$
\item $\E(\hat\mu_{ML}) = \E(\bar X) = \mu \Rightarrow$ оценка несмещённая

$\plim \hat \mu_{ML} = \plim \bar X = \mu \Rightarrow$ оценка состоятельная
\item $I(\mu) = n$
\item $\Var(\theta) \geq \frac{1}{I(\theta)}$
\item $\Var(\hat \mu_{ML}) = \frac{1}{n}$, так как неравенство Рао-Крамера выполнено
как равенство, оценка является эффективной.
\item $\theta = \E(X^2) = \Var(X) + \mu^2 = 1 + \mu^2$.
Тогда в силу инвариантности оценок максимального правдоподобия: $\hat\theta_{ML} = 1 + \hat\mu^2$.
\item $\E(\hat \theta_{ML}) = 1 + \E(\hat \mu^2) = 1 + \E((\bar X)^2)$

Пользуясь соотношением $\E((\bar X)^2) = \Var(\bar X) + (\E(\bar X))^2$,
получим: $\E(\hat \theta_{ML}) = 1 + \frac{1}{n} + \mu^2$, то есть оценка смещена.

Однако, $\lim_{n \to \infty} \left(1 + \frac{1}{n} + \mu^2\right) = 1 + \mu^2$, значит,
оценка асимптотически несмещена.
\item $\hat \theta_{ML} \approx 1 + \mu^2 + 2\mu(\hat \mu - \mu)$

$\Var(\hat \theta_{ML}) \approx 4 \mu^2 \Var(\hat \mu) = \frac{4 \mu^2}{n}$
\item Так как $\hat \theta_{ML}$ асимптотически несмещена, то для проверки
состоятельности достаточно показать, что
$\Var(\hat \theta_{ML}) = \frac{4\mu^2}{n} \to_{n \to \infty} 0$.
\end{enumerate}
\item[6.]
\begin{enumerate}
\item $\E(X_1) = \int_{0}^{\theta} \frac{2}{\theta^2}(\theta - x)x dx = \frac{\theta}{3}$

$\frac{\hat \theta_{MM}}{3} = \bar X \Rightarrow \hat \theta_{MM} = 3 \bar X$
\item Оценка $\hat \theta$ состоятельна. если $\plim \hat \theta_n = \theta$.

$\plim \hat \theta_{MM} = \plim 3 \bar X = 3 \E(X_1) = \theta \Rightarrow$ оценка состоятельна.
\end{enumerate}
\item[7.]
\begin{enumerate}
\item $\E\left(\frac{X_1 + X_2 + X_3}{3} \right) = \frac{1}{3} \cdot 3 \E(X_1) = 132.5$

$\Var\left(\frac{X_1 + X_2 + X_3}{3} \right) = \frac{1}{9} \Var(X_1 + X_2 + X_3) =
\frac{1}{9} (\Var(X_1) + \Var(X_2) + \Var(X_3) + 2 \Cov(X_1, X_2) + 2\Cov(X_1, X_3) + 2\Cov(X_2, X_3)) =
\frac{1}{9}(3\Var(X_1) + 6\Cov(X_1,X_2))$

$\Var(X_1) = \E(X_1^2) - \E(X_1)^2 = \frac{1}{4} \cdot 30^2 - \frac{1}{4} \cdot 500^2 - 132.5^2 = 45168.75$

$\Cov(X_1, X_1 + \ldots + X_4 = \Var(X_1) + 3\Cov(X_1,X_2) = 0 \Rightarrow \Cov(X_1,X_2) = -\frac{45168.75}{3} = -15056.25$

$\Var\left(\frac{X_1 + X_2 + X_3}{3} \right) = 5018.75$

\item $3/4$
\end{enumerate}
\item[8.] $\Delta_i = X_i - Y_i \sim \cN(\mu_x - \mu_y, \sigma^2)$

$\bar X = 297.5$, $\bar Y = 247.5$, $\bar \Delta = \bar X - \bar Y = 50$

$\hat \sigma^2 = \frac{1}{n-1} \sum_{i=1}^n (\Delta_i - \bar \Delta)^2 = 18266.(6)$.

Критическое значение — $t_{0.975, 3} = 3.182$ и доверительный интервал имеет вид:
\[
50 - 3.182 \sqrt{\frac{18266.(6)}{3}} < \mu_x - \mu_y < 50 + 3.182 \sqrt{\frac{18266.(6)}{3}}
\]
Так как $0$ входит в доверительный интервал, нельзя отвергнуть предположение о равенстве расхожов.
\item[9.]
\begin{enumerate}
\item $0.7 - 1.96 \sqrt{\frac{0.7 \cdot 0.3}{60}} < p < 0.7 + 1.96 \sqrt{\frac{0.7 \cdot 0.3}{60}} $
\item Да, так как $0.7667$ входит в доверительный интервал.
\item $\P(|p - \hat p| \leq 0.01) = 0.95$

$\P\left(\frac{|0.7 - p|}{\sqrt{\frac{0.7 \cdot 0.3}{n}}} < \frac{0.01}{\sqrt{\frac{0.7 \cdot 0.3}{n}}} \right) = 0.95$

$\frac{0.01}{\sqrt{\frac{0.7 \cdot 0.3}{n}}} = 1.96 \Rightarrow n \approx 8068$
\end{enumerate}
\end{enumerate}
