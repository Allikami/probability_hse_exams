\section{Контрольная работа 1. ИП}



\subsection[2017-2018]{\hyperref[sec:sol_kr_01_ip_2017_2018]{2017-2018}}
\label{sec:kr_01_ip_2017_2018}



Ровно 272 года назад императрица Елизавета повелела завезти во дворцы котов для ловли мышей.


\begin{enumerate}

\item В отсутствии кота Леопольда мыши Белый и Серый подкидывают по очереди игральный додекаэдр
%\footnote{Леопольд подсказывает по случаю праздника, что у додекаэдра 12 граней :)}
.
Сыр достаётся тому, кто первым выкинет число 6. Начинает подкидывать Белый.

\begin{enumerate}
  \item Какова вероятность того, что сыр достанется Белому?
  \item Сколько в среднем бросков продолжается игра?
  \item Какова дисперсия числа бросков?
\end{enumerate}

\item Микки Маус, Белый и Серый решили устроить труэль из любви к мышки Мии. Сначала стреляет Микки, затем Белый, затем Серый, затем снова Микки и так до тех пор, пока в живых не останется только один.

Прошлые данные говорят о том, что Микки попадает с вероятностью $1/3$, Белый — с вероятностью $2/3$, а Серый стреляет без промаха.

Найдите оптимальную стратегию каждого мыша.

\item Микки Маус, Белый и Серый пойманый злобным котом Леопольдом до начала труэли. И теперь Леопольд будет играть с ними в странную игру.

В комнате три закрытых внешне неотличимых коробки: с золотом, серебром и платиной. Общаться после начала игры мыши не могут, но могут заранее договориться о стратегии.

Правила игры таковы. Кот Леопольд будет заводить мышей в комнату по очереди. Каждый из мышей может открыть
две коробки по своему выбору. Перед следующим мышом коробки закрываются.

Если Микки откроет коробку с золотом, Белый
— с серебром, а Серый — с платиной, то они выигрывают. Если
хотя бы один из мышей не найдёт свой металл, то Леопольд их съест.
\begin{enumerate}
\item Какова оптимальная стратегия?
\item Какова вероятность выигрыша при использовании оптимальной стратегии?
\end{enumerate}

\item Накануне войны Жестокий Тиран Мышь очень большой страны издал указ. Отныне за каждого новорождённого мыша-мальчика семья получает денежную премию, но если в семье рождается вторая мышка-девочка, то всю семью убивают. Бедные жители страны запуганы и остро нуждаются в деньгах, поэтому в каждой семье мыши будут появляться до тех пор, пока не родится первая мышка-девочка.

\begin{enumerate}
  \item Каким будет среднее число детей в мышиной семье?
  \item Какой будет доля мышей-мальчиков в стране?
  \item Какой будет средняя доля мышей-мальчиков в случайной семье?
  \item Сколько в среднем мышей-мальчиков в случайно выбираемой семье?
\end{enumerate}

\item Вальяжный кот Василий положил на счёт в банке на Гаити один гурд. Сумма на счету растёт непрерывно с постоянной ставкой в течение очень длительного промежутка времени. В случайный момент этого промежутка кот Василий закрывает свой вклад.

Каков закон распределения первой цифры полученной Василием суммы?

\end{enumerate}
